\documentclass[a4paper,10pt]{article}
\usepackage{eumat}

\begin{document}
\begin{eulernotebook}
\begin{eulercomment}
\begin{eulercomment}
\eulerheading{EMT untuk Perhitungan Aljabar}
\begin{eulercomment}
Pada notebook ini Anda belajar menggunakan EMT untuk melakukan
berbagai perhitungan terkait dengan materi atau topik dalam Aljabar.
Kegiatan yang harus Anda lakukan adalah sebagai berikut:

- Membaca secara cermat dan teliti notebook ini;\\
- Menerjemahkan teks bahasa Inggris ke bahasa Indonesia;\\
- Mencoba contoh-contoh perhitungan (perintah EMT) dengan cara
meng-ENTER setiap perintah EMT yang ada (pindahkan kursor ke baris
perintah)\\
- Jika perlu Anda dapat memodifikasi perintah yang ada dan memberikan
keterangan/penjelasan tambahan terkait hasilnya.\\
- Menyisipkan baris-baris perintah baru untuk mengerjakan soal-soal
Aljabar dari file PDF yang saya berikan;\\
- Memberi catatan hasilnya.\\
- Jika perlu tuliskan soalnya pada teks notebook (menggunakan format
LaTeX).\\
- Gunakan tampilan hasil semua perhitungan yang eksak atau simbolik
dengan format LaTeX. (Seperti contoh-contoh pada notebook ini.)

\end{eulercomment}
\eulersubheading{Contoh pertama}
\begin{eulercomment}
Menyederhanakan bentuk aljabar:

\end{eulercomment}
\begin{eulerformula}
\[
6x^{-3}y^5\times -7x^2y^{-9}
\]
\end{eulerformula}
\begin{eulercomment}
\end{eulercomment}
\begin{eulerprompt}
>$&6*x^(-2)*y^4*-7*x^2*y^(-7)
\end{eulerprompt}
\begin{eulercomment}
Menjabarkan:

\end{eulercomment}
\begin{eulerformula}
\[
(6x^{-3}+y^5)(-7x^2-y^{-9})
\]
\end{eulerformula}
\begin{eulerprompt}
>$&showev('expand((8*x^(-3)+y^4)*(-7*x^2-y^(-7))))
\end{eulerprompt}
\begin{eulercomment}
\end{eulercomment}
\eulersubheading{Baris perintah}
\begin{eulercomment}
Baris perintah Euler terdiri dari satu atau beberapa perintah Euler
yang diikuti oleh titik koma ";" atau koma ",". Titik koma mencegah
pencetakan hasil. Koma setelah perintah terakhir dapat dihilangkan.

Baris perintah berikut hanya akan mencetak hasil ekspresi, bukan
perintah penugasan atau format.
\end{eulercomment}
\begin{eulerprompt}
>r:=2; h:=4; pi*r^2*h/3
\end{eulerprompt}
\begin{euleroutput}
  16.7551608191
\end{euleroutput}
\begin{eulercomment}
Perintah harus dipisahkan dengan spasi. Baris perintah berikut
mencetak dua hasilnya.
\end{eulercomment}
\begin{eulerprompt}
>pi*2*r*h, %+2*pi*r*h // Ingat tanda % menyatakan hasil perhitungan terakhir sebelumnya
\end{eulerprompt}
\begin{euleroutput}
  50.2654824574
  100.530964915
\end{euleroutput}
\begin{eulercomment}
Baris perintah dieksekusi sesuai urutan pengguna menekan tombol enter.
Jadi Anda akan mendapatkan nilai baru setiap kali Anda mengeksekusi
baris kedua.
\end{eulercomment}
\begin{eulerprompt}
>x := 1;
>x := cos(x) // nilai cosinus (x dalam radian)
\end{eulerprompt}
\begin{euleroutput}
  0.540302305868
\end{euleroutput}
\begin{eulerprompt}
>x := cos(x)
\end{eulerprompt}
\begin{euleroutput}
  0.857553215846
\end{euleroutput}
\begin{eulercomment}
Jika dua baris dihubungkan dengan "..." kedua baris akan selalu
dieksekusi secara bersamaan.
\end{eulercomment}
\begin{eulerprompt}
>x := 1.5; ...
>x := (x+2/x)/2, x := (x+2/x)/2, x := (x+2/x)/2, 
\end{eulerprompt}
\begin{euleroutput}
  1.41666666667
  1.41421568627
  1.41421356237
\end{euleroutput}
\begin{eulercomment}
Ini juga merupakan cara yang baik untuk membagi perintah yang panjang
menjadi dua baris atau lebih. Anda dapat menekan Ctrl+Return untuk
membagi baris menjadi dua pada posisi kursor saat ini, atau Ctlr+Back
untuk menggabungkan baris-baris tersebut.

Untuk melipat semua garis multi, tekan Ctrl+L. Kemudian, garis-garis
berikutnya hanya akan terlihat jika salah satunya menjadi fokus. Untuk
melipat satu garis multi, mulailah baris pertama dengan "\%+ ".
\end{eulercomment}
\begin{eulerprompt}
>%+ x=4+5; ...
\end{eulerprompt}
\begin{eulercomment}
Baris yang dimulai dengan \%\% tidak akan terlihat sama sekali.
\end{eulercomment}
\begin{euleroutput}
  81
\end{euleroutput}
\begin{eulercomment}
Euler mendukung perulangan dalam baris perintah, asalkan dapat
dimasukkan ke dalam satu baris atau beberapa baris. Dalam program,
pembatasan ini tentu saja tidak berlaku. Untuk informasi lebih lanjut,
lihat pengantar berikut.
\end{eulercomment}
\begin{eulerprompt}
>x=1; for i=1 to 5; x := (x+2/x)/2, end; // menghitung akar 2
\end{eulerprompt}
\begin{euleroutput}
  1.5
  1.41666666667
  1.41421568627
  1.41421356237
  1.41421356237
\end{euleroutput}
\begin{eulercomment}
Tidak apa-apa menggunakan beberapa baris. Pastikan baris diakhiri
dengan "...".
\end{eulercomment}
\begin{eulerprompt}
>x := 1.5; // comments go here before the ...
>repeat xnew:=(x+2/x)/2; until xnew~=x; ...
>   x := xnew; ...
>end; ...
>x,
\end{eulerprompt}
\begin{euleroutput}
  1.41421356237
\end{euleroutput}
\begin{eulercomment}
Struktur kondisional juga berfungsi.
\end{eulercomment}
\begin{eulerprompt}
>if E^pi>pi^E; then "Thought so!", endif;
\end{eulerprompt}
\begin{euleroutput}
  Thought so!
\end{euleroutput}
\begin{eulercomment}
Saat Anda menjalankan perintah, kursor dapat berada di posisi mana pun
di baris perintah. Anda dapat kembali ke perintah sebelumnya atau
melompat ke perintah berikutnya dengan tombol panah. Atau Anda dapat
mengklik bagian komentar di atas perintah untuk membuka perintah
tersebut.

Saat Anda menggerakkan kursor di sepanjang baris, pasangan tanda
kurung buka dan tutup akan disorot. Perhatikan juga baris status.
Setelah tanda kurung buka fungsi sqrt(), baris status akan menampilkan
teks bantuan untuk fungsi tersebut. Jalankan perintah dengan tombol
return.
\end{eulercomment}
\begin{eulerprompt}
>sqrt(sin(10°)/cos(20°))
\end{eulerprompt}
\begin{euleroutput}
  0.429875017772
\end{euleroutput}
\begin{eulercomment}
Untuk melihat bantuan untuk perintah terbaru, buka jendela bantuan
dengan F1. Di sana, Anda dapat memasukkan teks untuk dicari. Pada
baris kosong, bantuan untuk jendela bantuan akan ditampilkan. Anda
dapat menekan escape untuk menghapus baris, atau untuk menutup jendela
bantuan.

Anda dapat mengklik dua kali pada perintah apa pun untuk membuka
bantuan untuk perintah ini. Coba klik dua kali perintah exp di bawah
ini pada baris perintah.
\end{eulercomment}
\begin{eulerprompt}
>exp(log(2.5))
\end{eulerprompt}
\begin{euleroutput}
  2.5
\end{euleroutput}
\begin{eulercomment}
Anda juga dapat menyalin dan menempel di Euler. Gunakan Ctrl-C dan
Ctrl-V untuk ini. Untuk menandai teks, seret tetikus atau gunakan
shift bersamaan dengan tombol kursor apa pun. Selain itu, Anda dapat
menyalin tanda kurung yang disorot.
\end{eulercomment}
\begin{eulercomment}

\end{eulercomment}
\eulersubheading{Sintaks Dasar}
\begin{eulercomment}
Euler mengetahui fungsi matematika yang umum. Seperti yang telah Anda
lihat di atas, fungsi trigonometri bekerja dalam radian atau derajat.
Untuk mengonversi ke derajat, tambahkan simbol derajat (dengan tombol
F7) ke nilai, atau gunakan fungsi rad(x). Fungsi akar kuadrat disebut
sqrt di Euler. Tentu saja, x\textasciicircum{}(1/2) juga memungkinkan.

Untuk mengatur variabel, gunakan "=" atau ":=". Demi kejelasan,
pengantar ini menggunakan bentuk yang terakhir. Spasi tidak menjadi
masalah. Namun, spasi di antara perintah diharapkan.

Beberapa perintah dalam satu baris dipisahkan dengan "," atau ";".
Titik koma menghilangkan keluaran perintah. Di akhir baris perintah,
"," diasumsikan, jika ";" tidak ada.
\end{eulercomment}
\begin{eulerprompt}
>g:=9.81; t:=2.5; 1/2*g*t^2
\end{eulerprompt}
\begin{euleroutput}
  30.65625
\end{euleroutput}
\begin{eulercomment}
EMT menggunakan sintaks pemrograman untuk ekspresi. Untuk memasukkan

\end{eulercomment}
\begin{eulerformula}
\[
e^2 \cdot \left( \frac{1}{3+4 \log(0.6)}+\frac{1}{7} \right)
\]
\end{eulerformula}
\begin{eulercomment}
Anda harus menetapkan tanda kurung yang benar dan menggunakan / untuk
pecahan. Perhatikan tanda kurung yang disorot untuk mendapatkan
bantuan. Perhatikan bahwa konstanta Euler e diberi nama E dalam EMT.
\end{eulercomment}
\begin{eulerprompt}
>E^2*(1/(3+4*log(0.6))+1/7)
\end{eulerprompt}
\begin{euleroutput}
  8.77908249441
\end{euleroutput}
\begin{eulercomment}
Untuk menghitung ekspresi rumit seperti

\end{eulercomment}
\begin{eulerformula}
\[
\left(\frac{\frac17 + \frac18 + 2}{\frac13 + \frac12}\right)^2 \pi
\]
\end{eulerformula}
\begin{eulercomment}
Anda perlu memasukkannya dalam formulir baris.
\end{eulercomment}
\begin{eulerprompt}
>((1/7 + 1/8 + 2) / (1/3 + 1/2))^2 * pi
\end{eulerprompt}
\begin{euleroutput}
  23.2671801626
\end{euleroutput}
\begin{eulercomment}
Letakkan tanda kurung di sekitar sub-ekspresi yang perlu dihitung
terlebih dahulu dengan hati-hati. EMT membantu Anda dengan menyorot
ekspresi yang diakhiri tanda kurung tutup. Anda juga harus memasukkan
nama "pi" untuk huruf Yunani pi.

Hasil perhitungan ini adalah angka floating point. Secara default,
angka ini dicetak dengan akurasi sekitar 12 digit. Pada baris perintah
berikut, kita juga mempelajari cara merujuk ke hasil sebelumnya dalam
baris yang sama.
\end{eulercomment}
\begin{eulerprompt}
>1/3+1/7, fraction %
\end{eulerprompt}
\begin{euleroutput}
  0.47619047619
  10/21
\end{euleroutput}
\begin{eulercomment}
Perintah Euler dapat berupa ekspresi atau perintah primitif. Ekspresi
terdiri dari operator dan fungsi. Jika perlu, ekspresi harus berisi
tanda kurung untuk memaksakan urutan eksekusi yang benar. Jika ragu,
sebaiknya gunakan tanda kurung. Perhatikan bahwa EMT menampilkan tanda
kurung buka dan tutup saat mengedit baris perintah.
\end{eulercomment}
\begin{eulerprompt}
>(cos(pi/4)+1)^3*(sin(pi/4)+1)^2
\end{eulerprompt}
\begin{euleroutput}
  14.4978445072
\end{euleroutput}
\begin{eulercomment}
Operator numerik Euler meliputi:

\end{eulercomment}
\begin{eulerttcomment}
 + unary or operator plus
 - unary or operator minus
 *, /
 . the matrix product
 a^b power for positive a or integer b (a**b works too)
 n! the factorial operator
\end{eulerttcomment}
\begin{eulercomment}

and many more.

Berikut ini beberapa fungsi yang mungkin Anda perlukan. Masih banyak
lagi.

\end{eulercomment}
\begin{eulerttcomment}
 sin,cos,tan,atan,asin,acos,rad,deg
 log,exp,log10,sqrt,logbase
 bin,logbin,logfac,mod,floor,ceil,round,abs,sign
 conj,re,im,arg,conj,real,complex
 beta,betai,gamma,complexgamma,ellrf,ellf,ellrd,elle
 bitand,bitor,bitxor,bitnot
\end{eulerttcomment}
\begin{eulercomment}

Some commands have aliases, e.g. ln for log.
\end{eulercomment}
\begin{eulerprompt}
>ln(E^2), arctan(tan(0.5))
\end{eulerprompt}
\begin{euleroutput}
  2
  0.5
\end{euleroutput}
\begin{eulerprompt}
>sin(30°)
\end{eulerprompt}
\begin{euleroutput}
  0.5
\end{euleroutput}
\begin{eulercomment}
Pastikan untuk menggunakan tanda kurung (kurung bundar), jika ada
keraguan tentang urutan eksekusi! Berikut ini tidak sama dengan
(2\textasciicircum{}3)\textasciicircum{}4, yang merupakan default untuk 2\textasciicircum{}3\textasciicircum{}4 dalam EMT (beberapa sistem
numerik melakukannya dengan cara lain).
\end{eulercomment}
\begin{eulerprompt}
>2^3^4, (2^3)^4, 2^(3^4)
\end{eulerprompt}
\begin{euleroutput}
  2.41785163923e+24
  4096
  2.41785163923e+24
\end{euleroutput}
\begin{eulercomment}
Contoh Soal
\end{eulercomment}
\begin{eulerprompt}
>tan(90°)
\end{eulerprompt}
\begin{euleroutput}
  1.63312393532e+16
\end{euleroutput}
\begin{eulerprompt}
>10^5*9
\end{eulerprompt}
\begin{euleroutput}
  900000
\end{euleroutput}
\begin{eulercomment}
Tulis ekspresi yang setara tanpa negatif\\
\end{eulercomment}
\begin{eulerttcomment}
 eksponen dari fungsi berikut :
\end{eulerttcomment}
\begin{eulerformula}
\[
(4x^2y)^2
\]
\end{eulerformula}
\begin{eulerprompt}
>$&(4*x^2*y)^2
\end{eulerprompt}
\begin{eulercomment}
\end{eulercomment}
\eulersubheading{Bilangan Asli}
\begin{eulercomment}
Tipe data utama dalam Euler adalah bilangan riil. Bilangan riil
direpresentasikan dalam format IEEE dengan akurasi sekitar 16 digit
desimal.
\end{eulercomment}
\begin{eulerprompt}
>longest 1/3
\end{eulerprompt}
\begin{euleroutput}
       0.3333333333333333 
\end{euleroutput}
\begin{eulercomment}
Representasi ganda internal membutuhkan 8 byte.
\end{eulercomment}
\begin{eulerprompt}
>printdual(1/3)
\end{eulerprompt}
\begin{euleroutput}
  1.0101010101010101010101010101010101010101010101010101*2^-2
\end{euleroutput}
\begin{eulerprompt}
>printhex(1/3)
\end{eulerprompt}
\begin{euleroutput}
  5.5555555555554*16^-1
\end{euleroutput}
\eulersubheading{String}
\begin{eulercomment}
String dalam Euler didefinisikan dengan "...".
\end{eulercomment}
\begin{eulerprompt}
>"A string can contain anything."
\end{eulerprompt}
\begin{euleroutput}
  A string can contain anything.
\end{euleroutput}
\begin{eulercomment}
String dapat dirangkai dengan \textbar{} atau dengan +. Ini juga berlaku untuk
angka, yang dalam kasus tersebut diubah menjadi string.
\end{eulercomment}
\begin{eulerprompt}
>"The area of the circle with radius " + 2 + " cm is " + pi*4 + " cm^2."
\end{eulerprompt}
\begin{euleroutput}
  The area of the circle with radius 2 cm is 12.5663706144 cm^2.
\end{euleroutput}
\begin{eulercomment}
Fungsi cetak juga mengonversi angka menjadi string. Fungsi ini dapat
mengambil sejumlah digit dan sejumlah tempat (0 untuk keluaran padat),
dan optimalnya satu unit.
\end{eulercomment}
\begin{eulerprompt}
>"Golden Ratio : " + print((1+sqrt(5))/2,5,0)
\end{eulerprompt}
\begin{euleroutput}
  Golden Ratio : 1.61803
\end{euleroutput}
\begin{eulercomment}
Ada string khusus none, yang tidak dicetak. String ini dikembalikan
oleh beberapa fungsi, ketika hasilnya tidak penting. (Dikembalikan
secara otomatis, jika fungsi tersebut tidak memiliki pernyataan
return.)
\end{eulercomment}
\begin{eulerprompt}
>none
\end{eulerprompt}
\begin{eulercomment}
Untuk mengubah string menjadi angka, cukup evaluasi string tersebut.
Ini juga berlaku untuk ekspresi (lihat di bawah).
\end{eulercomment}
\begin{eulerprompt}
>"1234.5"()
\end{eulerprompt}
\begin{euleroutput}
  1234.5
\end{euleroutput}
\begin{eulercomment}
Untuk mendefinisikan vektor string, gunakan notasi vektor [...]
\end{eulercomment}
\begin{eulerprompt}
>v:=["affe","charlie","bravo"]
\end{eulerprompt}
\begin{euleroutput}
  affe
  charlie
  bravo
\end{euleroutput}
\begin{eulercomment}
Vektor string kosong dilambangkan dengan [none]. Vektor string dapat
dirangkai.
\end{eulercomment}
\begin{eulerprompt}
>w:=[none]; w|v|v
\end{eulerprompt}
\begin{euleroutput}
  affe
  charlie
  bravo
  affe
  charlie
  bravo
\end{euleroutput}
\begin{eulercomment}
String dapat berisi karakter Unicode. Secara internal, string ini
berisi kode UTF-8. Untuk membuat string seperti itu, gunakan u"..."
dan salah satu entitas HTML.

String Unicode dapat dirangkai seperti string lainnya.
\end{eulercomment}
\begin{eulerprompt}
>u"&alpha; = " + 45 + u"&deg;" // pdfLaTeX mungkin gagal menampilkan secara benar
\end{eulerprompt}
\begin{euleroutput}
  α = 45°
\end{euleroutput}
\begin{eulercomment}
I
\end{eulercomment}
\begin{eulercomment}
Dalam komentar, entitas yang sama seperti α, β dll. dapat
digunakan. Ini mungkin merupakan alternatif cepat untuk Latex.
(Rincian lebih lanjut ada di komentar di bawah).
\end{eulercomment}
\begin{eulercomment}
Ada beberapa fungsi untuk membuat atau menganalisis string unicode.
Fungsi strtochar() akan mengenali string Unicode dan menerjemahkannya
dengan benar..
\end{eulercomment}
\begin{eulerprompt}
>v=strtochar(u"&Auml; is a German letter")
\end{eulerprompt}
\begin{euleroutput}
  [196,  32,  105,  115,  32,  97,  32,  71,  101,  114,  109,  97,  110,
  32,  108,  101,  116,  116,  101,  114]
\end{euleroutput}
\begin{eulercomment}
Hasilnya adalah vektor angka Unicode. Fungsi kebalikannya adalah
chartoutf().
\end{eulercomment}
\begin{eulerprompt}
>v[1]=strtochar(u"&Uuml;")[1]; chartoutf(v)
\end{eulerprompt}
\begin{euleroutput}
  Ü is a German letter
\end{euleroutput}
\begin{eulercomment}
Fungsi utf() dapat menerjemahkan string dengan entitas dalam variabel
menjadi string Unicode.
\end{eulercomment}
\begin{eulerprompt}
>s="We have &alpha;=&beta;."; utf(s) // pdfLaTeX mungkin gagal menampilkan secara benar
\end{eulerprompt}
\begin{euleroutput}
  We have α=β.
\end{euleroutput}
\begin{eulercomment}
Dimungkinkan juga untuk menggunakan entitas numerik.
\end{eulercomment}
\begin{eulerprompt}
>u"&#196;hnliches"
\end{eulerprompt}
\begin{euleroutput}
  Ähnliches
\end{euleroutput}
\eulersubheading{Nilai Boolean}
\begin{eulercomment}
Nilai Boolean direpresentasikan dengan 1=benar atau 0=salah dalam
Euler. String dapat dibandingkan, seperti halnya angka.
\end{eulercomment}
\begin{eulerprompt}
>2<1, "apel"<"banana"
\end{eulerprompt}
\begin{euleroutput}
  0
  1
\end{euleroutput}
\begin{eulercomment}
"dan" adalah operator "\&\&" dan "atau" adalah operator "\textbar{}\textbar{}", seperti
dalam bahasa C. (Kata "dan" dan "atau" hanya dapat digunakan dalam
kondisi "jika".)
\end{eulercomment}
\begin{eulerprompt}
>2<E && E<3
\end{eulerprompt}
\begin{euleroutput}
  1
\end{euleroutput}
\begin{eulercomment}
Operator Boolean mematuhi aturan bahasa matriks.
\end{eulercomment}
\begin{eulerprompt}
>(1:10)>5, nonzeros(%)
\end{eulerprompt}
\begin{euleroutput}
  [0,  0,  0,  0,  0,  1,  1,  1,  1,  1]
  [6,  7,  8,  9,  10]
\end{euleroutput}
\begin{eulercomment}
Anda dapat menggunakan fungsi nonzeros() untuk mengekstrak elemen
tertentu dari sebuah vektor. Dalam contoh ini, kami menggunakan
kondisional isprime(n).
\end{eulercomment}
\begin{eulerprompt}
>N=2|3:2:99 // N berisi elemen 2 dan bilangan2 ganjil dari 3 s.d. 99
\end{eulerprompt}
\begin{euleroutput}
  [2,  3,  5,  7,  9,  11,  13,  15,  17,  19,  21,  23,  25,  27,  29,
  31,  33,  35,  37,  39,  41,  43,  45,  47,  49,  51,  53,  55,  57,
  59,  61,  63,  65,  67,  69,  71,  73,  75,  77,  79,  81,  83,  85,
  87,  89,  91,  93,  95,  97,  99]
\end{euleroutput}
\begin{eulerprompt}
>N[nonzeros(isprime(N))] //pilih anggota2 N yang prima
\end{eulerprompt}
\begin{euleroutput}
  [2,  3,  5,  7,  11,  13,  17,  19,  23,  29,  31,  37,  41,  43,  47,
  53,  59,  61,  67,  71,  73,  79,  83,  89,  97]
\end{euleroutput}
\eulersubheading{Format Keluaran}
\begin{eulercomment}
Format keluaran default EMT mencetak 12 digit. Untuk memastikan bahwa
kita melihat default, kita mengatur ulang formatnya.
\end{eulercomment}
\begin{eulerprompt}
>defformat; pi
\end{eulerprompt}
\begin{euleroutput}
  3.14159265359
\end{euleroutput}
\begin{eulercomment}
Secara internal, EMT menggunakan standar IEEE untuk angka ganda dengan
sekitar 16 digit desimal. Untuk melihat jumlah digit lengkap, gunakan
perintah "longestformat", atau kami menggunakan operator "longest"
untuk menampilkan hasil dalam format terpanjang.
\end{eulercomment}
\begin{eulerprompt}
>longest pi
\end{eulerprompt}
\begin{euleroutput}
        3.141592653589793 
\end{euleroutput}
\begin{eulercomment}
Berikut adalah representasi heksadesimal internal dari angka ganda.
\end{eulercomment}
\begin{eulerprompt}
>printhex(pi)
\end{eulerprompt}
\begin{euleroutput}
  3.243F6A8885A30*16^0
\end{euleroutput}
\begin{eulercomment}
Format keluaran dapat diubah secara permanen dengan perintah format.
\end{eulercomment}
\begin{eulerprompt}
>format(12,5); 1/3, pi, sin(1)
\end{eulerprompt}
\begin{euleroutput}
      0.33333 
      3.14159 
      0.84147 
\end{euleroutput}
\begin{eulercomment}
Format defaultnya adalah(12).
\end{eulercomment}
\begin{eulerprompt}
>format(12); 1/3
\end{eulerprompt}
\begin{euleroutput}
  0.333333333333
\end{euleroutput}
\begin{eulercomment}
Fungsi seperti "shortestformat", "shortformat", "longformat" bekerja
untuk vektor dengan cara berikut.
\end{eulercomment}
\begin{eulerprompt}
>shortestformat; random(3,8)
\end{eulerprompt}
\begin{euleroutput}
    0.66    0.2   0.89   0.28   0.53   0.31   0.44    0.3 
    0.28   0.88   0.27    0.7   0.22   0.45   0.31   0.91 
    0.19   0.46  0.095    0.6   0.43   0.73   0.47   0.32 
\end{euleroutput}
\begin{eulercomment}
Format default untuk skalar adalah format(12). Namun, ini dapat
diubah.
\end{eulercomment}
\begin{eulerprompt}
>setscalarformat(5); pi
\end{eulerprompt}
\begin{euleroutput}
  3.1416
\end{euleroutput}
\begin{eulercomment}
Fungsi "longestformat" juga mengatur format skalar.
\end{eulercomment}
\begin{eulerprompt}
>longestformat; pi
\end{eulerprompt}
\begin{euleroutput}
  3.141592653589793
\end{euleroutput}
\begin{eulercomment}
Sebagai referensi, berikut adalah daftar format output yang paling
penting.

format terpendek format pendek format panjang, format terpanjang\\
format(panjang,digit) format baik(panjang)\\
fracformat (panjang)\\
mengubah bentuk


Keakuratan internal EMT adalah sekitar 16 tempat desimal, yang
merupakan standar IEEE. Angka disimpan dalam format internal ini.

Namun, format output EMT dapat diatur dengan cara yang fleksibel.
\end{eulercomment}
\begin{eulerprompt}
>longestformat; pi,
\end{eulerprompt}
\begin{euleroutput}
  3.141592653589793
\end{euleroutput}
\begin{eulerprompt}
>format(10,5); pi
\end{eulerprompt}
\begin{euleroutput}
    3.14159 
\end{euleroutput}
\begin{eulercomment}
Standarnya adalah defformat().
\end{eulercomment}
\begin{eulerprompt}
>defformat; // default
\end{eulerprompt}
\begin{eulercomment}
Ada operator pendek yang hanya mencetak satu nilai. Operator
"terpanjang" akan mencetak semua digit angka yang valid.
\end{eulercomment}
\begin{eulerprompt}
>longest pi^2/2
\end{eulerprompt}
\begin{euleroutput}
        4.934802200544679 
\end{euleroutput}
\begin{eulercomment}
Ada juga operator pendek untuk mencetak hasil dalam format pecahan.
Kami telah menggunakannya di atas.
\end{eulercomment}
\begin{eulerprompt}
>fraction 1+1/2+1/3+1/4
\end{eulerprompt}
\begin{euleroutput}
  25/12
\end{euleroutput}
\begin{eulercomment}
Karena format internal menggunakan cara biner untuk menyimpan angka,
nilai 0,1 tidak akan terwakili secara tepat. Kesalahannya bertambah
sedikit, seperti yang Anda lihat dalam perhitungan berikut.
\end{eulercomment}
\begin{eulerprompt}
>longest 0.1+0.1+0.1+0.1+0.1+0.1+0.1+0.1+0.1+0.1-1
\end{eulerprompt}
\begin{euleroutput}
   -1.110223024625157e-16 
\end{euleroutput}
\begin{eulercomment}
Namun dengan "longformat" default, Anda tidak akan melihat hal ini.
Demi kenyamanan, output angka yang sangat kecil adalah 0.
\end{eulercomment}
\begin{eulerprompt}
>0.1+0.1+0.1+0.1+0.1+0.1+0.1+0.1+0.1+0.1-1
\end{eulerprompt}
\begin{euleroutput}
  0
\end{euleroutput}
\eulerheading{Ekspresi}
\begin{eulercomment}
String atau nama dapat digunakan untuk menyimpan ekspresi matematika,
yang dapat dievaluasi oleh EMT. Untuk ini, gunakan tanda kurung
setelah ekspresi. Jika Anda ingin menggunakan string sebagai ekspresi,
gunakan konvensi untuk menamainya "fx" atau "fxy", dst. Ekspresi lebih
diutamakan daripada fungsi.

Variabel global dapat digunakan dalam evaluasi.
\end{eulercomment}
\begin{eulerprompt}
>r:=2; fx:="pi*r^2"; longest fx()
\end{eulerprompt}
\begin{euleroutput}
        12.56637061435917 
\end{euleroutput}
\begin{eulercomment}
Parameter ditetapkan ke x, y, dan z dalam urutan tersebut. Parameter
tambahan dapat ditambahkan menggunakan parameter yang ditetapkan.
\end{eulercomment}
\begin{eulerprompt}
>fx:="a*sin(x)^2"; fx(5,a=-1)
\end{eulerprompt}
\begin{euleroutput}
  -0.919535764538
\end{euleroutput}
\begin{eulercomment}
Perhatikan bahwa ekspresi akan selalu menggunakan variabel global,
bahkan jika ada variabel dalam suatu fungsi dengan nama yang sama.
(Jika tidak, evaluasi ekspresi dalam fungsi dapat memberikan hasil
yang sangat membingungkan bagi pengguna yang memanggil fungsi
tersebut.)
\end{eulercomment}
\begin{eulerprompt}
>at:=4; function f(expr,x,at) := expr(x); ...
>f("at*x^2",3,5) // computes 4*3^2 not 5*3^2
\end{eulerprompt}
\begin{euleroutput}
  36
\end{euleroutput}
\begin{eulercomment}
Jika Anda ingin menggunakan nilai lain untuk "at" selain nilai global,
Anda perlu menambahkan "at=value".
\end{eulercomment}
\begin{eulerprompt}
>at:=4; function f(expr,x,a) := expr(x,at=a); ...
>f("at*x^2",3,5)
\end{eulerprompt}
\begin{euleroutput}
  45
\end{euleroutput}
\begin{eulercomment}
Sebagai referensi, kami mencatat bahwa koleksi panggilan (dibahas di
tempat lain) dapat berisi ekspresi. Jadi, kita dapat membuat contoh di
atas sebagai berikut.
\end{eulercomment}
\begin{eulerprompt}
>at:=4; function f(expr,x) := expr(x); ...
>f(\{\{"at*x^2",at=5\}\},3)
\end{eulerprompt}
\begin{euleroutput}
  45
\end{euleroutput}
\begin{eulercomment}
Ekspresi dalam x sering digunakan seperti fungsi.\\
Perlu dicatat bahwa mendefinisikan fungsi dengan nama yang sama
seperti ekspresi simbolik global akan menghapus variabel ini untuk
menghindari kebingungan antara ekspresi simbolik dan fungsi.
\end{eulercomment}
\begin{eulerprompt}
>f &= 5*x;
>function f(x) := 6*x;
>f(2)
\end{eulerprompt}
\begin{euleroutput}
  12
\end{euleroutput}
\begin{eulercomment}
Berdasarkan konvensi, ekspresi simbolik atau numerik harus diberi nama
fx, fxy, dst. Skema penamaan ini tidak boleh digunakan untuk fungsi.
\end{eulercomment}
\begin{eulerprompt}
>fx &= diff(x^x,x); $&fx
\end{eulerprompt}
\begin{eulercomment}
Bentuk khusus dari suatu ekspresi memperbolehkan variabel apa pun
sebagai parameter tanpa nama untuk evaluasi ekspresi, bukan hanya "x",
"y", dst. Untuk ini, awali ekspresi dengan "@(variabel) ...".
\end{eulercomment}
\begin{eulerprompt}
>"@(a,b) a^2+b^2", %(4,5)
\end{eulerprompt}
\begin{euleroutput}
  @(a,b) a^2+b^2
  41
\end{euleroutput}
\begin{eulercomment}
Hal ini memungkinkan untuk memanipulasi ekspresi dalam variabel lain
untuk fungsi EMT yang memerlukan ekspresi dalam "x".

Cara paling dasar untuk mendefinisikan fungsi sederhana adalah dengan
menyimpan rumusnya dalam ekspresi simbolik atau numerik. Jika variabel
utamanya adalah x, ekspresi tersebut dapat dievaluasi seperti halnya
fungsi.

Seperti yang Anda lihat dalam contoh berikut, variabel global terlihat
selama evaluasi.
\end{eulercomment}
\begin{eulerprompt}
>fx &= x^3-a*x;  ...
>a=1.2; fx(0.5)
\end{eulerprompt}
\begin{euleroutput}
  -0.475
\end{euleroutput}
\begin{eulercomment}
Semua variabel lain dalam ekspresi dapat ditentukan dalam evaluasi
menggunakan parameter yang ditetapkan.
\end{eulercomment}
\begin{eulerprompt}
>fx(0.5,a=1.1)
\end{eulerprompt}
\begin{euleroutput}
  -0.425
\end{euleroutput}
\begin{eulercomment}
Suatu ekspresi tidak harus simbolis. Hal ini diperlukan, jika ekspresi
tersebut memuat fungsi, yang hanya diketahui dalam kernel numerik,
bukan dalam Maxima.

\begin{eulercomment}
\eulerheading{Matematika Simbolis}
\begin{eulercomment}
EMT mengerjakan matematika simbolis dengan bantuan Maxima. Untuk
detailnya, mulailah dengan tutorial berikut, atau telusuri referensi
untuk Maxima. Para ahli di Maxima harus memperhatikan bahwa terdapat
perbedaan dalam sintaksis antara sintaksis asli Maxima dan sintaksis
default ekspresi simbolis dalam EMT.

Matematika simbolis terintegrasi dengan mulus ke dalam Euler dengan \&.
Setiap ekspresi yang dimulai dengan \& adalah ekspresi simbolis.
Ekspresi tersebut dievaluasi dan dicetak oleh Maxima.

Pertama-tama, Maxima memiliki aritmatika "tak terbatas" yang dapat
menangani angka yang sangat besar.
\end{eulercomment}
\begin{eulerprompt}
>$&44!
\end{eulerprompt}
\begin{eulercomment}
Dengan cara ini, Anda dapat menghitung hasil yang besar secara tepat.
Mari kita hitung\\
\end{eulercomment}
\begin{eulerformula}
\[
C(44,10) = \frac{44!}{34! \cdot 10!}
\]
\end{eulerformula}
\begin{eulerprompt}
>$& 44!/(34!*10!) // nilai C(44,10)
\end{eulerprompt}
\begin{eulercomment}
Tentu saja, Maxima memiliki fungsi yang lebih efisien untuk ini
(seperti halnya bagian numerik EMT).
\end{eulercomment}
\begin{eulerprompt}
>$binomial(44,10) //menghitung C(44,10) menggunakan fungsi binomial()
\end{eulerprompt}
\begin{eulercomment}
Untuk mempelajari lebih lanjut tentang fungsi tertentu, klik dua kali
pada fungsi tersebut. Misalnya, coba klik dua kali pada "\&binomial" di
baris perintah sebelumnya. Ini akan membuka dokumentasi Maxima
sebagaimana disediakan oleh penulis program tersebut.

Anda akan mempelajari bahwa hal berikut juga berfungsi.\\
\end{eulercomment}
\begin{eulerformula}
\[
C(x,3)=\frac{x!}{(x-3)!3!}=\frac{(x-2)(x-1)x}{6}
\]
\end{eulerformula}
\begin{eulerprompt}
>$binomial(x,3) // C(x,3)
\end{eulerprompt}
\begin{eulercomment}
Jika Anda ingin mengganti x dengan nilai tertentu, gunakan "with".
\end{eulercomment}
\begin{eulerprompt}
>$&binomial(x,3) with x=10 // substitusi x=10 ke C(x,3)
\end{eulerprompt}
\begin{eulercomment}
Dengan cara itu Anda dapat menggunakan solusi persamaan dalam
persamaan lain.

Ekspresi simbolik dicetak oleh Maxima dalam bentuk 2D. Alasannya
adalah adanya tanda simbolik khusus dalam string.

Seperti yang telah Anda lihat pada contoh sebelumnya dan berikutnya,
jika Anda telah menginstal LaTeX, Anda dapat mencetak ekspresi
simbolik dengan Latex. Jika tidak, perintah berikut akan mengeluarkan
pesan kesalahan.

Untuk mencetak ekspresi simbolik dengan LaTeX, gunakan \textdollar{} di depan \&
(atau Anda dapat menghilangkan \&) sebelum perintah. Jangan jalankan
perintah Maxima dengan \textdollar{}, jika Anda tidak menginstal LaTeX.
\end{eulercomment}
\begin{eulerprompt}
>$(3+x)/(x^2+1)
\end{eulerprompt}
\begin{eulercomment}
Ekspresi simbolik diurai oleh Euler. Jika Anda memerlukan sintaksis
yang kompleks dalam satu ekspresi, Anda dapat melampirkan ekspresi
tersebut dalam "...". Menggunakan lebih dari satu ekspresi sederhana
dimungkinkan, tetapi sangat tidak disarankan.
\end{eulercomment}
\begin{eulerprompt}
>&"v := 5; v^2"
\end{eulerprompt}
\begin{euleroutput}
  
                                    25
  
\end{euleroutput}
\begin{eulercomment}
Untuk kelengkapan, kami mencatat bahwa ekspresi simbolik dapat
digunakan dalam program, tetapi harus disertakan dalam tanda kutip.
Selain itu, akan jauh lebih efektif untuk memanggil Maxima pada waktu
kompilasi jika memungkinkan.
\end{eulercomment}
\begin{eulerprompt}
>$&expand((1+x)^4), $&factor(diff(%,x)) // diff: turunan, factor: faktor
\end{eulerprompt}
\begin{eulercomment}
Sekali lagi, \% merujuk pada hasil sebelumnya.

Untuk mempermudah, kami menyimpan solusi ke variabel simbolik.
Variabel simbolik didefinisikan dengan "\&=".
\end{eulercomment}
\begin{eulerprompt}
>fx &= (x+1)/(x^4+1); $&fx
\end{eulerprompt}
\begin{eulercomment}
Ekspresi simbolik dapat digunakan dalam ekspresi simbolik lainnya.
\end{eulercomment}
\begin{eulerprompt}
>$&factor(diff(fx,x))
\end{eulerprompt}
\begin{eulercomment}
Input langsung perintah Maxima juga tersedia. Awali baris perintah
dengan "::". Sintaks Maxima disesuaikan dengan sintaks EMT (disebut
"mode kompatibilitas").
\end{eulercomment}
\begin{eulerprompt}
>&factor(20!)
\end{eulerprompt}
\begin{euleroutput}
  
                           2432902008176640000
  
\end{euleroutput}
\begin{eulerprompt}
>::: factor(10!)
\end{eulerprompt}
\begin{euleroutput}
  
                                 8  4  2
                                2  3  5  7
  
\end{euleroutput}
\begin{eulerprompt}
>:: factor(20!)
\end{eulerprompt}
\begin{euleroutput}
  
                          18  8  4  2
                         2   3  5  7  11 13 17 19
  
\end{euleroutput}
\begin{eulercomment}
Jika Anda ahli dalam Maxima, Anda mungkin ingin menggunakan sintaksis
asli Maxima. Anda dapat melakukannya dengan ":::".
\end{eulercomment}
\begin{eulerprompt}
>::: av:g$ av^2;
\end{eulerprompt}
\begin{euleroutput}
  
                                     2
                                    g
  
\end{euleroutput}
\begin{eulerprompt}
>fx &= x^3*exp(x), $fx
\end{eulerprompt}
\begin{euleroutput}
  
                                   3  x
                                  x  E
  
\end{euleroutput}
\begin{eulercomment}
Variabel tersebut dapat digunakan dalam ekspresi simbolik lainnya.
Perhatikan bahwa dalam perintah berikut sisi kanan \&= dievaluasi
sebelum penugasan ke Fx.
\end{eulercomment}
\begin{eulerprompt}
>&(fx with x=5), $%, &float(%)
\end{eulerprompt}
\begin{euleroutput}
  
                                       5
                                  125 E
  
  
                            18551.64488782208
  
\end{euleroutput}
\begin{eulerprompt}
>fx(5)
\end{eulerprompt}
\begin{euleroutput}
  18551.6448878
\end{euleroutput}
\begin{eulercomment}
Untuk mengevaluasi ekspresi dengan nilai variabel tertentu, Anda dapat
menggunakan operator "with".

Baris perintah berikut juga menunjukkan bahwa Maxima dapat
mengevaluasi ekspresi secara numerik dengan float().
\end{eulercomment}
\begin{eulerprompt}
>&(fx with x=10)-(fx with x=5), &float(%)
\end{eulerprompt}
\begin{euleroutput}
  
                                  10        5
                            1000 E   - 125 E
  
  
                           2.20079141499189e+7
  
\end{euleroutput}
\begin{eulerprompt}
>$factor(diff(fx,x,2))
\end{eulerprompt}
\begin{eulercomment}
Untuk mendapatkan kode Latex untuk suatu ekspresi, Anda dapat
menggunakan perintah tex.
\end{eulercomment}
\begin{eulerprompt}
>tex(fx)
\end{eulerprompt}
\begin{euleroutput}
  x^3\(\backslash\),e^\{x\}
\end{euleroutput}
\begin{eulercomment}
Ekspresi simbolik dapat dievaluasi seperti halnya ekspresi numerik.
\end{eulercomment}
\begin{eulerprompt}
>fx(0.5)
\end{eulerprompt}
\begin{euleroutput}
  0.206090158838
\end{euleroutput}
\begin{eulercomment}
Dalam ekspresi simbolik, ini tidak berfungsi, karena Maxima tidak
mendukungnya. Sebagai gantinya, gunakan sintaks "with" (bentuk yang
lebih baik dari perintah at(...) dari Maxima).
\end{eulercomment}
\begin{eulerprompt}
>$&fx with x=1/2
\end{eulerprompt}
\begin{eulercomment}
Penugasan tersebut juga dapat bersifat simbolis.
\end{eulercomment}
\begin{eulerprompt}
>$&fx with x=1+t
\end{eulerprompt}
\begin{eulercomment}
Perintah solve memecahkan ekspresi simbolik untuk variabel dalam
Maxima. Hasilnya adalah vektor solusi.
\end{eulercomment}
\begin{eulerprompt}
>$&solve(x^2+x=4,x)
\end{eulerprompt}
\begin{eulercomment}
Bandingkan dengan perintah numerik "solve" di Euler, yang memerlukan
nilai awal, dan secara opsional nilai target.
\end{eulercomment}
\begin{eulerprompt}
>solve("x^2+x",1,y=4)
\end{eulerprompt}
\begin{euleroutput}
  1.56155281281
\end{euleroutput}
\begin{eulercomment}
Nilai numerik dari solusi simbolik dapat dihitung dengan mengevaluasi
hasil simbolik. Euler akan membaca ulang penugasan x= dst. Jika Anda
tidak memerlukan hasil numerik untuk perhitungan lebih lanjut, Anda
juga dapat membiarkan Maxima menemukan nilai numeriknya.
\end{eulercomment}
\begin{eulerprompt}
>sol &= solve(x^2+2*x=4,x); $&sol, sol(), $&float(sol)
\end{eulerprompt}
\begin{euleroutput}
  [-3.23607,  1.23607]
\end{euleroutput}
\begin{eulercomment}
Untuk mendapatkan solusi simbolis yang spesifik, seseorang dapat
menggunakan "with" dan indeks.
\end{eulercomment}
\begin{eulerprompt}
>$&solve(x^2+x=1,x), x2 &= x with %[2]; $&x2
\end{eulerprompt}
\begin{eulercomment}
Untuk menyelesaikan sistem persamaan, gunakan vektor persamaan.
Hasilnya adalah vektor solusi.
\end{eulercomment}
\begin{eulerprompt}
>sol &= solve([x+y=3,x^2+y^2=5],[x,y]); $&sol, $&x*y with sol[1]
\end{eulerprompt}
\begin{eulercomment}
Ekspresi simbolik dapat memiliki tanda, yang menunjukkan perlakuan
khusus di Maxima. Beberapa tanda dapat digunakan sebagai perintah
juga, yang lainnya tidak. Tanda ditambahkan dengan "\textbar{}" (bentuk yang
lebih baik dari "ev(...,flags)")
\end{eulercomment}
\begin{eulerprompt}
>$& diff((x^3-1)/(x+1),x) //turunan bentuk pecahan
>$& diff((x^3-1)/(x+1),x) | ratsimp //menyederhanakan pecahan
>$&factor(%)
\end{eulerprompt}
\eulerheading{Fungsi}
\begin{eulercomment}
Dalam EMT, fungsi adalah program yang didefinisikan dengan perintah
"function". Fungsi ini dapat berupa fungsi satu baris atau fungsi
multibaris.\\
Fungsi satu baris dapat berupa numerik atau simbolik. Fungsi satu
baris numerik didefinisikan oleh ":=".
\end{eulercomment}
\begin{eulerprompt}
>function f(x) := x*sqrt(x^2+1)
\end{eulerprompt}
\begin{eulercomment}
Sebagai gambaran umum, kami tampilkan semua definisi yang mungkin
untuk fungsi satu baris. Suatu fungsi dapat dievaluasi seperti fungsi
Euler bawaan lainnya.
\end{eulercomment}
\begin{eulerprompt}
>f(2)
\end{eulerprompt}
\begin{euleroutput}
  4.472135955
\end{euleroutput}
\begin{eulercomment}
Fungsi ini juga akan bekerja untuk vektor, mematuhi bahasa matriks
Euler, karena ekspresi yang digunakan dalam fungsi tersebut
divektorkan.
\end{eulercomment}
\begin{eulerprompt}
>f(0:0.1:1)
\end{eulerprompt}
\begin{euleroutput}
  [0,  0.100499,  0.203961,  0.313209,  0.430813,  0.559017,  0.699714,
  0.854459,  1.0245,  1.21083,  1.41421]
\end{euleroutput}
\begin{eulercomment}
Fungsi dapat diplot. Alih-alih ekspresi, kita hanya perlu memberikan
nama fungsi.

Berbeda dengan ekspresi simbolik atau numerik, nama fungsi harus
diberikan dalam bentuk string.
\end{eulercomment}
\begin{eulerprompt}
>solve("f",1,y=1)
\end{eulerprompt}
\begin{euleroutput}
  0.786151377757
\end{euleroutput}
\begin{eulercomment}
Secara default, jika Anda perlu menimpa fungsi bawaan, Anda harus
menambahkan kata kunci "overwrite". Menimpa fungsi bawaan berbahaya
dan dapat menyebabkan masalah bagi fungsi lain yang bergantung
padanya.

Anda masih dapat memanggil fungsi bawaan sebagai "\_...", jika itu
adalah fungsi di inti Euler.
\end{eulercomment}
\begin{eulerprompt}
>function overwrite sin (x) := _sin(x°) // redine sine in degrees
>sin(45)
\end{eulerprompt}
\begin{euleroutput}
  0.707106781187
\end{euleroutput}
\begin{eulercomment}
Sebaiknya kita hilangkan pendefinisian ulang dosa ini.
\end{eulercomment}
\begin{eulerprompt}
>forget sin; sin(pi/4)
\end{eulerprompt}
\begin{euleroutput}
  0.707106781187
\end{euleroutput}
\eulersubheading{Parameter Default}
\begin{eulercomment}
Fungsi numerik dapat memiliki parameter default.
\end{eulercomment}
\begin{eulerprompt}
>function f(x,a=1) := a*x^2
\end{eulerprompt}
\begin{eulercomment}
Mengabaikan parameter ini akan menggunakan nilai default.
\end{eulercomment}
\begin{eulerprompt}
>f(4)
\end{eulerprompt}
\begin{euleroutput}
  16
\end{euleroutput}
\begin{eulercomment}
Mengaturnya akan menimpa nilai default.
\end{eulercomment}
\begin{eulerprompt}
>f(4,5)
\end{eulerprompt}
\begin{euleroutput}
  80
\end{euleroutput}
\begin{eulercomment}
Parameter yang ditetapkan juga akan menimpanya. Ini digunakan oleh
banyak fungsi Euler seperti plot2d, plot3d.
\end{eulercomment}
\begin{eulerprompt}
>f(4,a=1)
\end{eulerprompt}
\begin{euleroutput}
  16
\end{euleroutput}
\begin{eulercomment}
Jika suatu variabel bukan parameter, maka variabel tersebut harus
bersifat global. Fungsi satu baris dapat melihat variabel global.
\end{eulercomment}
\begin{eulerprompt}
>function f(x) := a*x^2
>a=6; f(2)
\end{eulerprompt}
\begin{euleroutput}
  24
\end{euleroutput}
\begin{eulercomment}
Namun, parameter yang ditetapkan akan menggantikan nilai global.

Jika argumen tidak ada dalam daftar parameter yang telah ditetapkan
sebelumnya, argumen tersebut harus dideklarasikan dengan ":="!
\end{eulercomment}
\begin{eulerprompt}
>f(2,a:=5)
\end{eulerprompt}
\begin{euleroutput}
  20
\end{euleroutput}
\begin{eulercomment}
Fungsi simbolik didefinisikan dengan "\&=". Fungsi ini didefinisikan
dalam Euler dan Maxima, dan berfungsi di kedua dunia. Ekspresi yang
mendefinisikan dijalankan melalui Maxima sebelum definisi.
\end{eulercomment}
\begin{eulerprompt}
>function g(x) &= x^3-x*exp(-x); $&g(x)
\end{eulerprompt}
\begin{eulercomment}
Fungsi simbolik dapat digunakan dalam ekspresi simbolik.
\end{eulercomment}
\begin{eulerprompt}
>$&diff(g(x),x), $&% with x=4/3
\end{eulerprompt}
\begin{eulercomment}
Mereka juga dapat digunakan dalam ekspresi numerik. Tentu saja, ini
hanya akan berfungsi jika EMT dapat menginterpretasikan semua hal di
dalam fungsi tersebut.
\end{eulercomment}
\begin{eulerprompt}
>g(5+g(1))
\end{eulerprompt}
\begin{euleroutput}
  178.635099908
\end{euleroutput}
\begin{eulercomment}
Mereka dapat digunakan untuk mendefinisikan fungsi atau ekspresi
simbolis lainnya.
\end{eulercomment}
\begin{eulerprompt}
>function G(x) &= factor(integrate(g(x),x)); $&G(c) // integrate: mengintegralkan
>solve(&g(x),0.5)
\end{eulerprompt}
\begin{euleroutput}
  0.703467422498
\end{euleroutput}
\begin{eulercomment}
Berikut ini juga berfungsi, karena Euler menggunakan ekspresi simbolik
dalam fungsi g, jika tidak menemukan variabel simbolik g, dan jika ada
fungsi simbolik g.
\end{eulercomment}
\begin{eulerprompt}
>solve(&g,0.5)
\end{eulerprompt}
\begin{euleroutput}
  0.703467422498
\end{euleroutput}
\begin{eulerprompt}
>function P(x,n) &= (2*x-1)^n; $&P(x,n)
>function Q(x,n) &= (x+2)^n; $&Q(x,n)
>$&P(x,4), $&expand(%)
>P(3,4)
\end{eulerprompt}
\begin{euleroutput}
  625
\end{euleroutput}
\begin{eulerprompt}
>$&P(x,4)+ Q(x,3), $&expand(%)
>$&P(x,4)-Q(x,3), $&expand(%), $&factor(%)
>$&P(x,4)*Q(x,3), $&expand(%), $&factor(%)
>$&P(x,4)/Q(x,1), $&expand(%), $&factor(%)
>function f(x) &= x^3-x; $&f(x)
\end{eulerprompt}
\begin{eulercomment}
Dengan \&= fungsinya bersifat simbolis, dan dapat digunakan dalam
ekspresi simbolis lainnya.
\end{eulercomment}
\begin{eulerprompt}
>$&integrate(f(x),x)
\end{eulerprompt}
\begin{eulercomment}
Dengan := fungsinya adalah numerik. Contoh yang bagus adalah integral
tentu seperti

\end{eulercomment}
\begin{eulerformula}
\[
f(x) = \int_1^x t^t \, dt,
\]
\end{eulerformula}
\begin{eulercomment}
yang tidak dapat dievaluasi secara simbolis.

Jika kita mendefinisikan ulang fungsi tersebut dengan kata kunci
"map", fungsi tersebut dapat digunakan untuk vektor x. Secara
internal, fungsi tersebut dipanggil untuk semua nilai x satu kali, dan
hasilnya disimpan dalam sebuah vektor.
\end{eulercomment}
\begin{eulerprompt}
>function map f(x) := integrate("x^x",1,x)
>f(0:0.5:2)
\end{eulerprompt}
\begin{euleroutput}
  [-0.783431,  -0.410816,  0,  0.676863,  2.05045]
\end{euleroutput}
\begin{eulercomment}
Fungsi dapat memiliki nilai default untuk parameter.
\end{eulercomment}
\begin{eulerprompt}
>function mylog (x,base=10) := ln(x)/ln(base);
\end{eulerprompt}
\begin{eulercomment}
Sekarang fungsi tersebut dapat dipanggil dengan atau tanpa parameter
"dasar".
\end{eulercomment}
\begin{eulerprompt}
>mylog(100), mylog(2^6.7,2)
\end{eulerprompt}
\begin{euleroutput}
  2
  6.7
\end{euleroutput}
\begin{eulercomment}
Selain itu, dimungkinkan untuk menggunakan parameter yang ditetapkan.
\end{eulercomment}
\begin{eulerprompt}
>mylog(E^2,base=E)
\end{eulerprompt}
\begin{euleroutput}
  2
\end{euleroutput}
\begin{eulercomment}
Sering kali, kita ingin menggunakan fungsi untuk vektor di satu
tempat, dan untuk elemen individual di tempat lain. Hal ini
dimungkinkan dengan parameter vektor.
\end{eulercomment}
\begin{eulerprompt}
>function f([a,b]) &= a^2+b^2-a*b+b; $&f(a,b), $&f(x,y)
\end{eulerprompt}
\begin{eulercomment}
Fungsi simbolik semacam itu dapat digunakan untuk variabel simbolik.

Namun, fungsi tersebut juga dapat digunakan untuk vektor numerik.
\end{eulercomment}
\begin{eulerprompt}
>v=[3,4]; f(v)
\end{eulerprompt}
\begin{euleroutput}
  17
\end{euleroutput}
\begin{eulercomment}
Ada pula fungsi yang murni simbolis, yang tidak dapat digunakan secara
numerik.
\end{eulercomment}
\begin{eulerprompt}
>function lapl(expr,x,y) &&= diff(expr,x,2)+diff(expr,y,2)//turunan parsial kedua
\end{eulerprompt}
\begin{euleroutput}
  
                   diff(expr, y, 2) + diff(expr, x, 2)
  
\end{euleroutput}
\begin{eulerprompt}
>$&realpart((x+I*y)^4), $&lapl(%,x,y)
\end{eulerprompt}
\begin{eulercomment}
Namun tentu saja, mereka dapat digunakan dalam ekspresi simbolik atau
dalam definisi fungsi simbolik.
\end{eulercomment}
\begin{eulerprompt}
>function f(x,y) &= factor(lapl((x+y^2)^5,x,y)); $&f(x,y)
\end{eulerprompt}
\begin{eulercomment}
Singkatnya

- \&= mendefinisikan fungsi simbolik,\\
- := mendefinisikan fungsi numerik,\\
- \&\&= mendefinisikan fungsi simbolik murni.

\begin{eulercomment}
\eulerheading{Menyelesaikan Ekspresi}
\begin{eulercomment}
Ekspresi dapat diselesaikan secara numerik dan simbolik.

Untuk menyelesaikan ekspresi sederhana dari satu variabel, kita dapat
menggunakan fungsi solve(). Fungsi ini memerlukan nilai awal untuk
memulai pencarian. Secara internal, solve() menggunakan metode secant.
\end{eulercomment}
\begin{eulerprompt}
>solve("x^2-2",1)
\end{eulerprompt}
\begin{euleroutput}
  1.41421356237
\end{euleroutput}
\begin{eulercomment}
Ini juga berlaku untuk ekspresi simbolik. Ambil fungsi berikut.
\end{eulercomment}
\begin{eulerprompt}
>$&solve(x^2=2,x)
>$&solve(x^2-2,x)
>$&solve(a*x^2+b*x+c=0,x)
>$&solve([a*x+b*y=c,d*x+e*y=f],[x,y])
>px &= 4*x^8+x^7-x^4-x; $&px
\end{eulerprompt}
\begin{eulercomment}
Sekarang kita cari titik, di mana polinomialnya adalah 2. Dalam
solve(), nilai target default y=0 dapat diubah dengan variabel yang
ditetapkan.\\
Kita gunakan y=2 dan periksa dengan mengevaluasi polinomial pada hasil
sebelumnya.
\end{eulercomment}
\begin{eulerprompt}
>solve(px,1,y=2), px(%)
\end{eulerprompt}
\begin{euleroutput}
  0.966715594851
  2
\end{euleroutput}
\begin{eulercomment}
Memecahkan ekspresi simbolik dalam bentuk simbolik akan menghasilkan
daftar solusi. Kami menggunakan pemecah simbolik solve() yang
disediakan oleh Maxima.
\end{eulercomment}
\begin{eulerprompt}
>sol &= solve(x^2-x-1,x); $&sol
\end{eulerprompt}
\begin{eulercomment}
Cara termudah untuk mendapatkan nilai numerik adalah dengan
mengevaluasi solusi secara numerik seperti sebuah ekspresi.
\end{eulercomment}
\begin{eulerprompt}
>longest sol()
\end{eulerprompt}
\begin{euleroutput}
      -0.6180339887498949       1.618033988749895 
\end{euleroutput}
\begin{eulercomment}
Untuk menggunakan solusi secara simbolis dalam ekspresi lain, cara
termudah adalah "dengan".
\end{eulercomment}
\begin{eulerprompt}
>$&x^2 with sol[1], $&expand(x^2-x-1 with sol[2])
\end{eulerprompt}
\begin{eulercomment}
Memecahkan sistem persamaan secara simbolis dapat dilakukan dengan
vektor persamaan dan penyelesai simbolis solve(). Jawabannya adalah
daftar persamaan.
\end{eulercomment}
\begin{eulerprompt}
>$&solve([x+y=2,x^3+2*y+x=4],[x,y])
\end{eulerprompt}
\begin{eulercomment}
Fungsi f() dapat melihat variabel global. Namun, sering kali kita
ingin menggunakan parameter lokal.

\end{eulercomment}
\begin{eulerformula}
\[
a^x-x^a = 0.1
\]
\end{eulerformula}
\begin{eulercomment}
dengan a=3.
\end{eulercomment}
\begin{eulerprompt}
>function f(x,a) := x^a-a^x;
\end{eulerprompt}
\begin{eulercomment}
Salah satu cara untuk meneruskan parameter tambahan ke f() adalah
dengan menggunakan daftar dengan nama fungsi dan parameter (cara
lainnya adalah parameter titik koma).
\end{eulercomment}
\begin{eulerprompt}
>solve(\{\{"f",3\}\},2,y=0.1)
\end{eulerprompt}
\begin{euleroutput}
  2.54116291558
\end{euleroutput}
\begin{eulercomment}
Ini juga berlaku untuk ekspresi. Namun, elemen daftar bernama harus
digunakan. (Informasi lebih lanjut tentang daftar ada di tutorial
tentang sintaks EMT).
\end{eulercomment}
\begin{eulerprompt}
>solve(\{\{"x^a-a^x",a=3\}\},2,y=0.1)
\end{eulerprompt}
\begin{euleroutput}
  2.54116291558
\end{euleroutput}
\eulerheading{Menyelesaikan Pertidaksamaan}
\begin{eulercomment}
Untuk menyelesaikan pertidaksamaan, EMT tidak akan dapat melakukannya,
melainkan dengan bantuan Maxima, artinya secara eksak (simbolik).
Perintah Maxima yang digunakan adalah fourier\_elim(), yang harus
dipanggil dengan perintah "load(fourier\_elim)" terlebih dahulu.
\end{eulercomment}
\begin{eulerprompt}
>&load(fourier_elim)
\end{eulerprompt}
\begin{euleroutput}
  
          C:/Program Files/Euler x64/maxima/share/maxima/5.35.1/share/f\(\backslash\)
  ourier_elim/fourier_elim.lisp
  
\end{euleroutput}
\begin{eulerprompt}
>$&fourier_elim([x^2 - 1>0],[x]) // x^2-1 > 0
>$&fourier_elim([x^2 - 1<0],[x]) // x^2-1 < 0
>$&fourier_elim([x^2 - 1 # 0],[x]) // x^-1 <> 0
>$&fourier_elim([x # 6],[x])
>$&fourier_elim([x < 1, x > 1],[x]) // tidak memiliki penyelesaian
>$&fourier_elim([minf < x, x < inf],[x]) // solusinya R
>$&fourier_elim([x^3 - 1 > 0],[x])
>$&fourier_elim([cos(x) < 1/2],[x]) // ??? gagal
>$&fourier_elim([y-x < 5, x - y < 7, 10 < y],[x,y]) // sistem pertidaksamaan
>$&fourier_elim([y-x < 5, x - y < 7, 10 < y],[y,x])
>$&fourier_elim((x + y < 5) and (x - y >8),[x,y])
>$&fourier_elim(((x + y < 5) and x < 1) or  (x - y >8),[x,y])
>&fourier_elim([max(x,y) > 6, x # 8, abs(y-1) > 12],[x,y])
\end{eulerprompt}
\begin{euleroutput}
  
          [6 < x, x < 8, y < - 11] or [8 < x, y < - 11]
   or [x < 8, 13 < y] or [x = y, 13 < y] or [8 < x, x < y, 13 < y]
   or [y < x, 13 < y]
  
\end{euleroutput}
\begin{eulerprompt}
>$&fourier_elim([(x+6)/(x-9) <= 6],[x])
\end{eulerprompt}
\eulerheading{Bahasa Matriks}
\begin{eulercomment}
Dokumentasi inti EMT berisi pembahasan terperinci tentang bahasa
matriks Euler.

Vektor dan matriks dimasukkan dengan tanda kurung siku, elemen
dipisahkan dengan koma, baris dipisahkan dengan titik koma.
\end{eulercomment}
\begin{eulerprompt}
>A=[1,2;3,4]
\end{eulerprompt}
\begin{euleroutput}
              1             2 
              3             4 
\end{euleroutput}
\begin{eulercomment}
Produk matriks dilambangkan dengan sebuah titik.
\end{eulercomment}
\begin{eulerprompt}
>b=[3;4]
\end{eulerprompt}
\begin{euleroutput}
              3 
              4 
\end{euleroutput}
\begin{eulerprompt}
>b' // transpose b
\end{eulerprompt}
\begin{euleroutput}
  [3,  4]
\end{euleroutput}
\begin{eulerprompt}
>inv(A) //inverse A
\end{eulerprompt}
\begin{euleroutput}
             -2             1 
            1.5          -0.5 
\end{euleroutput}
\begin{eulerprompt}
>A.b //perkalian matriks
\end{eulerprompt}
\begin{euleroutput}
             11 
             25 
\end{euleroutput}
\begin{eulerprompt}
>A.inv(A)
\end{eulerprompt}
\begin{euleroutput}
              1             0 
              0             1 
\end{euleroutput}
\begin{eulercomment}
Poin utama dari bahasa matriks adalah bahwa semua fungsi dan operator
bekerja elemen demi elemen.
\end{eulercomment}
\begin{eulerprompt}
>A.A
\end{eulerprompt}
\begin{euleroutput}
              7            10 
             15            22 
\end{euleroutput}
\begin{eulerprompt}
>A^2 //perpangkatan elemen2 A
\end{eulerprompt}
\begin{euleroutput}
              1             4 
              9            16 
\end{euleroutput}
\begin{eulerprompt}
>A.A.A
\end{eulerprompt}
\begin{euleroutput}
             37            54 
             81           118 
\end{euleroutput}
\begin{eulerprompt}
>power(A,3) //perpangkatan matriks
\end{eulerprompt}
\begin{euleroutput}
             37            54 
             81           118 
\end{euleroutput}
\begin{eulerprompt}
>A/A //pembagian elemen-elemen matriks yang seletak
\end{eulerprompt}
\begin{euleroutput}
              1             1 
              1             1 
\end{euleroutput}
\begin{eulerprompt}
>A/b //pembagian elemen2 A oleh elemen2 b kolom demi kolom (karena b vektor kolom)
\end{eulerprompt}
\begin{euleroutput}
       0.333333      0.666667 
           0.75             1 
\end{euleroutput}
\begin{eulerprompt}
>A\(\backslash\)b // hasilkali invers A dan b, A^(-1)b 
\end{eulerprompt}
\begin{euleroutput}
             -2 
            2.5 
\end{euleroutput}
\begin{eulerprompt}
>inv(A).b
\end{eulerprompt}
\begin{euleroutput}
             -2 
            2.5 
\end{euleroutput}
\begin{eulerprompt}
>A\(\backslash\)A   //A^(-1)A
\end{eulerprompt}
\begin{euleroutput}
              1             0 
              0             1 
\end{euleroutput}
\begin{eulerprompt}
>inv(A).A
\end{eulerprompt}
\begin{euleroutput}
              1             0 
              0             1 
\end{euleroutput}
\begin{eulerprompt}
>A*A //perkalin elemen-elemen matriks seletak
\end{eulerprompt}
\begin{euleroutput}
              1             4 
              9            16 
\end{euleroutput}
\begin{eulercomment}
Ini bukan hasil perkalian matriks, tetapi perkalian elemen demi
elemen. Hal yang sama berlaku untuk vektor.
\end{eulercomment}
\begin{eulerprompt}
>b^2 // perpangkatan elemen-elemen matriks/vektor
\end{eulerprompt}
\begin{euleroutput}
              9 
             16 
\end{euleroutput}
\begin{eulercomment}
Jika salah satu operan merupakan vektor atau skalar, ia diekspansi
dengan cara alami.
\end{eulercomment}
\begin{eulerprompt}
>2*A
\end{eulerprompt}
\begin{euleroutput}
              2             4 
              6             8 
\end{euleroutput}
\begin{eulercomment}
Misalnya, jika operan adalah vektor kolom, elemen-elemennya diterapkan
ke semua baris A.
\end{eulercomment}
\begin{eulerprompt}
>[1,2]*A
\end{eulerprompt}
\begin{euleroutput}
              1             4 
              3             8 
\end{euleroutput}
\begin{eulercomment}
Jika itu adalah vektor baris maka diterapkan ke semua kolom A.
\end{eulercomment}
\begin{eulerprompt}
>A*[2,3]
\end{eulerprompt}
\begin{euleroutput}
              2             6 
              6            12 
\end{euleroutput}
\begin{eulercomment}
Seseorang dapat membayangkan perkalian ini seolah-olah vektor baris v
telah diduplikasi untuk membentuk matriks berukuran sama dengan A.
\end{eulercomment}
\begin{eulerprompt}
>dup([1,2],2) // dup: menduplikasi/menggandakan vektor [1,2] sebanyak 2 kali (baris)
\end{eulerprompt}
\begin{euleroutput}
              1             2 
              1             2 
\end{euleroutput}
\begin{eulerprompt}
>A*dup([1,2],2) 
\end{eulerprompt}
\begin{euleroutput}
              1             4 
              3             8 
\end{euleroutput}
\begin{eulercomment}
Hal ini juga berlaku untuk dua vektor, yang satu merupakan vektor
baris dan yang lainnya merupakan vektor kolom. Kita menghitung i*j
untuk i,j dari 1 hingga 5. Caranya adalah dengan mengalikan 1:5 dengan
transposenya. Bahasa matriks Euler secara otomatis menghasilkan tabel
nilai.
\end{eulercomment}
\begin{eulerprompt}
>(1:5)*(1:5)' // hasilkali elemen-elemen vektor baris dan vektor kolom
\end{eulerprompt}
\begin{euleroutput}
              1             2             3             4             5 
              2             4             6             8            10 
              3             6             9            12            15 
              4             8            12            16            20 
              5            10            15            20            25 
\end{euleroutput}
\begin{eulercomment}
Sekali lagi, ingatlah bahwa ini bukan produk matriks!
\end{eulercomment}
\begin{eulerprompt}
>(1:5).(1:5)' // hasilkali vektor baris dan vektor kolom
\end{eulerprompt}
\begin{euleroutput}
  55
\end{euleroutput}
\begin{eulerprompt}
>sum((1:5)*(1:5)) // sama hasilnya
\end{eulerprompt}
\begin{euleroutput}
  55
\end{euleroutput}
\begin{eulercomment}
Bahkan operator seperti \textless{} atau == bekerja dengan cara yang sama.
\end{eulercomment}
\begin{eulerprompt}
>(1:10)<6 // menguji elemen-elemen yang kurang dari 6
\end{eulerprompt}
\begin{euleroutput}
  [1,  1,  1,  1,  1,  0,  0,  0,  0,  0]
\end{euleroutput}
\begin{eulercomment}
Misalnya, kita dapat menghitung jumlah elemen yang memenuhi kondisi
tertentu dengan fungsi sum().
\end{eulercomment}
\begin{eulerprompt}
>sum((1:10)<6) // banyak elemen yang kurang dari 6
\end{eulerprompt}
\begin{euleroutput}
  5
\end{euleroutput}
\begin{eulercomment}
Euler memiliki operator perbandingan, seperti "==", yang memeriksa
kesetaraan.

Kita memperoleh vektor 0 dan 1, di mana 1 berarti benar.
\end{eulercomment}
\begin{eulerprompt}
>t=(1:10)^2; t==25 //menguji elemen2 t yang sama dengan 25 (hanya ada 1)
\end{eulerprompt}
\begin{euleroutput}
  [0,  0,  0,  0,  1,  0,  0,  0,  0,  0]
\end{euleroutput}
\begin{eulercomment}
Dari vektor tersebut, "nonzeros" memilih elemen yang bukan nol.

Dalam kasus ini, kita memperoleh indeks semua elemen yang lebih besar
dari 50.
\end{eulercomment}
\begin{eulerprompt}
>nonzeros(t>50) //indeks elemen2 t yang lebih besar daripada 50
\end{eulerprompt}
\begin{euleroutput}
  [8,  9,  10]
\end{euleroutput}
\begin{eulercomment}
Tentu saja, kita dapat menggunakan vektor indeks ini untuk mendapatkan
nilai yang sesuai dalam t.
\end{eulercomment}
\begin{eulerprompt}
>t[nonzeros(t>50)] //elemen2 t yang lebih besar daripada 50
\end{eulerprompt}
\begin{euleroutput}
  [64,  81,  100]
\end{euleroutput}
\begin{eulercomment}
Sebagai contoh, mari kita cari semua kuadrat angka 1 hingga 1000,
yaitu 5 modulo 11 dan 3 modulo 13.
\end{eulercomment}
\begin{eulerprompt}
>t=1:1000; nonzeros(mod(t^2,11)==5 && mod(t^2,13)==3)
\end{eulerprompt}
\begin{euleroutput}
  [4,  48,  95,  139,  147,  191,  238,  282,  290,  334,  381,  425,
  433,  477,  524,  568,  576,  620,  667,  711,  719,  763,  810,  854,
  862,  906,  953,  997]
\end{euleroutput}
\begin{eulercomment}
EMT tidak sepenuhnya efektif untuk komputasi integer. Ia menggunakan
floating point presisi ganda secara internal. Namun, ia sering kali
sangat berguna.

Kita dapat memeriksa keutamaan. Mari kita cari tahu, berapa banyak
kuadrat ditambah 1 yang merupakan bilangan prima.
\end{eulercomment}
\begin{eulerprompt}
>t=1:1000; length(nonzeros(isprime(t^2+1)))
\end{eulerprompt}
\begin{euleroutput}
  112
\end{euleroutput}
\begin{eulercomment}
Fungsi nonzeros() hanya berfungsi untuk vektor. Untuk matriks, ada
mnonzeros().
\end{eulercomment}
\begin{eulerprompt}
>seed(2); A=random(3,4)
\end{eulerprompt}
\begin{euleroutput}
       0.765761      0.401188      0.406347      0.267829 
        0.13673      0.390567      0.495975      0.952814 
       0.548138      0.006085      0.444255      0.539246 
\end{euleroutput}
\begin{eulercomment}
Mengembalikan indeks elemen, yang bukan nol.
\end{eulercomment}
\begin{eulerprompt}
>k=mnonzeros(A<0.4) //indeks elemen2 A yang kurang dari 0,4
\end{eulerprompt}
\begin{euleroutput}
              1             4 
              2             1 
              2             2 
              3             2 
\end{euleroutput}
\begin{eulercomment}
Indeks ini dapat digunakan untuk menetapkan elemen pada nilai
tertentu.
\end{eulercomment}
\begin{eulerprompt}
>mset(A,k,0) //mengganti elemen2 suatu matriks pada indeks tertentu
\end{eulerprompt}
\begin{euleroutput}
       0.765761      0.401188      0.406347             0 
              0             0      0.495975      0.952814 
       0.548138             0      0.444255      0.539246 
\end{euleroutput}
\begin{eulercomment}
Fungsi mset() juga dapat mengatur elemen pada indeks ke entri matriks
lainnya.
\end{eulercomment}
\begin{eulerprompt}
>mset(A,k,-random(size(A)))
\end{eulerprompt}
\begin{euleroutput}
       0.765761      0.401188      0.406347     -0.126917 
      -0.122404     -0.691673      0.495975      0.952814 
       0.548138     -0.483902      0.444255      0.539246 
\end{euleroutput}
\begin{eulercomment}
Dan adalah mungkin untuk mendapatkan unsur-unsur dalam sebuah vektor.
\end{eulercomment}
\begin{eulerprompt}
>mget(A,k)
\end{eulerprompt}
\begin{euleroutput}
  [0.267829,  0.13673,  0.390567,  0.006085]
\end{euleroutput}
\begin{eulercomment}
Fungsi lain yang berguna adalah extrema, yang mengembalikan nilai
minimal dan maksimal di setiap baris matriks dan posisinya.
\end{eulercomment}
\begin{eulerprompt}
>ex=extrema(A)
\end{eulerprompt}
\begin{euleroutput}
       0.267829             4      0.765761             1 
        0.13673             1      0.952814             4 
       0.006085             2      0.548138             1 
\end{euleroutput}
\begin{eulercomment}
Kita dapat menggunakan ini untuk mengekstrak nilai maksimal pada
setiap baris.
\end{eulercomment}
\begin{eulerprompt}
>ex[,3]'
\end{eulerprompt}
\begin{euleroutput}
  [0.765761,  0.952814,  0.548138]
\end{euleroutput}
\begin{eulercomment}
Ini tentu saja sama dengan fungsi max().
\end{eulercomment}
\begin{eulerprompt}
>max(A)'
\end{eulerprompt}
\begin{euleroutput}
  [0.765761,  0.952814,  0.548138]
\end{euleroutput}
\begin{eulercomment}
Tetapi dengan mget(), kita dapat mengekstrak indeks dan menggunakan
informasi ini untuk mengekstrak elemen pada posisi yang sama dari
matriks lain.
\end{eulercomment}
\begin{eulerprompt}
>j=(1:rows(A))'|ex[,4], mget(-A,j)
\end{eulerprompt}
\begin{euleroutput}
              1             1 
              2             4 
              3             1 
  [-0.765761,  -0.952814,  -0.548138]
\end{euleroutput}
\eulerheading{Fungsi Matriks Lainnya (Membangun Matriks)}
\begin{eulercomment}
Untuk membangun sebuah matriks, kita dapat menumpuk satu matriks di
atas matriks lainnya. Jika keduanya tidak memiliki jumlah kolom yang
sama, matriks yang lebih pendek akan diisi dengan 0.
\end{eulercomment}
\begin{eulerprompt}
>v=1:3; v_v
\end{eulerprompt}
\begin{euleroutput}
              1             2             3 
              1             2             3 
\end{euleroutput}
\begin{eulercomment}
Dengan cara yang sama, kita dapat menempelkan suatu matriks ke sisi
lain yang berdampingan, jika keduanya memiliki jumlah baris yang sama.
\end{eulercomment}
\begin{eulerprompt}
>A=random(3,4); A|v'
\end{eulerprompt}
\begin{euleroutput}
       0.032444     0.0534171      0.595713      0.564454             1 
        0.83916      0.175552      0.396988       0.83514             2 
      0.0257573      0.658585      0.629832      0.770895             3 
\end{euleroutput}
\begin{eulercomment}
Jika tidak memiliki jumlah baris yang sama, matriks yang lebih pendek
diisi dengan 0.

Ada pengecualian untuk aturan ini. Bilangan riil yang dilampirkan ke
matriks akan digunakan sebagai kolom yang diisi dengan bilangan riil
tersebut.
\end{eulercomment}
\begin{eulerprompt}
>A|1
\end{eulerprompt}
\begin{euleroutput}
       0.032444     0.0534171      0.595713      0.564454             1 
        0.83916      0.175552      0.396988       0.83514             1 
      0.0257573      0.658585      0.629832      0.770895             1 
\end{euleroutput}
\begin{eulercomment}
Dimungkinkan untuk membuat matriks dari vektor baris dan kolom.
\end{eulercomment}
\begin{eulerprompt}
>[v;v]
\end{eulerprompt}
\begin{euleroutput}
              1             2             3 
              1             2             3 
\end{euleroutput}
\begin{eulerprompt}
>[v',v']
\end{eulerprompt}
\begin{euleroutput}
              1             1 
              2             2 
              3             3 
\end{euleroutput}
\begin{eulercomment}
Tujuan utama dari ini adalah untuk menafsirkan vektor ekspresi untuk
vektor kolom.
\end{eulercomment}
\begin{eulerprompt}
>"[x,x^2]"(v')
\end{eulerprompt}
\begin{euleroutput}
              1             1 
              2             4 
              3             9 
\end{euleroutput}
\begin{eulercomment}
Untuk mendapatkan ukuran A, kita dapat menggunakan fungsi berikut.
\end{eulercomment}
\begin{eulerprompt}
>C=zeros(2,4); rows(C), cols(C), size(C), length(C)
\end{eulerprompt}
\begin{euleroutput}
  2
  4
  [2,  4]
  4
\end{euleroutput}
\begin{eulercomment}
Untuk vektor, ada length().
\end{eulercomment}
\begin{eulerprompt}
>length(2:10)
\end{eulerprompt}
\begin{euleroutput}
  9
\end{euleroutput}
\begin{eulercomment}
Ada banyak fungsi lain yang menghasilkan matriks.
\end{eulercomment}
\begin{eulerprompt}
>ones(2,2)
\end{eulerprompt}
\begin{euleroutput}
              1             1 
              1             1 
\end{euleroutput}
\begin{eulercomment}
Ini juga dapat digunakan dengan satu parameter. Untuk mendapatkan
vektor dengan angka selain 1, gunakan yang berikut ini.
\end{eulercomment}
\begin{eulerprompt}
>ones(5)*6
\end{eulerprompt}
\begin{euleroutput}
  [6,  6,  6,  6,  6]
\end{euleroutput}
\begin{eulercomment}
Matriks bilangan acak juga dapat dihasilkan dengan acak (distribusi
seragam) atau normal (distribusi Gauß).
\end{eulercomment}
\begin{eulerprompt}
>random(2,2)
\end{eulerprompt}
\begin{euleroutput}
        0.66566      0.831835 
          0.977      0.544258 
\end{euleroutput}
\begin{eulercomment}
Berikut adalah fungsi berguna lainnya, yang merestrukturisasi
elemen-elemen suatu matriks menjadi matriks lain.
\end{eulercomment}
\begin{eulerprompt}
>redim(1:9,3,3) // menyusun elemen2 1, 2, 3, ..., 9 ke bentuk matriks 3x3
\end{eulerprompt}
\begin{euleroutput}
              1             2             3 
              4             5             6 
              7             8             9 
\end{euleroutput}
\begin{eulercomment}
Dengan fungsi berikut, kita dapat menggunakan ini dan fungsi dup untuk
menulis fungsi rep(), yang mengulang vektor n kali.
\end{eulercomment}
\begin{eulerprompt}
>function rep(v,n) := redim(dup(v,n),1,n*cols(v))
\end{eulerprompt}
\begin{eulercomment}
Mari kita menguji.
\end{eulercomment}
\begin{eulerprompt}
>rep(1:3,5)
\end{eulerprompt}
\begin{euleroutput}
  [1,  2,  3,  1,  2,  3,  1,  2,  3,  1,  2,  3,  1,  2,  3]
\end{euleroutput}
\begin{eulercomment}
Fungsi multdup() menduplikasi elemen suatu vektor.
\end{eulercomment}
\begin{eulerprompt}
>multdup(1:3,5), multdup(1:3,[2,3,2])
\end{eulerprompt}
\begin{euleroutput}
  [1,  1,  1,  1,  1,  2,  2,  2,  2,  2,  3,  3,  3,  3,  3]
  [1,  1,  2,  2,  2,  3,  3]
\end{euleroutput}
\begin{eulercomment}
Fungsi flipx() dan flipy() membalikkan urutan baris atau kolom
matriks. Yaitu, fungsi flipx() membalik secara horizontal.
\end{eulercomment}
\begin{eulerprompt}
>flipx(1:5) //membalik elemen2 vektor baris
\end{eulerprompt}
\begin{euleroutput}
  [5,  4,  3,  2,  1]
\end{euleroutput}
\begin{eulercomment}
Untuk rotasi, Euler memiliki rotleft() dan rotright().
\end{eulercomment}
\begin{eulerprompt}
>rotleft(1:5) // memutar elemen2 vektor baris
\end{eulerprompt}
\begin{euleroutput}
  [2,  3,  4,  5,  1]
\end{euleroutput}
\begin{eulercomment}
Fungsi khusus adalah drop(v,i), yang menghapus elemen dengan indeks di
i dari vektor v.
\end{eulercomment}
\begin{eulerprompt}
>drop(10:20,3)
\end{eulerprompt}
\begin{euleroutput}
  [10,  11,  13,  14,  15,  16,  17,  18,  19,  20]
\end{euleroutput}
\begin{eulercomment}
Perhatikan bahwa vektor i dalam drop(v,i) merujuk pada indeks elemen
dalam v, bukan nilai elemen. Jika Anda ingin menghapus elemen, Anda
perlu menemukan elemen terlebih dahulu. Fungsi indexof(v,x) dapat
digunakan untuk menemukan elemen x dalam vektor v yang diurutkan.
\end{eulercomment}
\begin{eulerprompt}
>v=primes(50), i=indexof(v,10:20), drop(v,i)
\end{eulerprompt}
\begin{euleroutput}
  [2,  3,  5,  7,  11,  13,  17,  19,  23,  29,  31,  37,  41,  43,  47]
  [0,  5,  0,  6,  0,  0,  0,  7,  0,  8,  0]
  [2,  3,  5,  7,  23,  29,  31,  37,  41,  43,  47]
\end{euleroutput}
\begin{eulercomment}
Seperti yang Anda lihat, tidak ada salahnya menyertakan indeks di luar
rentang (seperti 0), indeks ganda, atau indeks yang tidak diurutkan.
\end{eulercomment}
\begin{eulerprompt}
>drop(1:10,shuffle([0,0,5,5,7,12,12]))
\end{eulerprompt}
\begin{euleroutput}
  [1,  2,  3,  4,  6,  8,  9,  10]
\end{euleroutput}
\begin{eulercomment}
Ada beberapa fungsi khusus untuk mengatur diagonal atau membuat
matriks diagonal.

Kita mulai dengan matriks identitas.
\end{eulercomment}
\begin{eulerprompt}
>A=id(5) // matriks identitas 5x5
\end{eulerprompt}
\begin{euleroutput}
              1             0             0             0             0 
              0             1             0             0             0 
              0             0             1             0             0 
              0             0             0             1             0 
              0             0             0             0             1 
\end{euleroutput}
\begin{eulercomment}
Kemudian kita atur diagonal bawah (-1) menjadi 1:4.
\end{eulercomment}
\begin{eulerprompt}
>setdiag(A,-1,1:4) //mengganti diagonal di bawah diagonal utama
\end{eulerprompt}
\begin{euleroutput}
              1             0             0             0             0 
              1             1             0             0             0 
              0             2             1             0             0 
              0             0             3             1             0 
              0             0             0             4             1 
\end{euleroutput}
\begin{eulercomment}
Perhatikan bahwa kita tidak mengubah matriks A. Kita mendapatkan
matriks baru sebagai hasil dari setdiag().

Berikut ini adalah fungsi yang mengembalikan matriks tri-diagonal.
\end{eulercomment}
\begin{eulerprompt}
>function tridiag (n,a,b,c) := setdiag(setdiag(b*id(n),1,c),-1,a); ...
>tridiag(5,1,2,3)
\end{eulerprompt}
\begin{euleroutput}
              2             3             0             0             0 
              1             2             3             0             0 
              0             1             2             3             0 
              0             0             1             2             3 
              0             0             0             1             2 
\end{euleroutput}
\begin{eulercomment}
Diagonal matriks juga dapat diekstraksi dari matriks. Untuk
menunjukkan hal ini, kami merestrukturisasi vektor 1:9 menjadi matriks
3x3.
\end{eulercomment}
\begin{eulerprompt}
>A=redim(1:9,3,3)
\end{eulerprompt}
\begin{euleroutput}
              1             2             3 
              4             5             6 
              7             8             9 
\end{euleroutput}
\begin{eulercomment}
Sekarang kita dapat mengekstrak diagonalnya.
\end{eulercomment}
\begin{eulerprompt}
>d=getdiag(A,0)
\end{eulerprompt}
\begin{euleroutput}
  [1,  5,  9]
\end{euleroutput}
\begin{eulercomment}
Misalnya, kita dapat membagi matriks berdasarkan diagonalnya. Bahasa
matriks memastikan bahwa vektor kolom d diterapkan ke matriks baris
demi baris.
\end{eulercomment}
\begin{eulerprompt}
>fraction A/d'
\end{eulerprompt}
\begin{euleroutput}
          1         2         3 
        4/5         1       6/5 
        7/9       8/9         1 
\end{euleroutput}
\eulerheading{Vektorisasi}
\begin{eulercomment}
Hampir semua fungsi di Euler juga berfungsi untuk masukan matriks dan
vektor, jika ini masuk akal.

Misalnya, fungsi sqrt() menghitung akar kuadrat dari semua elemen
vektor atau matriks.
\end{eulercomment}
\begin{eulerprompt}
>sqrt(1:3)
\end{eulerprompt}
\begin{euleroutput}
  [1,  1.41421,  1.73205]
\end{euleroutput}
\begin{eulercomment}
Jadi Anda dapat dengan mudah membuat tabel nilai. Ini adalah salah
satu cara untuk memplot fungsi (alternatifnya menggunakan ekspresi).
\end{eulercomment}
\begin{eulerprompt}
>x=1:0.01:5; y=log(x)/x^2; // terlalu panjang untuk ditampikan
\end{eulerprompt}
\begin{eulercomment}
Dengan ini dan operator titik dua a:delta:b, vektor nilai fungsi dapat
dibuat dengan mudah.

Dalam contoh berikut, kita buat vektor nilai t[i] dengan spasi 0,1
dari -1 hingga 1. Kemudian kita buat vektor nilai fungsi

\end{eulercomment}
\begin{eulerformula}
\[
s = t^3-t
\]
\end{eulerformula}
\begin{eulerprompt}
>t=-1:0.1:1; s=t^3-t
\end{eulerprompt}
\begin{euleroutput}
  [0,  0.171,  0.288,  0.357,  0.384,  0.375,  0.336,  0.273,  0.192,
  0.099,  0,  -0.099,  -0.192,  -0.273,  -0.336,  -0.375,  -0.384,
  -0.357,  -0.288,  -0.171,  0]
\end{euleroutput}
\begin{eulercomment}
EMT mengembangkan operator untuk skalar, vektor, dan matriks dengan
cara yang jelas.

Misalnya, vektor kolom dikalikan vektor baris akan berkembang menjadi
matriks, jika operator diterapkan. Berikut ini, v' adalah vektor yang
ditransposisikan (vektor kolom).
\end{eulercomment}
\begin{eulerprompt}
>shortest (1:5)*(1:5)'
\end{eulerprompt}
\begin{euleroutput}
       1      2      3      4      5 
       2      4      6      8     10 
       3      6      9     12     15 
       4      8     12     16     20 
       5     10     15     20     25 
\end{euleroutput}
\begin{eulercomment}
Perhatikan bahwa ini sangat berbeda dari perkalian matriks. Perkalian
matriks dilambangkan dengan titik "." dalam EMT.
\end{eulercomment}
\begin{eulerprompt}
>(1:5).(1:5)'
\end{eulerprompt}
\begin{euleroutput}
  55
\end{euleroutput}
\begin{eulercomment}
Secara default, vektor baris dicetak dalam format ringkas.
\end{eulercomment}
\begin{eulerprompt}
>[1,2,3,4]
\end{eulerprompt}
\begin{euleroutput}
  [1,  2,  3,  4]
\end{euleroutput}
\begin{eulercomment}
Untuk matriks, operator khusus . menunjukkan perkalian matriks, dan A'
menunjukkan transposisi. Matriks 1x1 dapat digunakan seperti bilangan
riil.
\end{eulercomment}
\begin{eulerprompt}
>v:=[1,2]; v.v', %^2
\end{eulerprompt}
\begin{euleroutput}
  5
  25
\end{euleroutput}
\begin{eulercomment}
Untuk mentranspos suatu matriks, kita menggunakan tanda apostrof.
\end{eulercomment}
\begin{eulerprompt}
>v=1:4; v'
\end{eulerprompt}
\begin{euleroutput}
              1 
              2 
              3 
              4 
\end{euleroutput}
\begin{eulercomment}
Jadi kita dapat menghitung matriks A dikali vektor b.
\end{eulercomment}
\begin{eulerprompt}
>A=[1,2,3,4;5,6,7,8]; A.v'
\end{eulerprompt}
\begin{euleroutput}
             30 
             70 
\end{euleroutput}
\begin{eulercomment}
Perhatikan bahwa v masih merupakan vektor baris. Jadi v'.v berbeda
dengan v.v'.
\end{eulercomment}
\begin{eulerprompt}
>v'.v
\end{eulerprompt}
\begin{euleroutput}
              1             2             3             4 
              2             4             6             8 
              3             6             9            12 
              4             8            12            16 
\end{euleroutput}
\begin{eulercomment}
v.v' menghitung norma v kuadrat untuk vektor baris v. Hasilnya adalah
vektor 1x1, yang berfungsi seperti bilangan real.
\end{eulercomment}
\begin{eulerprompt}
>v.v'
\end{eulerprompt}
\begin{euleroutput}
  30
\end{euleroutput}
\begin{eulercomment}
Ada juga fungsi norma (bersama dengan banyak fungsi Aljabar Linier
lainnya).
\end{eulercomment}
\begin{eulerprompt}
>norm(v)^2
\end{eulerprompt}
\begin{euleroutput}
  30
\end{euleroutput}
\begin{eulercomment}
Operator dan fungsi mematuhi bahasa matriks Euler.

Berikut ringkasan peraturannya.

- Suatu fungsi yang diterapkan pada vektor atau matriks diterapkan
pada setiap elemen.

- Operator yang mengoperasikan dua matriks dengan ukuran yang sama
diterapkan secara berpasangan pada elemen-elemen matriks.

- Jika kedua matriks mempunyai dimensi yang berbeda, keduanya
diekspansi secara rasional sehingga mempunyai ukuran yang sama.

Misalnya, nilai skalar dikalikan vektor dengan mengalikan nilai setiap
elemen vektor. Atau matriks dikalikan vektor (dengan *, bukan .)
memperluas vektor ke ukuran matriks dengan menduplikasinya.

Berikut ini adalah kasus sederhana dengan operator \textasciicircum{}.
\end{eulercomment}
\begin{eulerprompt}
>[1,2,3]^2
\end{eulerprompt}
\begin{euleroutput}
  [1,  4,  9]
\end{euleroutput}
\begin{eulercomment}
Ini kasus yang lebih rumit. Vektor baris dikalikan vektor kolom
memperluas keduanya dengan cara menduplikasi.
\end{eulercomment}
\begin{eulerprompt}
>v:=[1,2,3]; v*v'
\end{eulerprompt}
\begin{euleroutput}
              1             2             3 
              2             4             6 
              3             6             9 
\end{euleroutput}
\begin{eulercomment}
Perhatikan bahwa perkalian skalar menggunakan perkalian matriks, bukan
*!
\end{eulercomment}
\begin{eulerprompt}
>v.v'
\end{eulerprompt}
\begin{euleroutput}
  14
\end{euleroutput}
\begin{eulercomment}
Ada banyak fungsi matriks. Kami memberikan daftar singkat. Anda harus
membaca dokumentasi untuk informasi lebih lanjut tentang perintah ini.

\end{eulercomment}
\begin{eulerttcomment}
  sum,prod menghitung jumlah dan hasil kali baris
  cumsum,cumprod melakukan hal yang sama secara kumulatif
  menghitung nilai ekstrem setiap baris
  extreme mengembalikan vektor dengan informasi ekstrem
  diag(A,i) mengembalikan diagonal ke-i
  setdiag(A,i,v) menyetel diagonal ke-i
  id(n) matriks identitas
  det(A) determinannya
  charpoly(A) polinomial karakteristik
  nilai eigen(A) nilai eigen
\end{eulerttcomment}
\begin{eulerprompt}
>v*v, sum(v*v), cumsum(v*v)
\end{eulerprompt}
\begin{euleroutput}
  [1,  4,  9]
  14
  [1,  5,  14]
\end{euleroutput}
\begin{eulercomment}
Operator : menghasilkan vektor baris dengan spasi yang sama, opsional
dengan ukuran langkah.
\end{eulercomment}
\begin{eulerprompt}
>1:4, 1:2:10
\end{eulerprompt}
\begin{euleroutput}
  [1,  2,  3,  4]
  [1,  3,  5,  7,  9]
\end{euleroutput}
\begin{eulercomment}
Untuk menggabungkan matriks dan vektor terdapat operator "\textbar{}" Dan "\_".
\end{eulercomment}
\begin{eulerprompt}
>[1,2,3]|[4,5], [1,2,3]_1
\end{eulerprompt}
\begin{euleroutput}
  [1,  2,  3,  4,  5]
              1             2             3 
              1             1             1 
\end{euleroutput}
\begin{eulercomment}
Elemen-elemen matriks disebut dengan "A[i,j]".
\end{eulercomment}
\begin{eulerprompt}
>A:=[1,2,3;4,5,6;7,8,9]; A[2,3]
\end{eulerprompt}
\begin{euleroutput}
  6
\end{euleroutput}
\begin{eulercomment}
Untuk vektor baris atau kolom, v[i] adalah elemen ke-i dari vektor
tersebut. Untuk matriks, ini mengembalikan baris ke-i yang lengkap
dari matriks tersebut.
\end{eulercomment}
\begin{eulerprompt}
>v:=[2,4,6,8]; v[3], A[3]
\end{eulerprompt}
\begin{euleroutput}
  6
  [7,  8,  9]
\end{euleroutput}
\begin{eulercomment}
Indeks juga dapat berupa vektor baris dari indeks. : menunjukkan semua
indeks.
\end{eulercomment}
\begin{eulerprompt}
>v[1:2], A[:,2]
\end{eulerprompt}
\begin{euleroutput}
  [2,  4]
              2 
              5 
              8 
\end{euleroutput}
\begin{eulercomment}
Bentuk kependekan dari : menghilangkan indeks sepenuhnya.
\end{eulercomment}
\begin{eulerprompt}
>A[,2:3]
\end{eulerprompt}
\begin{euleroutput}
              2             3 
              5             6 
              8             9 
\end{euleroutput}
\begin{eulercomment}
Untuk tujuan vektorisasi, elemen matriks dapat diakses seolah-olah
elemen tersebut adalah vektor.
\end{eulercomment}
\begin{eulerprompt}
>A\{4\}
\end{eulerprompt}
\begin{euleroutput}
  4
\end{euleroutput}
\begin{eulercomment}
Matriks juga dapat diratakan menggunakan fungsi redim(). Ini
diimplementasikan dalam fungsi flatten().
\end{eulercomment}
\begin{eulerprompt}
>redim(A,1,prod(size(A))), flatten(A)
\end{eulerprompt}
\begin{euleroutput}
  [1,  2,  3,  4,  5,  6,  7,  8,  9]
  [1,  2,  3,  4,  5,  6,  7,  8,  9]
\end{euleroutput}
\begin{eulercomment}
Untuk menggunakan matriks pada tabel, mari kita atur ulang ke format
default, dan hitung tabel nilai sinus dan kosinus. Perhatikan bahwa
sudut dinyatakan dalam radian secara default.
\end{eulercomment}
\begin{eulerprompt}
>defformat; w=0°:45°:360°; w=w'; deg(w)
\end{eulerprompt}
\begin{euleroutput}
              0 
             45 
             90 
            135 
            180 
            225 
            270 
            315 
            360 
\end{euleroutput}
\begin{eulercomment}
Sekarang kita menambahkan kolom ke matriks.
\end{eulercomment}
\begin{eulerprompt}
>M = deg(w)|w|cos(w)|sin(w)
\end{eulerprompt}
\begin{euleroutput}
              0             0             1             0 
             45      0.785398      0.707107      0.707107 
             90        1.5708             0             1 
            135       2.35619     -0.707107      0.707107 
            180       3.14159            -1             0 
            225       3.92699     -0.707107     -0.707107 
            270       4.71239             0            -1 
            315       5.49779      0.707107     -0.707107 
            360       6.28319             1             0 
\end{euleroutput}
\begin{eulercomment}
Dengan menggunakan bahasa matriks, kita dapat menghasilkan beberapa
tabel dari beberapa fungsi sekaligus.

Dalam contoh berikut, kita menghitung t[j]\textasciicircum{}i untuk i dari 1 hingga n.
Kita mendapatkan sebuah matriks, yang setiap barisnya merupakan tabel
t\textasciicircum{}i untuk satu i. Artinya, matriks tersebut memiliki elemen \\
\end{eulercomment}
\begin{eulerformula}
\[
a_{i,j} = t_j^i, \quad 1 \le j \le 101, \quad 1 \le i \le n
\]
\end{eulerformula}
\begin{eulercomment}
Fungsi yang tidak berfungsi untuk masukan vektor harus "divektorkan".
Hal ini dapat dicapai dengan kata kunci "peta" dalam definisi fungsi.
Kemudian fungsi tersebut akan dievaluasi untuk setiap elemen parameter
vektor.

Integrasi numerik integral() hanya berfungsi untuk batas interval
skalar. Jadi kita perlu membuat vektorisasinya.
\end{eulercomment}
\begin{eulerprompt}
>function map f(x) := integrate("x^x",1,x)
\end{eulerprompt}
\begin{eulercomment}
Kata kunci "peta" membuat vektorisasi fungsi tersebut. Fungsinya
sekarang akan berfungsi\\
untuk vektor bilangan.
\end{eulercomment}
\begin{eulerprompt}
>f([1:5])
\end{eulerprompt}
\begin{euleroutput}
  [0,  2.05045,  13.7251,  113.336,  1241.03]
\end{euleroutput}
\eulerheading{Sub-Matriks dan Elemen Matriks}
\begin{eulercomment}
Untuk mengakses elemen matriks, gunakan notasi braket.
\end{eulercomment}
\begin{eulerprompt}
>A=[1,2,3;4,5,6;7,8,9], A[2,2]
\end{eulerprompt}
\begin{euleroutput}
              1             2             3 
              4             5             6 
              7             8             9 
  5
\end{euleroutput}
\begin{eulercomment}
Kita dapat mengakses baris matriks secara lengkap.
\end{eulercomment}
\begin{eulerprompt}
>A[2]
\end{eulerprompt}
\begin{euleroutput}
  [4,  5,  6]
\end{euleroutput}
\begin{eulercomment}
Dalam kasus vektor baris atau kolom, ini mengembalikan elemen vektor.
\end{eulercomment}
\begin{eulerprompt}
>v=1:3; v[2]
\end{eulerprompt}
\begin{euleroutput}
  2
\end{euleroutput}
\begin{eulercomment}
Untuk memastikan, Anda mendapatkan baris pertama untuk matriks 1xn dan
mxn, tentukan semua kolom menggunakan indeks kedua yang kosong.
\end{eulercomment}
\begin{eulerprompt}
>A[2,]
\end{eulerprompt}
\begin{euleroutput}
  [4,  5,  6]
\end{euleroutput}
\begin{eulercomment}
Jika indeks adalah vektor dari indeks, Euler akan mengembalikan baris
matriks yang sesuai.

Di sini kita menginginkan baris pertama dan kedua A.
\end{eulercomment}
\begin{eulerprompt}
>A[[1,2]]
\end{eulerprompt}
\begin{euleroutput}
              1             2             3 
              4             5             6 
\end{euleroutput}
\begin{eulercomment}
Kita bahkan dapat menyusun ulang A menggunakan vektor indeks.
Tepatnya, kita tidak mengubah A di sini, namun menghitung versi A yang
disusun ulang.
\end{eulercomment}
\begin{eulerprompt}
>A[[3,2,1]]
\end{eulerprompt}
\begin{euleroutput}
              7             8             9 
              4             5             6 
              1             2             3 
\end{euleroutput}
\begin{eulercomment}
Trik indeks juga berfungsi dengan kolom.

Contoh ini memilih semua baris A dan kolom kedua dan ketiga.
\end{eulercomment}
\begin{eulerprompt}
>A[1:3,2:3]
\end{eulerprompt}
\begin{euleroutput}
              2             3 
              5             6 
              8             9 
\end{euleroutput}
\begin{eulercomment}
Untuk singkatan ":" menunjukkan semua indeks baris atau kolom.
\end{eulercomment}
\begin{eulerprompt}
>A[:,3]
\end{eulerprompt}
\begin{euleroutput}
              3 
              6 
              9 
\end{euleroutput}
\begin{eulercomment}
Alternatifnya, biarkan indeks pertama kosong.
\end{eulercomment}
\begin{eulerprompt}
>A[,2:3]
\end{eulerprompt}
\begin{euleroutput}
              2             3 
              5             6 
              8             9 
\end{euleroutput}
\begin{eulercomment}
Kita juga bisa mendapatkan baris terakhir A.
\end{eulercomment}
\begin{eulerprompt}
>A[-1]
\end{eulerprompt}
\begin{euleroutput}
  [7,  8,  9]
\end{euleroutput}
\begin{eulercomment}
Sekarang mari kita ubah elemen A dengan menetapkan submatriks A ke
suatu nilai. Ini sebenarnya mengubah matriks A yang disimpan.
\end{eulercomment}
\begin{eulerprompt}
>A[1,1]=4
\end{eulerprompt}
\begin{euleroutput}
              4             2             3 
              4             5             6 
              7             8             9 
\end{euleroutput}
\begin{eulercomment}
Kita juga dapat memberikan nilai pada baris A.
\end{eulercomment}
\begin{eulerprompt}
>A[1]=[-1,-1,-1]
\end{eulerprompt}
\begin{euleroutput}
             -1            -1            -1 
              4             5             6 
              7             8             9 
\end{euleroutput}
\begin{eulercomment}
Kita bahkan dapat menetapkan sub-matriks jika ukurannya sesuai.
\end{eulercomment}
\begin{eulerprompt}
>A[1:2,1:2]=[5,6;7,8]
\end{eulerprompt}
\begin{euleroutput}
              5             6            -1 
              7             8             6 
              7             8             9 
\end{euleroutput}
\begin{eulercomment}
Selain itu, beberapa jalan pintas diperbolehkan.
\end{eulercomment}
\begin{eulerprompt}
>A[1:2,1:2]=0
\end{eulerprompt}
\begin{euleroutput}
              0             0            -1 
              0             0             6 
              7             8             9 
\end{euleroutput}
\begin{eulercomment}
Peringatan: Indeks di luar batas mengembalikan matriks kosong, atau
pesan kesalahan, bergantung pada pengaturan sistem. Standarnya adalah
pesan kesalahan. Namun perlu diingat bahwa indeks negatif dapat
digunakan untuk mengakses elemen matriks yang dihitung dari akhir.
\end{eulercomment}
\begin{eulerprompt}
>A[4]
\end{eulerprompt}
\begin{euleroutput}
  Row index 4 out of bounds!
  Error in:
  A[4] ...
      ^
\end{euleroutput}
\eulerheading{Menyortir dan Mengacak}
\begin{eulercomment}
Fungsi sort() mengurutkan vektor baris.
\end{eulercomment}
\begin{eulerprompt}
>sort([5,6,4,8,1,9])
\end{eulerprompt}
\begin{euleroutput}
  [1,  4,  5,  6,  8,  9]
\end{euleroutput}
\begin{eulercomment}
Seringkali perlu mengetahui indeks vektor yang diurutkan dalam vektor
aslinya. Ini dapat digunakan untuk menyusun ulang vektor lain dengan
cara yang sama.

Mari kita mengacak sebuah vektor.
\end{eulercomment}
\begin{eulerprompt}
>v=shuffle(1:10)
\end{eulerprompt}
\begin{euleroutput}
  [4,  5,  10,  6,  8,  9,  1,  7,  2,  3]
\end{euleroutput}
\begin{eulercomment}
Indeks berisi urutan v.
\end{eulercomment}
\begin{eulerprompt}
>\{vs,ind\}=sort(v); v[ind]
\end{eulerprompt}
\begin{euleroutput}
  [1,  2,  3,  4,  5,  6,  7,  8,  9,  10]
\end{euleroutput}
\begin{eulercomment}
Ini juga berfungsi untuk vektor string.
\end{eulercomment}
\begin{eulerprompt}
>s=["a","d","e","a","aa","e"]
\end{eulerprompt}
\begin{euleroutput}
  a
  d
  e
  a
  aa
  e
\end{euleroutput}
\begin{eulerprompt}
>\{ss,ind\}=sort(s); ss
\end{eulerprompt}
\begin{euleroutput}
  a
  a
  aa
  d
  e
  e
\end{euleroutput}
\begin{eulercomment}
Seperti yang Anda lihat, posisi entri ganda agak acak.
\end{eulercomment}
\begin{eulerprompt}
>ind
\end{eulerprompt}
\begin{euleroutput}
  [4,  1,  5,  2,  6,  3]
\end{euleroutput}
\begin{eulercomment}
Fungsi unik mengembalikan daftar elemen unik vektor yang diurutkan.
\end{eulercomment}
\begin{eulerprompt}
>intrandom(1,10,10), unique(%)
\end{eulerprompt}
\begin{euleroutput}
  [4,  4,  9,  2,  6,  5,  10,  6,  5,  1]
  [1,  2,  4,  5,  6,  9,  10]
\end{euleroutput}
\begin{eulercomment}
Ini juga berfungsi untuk vektor string.
\end{eulercomment}
\begin{eulerprompt}
>unique(s)
\end{eulerprompt}
\begin{euleroutput}
  a
  aa
  d
  e
\end{euleroutput}
\eulerheading{Aljabar Linier}
\begin{eulercomment}
EMT memiliki banyak sekali fungsi untuk menyelesaikan masalah sistem
linier, sistem sparse, atau regresi.

Untuk sistem linier Ax=b, Anda dapat menggunakan algoritma Gauss,
matriks invers, atau linear fit. Operator A\textbackslash{}b menggunakan versi
algoritma Gauss.
\end{eulercomment}
\begin{eulerprompt}
>A=[1,2;3,4]; b=[5;6]; A\(\backslash\)b
\end{eulerprompt}
\begin{euleroutput}
             -4 
            4.5 
\end{euleroutput}
\begin{eulercomment}
Contoh lain, kita membuat matriks berukuran 200x200 dan jumlah
baris-barisnya. Kemudian kita selesaikan Ax=b menggunakan matriks
invers. Kami mengukur kesalahan sebagai deviasi maksimal semua elemen
dari 1, yang tentu saja merupakan solusi yang tepat.
\end{eulercomment}
\begin{eulerprompt}
>A=normal(200,200); b=sum(A); longest totalmax(abs(inv(A).b-1))
\end{eulerprompt}
\begin{euleroutput}
    8.790745908981989e-13 
\end{euleroutput}
\begin{eulercomment}
Jika sistem tidak mempunyai solusi, kecocokan linier meminimalkan
norma kesalahan Ax-b.
\end{eulercomment}
\begin{eulerprompt}
>A=[1,2,3;4,5,6;7,8,9]
\end{eulerprompt}
\begin{euleroutput}
              1             2             3 
              4             5             6 
              7             8             9 
\end{euleroutput}
\begin{eulercomment}
Penentu matriks ini adalah 0.
\end{eulercomment}
\begin{eulerprompt}
>det(A)
\end{eulerprompt}
\begin{euleroutput}
  0
\end{euleroutput}
\eulerheading{Matriks Simbolik}
\begin{eulercomment}
Maxima memiliki matriks simbolik. Tentu saja Maxima dapat digunakan
untuk permasalahan aljabar linier sederhana seperti itu. Kita dapat
mendefinisikan matriks untuk Euler dan Maxima dengan \&:=, lalu
menggunakannya dalam ekspresi simbolik. Bentuk [...] yang biasa untuk
mendefinisikan matriks dapat digunakan di Euler untuk mendefinisikan
matriks simbolik.
\end{eulercomment}
\begin{eulerprompt}
>A &= [a,1,1;1,a,1;1,1,a]; $A
>$&det(A), $&factor(%)
>$&invert(A) with a=0
>A &= [1,a;b,2]; $A
\end{eulerprompt}
\begin{eulercomment}
Seperti semua variabel simbolik, matriks ini dapat digunakan dalam
ekspresi simbolik lainnya.
\end{eulercomment}
\begin{eulerprompt}
>$&det(A-x*ident(2)), $&solve(%,x)
\end{eulerprompt}
\begin{eulercomment}
Nilai eigen juga dapat dihitung secara otomatis. Hasilnya adalah
sebuah vektor dengan dua vektor nilai eigen dan multiplisitas.
\end{eulercomment}
\begin{eulerprompt}
>$&eigenvalues([a,1;1,a])
\end{eulerprompt}
\begin{eulercomment}
Untuk mengekstrak vektor eigen tertentu memerlukan pengindeksan yang
cermat.
\end{eulercomment}
\begin{eulerprompt}
>$&eigenvectors([a,1;1,a]), &%[2][1][1]
\end{eulerprompt}
\begin{euleroutput}
  
                                 [1, - 1]
  
\end{euleroutput}
\begin{eulercomment}
Matriks simbolik dapat dievaluasi dalam Euler secara numerik sama
seperti ekspresi simbolik lainnya.
\end{eulercomment}
\begin{eulerprompt}
>A(a=4,b=5)
\end{eulerprompt}
\begin{euleroutput}
              1             4 
              5             2 
\end{euleroutput}
\begin{eulercomment}
Dalam ekspresi simbolik, gunakan dengan.
\end{eulercomment}
\begin{eulerprompt}
>$&A with [a=4,b=5]
\end{eulerprompt}
\begin{eulercomment}
Akses ke deretan matriks simbolik berfungsi sama seperti matriks
numerik.
\end{eulercomment}
\begin{eulerprompt}
>$&A[1]
\end{eulerprompt}
\begin{eulercomment}
Ekspresi simbolis dapat berisi tugas. Dan itu mengubah matriks A.
\end{eulercomment}
\begin{eulerprompt}
>&A[1,1]:=t+1; $&A
\end{eulerprompt}
\begin{eulercomment}
Ada fungsi simbolik di Maxima untuk membuat vektor dan matriks. Untuk
ini, lihat dokumentasi Maxima atau tutorial tentang Maxima di EMT.
\end{eulercomment}
\begin{eulerprompt}
>v &= makelist(1/(i+j),i,1,3); $v
\end{eulerprompt}
\begin{eulerttcomment}
 
\end{eulerttcomment}
\begin{eulerprompt}
>B &:= [1,2;3,4]; $B, $&invert(B)
\end{eulerprompt}
\begin{eulercomment}
Hasilnya dapat dievaluasi secara numerik dalam Euler. Untuk informasi
lebih lanjut tentang Maxima, lihat pengenalan Maxima.
\end{eulercomment}
\begin{eulerprompt}
>$&invert(B)()
\end{eulerprompt}
\begin{euleroutput}
             -2             1 
            1.5          -0.5 
\end{euleroutput}
\begin{eulercomment}
Euler juga memiliki fungsi kuat xinv(), yang melakukan upaya lebih
besar dan mendapatkan hasil yang lebih tepat.

Perhatikan, bahwa dengan \&:= matriks B telah didefinisikan sebagai
simbolik dalam ekspresi simbolik dan numerik dalam ekspresi numerik.
Jadi kita bisa menggunakannya di sini.
\end{eulercomment}
\begin{eulerprompt}
>longest B.xinv(B)
\end{eulerprompt}
\begin{euleroutput}
                        1                       0 
                        0                       1 
\end{euleroutput}
\begin{eulercomment}
Misalnya. nilai eigen dari A dapat dihitung secara numerik.
\end{eulercomment}
\begin{eulerprompt}
>A=[1,2,3;4,5,6;7,8,9]; real(eigenvalues(A))
\end{eulerprompt}
\begin{euleroutput}
  [16.1168,  -1.11684,  0]
\end{euleroutput}
\begin{eulercomment}
Atau secara simbolis. Lihat tutorial tentang Maxima untuk detailnya.
\end{eulercomment}
\begin{eulerprompt}
>$&eigenvalues(@A)
\end{eulerprompt}
\eulerheading{Nilai Numerik dalam Ekspresi simbolik}
\begin{eulercomment}
Ekspresi simbolik hanyalah string yang berisi ekspresi. Jika kita
ingin mendefinisikan nilai untuk ekspresi simbolik dan ekspresi
numerik, kita harus menggunakan "\&:=".
\end{eulercomment}
\begin{eulerprompt}
>A &:= [1,pi;4,5]
\end{eulerprompt}
\begin{euleroutput}
              1       3.14159 
              4             5 
\end{euleroutput}
\begin{eulercomment}
Masih terdapat perbedaan antara bentuk numerik dan simbolik. Saat
mentransfer matriks ke bentuk simbolik, pendekatan pecahan untuk real
akan digunakan.
\end{eulercomment}
\begin{eulerprompt}
>$&A
\end{eulerprompt}
\begin{eulercomment}
Untuk menghindari hal ini, ada fungsi "mxmset(variabel)".
\end{eulercomment}
\begin{eulerprompt}
>mxmset(A); $&A
\end{eulerprompt}
\begin{eulercomment}
Maxima juga dapat menghitung dengan bilangan floating point, bahkan
dengan bilangan mengambang besar dengan 32 digit. Namun, evaluasinya
jauh lebih lambat.
\end{eulercomment}
\begin{eulerprompt}
>$&bfloat(sqrt(2)), $&float(sqrt(2))
\end{eulerprompt}
\begin{eulercomment}
Ketepatan angka floating point besar dapat diubah.
\end{eulercomment}
\begin{eulerprompt}
>&fpprec:=100; &bfloat(pi)
\end{eulerprompt}
\begin{euleroutput}
  
          3.14159265358979323846264338327950288419716939937510582097494\(\backslash\)
  4592307816406286208998628034825342117068b0
  
\end{euleroutput}
\begin{eulercomment}
Variabel numerik dapat digunakan dalam ekspresi simbolik apa pun
menggunakan "@var".

Perhatikan bahwa ini hanya diperlukan, jika variabel telah
didefinisikan dengan ":=" atau "=" sebagai variabel numerik.
\end{eulercomment}
\begin{eulerprompt}
>B:=[1,pi;3,4]; $&det(@B)
\end{eulerprompt}
\eulerheading{Demo - Suku Bunga}
\begin{eulercomment}
Di bawah ini, kami menggunakan Euler Math Toolbox (EMT) untuk
menghitung suku bunga. Kami melakukannya secara numerik dan simbolis
untuk menunjukkan kepada Anda bagaimana Euler dapat digunakan untuk
memecahkan masalah kehidupan nyata.

Asumsikan Anda memiliki modal awal sebesar 5.000 (katakanlah dalam
dolar).
\end{eulercomment}
\begin{eulerprompt}
>K=5000
\end{eulerprompt}
\begin{euleroutput}
  5000
\end{euleroutput}
\begin{eulercomment}
Sekarang kami mengasumsikan tingkat bunga 3\% per tahun. Mari kita
tambahkan satu tarif sederhana dan hitung hasilnya.
\end{eulercomment}
\begin{eulerprompt}
>K*1.03
\end{eulerprompt}
\begin{euleroutput}
  5150
\end{euleroutput}
\begin{eulercomment}
Euler juga akan memahami sintaks berikut.
\end{eulercomment}
\begin{eulerprompt}
>K+K*3%
\end{eulerprompt}
\begin{euleroutput}
  5150
\end{euleroutput}
\begin{eulercomment}
Namun lebih mudah menggunakan faktor tersebut
\end{eulercomment}
\begin{eulerprompt}
>q=1+3%, K*q
\end{eulerprompt}
\begin{euleroutput}
  1.03
  5150
\end{euleroutput}
\begin{eulercomment}
Selama 10 tahun, kita cukup mengalikan faktor-faktornya dan
mendapatkan nilai akhir dengan tingkat bunga majemuk.
\end{eulercomment}
\begin{eulerprompt}
>K*q^10
\end{eulerprompt}
\begin{euleroutput}
  6719.58189672
\end{euleroutput}
\begin{eulercomment}
Untuk keperluan kita, kita dapat mengatur format menjadi 2 digit
setelah titik desimal.
\end{eulercomment}
\begin{eulerprompt}
>format(12,2); K*q^10
\end{eulerprompt}
\begin{euleroutput}
      6719.58 
\end{euleroutput}
\begin{eulercomment}
Mari kita cetak yang dibulatkan menjadi 2 digit dalam satu kalimat
lengkap.
\end{eulercomment}
\begin{eulerprompt}
>"Starting from " + K + "$ you get " + round(K*q^10,2) + "$."
\end{eulerprompt}
\begin{euleroutput}
  Starting from 5000$ you get 6719.58$.
\end{euleroutput}
\begin{eulercomment}
Bagaimana jika kita ingin mengetahui hasil antara dari tahun 1 sampai
tahun ke 9? Untuk ini, bahasa matriks Euler sangat membantu. Anda
tidak perlu menulis satu perulangan, tetapi cukup masuk
\end{eulercomment}
\begin{eulerprompt}
>K*q^(0:10)
\end{eulerprompt}
\begin{euleroutput}
  Real 1 x 11 matrix
  
      5000.00     5150.00     5304.50     5463.64     ...
\end{euleroutput}
\begin{eulercomment}
Bagaimana keajaiban ini terjadi? Pertama, ekspresi 0:10 mengembalikan
vektor bilangan bulat.
\end{eulercomment}
\begin{eulerprompt}
>short 0:10
\end{eulerprompt}
\begin{euleroutput}
  [0,  1,  2,  3,  4,  5,  6,  7,  8,  9,  10]
\end{euleroutput}
\begin{eulercomment}
Kemudian semua operator dan fungsi di Euler dapat diterapkan pada
vektor elemen demi elemen. Jadi
\end{eulercomment}
\begin{eulerprompt}
>short q^(0:10)
\end{eulerprompt}
\begin{euleroutput}
  [1,  1.03,  1.0609,  1.0927,  1.1255,  1.1593,  1.1941,  1.2299,
  1.2668,  1.3048,  1.3439]
\end{euleroutput}
\begin{eulercomment}
adalah vektor faktor q\textasciicircum{}0 sampai q\textasciicircum{}10. Ini dikalikan dengan K, dan kita
mendapatkan vektor nilainya.
\end{eulercomment}
\begin{eulerprompt}
>VK=K*q^(0:10);
\end{eulerprompt}
\begin{eulercomment}
Tentu saja, cara realistis untuk menghitung suku bunga ini adalah
dengan membulatkan ke sen terdekat setiap tahunnya. Mari kita
tambahkan fungsi untuk ini.
\end{eulercomment}
\begin{eulerprompt}
>function oneyear (K) := round(K*q,2)
\end{eulerprompt}
\begin{eulercomment}
Mari kita bandingkan kedua hasil tersebut, dengan dan tanpa
pembulatan.
\end{eulercomment}
\begin{eulerprompt}
>longest oneyear(1234.57), longest 1234.57*q
\end{eulerprompt}
\begin{euleroutput}
                  1271.61 
                1271.6071 
\end{euleroutput}
\begin{eulercomment}
Sekarang tidak ada rumus sederhana untuk tahun ke-n, dan kita harus
mengulanginya selama bertahun-tahun. Euler memberikan banyak solusi
untuk ini.

Cara termudah adalah fungsi iterate, yang mengulangi fungsi tertentu
beberapa kali.
\end{eulercomment}
\begin{eulerprompt}
>VKr=iterate("oneyear",5000,10)
\end{eulerprompt}
\begin{euleroutput}
  Real 1 x 11 matrix
  
      5000.00     5150.00     5304.50     5463.64     ...
\end{euleroutput}
\begin{eulercomment}
Kami dapat mencetaknya dengan cara yang ramah, menggunakan format kami
dengan tempat desimal tetap.
\end{eulercomment}
\begin{eulerprompt}
>VKr'
\end{eulerprompt}
\begin{euleroutput}
      5000.00 
      5150.00 
      5304.50 
      5463.64 
      5627.55 
      5796.38 
      5970.27 
      6149.38 
      6333.86 
      6523.88 
      6719.60 
\end{euleroutput}
\begin{eulercomment}
Untuk mendapatkan elemen vektor tertentu, kami menggunakan indeks
dalam tanda kurung siku.
\end{eulercomment}
\begin{eulerprompt}
>VKr[2], VKr[1:3]
\end{eulerprompt}
\begin{euleroutput}
      5150.00 
      5000.00     5150.00     5304.50 
\end{euleroutput}
\begin{eulercomment}
Anehnya, kita juga bisa menggunakan vektor indeks. Ingatlah bahwa 1:3
menghasilkan vektor [1,2,3].

Mari kita bandingkan elemen terakhir dari nilai yang dibulatkan dengan
nilai penuh.
\end{eulercomment}
\begin{eulerprompt}
>VKr[-1], VK[-1]
\end{eulerprompt}
\begin{euleroutput}
      6719.60 
      6719.58 
\end{euleroutput}
\begin{eulercomment}
Perbedaannya sangat kecil.

\begin{eulercomment}
\eulerheading{Memecahkan Persamaan}
\begin{eulercomment}
Sekarang kita mengambil fungsi yang lebih maju, yang menambahkan
tingkat uang tertentu setiap tahunnya.
\end{eulercomment}
\begin{eulerprompt}
>function onepay (K) := K*q+R
\end{eulerprompt}
\begin{eulercomment}
Kita tidak perlu menentukan q atau R untuk definisi fungsi. Hanya jika
kita menjalankan perintah, kita harus mendefinisikan nilai-nilai ini.
Kami memilih R=200.
\end{eulercomment}
\begin{eulerprompt}
>R=200; iterate("onepay",5000,10)
\end{eulerprompt}
\begin{euleroutput}
  Real 1 x 11 matrix
  
      5000.00     5350.00     5710.50     6081.82     ...
\end{euleroutput}
\begin{eulercomment}
Bagaimana jika kita menghapus jumlah yang sama setiap tahun?
\end{eulercomment}
\begin{eulerprompt}
>R=-200; iterate("onepay",5000,10)
\end{eulerprompt}
\begin{euleroutput}
  Real 1 x 11 matrix
  
      5000.00     4950.00     4898.50     4845.45     ...
\end{euleroutput}
\begin{eulercomment}
Kami melihat uangnya berkurang. Jelasnya, jika kita hanya mendapat
bunga sebesar 150 pada tahun pertama, namun menghapus 200, kita
kehilangan uang setiap tahunnya.

Bagaimana kita dapat menentukan berapa tahun uang tersebut akan
bertahan? Kita harus menulis satu lingkaran untuk ini. Cara termudah
adalah dengan melakukan iterasi cukup lama.
\end{eulercomment}
\begin{eulerprompt}
>VKR=iterate("onepay",5000,50)
\end{eulerprompt}
\begin{euleroutput}
  Real 1 x 51 matrix
  
      5000.00     4950.00     4898.50     4845.45     ...
\end{euleroutput}
\begin{eulercomment}
Dengan menggunakan bahasa matriks, kita dapat menentukan nilai negatif
pertama dengan cara berikut.
\end{eulercomment}
\begin{eulerprompt}
>min(nonzeros(VKR<0))
\end{eulerprompt}
\begin{euleroutput}
        48.00 
\end{euleroutput}
\begin{eulercomment}
Alasannya adalah bukan nol (VKR\textless{}0) mengembalikan vektor indeks i,
dengan VKR[i]\textless{}0, dan min menghitung indeks minimal.

Karena vektor selalu dimulai dengan indeks 1, maka jawabannya adalah
47 tahun.

Fungsi iterate() memiliki satu trik lagi. Ini dapat mengambil kondisi
akhir sebagai argumen. Kemudian akan mengembalikan nilai dan jumlah
iterasi.
\end{eulercomment}
\begin{eulerprompt}
>\{x,n\}=iterate("onepay",5000,till="x<0"); x, n,
\end{eulerprompt}
\begin{euleroutput}
       -19.83 
        47.00 
\end{euleroutput}
\begin{eulercomment}
Mari kita coba menjawab pertanyaan yang lebih ambigu. Asumsikan kita
mengetahui bahwa nilainya adalah 0 setelah 50 tahun. Berapa tingkat
bunganya?

Ini adalah pertanyaan yang hanya bisa dijawab secara numerik. Di bawah
ini, kita akan mendapatkan rumus yang diperlukan. Kemudian Anda akan
melihat bahwa tidak ada rumus yang mudah untuk menentukan tingkat suku
bunga. Namun untuk saat ini, kami menargetkan solusi numerik.

Langkah pertama adalah mendefinisikan fungsi yang melakukan iterasi
sebanyak n kali. Kami menambahkan semua parameter ke fungsi ini.
\end{eulercomment}
\begin{eulerprompt}
>function f(K,R,P,n) := iterate("x*(1+P/100)+R",K,n;P,R)[-1]
\end{eulerprompt}
\begin{eulercomment}
Iterasinya sama seperti di atas

\end{eulercomment}
\begin{eulerformula}
\[
x_{n+1} = x_n \cdot \kiri(1+ \frac{P}{100}\kanan) + R
\]
\end{eulerformula}
\begin{eulercomment}
Namun kami tidak lagi menggunakan nilai global R dalam ekspresi kami.
Fungsi seperti iterate() memiliki trik khusus di Euler. Anda dapat
meneruskan nilai variabel dalam ekspresi sebagai parameter titik koma.
Dalam hal ini P dan R.

Apalagi kami hanya tertarik pada nilai terakhir. Jadi kita ambil
indeks [-1].

Mari kita coba tes.
\end{eulercomment}
\begin{eulerprompt}
>f(5000,-200,3,47)
\end{eulerprompt}
\begin{euleroutput}
       -19.83 
\end{euleroutput}
\begin{eulercomment}
Sekarang kita bisa menyelesaikan masalah kita.
\end{eulercomment}
\begin{eulerprompt}
>solve("f(5000,-200,x,50)",3)
\end{eulerprompt}
\begin{euleroutput}
         3.15 
\end{euleroutput}
\begin{eulercomment}
Rutinitas penyelesaian menyelesaikan ekspresi=0 untuk variabel x.
Jawabannya adalah 3,15\% per tahun. Kami mengambil nilai awal 3\% untuk
algoritma. Fungsi solve() selalu membutuhkan nilai awal.

Kita dapat menggunakan fungsi yang sama untuk menyelesaikan pertanyaan
berikut: Berapa banyak yang dapat kita keluarkan per tahun sehingga
modal awal habis setelah 20 tahun dengan asumsi tingkat bunga 3\% per
tahun.
\end{eulercomment}
\begin{eulerprompt}
>solve("f(5000,x,3,20)",-200)
\end{eulerprompt}
\begin{euleroutput}
      -336.08 
\end{euleroutput}
\begin{eulercomment}
Perhatikan bahwa Anda tidak dapat menyelesaikan jumlah tahun, karena
fungsi kita mengasumsikan n sebagai nilai bilangan bulat.

\end{eulercomment}
\eulersubheading{Solusi Simbolis Masalah Suku Bunga}
\begin{eulercomment}
Kita dapat menggunakan bagian simbolis dari Euler untuk mempelajari
masalahnya. Pertama kita mendefinisikan fungsi onepay() kita secara
simbolis.
\end{eulercomment}
\begin{eulerprompt}
>function op(K) &= K*q+R; $&op(K)
\end{eulerprompt}
\begin{eulercomment}
Sekarang kita dapat mengulanginya.
\end{eulercomment}
\begin{eulerprompt}
>$&op(op(op(op(K)))), $&expand(%)
\end{eulerprompt}
\begin{eulercomment}
Kami melihat sebuah pola. Setelah n periode yang kita miliki

\end{eulercomment}
\begin{eulerformula}
\[
K_n = q^n K + R (1+q+\ldots+q^{n-1}) = q^n K + \frac{q^n-1}{q-1} R
\]
\end{eulerformula}
\begin{eulercomment}
Rumusnya adalah rumus jumlah geometri yang diketahui Maxima.
\end{eulercomment}
\begin{eulerprompt}
>&sum(q^k,k,0,n-1); $& % = ev(%,simpsum)
\end{eulerprompt}
\begin{eulercomment}
Ini agak rumit. Jumlahnya dievaluasi dengan tanda "simpsum" untuk
menguranginya menjadi hasil bagi.

Mari kita membuat fungsi untuk ini.
\end{eulercomment}
\begin{eulerprompt}
>function fs(K,R,P,n) &= (1+P/100)^n*K + ((1+P/100)^n-1)/(P/100)*R; $&fs(K,R,P,n)
\end{eulerprompt}
\begin{eulercomment}
Fungsinya sama dengan fungsi f kita sebelumnya. Tapi ini lebih
efektif.
\end{eulercomment}
\begin{eulerprompt}
>longest f(5000,-200,3,47), longest fs(5000,-200,3,47)
\end{eulerprompt}
\begin{euleroutput}
       -19.82504734650985 
       -19.82504734652684 
\end{euleroutput}
\begin{eulercomment}
Sekarang kita dapat menggunakannya untuk menanyakan waktu n. Kapan
modal kita habis? Perkiraan awal kami adalah 30 tahun.
\end{eulercomment}
\begin{eulerprompt}
>solve("fs(5000,-330,3,x)",30)
\end{eulerprompt}
\begin{euleroutput}
        20.51 
\end{euleroutput}
\begin{eulercomment}
Jawaban ini mengatakan akan menjadi negatif setelah 21 tahun.

Kita juga dapat menggunakan sisi simbolis Euler untuk menghitung rumus
pembayaran.

Asumsikan kita mendapatkan pinjaman sebesar K, dan membayar n
pembayaran sebesar R (dimulai setelah tahun pertama) meninggalkan sisa
hutang sebesar Kn (pada saat pembayaran terakhir). Rumusnya jelas
\end{eulercomment}
\begin{eulerprompt}
>equ &= fs(K,R,P,n)=Kn; $&equ
\end{eulerprompt}
\begin{eulercomment}
Biasanya rumus ini diberikan dalam bentuk

\end{eulercomment}
\begin{eulerformula}
\[
i = \frac{P}{100}
\]
\end{eulerformula}
\begin{eulerprompt}
>equ &= (equ with P=100*i); $&equ
\end{eulerprompt}
\begin{eulercomment}
Kita dapat menyelesaikan nilai R secara simbolis.
\end{eulercomment}
\begin{eulerprompt}
>$&solve(equ,R)
\end{eulerprompt}
\begin{eulercomment}
Seperti yang Anda lihat dari rumusnya, fungsi ini mengembalikan
kesalahan floating point untuk i=0. Euler tetap merencanakannya.

Tentu saja, kami memiliki batasan berikut.
\end{eulercomment}
\begin{eulerprompt}
>$&limit(R(5000,0,x,10),x,0)
\end{eulerprompt}
\begin{eulercomment}
Yang jelas tanpa bunga kita harus membayar kembali 10 tarif 500.

Persamaan tersebut juga dapat diselesaikan untuk n. Akan terlihat
lebih bagus jika kita menerapkan beberapa penyederhanaan padanya.
\end{eulercomment}
\begin{eulerprompt}
>fn &= solve(equ,n) | ratsimp; $&fn
\end{eulerprompt}
\eulersubheading{Contoh Latihan Soal}
\eulersubheading{ Exercise R.2}
\begin{eulercomment}
1. Sederhanakan bentuk eksponen dari fungsi berikut :

\end{eulercomment}
\begin{eulerformula}
\[
\left( \frac {24a^{10}b^{-8}c^7}{12a^6b^{-3}c^5} \right)^{-5}
\]
\end{eulerformula}
\begin{eulercomment}
Penyelesaian :
\end{eulercomment}
\begin{eulerprompt}
>$&((24*a^10*b^(-8)*c^7) / (12*a^6*b^(-3)*c^5))^(-5)
\end{eulerprompt}
\begin{eulercomment}
2. Sederhanakan bentuk fungsi berikut :

\end{eulercomment}
\begin{eulerformula}
\[
\left( \frac {125p^{12}q^{-14}r^{22}}{25p^8q^6r^{-15}} \right)^{-4}
\]
\end{eulerformula}
\begin{eulercomment}
Penyelesaian :
\end{eulercomment}
\begin{eulerprompt}
>$& ((125*p^12*q^-14*r^22) / (25*p^8*q^6*r^-15))^(-4)
\end{eulerprompt}
\begin{eulercomment}
3. Hitunglah operasi di bawah ini :\\
\end{eulercomment}
\begin{eulerformula}
\[
\frac {4(8-6)^2-4\cdot3+2\cdot8}{3^1+19^0}
\]
\end{eulerformula}
\begin{eulercomment}
Penyelesaian :
\end{eulercomment}
\begin{eulerprompt}
>&(4*(8-6)^2-4*3+2*8)/(3^1+19^0)
\end{eulerprompt}
\begin{euleroutput}
  
                                    5
  
\end{euleroutput}
\begin{eulercomment}
4. Hitunglah operasi di bawah ini :\\
\end{eulercomment}
\begin{eulerformula}
\[
\frac{[4(8-6)^2+4](3-2\cdot8)}{2^2(2^3+5)}
\]
\end{eulerformula}
\begin{eulercomment}
Penyelesaian :
\end{eulercomment}
\begin{eulerprompt}
>&((4*(8-6)^2+4)*(3-2*8))/(2^2*(2^3+5))
\end{eulerprompt}
\begin{euleroutput}
  
                                   - 5
  
\end{euleroutput}
\eulersubheading{Exercise Set R.3}
\begin{eulercomment}
6. Selesaikan soal di bawah ini\\
\end{eulercomment}
\begin{eulerformula}
\[
{(3x^{2}-2x-x^{3}+2)}-{(5x^{2}-8x-x^{3}+4)}
\]
\end{eulerformula}
\begin{eulercomment}
Penyelesaian :
\end{eulercomment}
\begin{eulerprompt}
>$&(3*x^2-2*x-x^3+2)-(5*x^2-8*x-x^3+4)
\end{eulerprompt}
\begin{eulercomment}
7. Selesaikan soal operasi di bawah ini!\\
\end{eulercomment}
\begin{eulerformula}
\[
(2x+3y+z-7)+(4x-2y-z+8)+(-3x+y-2z-4)
\]
\end{eulerformula}
\begin{eulercomment}
Penyelesaian :
\end{eulercomment}
\begin{eulerprompt}
>$&(2*x+3*y+z-7)+(4*x-2*y-z+8)+(-3*x+y-2*z-4)
\end{eulerprompt}
\begin{eulercomment}
8. Selesaikan soal operasi di bawah ini!\\
\end{eulercomment}
\begin{eulerformula}
\[
(2x^2+12xy-11)+(6x^2-2x+4)+(-x^2-y-2)
\]
\end{eulerformula}
\begin{eulercomment}
Penyelesaian :
\end{eulercomment}
\begin{eulerprompt}
>$&(2*x^2+12*x*y-11)+(6*x^2-2*x+4)+(-x^2-y-2)
\end{eulerprompt}
\begin{eulercomment}
9. Selesaikan soal operasi di bawah ini!\\
\end{eulercomment}
\begin{eulerformula}
\[
(5x^2+4xy-3y^2+2)-(9x^2-4xy+2y^2-1)
\]
\end{eulerformula}
\begin{eulercomment}
Penyelesaian :
\end{eulercomment}
\begin{eulerprompt}
>$&(5*x^2+4*x*y-3*y^2+2)-(9*x^2-4*x*y+2*y^2-1)
\end{eulerprompt}
\begin{eulercomment}
10. Selesaikan soal operasi di bawah ini!\\
\end{eulercomment}
\begin{eulerformula}
\[
(x^4-3x^2+4x)-(3x^3+x^2-5x+3)
\]
\end{eulerformula}
\begin{eulercomment}
Penyelesaian :
\end{eulercomment}
\begin{eulerprompt}
>$&(x^4-3*x^2+4*x)-(3*x^3+x^2-5*x+3)
\end{eulerprompt}
\begin{eulercomment}
11. Asumsikan bahwa semua eksponen adalah bilangan asli. Maka
selesaikan soal di bawah ini :\\
\end{eulercomment}
\begin{eulerformula}
\[
(a^n+b^n)(a^n-b^n)
\]
\end{eulerformula}
\begin{eulercomment}
Penyelesaian :
\end{eulercomment}
\begin{eulerprompt}
>$&expand ((a^n+b^n)*(a^n-b^n))
\end{eulerprompt}
\begin{eulercomment}
12. Asumsikan bahwa semua eksponen adalah bilangan asli. Maka
selesaikan soal di bawah ini:\\
\end{eulercomment}
\begin{eulerformula}
\[
(a+b+c)^2
\]
\end{eulerformula}
\begin{eulercomment}
Penyelesaian :
\end{eulercomment}
\begin{eulerprompt}
>$&expand((a+b+c)^2)
\end{eulerprompt}
\begin{eulercomment}
13. Asumsikan bahwa semua eksponen adalah bilangan asli. Kalikan\\
\end{eulercomment}
\begin{eulerformula}
\[
(x^{3m}-t^{5n})^2
\]
\end{eulerformula}
\begin{eulercomment}
Penyelesaian :
\end{eulercomment}
\begin{eulerprompt}
>$&expand((x^(3*m)-t^(5*n))^2)
\end{eulerprompt}
\begin{eulercomment}
\end{eulercomment}
\eulersubheading{Exercise Set R.4}
\begin{eulercomment}
14. Selesaikan soal berikut!\\
\end{eulercomment}
\begin{eulerformula}
\[
t^2+8t+15
\]
\end{eulerformula}
\begin{eulercomment}
Penyelesaian :
\end{eulercomment}
\begin{eulerprompt}
>$&factor(t^2+8*t+15)
\end{eulerprompt}
\begin{eulercomment}
15. Selesaikan soal berikut!\\
\end{eulercomment}
\begin{eulerformula}
\[
y^2+12y+27
\]
\end{eulerformula}
\begin{eulercomment}
Penyelesaian :
\end{eulercomment}
\begin{eulerprompt}
>$&factor(y^2+12*y+27)
\end{eulerprompt}
\begin{eulercomment}
16. Selesaikan soal di bawah ini!\\
\end{eulercomment}
\begin{eulerformula}
\[
x^3-4x^2+5x-20
\]
\end{eulerformula}
\begin{eulercomment}
Penyelesaian :
\end{eulercomment}
\begin{eulerprompt}
>$&factor(x^3-4*x^2+5*x-20)
\end{eulerprompt}
\begin{eulercomment}
17. Selesaikan soal di bawah ini!\\
\end{eulercomment}
\begin{eulerformula}
\[
m^2-9n^2
\]
\end{eulerformula}
\begin{eulercomment}
Penyelesaian :
\end{eulercomment}
\begin{eulerprompt}
>$&factor(m^2-9*n^2)
\end{eulerprompt}
\begin{eulercomment}
18. Selesaikan soal di bawah ini!\\
\end{eulercomment}
\begin{eulerformula}
\[
z^3+3z^2-3z-9
\]
\end{eulerformula}
\begin{eulercomment}
Penyelesaian :
\end{eulercomment}
\begin{eulerprompt}
>$&factor(z^3+3*z^2-3*z-9)
\end{eulerprompt}
\begin{eulercomment}
19. Selesaikan soal pefaktoran berikut :\\
\end{eulercomment}
\begin{eulerformula}
\[
18a^2b-15ab^2
\]
\end{eulerformula}
\begin{eulercomment}
Penyelesaian :
\end{eulercomment}
\begin{eulerprompt}
>$&factor(18*a^2*b-15*a*b^2)
\end{eulerprompt}
\begin{eulercomment}
20. Selesaikan soal pemfaktoran berikut :\\
\end{eulercomment}
\begin{eulerformula}
\[
250z^4-2z
\]
\end{eulerformula}
\begin{eulercomment}
Penyelesaian :
\end{eulercomment}
\begin{eulerprompt}
>$&factor(250*z^4-2*z)
\end{eulerprompt}
\begin{eulercomment}
21. Selesaikan soal di bawah ini :\\
\end{eulercomment}
\begin{eulerformula}
\[
m^6+8m^3-20
\]
\end{eulerformula}
\begin{eulercomment}
Penyelesaian :
\end{eulercomment}
\begin{eulerprompt}
>$&factor(m^6+8*m^3-20)
\end{eulerprompt}
\begin{eulercomment}
22. Selesaikan soal berikut:\\
\end{eulercomment}
\begin{eulerformula}
\[
25ab^4-25az^4
\]
\end{eulerformula}
\begin{eulercomment}
Penyelesaian :
\end{eulercomment}
\begin{eulerprompt}
>$&factor(25*a*b^4-25*a*z^4)
\end{eulerprompt}
\begin{eulercomment}
23. Selesaikan soal dibawah ini :\\
\end{eulercomment}
\begin{eulerformula}
\[
16a^7b+54ab^7
\]
\end{eulerformula}
\begin{eulercomment}
Penyelesaian :
\end{eulercomment}
\begin{eulerprompt}
>$&factor(16*a^7*b+54*a*b^7)
\end{eulerprompt}
\begin{eulercomment}
24. Selesaikan soal dibawah ini :\\
\end{eulercomment}
\begin{eulerformula}
\[
11x^2+x^4-80
\]
\end{eulerformula}
\begin{eulercomment}
Penyelesaian :
\end{eulercomment}
\begin{eulerprompt}
>$&factor(11*x^2*+x^4-80)
\end{eulerprompt}
\begin{eulercomment}
25. Selesaikan soal dibawah ini :\\
\end{eulercomment}
\begin{eulerformula}
\[
x^4-37x^2+36
\]
\end{eulerformula}
\begin{eulercomment}
Penyelesaian :
\end{eulercomment}
\begin{eulerprompt}
>$&factor(x^4-37*x^2+36)
\end{eulerprompt}
\begin{eulercomment}
26. Selesaikan soal dibawah ini :\\
\end{eulercomment}
\begin{eulerformula}
\[
2x^2-288
\]
\end{eulerformula}
\begin{eulercomment}
Penyelesaian :
\end{eulercomment}
\begin{eulerprompt}
>$&factor(2*x^2-288)
\end{eulerprompt}
\begin{eulercomment}
27. Selesaikan soal dibawah ini :\\
\end{eulercomment}
\begin{eulerformula}
\[
(x+0.001)^2-x^2
\]
\end{eulerformula}
\begin{eulercomment}
Penyelesaian :
\end{eulercomment}
\begin{eulerprompt}
>$&factor(x+0.001)^2-x^2
\end{eulerprompt}
\begin{eulercomment}
28. Selesaikan soal dibawah ini :\\
\end{eulercomment}
\begin{eulerformula}
\[
(x+h)^3-x^3
\]
\end{eulerformula}
\begin{eulercomment}
Penyelesaian :
\end{eulercomment}
\begin{eulerprompt}
>$&factor((x+h)^3-(x^3))
\end{eulerprompt}
\eulersubheading{Exercise Set R.5}
\begin{eulercomment}
29. Selesaikan soal dibawah ini :\\
\end{eulercomment}
\begin{eulerformula}
\[
7(3x+6)=11-(x+2)
\]
\end{eulerformula}
\begin{eulercomment}
Penyelesaian :
\end{eulercomment}
\begin{eulerprompt}
>$&solve(7*(3*x+6)=11-(x+2))
\end{eulerprompt}
\begin{eulercomment}
30. Selesaikan soal dibawah ini :\\
\end{eulercomment}
\begin{eulerformula}
\[
9(2x+8)=20-(x+5)
\]
\end{eulerformula}
\begin{eulercomment}
Penyelesaian :
\end{eulercomment}
\begin{eulerprompt}
>$&solve(9*(2*x+8)=20-(x+5))
\end{eulerprompt}
\begin{eulercomment}
31. Selesaikan soal dibawah ini :\\
\end{eulercomment}
\begin{eulerformula}
\[
4(3y-1)-6=5(y+2)
\]
\end{eulerformula}
\begin{eulercomment}
Penyelesaian :
\end{eulercomment}
\begin{eulerprompt}
>$&solve(4*(2*y-1)-6=5*(y+2))
\end{eulerprompt}
\begin{eulercomment}
32. Selesaikan soal dibawah ini :\\
\end{eulercomment}
\begin{eulerformula}
\[
3(2n-5)-7=4(n-9)
\]
\end{eulerformula}
\begin{eulercomment}
Penyelesaian :
\end{eulercomment}
\begin{eulerprompt}
>$&solve(3*(2*n-5)-7=4*(n-9))
\end{eulerprompt}
\begin{eulercomment}
\end{eulercomment}
\eulersubheading{Exercise Set R.6}
\begin{eulercomment}
33. Selesaikan soal dibawah ini :\\
\end{eulercomment}
\begin{eulerformula}
\[
\left[\frac{x+1}{x-1} + 1\right]^5
\]
\end{eulerformula}
\begin{eulercomment}
Penyelesaian :
\end{eulercomment}
\begin{eulerprompt}
>$&solve(((x+1)/(x-1))+1)^5
\end{eulerprompt}
\begin{eulercomment}
34. Selesaikan soal dibawah ini :\\
\end{eulercomment}
\begin{eulerformula}
\[
\left( \frac{x^2-4}{x^2-4x+4}\right)
\]
\end{eulerformula}
\begin{eulercomment}
Penyelesaian :
\end{eulercomment}
\begin{eulerprompt}
>$&solve((x^2-4)/(x^2-4*x+4))
\end{eulerprompt}
\begin{eulercomment}
35. Selesaikan soal dibawah ini :\\
\end{eulercomment}
\begin{eulerformula}
\[
\left[\frac{-5}{3}\right]
\]
\end{eulerformula}
\begin{eulercomment}
Penyelesaian :
\end{eulercomment}
\begin{eulerprompt}
>$&solve((-5)/(3))
\end{eulerprompt}
\begin{eulercomment}
36. Selesaikan soal dibawah ini :\\
\end{eulercomment}
\begin{eulerformula}
\[
\left[\frac{4}{7-x}\right]
\]
\end{eulerformula}
\begin{eulercomment}
Penyelesaian :
\end{eulercomment}
\begin{eulerprompt}
>$&solve((4)/(7-x))
\end{eulerprompt}
\eulersubheading{Exercise Set 2.3}
\begin{eulercomment}
37. Selesaikan soal dibawah ini :\\
\end{eulercomment}
\begin{eulerformula}
\[
(f \circ g)(-1)
\]
\end{eulerformula}
\begin{eulercomment}
Penyelesaian :



\end{eulercomment}
\eulersubheading{Soal Bilangan Asli}
\begin{eulercomment}
38. Tuliskan bilangan asli dari\\
\end{eulercomment}
\begin{eulerformula}
\[
\frac{1}{8}
\]
\end{eulerformula}
\begin{eulercomment}
Penyelesaian :
\end{eulercomment}
\begin{eulerprompt}
>longest 1/8
\end{eulerprompt}
\begin{euleroutput}
                    0.125 
\end{euleroutput}
\begin{eulerprompt}
>printdual (1/8)
\end{eulerprompt}
\begin{euleroutput}
  1.0000000000000000000000000000000000000000000000000000*2^-3
\end{euleroutput}
\begin{eulerprompt}
>printhex (1/8)
\end{eulerprompt}
\begin{euleroutput}
  2.0000000000000*16^-1
\end{euleroutput}
\begin{eulercomment}
39. Tuliskan bilangan asli dari\\
\end{eulercomment}
\begin{eulerformula}
\[
\frac{13}{23}
\]
\end{eulerformula}
\begin{eulercomment}
Penyelesaian :
\end{eulercomment}
\begin{eulerprompt}
>longest 13/23
\end{eulerprompt}
\begin{euleroutput}
       0.5652173913043478 
\end{euleroutput}
\begin{eulerprompt}
>printdual (13/23)
\end{eulerprompt}
\begin{euleroutput}
  1.0010000101100100001011001000010110010000101100100001*2^-1
\end{euleroutput}
\begin{eulerprompt}
>printhex (13/23)
\end{eulerprompt}
\begin{euleroutput}
  9.0B21642C85908*16^-1
\end{euleroutput}
\eulersubheading{Soal String}
\begin{eulercomment}
40. Tuliskan string dari "Aku suka mata kuliah Aplikom"

Penyelesaian :
\end{eulercomment}
\begin{eulerprompt}
>"Aku suka mata kuliah Aplikom"
\end{eulerprompt}
\begin{euleroutput}
  Aku suka mata kuliah Aplikom
\end{euleroutput}
\eulersubheading{Soal Nilai Boolean}
\begin{eulercomment}
41. Apakah benar bahwa:\\
\end{eulercomment}
\begin{eulerformula}
\[
5 > 3 and 2 < 1
\]
\end{eulerformula}
\begin{eulercomment}
Penyelesaian :
\end{eulercomment}
\begin{eulerprompt}
>5 > 3 && 2 < 1
\end{eulerprompt}
\begin{euleroutput}
         0.00 
\end{euleroutput}
\eulersubheading{Soal Keluaran}
\begin{eulercomment}
42. Tulisan format keluaran dari nilai :\\
\end{eulercomment}
\begin{eulerformula}
\[
tan (180)
\]
\end{eulerformula}
\begin{eulercomment}
Penyelesaian :
\end{eulercomment}
\begin{eulerprompt}
>defformat; tan(180)
\end{eulerprompt}
\begin{euleroutput}
  1.33869021035
\end{euleroutput}
\begin{eulerprompt}
>longest tan(180)
\end{eulerprompt}
\begin{euleroutput}
        1.338690210351154 
\end{euleroutput}
\begin{eulerprompt}
>printhex (tan(180))
\end{eulerprompt}
\begin{euleroutput}
  1.56B466D0EEFE6*16^0
\end{euleroutput}
\begin{eulerprompt}
>longestformat; tan(180)
\end{eulerprompt}
\begin{euleroutput}
  1.338690210351154
\end{euleroutput}
\eulersubheading{Soal Matematika Simbolik}
\begin{eulercomment}
43. Hitunglah hasil dari soal berikut :\\
\end{eulercomment}
\begin{eulerformula}
\[
C(98,10) = \frac{70!}{56! \cdot 10!}
\]
\end{eulerformula}
\begin{eulercomment}
Penyelesaian :
\end{eulercomment}
\begin{eulerprompt}
>$& 70!/(56!*10!) // nilai C(98,10)
>$binomial(98,10) //menghitung C(98,10) menggunakan fungsi binomial()
\end{eulerprompt}
\begin{eulercomment}
44. Hitunglah hasil dari soal berikut :\\
\end{eulercomment}
\begin{eulerformula}
\[
C(x,9)=\frac{x!}{(x-8)!7!}=\frac{(x-4)(x-2)x}{8}
\]
\end{eulerformula}
\begin{eulercomment}
Penyelesaian :
\end{eulercomment}
\begin{eulerprompt}
>$binomial(x,9) // c(x,9)
>$&binomial(x,9) with x=10 // substitusi x=10 ke C(x,9)
>&"v := 8; v^2"
\end{eulerprompt}
\begin{euleroutput}
  
                                    64
  
\end{euleroutput}
\begin{eulercomment}
45. Hitunglah faktor dari 90

Penuyelesaian :
\end{eulercomment}
\begin{eulerprompt}
>&factor(90!)
\end{eulerprompt}
\begin{euleroutput}
  
          1485715964481761497309522733620825737885569961284688766942216\(\backslash\)
  863704985393094065876545992131370884059645617234469978112000000000000\(\backslash\)
  000000000
  
\end{euleroutput}
\begin{eulerprompt}
>::: factor(90!)
\end{eulerprompt}
\begin{euleroutput}
  
           86  44  21  13   8   6   5   4   3   3   2   2   2   2
          2   3   5   7   11  13  17  19  23  29  31  37  41  43  47 53
                                                 59 61 67 71 73 79 83 89
  
\end{euleroutput}
\eulersubheading{Soal Fungsi}
\begin{eulercomment}
46.Tentukan hasil dari fungsi berikut :\\
\end{eulercomment}
\begin{eulerformula}
\[
x^8+6
\]
\end{eulerformula}
\begin{eulercomment}
Penyelesaian :


\end{eulercomment}
\eulersubheading{Soal Matriks}
\begin{eulercomment}
46. Hitunglah\\
\end{eulercomment}
\begin{eulerformula}
\[
A=[5,6;7,8]
\]
\end{eulerformula}
\begin{eulercomment}
Penyelesaian :
\end{eulercomment}
\begin{eulerprompt}
>A=[5,6;7,8]
\end{eulerprompt}
\begin{euleroutput}
                        5                       6 
                        7                       8 
\end{euleroutput}
\begin{eulerprompt}
>a=[5;6]
\end{eulerprompt}
\begin{euleroutput}
                        5 
                        6 
\end{euleroutput}
\begin{eulerprompt}
>a'
\end{eulerprompt}
\begin{euleroutput}
  [5,  6]
\end{euleroutput}
\begin{eulerprompt}
>inv(A)
\end{eulerprompt}
\begin{euleroutput}
       -4.000000000000002       3.000000000000001 
        3.500000000000002      -2.500000000000001 
\end{euleroutput}
\begin{eulerprompt}
>A.a
\end{eulerprompt}
\begin{euleroutput}
                       61 
                       83 
\end{euleroutput}
\begin{eulerprompt}
>A^2
\end{eulerprompt}
\begin{euleroutput}
                       25                      36 
                       49                      64 
\end{euleroutput}
\begin{eulerprompt}
>A.A.A.A
\end{eulerprompt}
\begin{euleroutput}
                    11587                   13494 
                    15743                   18334 
\end{euleroutput}
\begin{eulerprompt}
>power(A,8)
\end{eulerprompt}
\begin{euleroutput}
                346694611               403753974 
                471046303               548571598 
\end{euleroutput}
\begin{eulerprompt}
>A/A
\end{eulerprompt}
\begin{euleroutput}
                        1                       1 
                        1                       1 
\end{euleroutput}
\begin{eulerprompt}
>A/a
\end{eulerprompt}
\begin{euleroutput}
                        1                     1.2 
        1.166666666666667       1.333333333333333 
\end{euleroutput}
\begin{eulerprompt}
>A\(\backslash\)a
\end{eulerprompt}
\begin{euleroutput}
       -2.000000000000002 
        2.500000000000002 
\end{euleroutput}
\begin{eulerprompt}
>inv(A).a
\end{eulerprompt}
\begin{euleroutput}
                       -2 
                      2.5 
\end{euleroutput}
\begin{eulerprompt}
>A\(\backslash\)A  //A^(-1)A
\end{eulerprompt}
\begin{euleroutput}
                        1                       0 
                        0                       1 
\end{euleroutput}
\begin{eulerprompt}
>inv(A).A
\end{eulerprompt}
\begin{euleroutput}
        1.000000000000004                       0 
   -3.552713678800501e-15                       1 
\end{euleroutput}
\begin{eulerprompt}
>A*A
\end{eulerprompt}
\begin{euleroutput}
                       25                      36 
                       49                      64 
\end{euleroutput}
\begin{eulerprompt}
>a^2
\end{eulerprompt}
\begin{euleroutput}
                       25 
                       36 
\end{euleroutput}
\begin{eulerprompt}
>A|1
\end{eulerprompt}
\begin{euleroutput}
  Real 2 x 3 matrix
  
                        5     ...
                        7     ...
\end{euleroutput}
\eulersubheading{Soal Masalah suku bunga}
\begin{eulercomment}
47.  Sebuah motor bekas akan dijual dengan harga \textdollar{}78,000 dengan uang
muka sebanyak \textdollar{}12,000. Jika seseorang meminjam uang untuk membeli
motor tersebut dengan lama pinjaman 20 tahun dan suku bunga 5,4\%, maka
berapa jumlah yang harus dibayarkan oleh peminjam di setiap bulannya

Penyelesaian :
\end{eulercomment}
\begin{eulerprompt}
>K=78000-12000, q=1+5.4%, format(12,2); (K*q^20)/(12*20)
\end{eulerprompt}
\begin{euleroutput}
  66000
  1.054
       787.31 
\end{euleroutput}
\begin{eulercomment}
48.  Sebuah tanah pekarangan akan dijual dengan harga \textdollar{}125,000 dengan
uang muka sebanyak \textdollar{}100,000. Jika seseorang meminjam uang untuk
membeli tanah pekarangan tersebut dengan lama pinjaman 35 tahun dan
suku bunga 8,4\%, maka berapa jumlah yang harus dibayarkan oleh
peminjam di setiap bulannya

Penyelesaian :
\end{eulercomment}
\begin{eulerprompt}
>K=125000-100000, q=1+8.4%, format(12,2); (K*q^35)/(12*35)
\end{eulerprompt}
\begin{euleroutput}
     25000.00 
         1.08 
      1001.65 
\end{euleroutput}
\begin{eulercomment}
49.  Sebuah rumah akan dijual dengan harga \textdollar{}72,000 dengan uang muka
sebanyak \textdollar{}50,000. Jika seseorang meminjam uang untuk membeli rumah
tersebut dengan lama pinjaman 15 tahun dan suku bunga 6,4\%, maka
berapa jumlah yang harus dibayarkan oleh peminjam di setiap bulannya

Penyelesaian :
\end{eulercomment}
\begin{eulerprompt}
>K=72000-50000, q=1+6.4%, format(12,2); (K*q^15)/(12*15)
\end{eulerprompt}
\begin{euleroutput}
     22000.00 
         1.06 
       309.94 
\end{euleroutput}
\begin{eulercomment}
50.  Sebuah laptop bekas akan dijual dengan harga \textdollar{}45,000 dengan uang
muka sebanyak \textdollar{}30,000. Jika seseorang meminjam uang untuk membeli
laptop bekas tersebut dengan lama pinjaman 5 tahun dan suku bunga
2,4\%, maka berapa jumlah yang harus dibayarkan oleh peminjam di setiap
bulannya

Penyelesaian :
\end{eulercomment}
\begin{eulerprompt}
>K=45000-30000, q=1+2.4%, format(12,2); (K*q^5)/(12*5)
\end{eulerprompt}
\begin{euleroutput}
     15000.00 
         1.02 
       281.47 
\end{euleroutput}
\begin{eulercomment}
Nama : Aulia Agnes Luistin\\
NIM : 23030630001\\
Kelas : Matematika B




\begin{eulercomment}
\eulerheading{Menggambar Grafik 2D dengan EMT}
\begin{eulercomment}
Notebook ini menjelaskan tentang cara menggambar berbagaikurva dan
grafik 2D dengan software EMT. EMT menyediakan fungsi plot2d() untuk
menggambar berbagai kurva dan grafik dua dimensi (2D).\\
\end{eulercomment}
\eulersubheading{Plot Dasar}
\begin{eulercomment}
Ada fungsi plot yang sangat mendasar. Terdapat koordinat layar yang
selalu berkisar antara 0 hingga 1024 di setiap sumbu, tidak peduli
apakah layarnya berbentuk persegi atau tidak. Semut terdapat koordinat
plot yang dapat diatur dengan setplot(). Pemetaan antar koordinat
bergantung pada jendela plot saat ini. Misalnya, shrinkwindow()
default menyisakan ruang untuk label sumbu dan judul plot.

Dalam contoh ini, kita hanya menggambar beberapa garis acak dengan
berbagai warna. Untuk rincian tentang fungsi-fungsi ini, pelajari
fungsi inti EMT.
\end{eulercomment}
\begin{eulerprompt}
>clg; // untuk membersihkan layar
>window(0,0,1024,1024); // gunakan semua window
>setplot(0,1,0,1); // koordinat set plot
>hold on; // untuk memulai overwrite mode
>n=100; X=random(n,2); Y=random(n,2);  // untuk membuat koordinat acak
>colors=rgb(random(n),random(n),random(n)); // get random colors
>loop 1 to n; color(colors[#]); plot(X[#],Y[#]); end; // plot
>hold off; // mengakhiri overwrite mode
>insimg; //memasukkan ke notebook
>reset;
\end{eulerprompt}
\begin{eulercomment}
Grafik perlu ditahan, karena perintah plot() akan menghapus jendela
plot.

Untuk menghapus semua yang kami lakukan, kami menggunakan reset().

Untuk menampilkan gambar hasil plot di layar notebook, perintah
plot2d() dapat diakhiri dengan titik dua (:). Cara lainnya adalah
perintah plot2d() diakhiri dengan titik koma (;), kemudian menggunakan
perintah insimg() untuk menampilkan gambar hasil plot.

Contoh lain, kita menggambar plot sebagai sisipan di plot lain. Hal
ini dilakukan dengan mendefinisikan jendela plot yang lebih kecil.
Perhatikan bahwa jendela ini tidak memberikan ruang untuk label sumbu
di luar jendela plot. Kita harus menambahkan beberapa margin untuk ini
sesuai kebutuhan. Perhatikan bahwa kita menyimpan dan memulihkan
jendela penuh, dan menahan plot saat ini sementara kita memplot inset.
\end{eulercomment}
\begin{eulerprompt}
>plot2d("x^3-x");
>xw=200; yw=100; ww=300; hw=300;
>ow=window();
>window(xw,yw,xw+ww,yw+hw);
>hold on;
>barclear(xw-50,yw-10,ww+60,ww+60);
>plot2d("x^4-x",grid=6):
>hold off;
>window("x^4-x",grid=6):...
\end{eulerprompt}
\begin{euleroutput}
  Argument grid not in parameter list of function window.
  
  Parameters:
  
  
\end{euleroutput}
\begin{eulercomment}
Plot dengan banyak gambar dicapai dengan cara yang sama. Ada fungsi
utilitas figure() untuk ini.

\end{eulercomment}
\eulersubheading{Aspek Plot}
\begin{eulercomment}
Plot default menggunakan jendela plot persegi. Anda dapat mengubahnya
dengan fungsi aspek(). Jangan lupa untuk mengatur ulang aspeknya
nanti. Anda juga dapat mengubah default ini di menu dengan "Set
Aspect" ke rasio aspek tertentu atau ke ukuran jendela grafik saat
ini.

Tapi Anda juga bisa mengubahnya untuk satu plot. Untuk ini, ukuran
area plot saat ini diubah, dan jendela diatur sehingga label memiliki
cukup ruang.
\end{eulercomment}
\begin{eulerprompt}
>aspect(2); // rasio panjang dan lebar 2:1
\end{eulerprompt}
\begin{euleroutput}
  Error in:
  aspect(2); // rasio panjang dan lebar 2:1 ...
        ^
\end{euleroutput}
\begin{eulerprompt}
>plot2d(["sin(x)","cos(x)"],0,2pi):
>aspect(2); // rasio panjang dan lebar 2:1
>plot2d(["sin(x)","cos(x)"],0,2pi):
>aspect(3); // rasio panjang dan lebar 2:1
>plot2d(["sin(x)","cos(x)"],0,2pi):
>aspect();
>reset;
\end{eulerprompt}
\begin{eulercomment}
Fungsi reset() mengembalikan default plot termasuk rasio aspek.\\
Plot 2D di Euler

EMT Math Toolbox memiliki plot dalam 2D, baik untuk data maupun
fungsi. EMT menggunakan fungsi plot2d. Fungsi ini dapat memplot fungsi
dan data.

Dimungkinkan untuk membuat plot di Maxima menggunakan Gnuplot atau
dengan Python menggunakan Math Plot Lib.

Euler dapat membuat plot 2D

- ekspresi\\
- fungsi, variabel, atau kurva berparameter,\\
- vektor nilai x-y,\\
- awan titik di pesawat,\\
- kurva implisit dengan level atau wilayah level.\\
- Fungsi kompleks

Gaya plot mencakup berbagai gaya untuk garis dan titik, plot batang,
dan plot berbayang.\\
Plot Ekspresi atau Variabel

Ekspresi tunggal dalam "x" (misalnya "4*x\textasciicircum{}2") atau nama suatu fungsi
(misalnya "f") menghasilkan grafik fungsi tersebut.

Berikut adalah contoh paling dasar, yang menggunakan rentang default
dan menetapkan rentang y yang tepat agar sesuai dengan plot fungsinya.

Catatan: Jika Anda mengakhiri baris perintah dengan titik dua ":",
plot akan dimasukkan ke dalam jendela teks. Jika tidak, tekan TAB
untuk melihat plot jika jendela plot tertutup.
\end{eulercomment}
\begin{eulerprompt}
>plot2d("x^2"):
>aspect(1.5); plot2d("x^3-x"):
>a:=5.6; plot2d("exp(-a*x^2)/a"); insimg(30); // menampilkan gambar hasil plot setinggi 25 baris
\end{eulerprompt}
\begin{eulercomment}
Dari beberapa contoh sebelumnya Anda dapat melihat bahwa aslinya
gambar plot menggunakan sumbu X dengan rentang nilai dari -2 sampai
dengan 2. Untuk mengubah rentang nilai X dan Y, Anda dapat menambahkan
nilai-nilai batas X (dan Y) di belakang ekspresi yang digambar.

Rentang plot diatur dengan parameter yang ditetapkan sebagai berikut

- a,b: rentang x (default -2,2)\\
- c,d: rentang y (default: skala dengan nilai)\\
- r: alternatifnya radius di sekitar pusat plot\\
- cx,cy: koordinat pusat plot (default 0,0)
\end{eulercomment}
\begin{eulerprompt}
>plot2d("x^3-x",-1,2):
>plot2d("sin(x)",-2*pi,2*pi): // plot sin(x) pada interval [-2pi, 2pi]
>plot2d("cos(x)","sin(3*x)",xmin=0,xmax=2pi):
\end{eulerprompt}
\begin{eulercomment}
Alternatif untuk titik dua adalah perintah insimg(baris), yang
menyisipkan plot yang menempati sejumlah baris teks tertentu.

Dalam opsi, plot dapat diatur agar muncul

- di jendela terpisah yang dapat diubah ukurannya,\\
- di jendela buku catatan.

Lebih banyak gaya dapat dicapai dengan perintah plot tertentu.

Bagaimanapun, tekan tombol tabulator untuk melihat plotnya, jika
tersembunyi.

Untuk membagi jendela menjadi beberapa plot, gunakan perintah
figure(). Dalam contoh, kita memplot x\textasciicircum{}1 hingga x\textasciicircum{}4 menjadi 4 bagian
jendela. gambar(0) mengatur ulang jendela default.
\end{eulercomment}
\begin{eulerprompt}
>reset;
>figure(2,2); ...
>for n=1 to 4; figure(n); plot2d("x^"+n); end; ...
>figure(0):
\end{eulerprompt}
\begin{eulercomment}
Di plot2d(), ada gaya alternatif yang tersedia dengan grid=x. Untuk
gambaran umum, kami menampilkan berbagai gaya kisi dalam satu gambar
(lihat di bawah untuk perintah figure()). Gaya grid=0 tidak
disertakan. Ini tidak menunjukkan kisi dan bingkai.
\end{eulercomment}
\begin{eulerprompt}
>figure(3,3); ...
>for k=1:9; figure(k); plot2d("x^3-x",-2,1,grid=k); end; ...
>figure(0):
\end{eulerprompt}
\begin{eulercomment}
Jika argumen pada plot2d() adalah ekspresi yang diikuti oleh empat
angka, angka-angka tersebut adalah rentang x dan y untuk plot
tersebut.

Alternatifnya, a, b, c, d dapat ditentukan sebagai parameter yang
ditetapkan sebagai a=... dll.

Pada contoh berikut, kita mengubah gaya kisi, menambahkan label, dan
menggunakan label vertikal untuk sumbu y.
\end{eulercomment}
\begin{eulerprompt}
>aspect(1.5); plot2d("sin(x)",0,2pi,-1.2,1.2,grid=3,xl="x",yl="sin(x)"):
>plot2d("sin(x)+cos(2*x)",0,4pi):
\end{eulerprompt}
\begin{eulercomment}
Gambar yang dihasilkan dengan memasukkan plot ke dalam jendela teks
disimpan di direktori yang sama dengan buku catatan, secara default di
subdirektori bernama "gambar". Mereka juga digunakan oleh ekspor HTML.

Anda cukup menandai gambar apa saja dan menyalinnya ke clipboard
dengan Ctrl-C. Tentu saja, Anda juga dapat mengekspor grafik saat ini
dengan fungsi di menu File.

Fungsi atau ekspresi di plot2d dievaluasi secara adaptif. Agar lebih
cepat, nonaktifkan plot adaptif dengan \textless{}adaptive dan tentukan jumlah
subinterval dengan n=... Hal ini hanya diperlukan dalam kasus yang
jarang terjadi.
\end{eulercomment}
\begin{eulerprompt}
>plot2d("sign(x)*exp(-x^2)",-1,1,<adaptive,n=10000):
>plot2d("x^x",r=1.2,cx=1,cy=1):
\end{eulerprompt}
\begin{eulercomment}
Perhatikan bahwa x\textasciicircum{}x tidak ditentukan untuk x\textless{}=0. Fungsi plot2d
menangkap kesalahan ini, dan mulai membuat plot segera setelah
fungsinya ditentukan. Ini berfungsi untuk semua fungsi yang
mengembalikan NAN di luar jangkauan definisinya.
\end{eulercomment}
\begin{eulerprompt}
>plot2d("log(x)",-0.1,2):
\end{eulerprompt}
\begin{eulercomment}
Parameter square=true (atau \textgreater{}square) memilih rentang y secara otomatis
sehingga hasilnya adalah jendela plot persegi. Perhatikan bahwa secara
default, Euler menggunakan spasi persegi di dalam jendela plot.
\end{eulercomment}
\begin{eulerprompt}
>plot2d("x^3-x",>square):
>plot2d(''integrate("sin(x)*exp(-x^2)",0,x)'',0,2): // plot integral
\end{eulerprompt}
\begin{eulercomment}
Jika Anda memerlukan lebih banyak ruang untuk label y, panggil
shrinkwindow() dengan parameter lebih kecil, atau tetapkan nilai
positif untuk "lebih kecil" di plot2d().
\end{eulercomment}
\begin{eulerprompt}
>plot2d("gamma(x)",1,10,yl="y-values",smaller=6,<vertical):
\end{eulerprompt}
\begin{eulercomment}
Ekspresi simbolik juga dapat digunakan karena disimpan sebagai
ekspresi string sederhana.
\end{eulercomment}
\begin{eulerprompt}
>x=linspace(0,2pi,1000); plot2d(sin(5x),cos(7x)):
>a:=5.6; expr &= exp(-a*x^2)/a; // define expression
>plot2d(expr,-2,2): // plot from -2 to 2
>plot2d(expr,r=1,thickness=2): // plot in a square around (0,0)
>plot2d(&diff(expr,x),>add,style="--",color=red): // add another plot
>plot2d(&diff(expr,x,2),a=-2,b=2,c=-2,d=1): // plot in rectangle
>plot2d(&diff(expr,x),a=-2,b=2,>square): // keep plot square
>plot2d("x^2",0,1,steps=1,color=red,n=10):
>plot2d("x^2",>add,steps=2,color=blue,n=10):
>a=0.3; plot2d("f",0,1;a): // plot with a=0.3
\end{eulerprompt}
\begin{euleroutput}
  Function f needs at least 4 arguments!
  Use: f (K, R, P, n) 
   %ploteval:
      y0=f$(x[1],args());
  adaptiveevalone:
      s=%ploteval(g$,t;args());
  Try "trace errors" to inspect local variables after errors.
  plot2d:
      dw/n,dw/n^2,dw/n,auto;args());
\end{euleroutput}
\eulerheading{Fungsi dalam satu Parameter}
\begin{eulercomment}
Fungsi plot yang paling penting untuk plot planar adalah plot2d().
Fungsi ini diimplementasikan dalam bahasa Euler di file "plot.e", yang
dimuat di awal program.

Berikut beberapa contoh penggunaan suatu fungsi. Seperti biasa di EMT,
fungsi yang berfungsi untuk fungsi atau ekspresi lain, Anda bisa
meneruskan parameter tambahan (selain x) yang bukan variabel global ke
fungsi dengan parameter titik koma atau dengan kumpulan panggilan.
\end{eulercomment}
\begin{eulerprompt}
>function f(x,a) := (x^2/a+a*x^2-x); // define a function
>a=0.3; plot2d("f",0,1;a): // plot with a=0.3
>plot2d("f",0,1;0.4): // plot with a=0.4
>plot2d(\{\{"f",0.2\}\},0,1): // plot with a=0.2
>plot2d(\{\{"f(x,b)",b=0.1\}\},0,1): // plot with 0.1
>function f(x) := x^3-x; ...
>plot2d("f",r=1):
\end{eulerprompt}
\begin{eulercomment}
Berikut ini ringkasan fungsi yang diterima

- ekspresi atau ekspresi simbolik di x\\
- fungsi atau fungsi simbolik dengan nama "f"\\
- fungsi simbolik hanya dengan nama f

Fungsi plot2d() juga menerima fungsi simbolik. Untuk fungsi simbolik,
namanya saja yang berfungsi.
\end{eulercomment}
\begin{eulerprompt}
>function f(x) &= diff(x^x,x)
\end{eulerprompt}
\begin{euleroutput}
  
                              x
                             x  (log(x) + 1)
  
\end{euleroutput}
\begin{eulerprompt}
>plot2d(f,0,2):
\end{eulerprompt}
\begin{eulercomment}
Tentu saja, untuk ekspresi atau ekspresi simbolik, nama variabel sudah
cukup untuk memplotnya.
\end{eulercomment}
\begin{eulerprompt}
>expr &= sin(x)*exp(-x)
\end{eulerprompt}
\begin{euleroutput}
  
                                - x
                               E    sin(x)
  
\end{euleroutput}
\begin{eulerprompt}
>plot2d(expr,0,3pi):
>function f(x) &= x^x;
>plot2d(f,r=1,cx=1,cy=1,color=blue,thickness=2);
>plot2d(&diff(f(x),x),>add,color=red,style="-.-"):
\end{eulerprompt}
\begin{eulercomment}
Untuk gaya garis ada berbagai pilihan.

- gaya="...". Pilih dari "-", "--", "-.", ".", ".-.", "-.-".\\
- Warna: Lihat di bawah untuk warna.\\
- ketebalan: Defaultnya adalah 1.

Warna dapat dipilih sebagai salah satu warna default, atau sebagai
warna RGB.

- 0..15: indeks warna default.\\
- konstanta warna: putih, hitam, merah, hijau, biru, cyan, zaitun,
abu-abu muda, abu-abu, abu-abu tua, oranye, hijau muda, pirus, biru
muda, oranye muda, kuning\\
- rgb(merah,hijau,biru): parameternya real di [0,1].
\end{eulercomment}
\begin{eulerprompt}
>plot2d("exp(-x^2)",r=2,color=red,thickness=3,style="--"):
\end{eulerprompt}
\begin{eulercomment}
Berikut adalah tampilan warna EMT yang telah ditentukan sebelumnya.
\end{eulercomment}
\begin{eulerprompt}
>aspect(2); columnsplot(ones(1,16),lab=0:15,grid=0,color=0:15):
\end{eulerprompt}
\begin{eulercomment}
Tapi Anda bisa menggunakan warna apa saja.
\end{eulercomment}
\begin{eulerprompt}
>columnsplot(ones(1,16),grid=0,color=rgb(0,0,linspace(0,1,15))):
\end{eulerprompt}
\eulerheading{Menggambar Beberapa Kurva pada bidang koordinat yang sama}
\begin{eulercomment}
Plot lebih dari satu fungsi (multiple function) ke dalam satu jendela
dapat dilakukan dengan berbagai cara. Salah satu metodenya adalah
menggunakan \textgreater{}add untuk beberapa panggilan ke plot2d secara
keseluruhan, kecuali panggilan pertama. Kami telah menggunakan fitur
ini pada contoh di atas.
\end{eulercomment}
\begin{eulerprompt}
>aspect(); plot2d("cos(x)",r=2,grid=6); plot2d("x",style=".",>add):
>aspect(1.5); plot2d("sin(x)",0,2pi); plot2d("cos(x)",color=blue,style="--",>add):
\end{eulerprompt}
\begin{eulercomment}
Salah satu kegunaan \textgreater{}add adalah untuk menambahkan titik pada kurva.
\end{eulercomment}
\begin{eulerprompt}
>plot2d("sin(x)",0,pi); plot2d(2,sin(2),>points,>add):
\end{eulerprompt}
\begin{eulercomment}
Kita tambahkan titik perpotongan dengan label (pada posisi "cl" untuk
kiri tengah), dan masukkan hasilnya ke dalam buku catatan. Kami juga
menambahkan judul pada plot.
\end{eulercomment}
\begin{eulerprompt}
>plot2d(["cos(x)","x"],r=1.1,cx=0.5,cy=0.5, ...
>  color=[black,blue],style=["-","."], ...
>  grid=1);
>x0=solve("cos(x)-x",1);  ...
>  plot2d(x0,x0,>points,>add,title="Intersection Demo");  ...
>  label("cos(x) = x",x0,x0,pos="cl",offset=20):
\end{eulerprompt}
\begin{eulercomment}
Dalam demo berikut, kita memplot fungsi sin(x)=sin(x)/x dan ekspansi
Taylor ke-8 dan ke-16. Kami menghitung perluasan ini menggunakan
Maxima melalui ekspresi simbolik.\\
Plot ini dilakukan dalam perintah multi-baris berikut dengan tiga
panggilan ke plot2d(). Yang kedua dan ketiga memiliki kumpulan tanda
\textgreater{}add, yang membuat plot menggunakan rentang sebelumnya.

Kami menambahkan kotak label yang menjelaskan fungsinya.
\end{eulercomment}
\begin{eulerprompt}
>$taylor(sin(x)/x,x,0,4)
>plot2d("sinc(x)",0,4pi,color=green,thickness=2); ...
>  plot2d(&taylor(sin(x)/x,x,0,8),>add,color=blue,style="--"); ...
>  plot2d(&taylor(sin(x)/x,x,0,16),>add,color=red,style="-.-"); ...
>  labelbox(["sinc","T8","T16"],styles=["-","--","-.-"], ...
>    colors=[black,blue,red]):
\end{eulerprompt}
\begin{eulercomment}
Dalam contoh berikut, kami menghasilkan Polinomial Bernstein.

\end{eulercomment}
\begin{eulerformula}
\[
B_i(x) = \binom{n}{i} x^i (1-x)^{n-i}
\]
\end{eulerformula}
\begin{eulerprompt}
>plot2d("(1-x)^10",0,1); // plot first function
>for i=1 to 10; plot2d("bin(10,i)*x^i*(1-x)^(10-i)",>add); end;
>insimg;
\end{eulerprompt}
\begin{eulercomment}
Cara kedua adalah dengan menggunakan pasangan matriks bernilai x dan
matriks bernilai y yang berukuran sama.

Kami menghasilkan matriks nilai dengan satu Polinomial Bernstein di
setiap baris. Untuk ini, kita cukup menggunakan vektor kolom i. Lihat
pendahuluan tentang bahasa matriks untuk mempelajari lebih detail.
\end{eulercomment}
\begin{eulerprompt}
>x=linspace(0,1,500);
>n=10; k=(0:n)'; // n is row vector, k is column vector
>y=bin(n,k)*x^k*(1-x)^(n-k); // y is a matrix then
>plot2d(x,y):
\end{eulerprompt}
\begin{eulercomment}
Perhatikan bahwa parameter warna dapat berupa vektor. Kemudian setiap
warna digunakan untuk setiap baris matriks.
\end{eulercomment}
\begin{eulerprompt}
>x=linspace(0,1,200); y=x^(1:10)'; plot2d(x,y,color=1:10):
\end{eulerprompt}
\begin{eulercomment}
Metode lain adalah menggunakan vektor ekspresi (string). Anda kemudian
dapat menggunakan susunan warna, susunan gaya, dan susunan ketebalan
dengan panjang yang sama.
\end{eulercomment}
\begin{eulerprompt}
>plot2d(["sin(x)","cos(x)"],0,2pi,color=4:5): 
>plot2d(["sin(x)","cos(x)"],0,2pi): // plot vector of expressions
\end{eulerprompt}
\begin{eulercomment}
Kita bisa mendapatkan vektor seperti itu dari Maxima menggunakan
makelist() dan mxm2str().
\end{eulercomment}
\begin{eulerprompt}
>v &= makelist(binomial(10,i)*x^i*(1-x)^(10-i),i,0,10) // make list
\end{eulerprompt}
\begin{euleroutput}
  
                  10            9              8  2             7  3
          [(1 - x)  , 10 (1 - x)  x, 45 (1 - x)  x , 120 (1 - x)  x , 
             6  4             5  5             4  6             3  7
  210 (1 - x)  x , 252 (1 - x)  x , 210 (1 - x)  x , 120 (1 - x)  x , 
            2  8              9   10
  45 (1 - x)  x , 10 (1 - x) x , x  ]
  
\end{euleroutput}
\begin{eulerprompt}
>mxm2str(v) // get a vector of strings from the symbolic vector
\end{eulerprompt}
\begin{euleroutput}
  (1-x)^10
  10*(1-x)^9*x
  45*(1-x)^8*x^2
  120*(1-x)^7*x^3
  210*(1-x)^6*x^4
  252*(1-x)^5*x^5
  210*(1-x)^4*x^6
  120*(1-x)^3*x^7
  45*(1-x)^2*x^8
  10*(1-x)*x^9
  x^10
\end{euleroutput}
\begin{eulerprompt}
>plot2d(mxm2str(v),0,1): // plot functions
\end{eulerprompt}
\begin{eulercomment}
Alternatif lain adalah dengan menggunakan bahasa matriks Euler.

Jika suatu ekspresi menghasilkan matriks fungsi, dengan satu fungsi di
setiap baris, semua fungsi tersebut akan diplot ke dalam satu plot.

Untuk ini, gunakan vektor parameter dalam bentuk vektor kolom. Jika
array warna ditambahkan maka akan digunakan untuk setiap baris plot.
\end{eulercomment}
\begin{eulerprompt}
>n=(1:10)'; plot2d("x^n",0,1,color=1:10):
\end{eulerprompt}
\begin{eulercomment}
Ekspresi dan fungsi satu baris dapat melihat variabel global.

Jika Anda tidak dapat menggunakan variabel global, Anda perlu
menggunakan fungsi dengan parameter tambahan, dan meneruskan parameter
ini sebagai parameter titik koma.

Berhati-hatilah, untuk meletakkan semua parameter yang ditetapkan di
akhir perintah plot2d. Dalam contoh ini kita meneruskan a=5 ke fungsi
f, yang kita plot dari -10 hingga 10.
\end{eulercomment}
\begin{eulerprompt}
>function f(x,a) := 1/a*exp(-x^2/a); ...
>plot2d("f",-10,10;5,thickness=2,title="a=5"):
\end{eulerprompt}
\begin{eulercomment}
Alternatifnya, gunakan koleksi dengan nama fungsi dan semua parameter
tambahan. Daftar khusus ini disebut kumpulan panggilan, dan ini adalah
cara yang lebih disukai untuk meneruskan argumen ke suatu fungsi yang
kemudian diteruskan sebagai argumen ke fungsi lain.

Pada contoh berikut, kita menggunakan loop untuk memplot beberapa
fungsi (lihat tutorial tentang pemrograman loop).
\end{eulercomment}
\begin{eulerprompt}
>plot2d(\{\{"f",1\}\},-10,10); ...
>for a=2:10; plot2d(\{\{"f",a\}\},>add); end:
\end{eulerprompt}
\begin{eulercomment}
Kita dapat mencapai hasil yang sama dengan cara berikut menggunakan
bahasa matriks EMT. Setiap baris matriks f(x,a) merupakan satu fungsi.
Selain itu, kita dapat mengatur warna untuk setiap baris matriks. Klik
dua kali pada fungsi getspectral() untuk penjelasannya.
\end{eulercomment}
\begin{eulerprompt}
>x=-10:0.01:10; a=(1:10)'; plot2d(x,f(x,a),color=getspectral(a/10)):
\end{eulerprompt}
\eulersubheading{Label Teks}
\begin{eulercomment}
Dekorasi sederhana pun bisa

- judul dengan judul = "..."\\
- label x dan y dengan xl="...", yl="..."\\
- label teks lain dengan label("...",x,y)

Perintah label akan memplot ke plot saat ini pada koordinat plot
(x,y). Hal ini memerlukan argumen posisional.
\end{eulercomment}
\begin{eulerprompt}
>plot2d("x^3-x",-1,2,title="y=x^3-x",yl="y",xl="x"):
>expr := "log(x)/x"; ...
>  plot2d(expr,0.5,5,title="y="+expr,xl="x",yl="y"); ...
>  label("(1,0)",1,0); label("Max",E,expr(E),pos="lc"):
\end{eulerprompt}
\begin{eulercomment}
Ada juga fungsi labelbox(), yang dapat menampilkan fungsi dan teks.
Dibutuhkan vektor string dan warna, satu item untuk setiap fungsi.
\end{eulercomment}
\begin{eulerprompt}
>function f(x) &= x^2*exp(-x^2);  ...
>plot2d(&f(x),a=-3,b=3,c=-1,d=1);  ...
>plot2d(&diff(f(x),x),>add,color=blue,style="--"); ...
>labelbox(["function","derivative"],styles=["-","--"], ...
>   colors=[black,blue],w=0.4):
\end{eulerprompt}
\begin{eulercomment}
Kotak ini berlabuh di kanan atas secara default, tetapi \textgreater{}kiri berlabuh
di kiri atas. Anda dapat memindahkannya ke tempat mana pun yang Anda
suka. Posisi jangkar berada di pojok kanan atas kotak, dan angkanya
merupakan pecahan dari ukuran jendela grafis. Lebarnya otomatis.

Untuk plot titik, kotak label juga berfungsi. Tambahkan parameter
\textgreater{}points, atau vektor bendera, satu untuk setiap label.

Pada contoh berikut, hanya ada satu fungsi. Jadi kita bisa menggunakan
string sebagai pengganti vektor string. Kami mengatur warna teks
menjadi hitam untuk contoh ini.
\end{eulercomment}
\begin{eulerprompt}
>n=10; plot2d(0:n,bin(n,0:n),>addpoints); ...
>labelbox("Binomials",styles="[]",>points,x=0.1,y=0.1, ...
>tcolor=black,>left):
\end{eulerprompt}
\begin{eulercomment}
Gaya plot ini juga tersedia di statplot(). Seperti di plot2d() warna
dapat diatur untuk setiap baris plot. Masih banyak lagi plot khusus
untuk keperluan statistik (lihat tutorial tentang statistik).
\end{eulercomment}
\begin{eulerprompt}
>statplot(1:10,random(2,10),color=[red,blue]):
\end{eulerprompt}
\begin{eulercomment}
Fitur serupa adalah fungsi textbox().

Lebarnya secara default adalah lebar maksimal baris teks. Tapi itu
bisa diatur oleh pengguna juga.
\end{eulercomment}
\begin{eulerprompt}
>function f(x) &= exp(-x)*sin(2*pi*x); ...
>plot2d("f(x)",0,2pi); ...
>textbox(latex("\(\backslash\)text\{Example of a damped oscillation\}\(\backslash\) f(x)=e^\{-x\}sin(2\(\backslash\)pi x)"),w=0.85):
\end{eulerprompt}
\begin{eulercomment}
Label teks, judul, kotak label, dan teks lainnya dapat berisi string
Unicode (lihat sintaks EMT untuk mengetahui lebih lanjut tentang
string Unicode).
\end{eulercomment}
\begin{eulerprompt}
>plot2d("x^3-x",title=u"x &rarr; x&sup3; - x"):
\end{eulerprompt}
\begin{eulercomment}
Label pada sumbu x dan y bisa vertikal, begitu juga dengan sumbunya.
\end{eulercomment}
\begin{eulerprompt}
>plot2d("sinc(x)",0,2pi,xl="x",yl=u"x &rarr; sinc(x)",>vertical):
\end{eulerprompt}
\eulersubheading{LaTeX}
\begin{eulercomment}
Anda juga dapat memplot rumus LaTeX jika Anda telah menginstal sistem
LaTeX. Saya merekomendasikan MiKTeX. Jalur ke biner "lateks" dan
"dvipng" harus berada di jalur sistem, atau Anda harus mengatur LaTeX
di menu opsi.

Perhatikan, penguraian LaTeX lambat. Jika Anda ingin menggunakan LaTeX
dalam plot animasi, Anda harus memanggil latex() sebelum loop satu
kali dan menggunakan hasilnya (gambar dalam matriks RGB).

Pada plot berikut, kami menggunakan LaTeX untuk label x dan y, label,
kotak label, dan judul plot.
\end{eulercomment}
\begin{eulerprompt}
>plot2d("exp(-x)*sin(x)/x",a=0,b=2pi,c=0,d=1,grid=6,color=blue, ...
>  title=latex("\(\backslash\)text\{Function $\(\backslash\)Phi$\}"), ...
>  xl=latex("\(\backslash\)phi"),yl=latex("\(\backslash\)Phi(\(\backslash\)phi)")); ...
>textbox( ...
>  latex("\(\backslash\)Phi(\(\backslash\)phi) = e^\{-\(\backslash\)phi\} \(\backslash\)frac\{\(\backslash\)sin(\(\backslash\)phi)\}\{\(\backslash\)phi\}"),x=0.8,y=0.5); ...
>label(latex("\(\backslash\)Phi",color=blue),1,0.4):
\end{eulerprompt}
\begin{eulercomment}
Seringkali, kita menginginkan spasi dan label teks yang tidak
konformal pada sumbu x. Kita bisa menggunakan xaxis() dan yaxis()
seperti yang akan kita tunjukkan nanti.

Cara termudah adalah membuat plot kosong dengan bingkai menggunakan
grid=4, lalu menambahkan grid dengan ygrid() dan xgrid(). Pada contoh
berikut, kami menggunakan tiga string LaTeX untuk label pada sumbu x
dengan xtick().
\end{eulercomment}
\begin{eulerprompt}
>plot2d("sinc(x)",0,2pi,grid=4,<ticks); ...
>ygrid(-2:0.5:2,grid=6); ...
>xgrid([0:2]*pi,<ticks,grid=6);  ...
>xtick([0,pi,2pi],["0","\(\backslash\)pi","2\(\backslash\)pi"],>latex):
\end{eulerprompt}
\begin{eulercomment}
Tentu saja fungsinya juga bisa digunakan.
\end{eulercomment}
\begin{eulerprompt}
>function map f(x) ...
\end{eulerprompt}
\begin{eulerudf}
  if x>0 then return x^4
  else return x^2
  endif
  endfunction
\end{eulerudf}
\begin{eulercomment}
Parameter "peta" membantu menggunakan fungsi untuk vektor. Untuk\\
plot, itu tidak perlu. Tapi untuk menunjukkan vektorisasi itu\\
berguna, kita menambahkan beberapa poin penting ke plot di x=-1, x=0
dan x=1.

Pada plot berikut, kami juga memasukkan beberapa kode LaTeX. Kami
menggunakannya untuk\\
dua label dan kotak teks. Tentu saja, Anda hanya bisa menggunakannya\\
LaTeX jika Anda telah menginstal LaTeX dengan benar.
\end{eulercomment}
\begin{eulerprompt}
>plot2d("f",-1,1,xl="x",yl="f(x)",grid=6);  ...
>plot2d([-1,0,1],f([-1,0,1]),>points,>add); ...
>label(latex("x^3"),0.72,f(0.72)); ...
>label(latex("x^2"),-0.52,f(-0.52),pos="ll"); ...
>textbox( ...
>  latex("f(x)=\(\backslash\)begin\{cases\} x^3 & x>0 \(\backslash\)\(\backslash\) x^2 & x \(\backslash\)le 0\(\backslash\)end\{cases\}"), ...
>  x=0.7,y=0.2):
\end{eulerprompt}
\eulersubheading{Interaksi Pengguna}
\begin{eulercomment}
Saat memplot suatu fungsi atau ekspresi, parameter \textgreater{}pengguna
memungkinkan pengguna untuk memperbesar dan menggeser plot dengan
tombol kursor atau mouse. Pengguna bisa

- perbesar dengan + atau -\\
- pindahkan plot dengan tombol kursor\\
- pilih jendela plot dengan mouse\\
- atur ulang tampilan dengan spasi\\
- keluar dengan kembali

Tombol spasi akan mengatur ulang plot ke jendela plot aslinya.

Saat memplot data, flag \textgreater{}user hanya akan menunggu penekanan tombol.
\end{eulercomment}
\begin{eulerprompt}
>plot2d(\{\{"x^3-a*x",a=1\}\},>user,title="Press any key!"):
>plot2d("exp(x)*sin(x)",user=true, ...
>  title="+/- or cursor keys (return to exit)"):
\end{eulerprompt}
\begin{eulercomment}
Berikut ini menunjukkan cara interaksi pengguna tingkat lanjut (lihat
tutorial tentang pemrograman untuk detailnya).

Fungsi bawaan mousedrag() menunggu aktivitas mouse atau keyboard. Ini
melaporkan mouse ke bawah, mouse digerakkan atau mouse ke atas, dan
penekanan tombol. Fungsi dragpoints() memanfaatkan ini, dan
memungkinkan pengguna menyeret titik mana pun dalam plot.

Kita membutuhkan fungsi plot terlebih dahulu. Misalnya, kita melakukan
interpolasi pada 5 titik dengan polinomial. Fungsi tersebut harus
diplot ke dalam area plot yang tetap.
\end{eulercomment}
\begin{eulerprompt}
>function plotf(xp,yp,select) ...
\end{eulerprompt}
\begin{eulerudf}
    d=interp(xp,yp);
    plot2d("interpval(xp,d,x)";d,xp,r=2);
    plot2d(xp,yp,>points,>add);
    if select>0 then
      plot2d(xp[select],yp[select],color=red,>points,>add);
    endif;
    title("Drag one point, or press space or return!");
  endfunction
\end{eulerudf}
\begin{eulercomment}
Perhatikan parameter titik koma di plot2d (d dan xp), yang diteruskan
ke evaluasi fungsi interp(). Tanpa ini, kita harus menulis fungsi
plotinterp() terlebih dahulu, mengakses nilainya secara global.

Sekarang kita menghasilkan beberapa nilai acak, dan membiarkan
pengguna menyeret titiknya.
\end{eulercomment}
\begin{eulerprompt}
>t=-1:0.5:1; dragpoints("plotf",t,random(size(t))-0.5):
\end{eulerprompt}
\begin{eulercomment}
Ada juga fungsi yang memplot fungsi lain bergantung pada vektor
parameter, dan memungkinkan pengguna menyesuaikan parameter ini.

Pertama kita membutuhkan fungsi plot.
\end{eulercomment}
\begin{eulerprompt}
>function plotf([a,b]) := plot2d("exp(a*x)*cos(2pi*b*x)",0,2pi;a,b);
\end{eulerprompt}
\begin{eulercomment}
Kemudian kita memerlukan nama untuk parameter, nilai awal dan matriks
rentang nx2, opsional garis judul.\\
Ada penggeser interaktif, yang dapat menetapkan nilai oleh pengguna.
Fungsi dragvalues() menyediakan ini.
\end{eulercomment}
\begin{eulerprompt}
>dragvalues("plotf",["a","b"],[-1,2],[[-2,2];[1,10]], ...
>  heading="Drag these values:",hcolor=black):
\end{eulerprompt}
\begin{eulercomment}
Dimungkinkan untuk membatasi nilai yang diseret menjadi bilangan
bulat. Sebagai contoh, kita menulis fungsi plot, yang memplot
polinomial Taylor berderajat n ke fungsi kosinus.
\end{eulercomment}
\begin{eulerprompt}
>function plotf(n) ...
\end{eulerprompt}
\begin{eulerudf}
  plot2d("cos(x)",0,2pi,>square,grid=6);
  plot2d(&"taylor(cos(x),x,0,@n)",color=blue,>add);
  textbox("Taylor polynomial of degree "+n,0.1,0.02,style="t",>left);
  endfunction
\end{eulerudf}
\begin{eulercomment}
Sekarang kita izinkan derajat n bervariasi dari 0 hingga 20 dalam 20
perhentian. Hasil dragvalues() digunakan untuk memplot sketsa dengan n
ini, dan untuk memasukkan plot ke dalam buku catatan.
\end{eulercomment}
\begin{eulerprompt}
>nd=dragvalues("plotf","degree",2,[0,20],20,y=0.8, ...
>   heading="Drag the value:"); ...
>plotf(nd):
\end{eulerprompt}
\begin{eulercomment}
Berikut ini adalah demonstrasi sederhana dari fungsinya. Pengguna
dapat menggambar jendela plot, meninggalkan jejak titik.
\end{eulercomment}
\begin{eulerprompt}
>function dragtest ...
\end{eulerprompt}
\begin{eulerudf}
    plot2d(none,r=1,title="Drag with the mouse, or press any key!");
    start=0;
    repeat
      \{flag,m,time\}=mousedrag();
      if flag==0 then return; endif;
      if flag==2 then
        hold on; mark(m[1],m[2]); hold off;
      endif;
    end
  endfunction
\end{eulerudf}
\begin{eulerprompt}
>dragtest // lihat hasilnya dan cobalah lakukan!
\end{eulerprompt}
\eulersubheading{Gaya Plot 2D}
\begin{eulercomment}
Secara default, EMT menghitung tick sumbu otomatis dan menambahkan
label ke setiap tick. Ini dapat diubah dengan parameter grid. Gaya
default sumbu dan label dapat diubah. Selain itu, label dan judul
dapat ditambahkan secara manual. Untuk menyetel ulang ke gaya default,
gunakan reset().
\end{eulercomment}
\begin{eulerprompt}
>aspect();
>figure(3,4); ...
> figure(1); plot2d("x^3-x",grid=0); ... // no grid, frame or axis
> figure(2); plot2d("x^3-x",grid=1); ... // x-y-axis
> figure(3); plot2d("x^3-x",grid=2); ... // default ticks
> figure(4); plot2d("x^3-x",grid=3); ... // x-y- axis with labels inside
> figure(5); plot2d("x^3-x",grid=4); ... // no ticks, only labels
> figure(6); plot2d("x^3-x",grid=5); ... // default, but no margin
> figure(7); plot2d("x^3-x",grid=6); ... // axes only
> figure(8); plot2d("x^3-x",grid=7); ... // axes only, ticks at axis
> figure(9); plot2d("x^3-x",grid=8); ... // axes only, finer ticks at axis
> figure(10); plot2d("x^3-x",grid=9); ... // default, small ticks inside
> figure(11); plot2d("x^3-x",grid=10); ...// no ticks, axes only
> figure(0):
\end{eulerprompt}
\begin{eulercomment}
Parameter \textless{}frame mematikan frame, dan framecolor=blue mengatur frame
menjadi warna biru.

Jika Anda menginginkan tanda centang Anda sendiri, Anda dapat
menggunakan style=0, dan menambahkan semuanya nanti.
\end{eulercomment}
\begin{eulerprompt}
>aspect(1.5); 
>plot2d("x^3-x",grid=0); // plot
>frame; xgrid([-1,0,1]); ygrid(0): // add frame and grid
\end{eulerprompt}
\begin{eulercomment}
Untuk judul plot dan label sumbu, lihat contoh berikut.
\end{eulercomment}
\begin{eulerprompt}
>plot2d("exp(x)",-1,1);
>textcolor(black); // set the text color to black
>title(latex("y=e^x")); // title above the plot
>xlabel(latex("x")); // "x" for x-axis
>ylabel(latex("y"),>vertical); // vertical "y" for y-axis
>label(latex("(0,1)"),0,1,color=blue): // label a point
\end{eulerprompt}
\begin{eulercomment}
Sumbu dapat digambar secara terpisah dengan xaxis() dan yaxis().
\end{eulercomment}
\begin{eulerprompt}
>plot2d("x^3-x",<grid,<frame);
>xaxis(0,xx=-2:1,style="->"); yaxis(0,yy=-5:5,style="->"):
\end{eulerprompt}
\begin{eulercomment}
Teks pada plot dapat diatur dengan label(). Dalam contoh berikut, "lc"
berarti bagian tengah bawah. Ini menetapkan posisi label relatif
terhadap koordinat plot.
\end{eulercomment}
\begin{eulerprompt}
>function f(x) &= x^3-x
\end{eulerprompt}
\begin{euleroutput}
  
                                   3
                                  x  - x
  
\end{euleroutput}
\begin{eulerprompt}
>plot2d(f,-1,1,>square);
>x0=fmin(f,0,1); // compute point of minimum
>label("Rel. Min.",x0,f(x0),pos="lc"): // add a label there
\end{eulerprompt}
\begin{eulercomment}
Ada juga kotak teks.
\end{eulercomment}
\begin{eulerprompt}
>plot2d(&f(x),-1,1,-2,2); // function
>plot2d(&diff(f(x),x),>add,style="--",color=red); // derivative
>labelbox(["f","f'"],["-","--"],[black,red]): // label box
>plot2d(["exp(x)","1+x"],color=[black,blue],style=["-","-.-"]):
>gridstyle("->",color=gray,textcolor=gray,framecolor=gray);  ...
> plot2d("x^3-x",grid=1);   ...
> settitle("y=x^3-x",color=black); ...
> label("x",2,0,pos="bc",color=gray);  ...
> label("y",0,6,pos="cl",color=gray); ...
> reset():
\end{eulerprompt}
\begin{eulercomment}
Untuk kontrol lebih lanjut, sumbu x dan sumbu y dapat dilakukan secara
manual.

Perintah fullwindow() memperluas jendela plot karena kita tidak lagi
memerlukan tempat untuk label di luar jendela plot. Gunakan
shrinkwindow() atau reset() untuk menyetel ulang ke default.
\end{eulercomment}
\begin{eulerprompt}
>fullwindow; ...
> gridstyle(color=darkgray,textcolor=darkgray); ...
> plot2d(["2^x","1","2^(-x)"],a=-2,b=2,c=0,d=4,<grid,color=4:6,<frame); ...
> xaxis(0,-2:1,style="->"); xaxis(0,2,"x",<axis); ...
> yaxis(0,4,"y",style="->"); ...
> yaxis(-2,1:4,>left); ...
> yaxis(2,2^(-2:2),style=".",<left); ...
> labelbox(["2^x","1","2^-x"],colors=4:6,x=0.8,y=0.2); ...
> reset:
\end{eulerprompt}
\begin{eulercomment}
Berikut adalah contoh lain, di mana string Unicode digunakan dan
sumbunya berada di luar area plot.
\end{eulercomment}
\begin{eulerprompt}
>aspect(1.5); 
>plot2d(["sin(x)","cos(x)"],0,2pi,color=[red,green],<grid,<frame); ...
> xaxis(-1.1,(0:2)*pi,xt=["0",u"&pi;",u"2&pi;"],style="-",>ticks,>zero);  ...
> xgrid((0:0.5:2)*pi,<ticks); ...
> yaxis(-0.1*pi,-1:0.2:1,style="-",>zero,>grid); ...
> labelbox(["sin","cos"],colors=[red,green],x=0.5,y=0.2,>left); ...
> xlabel(u"&phi;"); ylabel(u"f(&phi;)"):
\end{eulerprompt}
\eulerheading{Merencanakan Data 2D}
\begin{eulercomment}
Jika x dan y adalah vektor data, maka data tersebut akan digunakan
sebagai koordinat x dan y pada suatu kurva. Dalam hal ini, a, b, c,
dan d, atau radius r dapat ditentukan, atau jendela plot akan
menyesuaikan secara otomatis dengan data. Alternatifnya, \textgreater{}persegi
dapat diatur untuk mempertahankan rasio aspek persegi.

Merencanakan ekspresi hanyalah singkatan dari plot data. Untuk plot
data, Anda memerlukan satu atau beberapa baris nilai x, dan satu atau
beberapa baris nilai y. Dari rentang dan nilai x, fungsi plot2d akan
menghitung data yang akan diplot, secara default dengan evaluasi
fungsi yang adaptif. Untuk plot titik gunakan "\textgreater{}titik", untuk garis
dan titik campuran gunakan "\textgreater{}addpoints".

Tapi Anda bisa memasukkan data secara langsung.

- Gunakan vektor baris untuk x dan y untuk satu fungsi.\\
- Matriks untuk x dan y diplot baris demi baris.

Berikut adalah contoh dengan satu baris untuk x dan y.
\end{eulercomment}
\begin{eulerprompt}
>x=-10:0.1:10; y=exp(-x^2)*x; plot2d(x,y):
\end{eulerprompt}
\begin{eulercomment}
Data juga dapat diplot sebagai poin. Gunakan points=true untuk ini.
Plotnya berfungsi seperti poligon, tetapi hanya menggambar sudutnya
saja.

- style="...": Pilih dari "[]", "\textless{}\textgreater{}", "o", ".", "..", "+", "*", "[]#",
"\textless{} \textgreater{}#", "o#", "..#", "#", "\textbar{}".

Untuk memplot kumpulan titik, gunakan \textgreater{}titik. Jika warna merupakan
vektor warna, masing-masing titik\\
mendapat warna berbeda. Untuk matriks koordinat dan vektor kolom,
warna diterapkan pada baris matriks.\\
Parameter \textgreater{}addpoints menambahkan titik ke segmen garis untuk plot
data.
\end{eulercomment}
\begin{eulerprompt}
>xdata=[1,1.5,2.5,3,4]; ydata=[3,3.1,2.8,2.9,2.7]; // data
>plot2d(xdata,ydata,a=0.5,b=4.5,c=2.5,d=3.5,style="."); // lines
>plot2d(xdata,ydata,>points,>add,style="o"): // add points
>p=polyfit(xdata,ydata,1); // get regression line
>plot2d("polyval(p,x)",>add,color=red): // add plot of line
\end{eulerprompt}
\eulerheading{Menggambar Daerah Yang Dibatasi Kurva}
\begin{eulercomment}
Plot data sebenarnya berbentuk poligon. Kita juga dapat memplot kurva
atau kurva terisi.

- terisi=benar mengisi plot.\\
- style="...": Pilih dari "#", "/", "\textbackslash{}", "\textbackslash{}/".\\
- Fillcolor : Lihat di atas untuk mengetahui warna yang tersedia.

Warna isian ditentukan oleh argumen "fillcolor", dan pada \textless{}outline
opsional, mencegah menggambar batas untuk semua gaya kecuali gaya
default.
\end{eulercomment}
\begin{eulerprompt}
>t=linspace(0,2pi,1000); // parameter for curve
>x=sin(t)*exp(t/pi); y=cos(t)*exp(t/pi); // x(t) and y(t)
>figure(1,2); aspect(16/9)
>figure(1); plot2d(x,y,r=10); // plot curve
>figure(2); plot2d(x,y,r=10,>filled,style="/",fillcolor=red); // fill curve
>figure(0):
\end{eulerprompt}
\begin{eulercomment}
Dalam contoh berikut kita memplot elips terisi dan dua segi enam
terisi menggunakan kurva tertutup dengan 6 titik dengan gaya isian
berbeda.
\end{eulercomment}
\begin{eulerprompt}
>x=linspace(0,2pi,1000); plot2d(sin(x),cos(x)*0.5,r=1,>filled,style="/"):
>t=linspace(0,2pi,6); ...
>plot2d(cos(t),sin(t),>filled,style="/",fillcolor=red,r=1.2):
>t=linspace(0,2pi,6); plot2d(cos(t),sin(t),>filled,style="#"):
\end{eulerprompt}
\begin{eulercomment}
Contoh lainnya adalah septagon yang kita buat dengan 7 titik pada
lingkaran satuan.
\end{eulercomment}
\begin{eulerprompt}
>t=linspace(0,2pi,7);  ...
> plot2d(cos(t),sin(t),r=1,>filled,style="/",fillcolor=red):
\end{eulerprompt}
\begin{eulercomment}
Berikut adalah himpunan nilai maksimal dari empat kondisi linier yang
kurang dari atau sama dengan 3. Ini adalah A[k].v\textless{}=3 untuk semua baris
A. Untuk mendapatkan sudut yang bagus, kita menggunakan n yang relatif
besar.
\end{eulercomment}
\begin{eulerprompt}
>A=[2,1;1,2;-1,0;0,-1];
>function f(x,y) := max([x,y].A');
>plot2d("f",r=4,level=[0;3],color=green,n=111):
\end{eulerprompt}
\begin{eulercomment}
Poin utama dari bahasa matriks adalah memungkinkan pembuatan tabel
fungsi dengan mudah.
\end{eulercomment}
\begin{eulerprompt}
>t=linspace(0,2pi,1000); x=cos(3*t); y=sin(4*t);
\end{eulerprompt}
\begin{eulercomment}
Kami sekarang memiliki nilai vektor x dan y. plot2d() dapat memplot
nilai-nilai ini\\
sebagai kurva yang menghubungkan titik-titik tersebut. Plotnya bisa
diisi. Dalam hal ini\\
ini menghasilkan hasil yang bagus karena aturan belitan, yang
digunakan untuk\\
isi.
\end{eulercomment}
\begin{eulerprompt}
>plot2d(x,y,<grid,<frame,>filled):
\end{eulerprompt}
\begin{eulercomment}
Vektor interval diplot terhadap nilai x sebagai wilayah terisi\\
antara nilai interval yang lebih rendah dan lebih tinggi.

Hal ini dapat berguna untuk memplot kesalahan perhitungan. Tapi itu
bisa\\
juga dapat digunakan untuk memplot kesalahan statistik.
\end{eulercomment}
\begin{eulerprompt}
>t=0:0.1:1; ...
> plot2d(t,interval(t-random(size(t)),t+random(size(t))),style="|");  ...
> plot2d(t,t,add=true):
\end{eulerprompt}
\begin{eulercomment}
Jika x adalah vektor yang diurutkan, dan y adalah vektor interval,
maka plot2d akan memplot rentang interval yang terisi pada bidang.
Gaya isiannya sama dengan gaya poligon.
\end{eulercomment}
\begin{eulerprompt}
>t=-1:0.01:1; x=~t-0.01,t+0.01~; y=x^3-x;
>plot2d(t,y):
\end{eulerprompt}
\begin{eulercomment}
Dimungkinkan untuk mengisi wilayah nilai untuk fungsi tertentu. Untuk\\
ini, level harus berupa matriks 2xn. Baris pertama adalah batas bawah\\
dan baris kedua berisi batas atas.
\end{eulercomment}
\begin{eulerprompt}
>expr := "2*x^2+x*y+3*y^4+y"; // define an expression f(x,y)
>plot2d(expr,level=[0;1],style="-",color=blue): // 0 <= f(x,y) <= 1
\end{eulerprompt}
\begin{eulercomment}
Kita juga dapat mengisi rentang nilai seperti

lateks: -1 \textbackslash{}le (x\textasciicircum{}2+y\textasciicircum{}2)\textasciicircum{}2-x\textasciicircum{}2+y\textasciicircum{}2 \textbackslash{}le 0.
\end{eulercomment}
\begin{eulerprompt}
>plot2d("(x^2+y^2)^2-x^2+y^2",r=1.2,level=[-1;0],style="/"):
>plot2d("cos(x)","sin(x)^3",xmin=0,xmax=2pi,>filled,style="/"):
\end{eulerprompt}
\eulerheading{Grafik Fungsi Parametrik}
\begin{eulercomment}
Nilai x tidak perlu diurutkan. (x,y) hanya menggambarkan sebuah kurva.
Jika x diurutkan, kurva tersebut merupakan grafik suatu fungsi.

Dalam contoh berikut, kita memplot spiral

\end{eulercomment}
\begin{eulerformula}
\[
\gamma(t) = t \cdot (\cos(2\pi t),\sin(2\pi t))
\]
\end{eulerformula}
\begin{eulercomment}
Kita perlu menggunakan banyak titik untuk tampilan yang halus atau
fungsi adaptif() untuk mengevaluasi ekspresi (lihat fungsi adaptif()
untuk lebih jelasnya).
\end{eulercomment}
\begin{eulerprompt}
>t=linspace(0,1,1000); ...
>plot2d(t*cos(2*pi*t),t*sin(2*pi*t),r=1):
\end{eulerprompt}
\begin{eulercomment}
Sebagai alternatif, dimungkinkan untuk menggunakan dua ekspresi untuk
kurva. Berikut ini plot kurva yang sama seperti di atas.
\end{eulercomment}
\begin{eulerprompt}
>plot2d("x*cos(2*pi*x)","x*sin(2*pi*x)",xmin=0,xmax=1,r=1):
>t=linspace(0,1,1000); r=exp(-t); x=r*cos(2pi*t); y=r*sin(2pi*t);
>plot2d(x,y,r=1):
\end{eulerprompt}
\begin{eulercomment}
Dalam contoh berikutnya, kami memplot kurva

\end{eulercomment}
\begin{eulerformula}
\[
\gamma(t) = (r(t) \cos(t), r(t) \sin(t))
\]
\end{eulerformula}
\begin{eulercomment}
dengan

\end{eulercomment}
\begin{eulerformula}
\[
r(t) = 1 + \dfrac{\sin(3t)}{2}.
\]
\end{eulerformula}
\begin{eulerprompt}
>t=linspace(0,2pi,1000); r=1+sin(3*t)/2; x=r*cos(t); y=r*sin(t); ...
>plot2d(x,y,>filled,fillcolor=red,style="/",r=1.5):
\end{eulerprompt}
\eulerheading{Menggambar Grafik Bilangan Kompleks}
\begin{eulercomment}
Serangkaian bilangan kompleks juga dapat diplot. Kemudian titik-titik
grid akan dihubungkan. Jika sejumlah garis kisi ditentukan (atau
vektor garis kisi 1x2) dalam argumen cgrid, hanya garis kisi tersebut
yang terlihat.

Matriks bilangan kompleks secara otomatis akan diplot sebagai
kisi-kisi pada bidang kompleks.

Pada contoh berikut, kita memplot gambar lingkaran satuan di bawah
fungsi eksponensial. Parameter cgrid menyembunyikan beberapa kurva
grid.
\end{eulercomment}
\begin{eulerprompt}
>aspect(); r=linspace(0,1,50); a=linspace(0,2pi,80)'; z=r*exp(I*a);...
>plot2d(z,a=-1.25,b=1.25,c=-1.25,d=1.25,cgrid=10):
>aspect(1.25); r=linspace(0,1,50); a=linspace(0,2pi,200)'; z=r*exp(I*a);
>plot2d(exp(z),cgrid=[40,10]):
>r=linspace(0,1,10); a=linspace(0,2pi,40)'; z=r*exp(I*a);
>plot2d(exp(z),>points,>add):
\end{eulerprompt}
\begin{eulercomment}
Sebuah vektor bilangan kompleks secara otomatis diplot sebagai kurva
pada bidang kompleks dengan\\
bagian real dan bagian imajiner.\\
Dalam contoh, kami memplot lingkaran satuan dengan

\end{eulercomment}
\begin{eulerformula}
\[
\gamma(t) = e^{it}
\]
\end{eulerformula}
\begin{eulerprompt}
>t=linspace(0,2pi,1000); ...
>plot2d(exp(I*t)+exp(4*I*t),r=2):
\end{eulerprompt}
\eulerheading{Plot Statistik}
\begin{eulercomment}
Ada banyak fungsi yang dikhususkan pada plot statistik. Salah satu
plot yang sering digunakan adalah plot kolom.

Jumlah kumulatif dari nilai terdistribusi normal 0-1 menghasilkan
jalan acak.
\end{eulercomment}
\begin{eulerprompt}
>plot2d(cumsum(randnormal(1,1000))):
\end{eulerprompt}
\begin{eulercomment}
Penggunaan dua baris menunjukkan jalan dalam dua dimensi.
\end{eulercomment}
\begin{eulerprompt}
>X=cumsum(randnormal(2,1000)); plot2d(X[1],X[2]):
>columnsplot(cumsum(random(10)),style="/",color=blue):
\end{eulerprompt}
\begin{eulercomment}
Itu juga dapat menampilkan string sebagai label.
\end{eulercomment}
\begin{eulerprompt}
>months=["Jan","Feb","Mar","Apr","May","Jun", ...
>  "Jul","Aug","Sep","Oct","Nov","Dec"];
>values=[10,12,12,18,22,28,30,26,22,18,12,8];
>columnsplot(values,lab=months,color=red,style="-");
>title("Temperature"):
>k=0:10;
>plot2d(k, bin(10,k),>bar):
>plot2d(k,bin(10,k)); plot2d(k,bin(10,k),>points,>add):
>plot2d(normal(1000),normal(1000),>points,grid=6,style=".."):
>plot2d(normal(1,1000),>distribution,style="O"):
>plot2d("qnormal",0,5;2.5,0.5,>filled):
\end{eulerprompt}
\begin{eulercomment}
Untuk memplot distribusi statistik eksperimental, Anda dapat
menggunakan distribution=n dengan plot2d.
\end{eulercomment}
\begin{eulerprompt}
>w=randexponential(1,1000); // exponential distribution
>plot2d(w,>distribution): // or distribution=n with n intervals
\end{eulerprompt}
\begin{eulercomment}
Atau Anda dapat menghitung distribusi dari data dan memplot hasilnya
dengan \textgreater{}bar di plot3d, atau dengan plot kolom.
\end{eulercomment}
\begin{eulerprompt}
>w=normal(1000); // 0-1-normal distribution
>\{x,y\}=histo(w,10,v=[-6,-4,-2,-1,0,1,2,4,6]); // interval bounds v
>plot2d(x,y,>bar):
\end{eulerprompt}
\begin{eulercomment}
Fungsi statplot() mengatur gaya dengan string sederhana.
\end{eulercomment}
\begin{eulerprompt}
>statplot(1:10,cumsum(random(10)),"b"):
>n=10; i=0:n; ...
>plot2d(i,bin(n,i)/2^n,a=0,b=10,c=0,d=0.3); ...
>plot2d(i,bin(n,i)/2^n,points=true,style="ow",add=true,color=blue):
\end{eulerprompt}
\begin{eulercomment}
Selain itu, data dapat diplot sebagai batang. Dalam hal ini, x harus
diurutkan dan satu elemen lebih panjang dari y. Batangnya akan
memanjang dari x[i] hingga x[i+1] dengan nilai y[i]. Jika x berukuran
sama dengan y, maka x akan diperpanjang satu elemen dengan spasi
terakhir.

Gaya isian dapat digunakan seperti di atas.
\end{eulercomment}
\begin{eulerprompt}
>n=10; k=bin(n,0:n); ...
>plot2d(-0.5:n+0.5,k,bar=true,fillcolor=lightgray):
\end{eulerprompt}
\begin{eulercomment}
Data untuk plot batang (batang=1) dan histogram (histogram=1) dapat
diberikan secara eksplisit dalam xv dan yv, atau dapat dihitung dari
distribusi empiris dalam xv dengan \textgreater{}distribusi (atau distribusi=n).
Histogram nilai xv akan dihitung secara otomatis dengan \textgreater{}histogram.
Jika \textgreater{}even ditentukan, nilai xv akan dihitung dalam interval bilangan
bulat.
\end{eulercomment}
\begin{eulerprompt}
>plot2d(normal(10000),distribution=50):
>k=0:10; m=bin(10,k); x=(0:11)-0.5; plot2d(x,m,>bar):
>columnsplot(m,k):
>plot2d(random(600)*6,histogram=6):
\end{eulerprompt}
\begin{eulercomment}
Untuk distribusi, terdapat parameter distribution=n, yang menghitung
nilai secara otomatis dan mencetak distribusi relatif dengan n
sub-interval.
\end{eulercomment}
\begin{eulerprompt}
>plot2d(normal(1,1000),distribution=10,style="\(\backslash\)/"):
\end{eulerprompt}
\begin{eulercomment}
Dengan parameter even=true, ini akan menggunakan interval bilangan
bulat.
\end{eulercomment}
\begin{eulerprompt}
>plot2d(intrandom(1,1000,10),distribution=10,even=true):
\end{eulerprompt}
\begin{eulercomment}
Perhatikan bahwa ada banyak plot statistik yang mungkin berguna.
Silahkan lihat tutorial tentang statistik.
\end{eulercomment}
\begin{eulerprompt}
>columnsplot(getmultiplicities(1:6,intrandom(1,6000,6))):
>plot2d(normal(1,1000),>distribution); ...
>  plot2d("qnormal(x)",color=red,thickness=2,>add):
\end{eulerprompt}
\begin{eulercomment}
Ada juga banyak plot khusus untuk statistik. Plot kotak menunjukkan
kuartil distribusi ini dan banyak outlier. Menurut definisinya,
outlier dalam plot kotak adalah data yang melebihi 1,5 kali rentang
50\% tengah plot.
\end{eulercomment}
\begin{eulerprompt}
>M=normal(5,1000); boxplot(quartiles(M)):
\end{eulerprompt}
\eulerheading{Fungsi Implisit}
\begin{eulercomment}
Plot implisit menunjukkan penyelesaian garis level f(x,y)=level,
dengan "level" dapat berupa nilai tunggal atau vektor nilai. Jika
level = "auto", akan ada garis level nc, yang akan tersebar antara
fungsi minimum dan maksimum secara merata. Warna yang lebih gelap atau
terang dapat ditambahkan dengan \textgreater{}hue untuk menunjukkan nilai fungsi.
Untuk fungsi implisit, xv harus berupa fungsi atau ekspresi parameter
x dan y, atau alternatifnya, xv dapat berupa matriks nilai.

Euler dapat menandai garis level

\end{eulercomment}
\begin{eulerformula}
\[
f(x,y) = c
\]
\end{eulerformula}
\begin{eulercomment}
dari fungsi apa pun.

Untuk menggambar himpunan f(x,y)=c untuk satu atau lebih konstanta c,
Anda dapat menggunakan plot2d() dengan plot implisitnya pada bidang.
Parameter c adalah level=c, dimana c dapat berupa vektor garis level.
Selain itu, skema warna dapat digambar di latar belakang untuk
menunjukkan nilai fungsi setiap titik dalam plot. Parameter "n"
menentukan kehalusan plot.
\end{eulercomment}
\begin{eulerprompt}
>aspect(1.5); 
>plot2d("x^2+y^2-x*y-x",r=1.5,level=0,contourcolor=red):
>expr := "2*x^2+x*y+3*y^4+y"; // define an expression f(x,y)
>plot2d(expr, level=0): // Solutions of f(x,y)=0
>plot2d(expr,level=0:0.5:20,>hue,contourcolor=white,n=200): // nice
>plot2d(expr,level=0:0.5:20,>hue,>spectral,n=200,grid=4): // nicer
\end{eulerprompt}
\begin{eulercomment}
Ini juga berfungsi untuk plot data. Namun Anda harus menentukan
rentangnya\\
untuk label sumbu.
\end{eulercomment}
\begin{eulerprompt}
>x=-2:0.05:1; y=x'; z=expr(x,y);
>plot2d(z,level=0,a=-1,b=2,c=-2,d=1,>hue):
>plot2d("x^3-y^2",>contour,>hue,>spectral):
>plot2d("x^3-y^2",level=0,contourwidth=3,>add,contourcolor=red):
>z=z+normal(size(z))*0.2;
>plot2d(z,level=0.5,a=-1,b=2,c=-2,d=1):
>plot2d(expr,level=[0:0.2:5;0.05:0.2:5.05],color=lightgray):
>plot2d("x^2+y^3+x*y",level=1,r=4,n=100):
>plot2d("x^2+2*y^2-x*y",level=0:0.1:10,n=100,contourcolor=white,>hue):
\end{eulerprompt}
\begin{eulercomment}
Dimungkinkan juga untuk mengisi set

\end{eulercomment}
\begin{eulerformula}
\[
a \le f(x,y) \le b
\]
\end{eulerformula}
\begin{eulercomment}
dengan rentang level.

Dimungkinkan untuk mengisi wilayah nilai untuk fungsi tertentu. Untuk
ini, level harus berupa matriks 2xn. Baris pertama adalah batas bawah
dan baris kedua berisi batas atas.
\end{eulercomment}
\begin{eulerprompt}
>plot2d(expr,level=[0;1],style="-",color=blue): // 0 <= f(x,y) <= 1
\end{eulerprompt}
\begin{eulercomment}
Plot implisit juga dapat menunjukkan rentang level. Maka level harus
berupa matriks interval level 2xn, di mana baris pertama berisi awal
dan baris kedua berisi akhir setiap interval. Alternatifnya, vektor
baris sederhana dapat digunakan untuk level, dan parameter dl
memperluas nilai level ke interval.
\end{eulercomment}
\begin{eulerprompt}
>plot2d("x^4+y^4",r=1.5,level=[0;1],color=blue,style="/"):
>plot2d("x^2+y^3+x*y",level=[0,2,4;1,3,5],style="/",r=2,n=100):
>plot2d("x^2+y^3+x*y",level=-10:20,r=2,style="-",dl=0.1,n=100):
>plot2d("sin(x)*cos(y)",r=pi,>hue,>levels,n=100):
\end{eulerprompt}
\begin{eulercomment}
Dimungkinkan juga untuk menandai suatu wilayah

\end{eulercomment}
\begin{eulerformula}
\[
a \le f(x,y) \le b.
\]
\end{eulerformula}
\begin{eulercomment}
Hal ini dilakukan dengan menambahkan level dengan dua baris.
\end{eulercomment}
\begin{eulerprompt}
>plot2d("(x^2+y^2-1)^3-x^2*y^3",r=1.3, ...
>  style="#",color=red,<outline, ...
>  level=[-2;0],n=100):
\end{eulerprompt}
\begin{eulercomment}
Dimungkinkan untuk menentukan level tertentu. Misalnya, kita dapat
memplot solusi persamaan seperti

\end{eulercomment}
\begin{eulerformula}
\[
x^3-xy+x^2y^2=6
\]
\end{eulerformula}
\begin{eulerprompt}
>plot2d("x^3-x*y+x^2*y^2",r=6,level=1,n=100):
>function starplot1 (v, style="/", color=green, lab=none) ...
\end{eulerprompt}
\begin{eulerudf}
    if !holding() then clg; endif;
    w=window(); window(0,0,1024,1024);
    h=holding(1);
    r=max(abs(v))*1.2;
    setplot(-r,r,-r,r);
    n=cols(v); t=linspace(0,2pi,n);
    v=v|v[1]; c=v*cos(t); s=v*sin(t);
    cl=barcolor(color); st=barstyle(style);
    loop 1 to n
      polygon([0,c[#],c[#+1]],[0,s[#],s[#+1]],1);
      if lab!=none then
        rlab=v[#]+r*0.1;
        \{col,row\}=toscreen(cos(t[#])*rlab,sin(t[#])*rlab);
        ctext(""+lab[#],col,row-textheight()/2);
      endif;
    end;
    barcolor(cl); barstyle(st);
    holding(h);
    window(w);
  endfunction
\end{eulerudf}
\begin{eulercomment}
Tidak ada tanda centang kotak atau sumbu di sini. Selain itu, kami
menggunakan jendela penuh untuk plotnya.

Kami memanggil reset sebelum kami menguji plot ini untuk mengembalikan
default grafis. Ini tidak perlu dilakukan jika Anda yakin plot Anda
berhasil.
\end{eulercomment}
\begin{eulerprompt}
>reset; starplot1(normal(1,10)+5,color=red,lab=1:10):
\end{eulerprompt}
\begin{eulercomment}
Terkadang, Anda mungkin ingin merencanakan sesuatu yang plot2d tidak
bisa lakukan, tapi hampir.

Dalam fungsi berikut, kita membuat plot impuls logaritmik. plot2d
dapat melakukan plot logaritmik, tetapi tidak untuk batang impuls.
\end{eulercomment}
\begin{eulerprompt}
>function logimpulseplot1 (x,y) ...
\end{eulerprompt}
\begin{eulerudf}
    \{x0,y0\}=makeimpulse(x,log(y)/log(10));
    plot2d(x0,y0,>bar,grid=0);
    h=holding(1);
    frame();
    xgrid(ticks(x));
    p=plot();
    for i=-10 to 10;
      if i<=p[4] and i>=p[3] then
         ygrid(i,yt="10^"+i);
      endif;
    end;
    holding(h);
  endfunction
\end{eulerudf}
\begin{eulercomment}
Mari kita uji dengan nilai yang terdistribusi secara eksponensial.
\end{eulercomment}
\begin{eulerprompt}
>aspect(1.5); x=1:10; y=-log(random(size(x)))*200; ...
>logimpulseplot1(x,y):
\end{eulerprompt}
\begin{eulercomment}
Mari kita menganimasikan kurva 2D menggunakan plot langsung. Perintah
plot(x,y) hanya memplot kurva ke dalam jendela plot. setplot(a,b,c,d)
menyetel jendela ini.

Fungsi wait(0) memaksa plot muncul di jendela grafis. Jika tidak,
pengundian ulang akan dilakukan dalam interval waktu yang jarang.
\end{eulercomment}
\begin{eulerprompt}
>function animliss (n,m) ...
\end{eulerprompt}
\begin{eulerudf}
  t=linspace(0,2pi,500);
  f=0;
  c=framecolor(0);
  l=linewidth(2);
  setplot(-1,1,-1,1);
  repeat
    clg;
    plot(sin(n*t),cos(m*t+f));
    wait(0);
    if testkey() then break; endif;
    f=f+0.02;
  end;
  framecolor(c);
  linewidth(l);
  endfunction
\end{eulerudf}
\begin{eulercomment}
Tekan tombol apa saja untuk menghentikan animasi ini.
\end{eulercomment}
\begin{eulerprompt}
>animliss(2,3); // lihat hasilnya, jika sudah puas, tekan ENTER
\end{eulerprompt}
\eulerheading{Plot Logaritmik}
\begin{eulercomment}
EMT menggunakan parameter "logplot" untuk skala logaritmik.\\
Plot logaritma dapat diplot menggunakan skala logaritma di y dengan
logplot=1, atau menggunakan skala logaritma di x dan y dengan
logplot=2, atau di x dengan logplot=3.

\end{eulercomment}
\begin{eulerttcomment}
 - logplot=1: y-logaritma
 - logplot=2: x-y-logaritma
 - logplot=3: x-logaritma
\end{eulerttcomment}
\begin{eulerprompt}
>plot2d("exp(x^3-x)*x^2",1,5,logplot=1):
>plot2d("exp(x+sin(x))",0,100,logplot=1):
>plot2d("exp(x+sin(x))",10,100,logplot=2):
>plot2d("gamma(x)",1,10,logplot=1):
>plot2d("log(x*(2+sin(x/100)))",10,1000,logplot=3):
\end{eulerprompt}
\begin{eulercomment}
Ini juga berfungsi dengan plot data.
\end{eulercomment}
\begin{eulerprompt}
>x=10^(1:20); y=x^2-x;
>plot2d(x,y,logplot=2):
\end{eulerprompt}
\eulersubheading{Contoh Latihan Soal}
\begin{eulercomment}
1. Buatlah plot 2 dimensi dari fungsi berikut\\
\end{eulercomment}
\begin{eulerformula}
\[
y=a*{(x^2)/a}
\]
\end{eulerformula}
\begin{eulercomment}
dengan a=4 dan interval dari x=-1 sampai x=1. Kemudian buatlah plot
tersebut dalam grid 3 dan ketebalan 3!

Penyelesaian:
\end{eulercomment}
\begin{eulerprompt}
>a:=4; expr &= exp(a*(x^2)/a);
>plot2d(expr,-1,1,grid=3,thickness=3):
\end{eulerprompt}
\begin{eulercomment}
2. Buatlah plot 2 dimensi dari fungsi berikut\\
\end{eulercomment}
\begin{eulerformula}
\[
y(x,a)=x^2-a*x
\]
\end{eulerformula}
\begin{eulercomment}
dengan nilai parameter adalah 5, dengan interval x nya dari 1 sampai 2

Penyelesaian:
\end{eulercomment}
\begin{eulerprompt}
>a:=5; expr &= exp((x^2)-a*x);
>plot2d(expr,1,2):
\end{eulerprompt}
\begin{eulercomment}
3. Selidiki dimanakah fungsi f(x) berikut naik dan turun\\
\end{eulercomment}
\begin{eulerformula}
\[
f(x)=(1/3)x^3*-x^2-3*x+4
\]
\end{eulerformula}
\begin{eulercomment}
Penyelesaian:
\end{eulercomment}
\begin{eulerprompt}
>function f(x):=(1/3)*x^3-x^2-3*x+4;
>plot2d("f",-4,5):
>a:=8; expr &= exp((x^5)-a*x);
>plot2d(expr,1,2):
\end{eulerprompt}
\begin{eulercomment}
5. Buatlah plot 2 dimensi dari fungsi berikut\\
\end{eulercomment}
\begin{eulerformula}
\[
y(x,a)=x^4-a*x
\]
\end{eulerformula}
\begin{eulercomment}
dengan nilai parameter adalah 5, dengan interval x nya dari 1 sampai 2

Penyelesaian:
\end{eulercomment}
\begin{eulerprompt}
>a:=5; expr &= exp((x^4)-a*x);
>plot2d(expr,1,2):
\end{eulerprompt}
\begin{eulercomment}
\end{eulercomment}
\eulersubheading{Rujukan Lengkap Fungsi}
\begin{eulercomment}
plot2d() function plot2d (xv, yv, btest, a, b, c, d, xmin, xmax, r, n,
..\\
logplot, kisi, bingkai, warna bingkai, kotak, warna, ketebalan, gaya,
..\\
otomatis, tambahkan, pengguna, delta, poin, titik tambahan, gaya
titik, bilah, histogram, ..\\
distribusi, genap, langkah, sendiri, adaptif, rona, level, kontur, ..\\
nc, terisi, fillcolor, outline, title, xl, yl, maps, contourcolor, ..\\
contourwidth, ticks, margin, clipping, cx, cy, insimg, spectral, ..\\
cgrid, vertikal, lebih kecil, dl, niveau, level)\\
Fungsi plot serbaguna untuk plot di pesawat (plot 2D). Fungsi ini
dapat melakukan\\
plot fungsi satu variabel, plot data, kurva bidang, plot batang,
kisi-kisi bilangan kompleks, dan plot\\
implisit fungsi dua variabel.\\
Parameter\\
x,y : persamaan, fungsi atau vektor data\\
a,b,c,d : Area plot (default a=-2,b=2)\\
r : jika r diset, maka a=cx-r, b=cx+r, c=cy-r, d=cy+r\\
r dapat berupa vektor [rx,ry] atau vektor\\
[rx1,rx2,ry1,ry2].\\
xmin,xmax : rentang parameter untuk kurva\\
auto : Menentukan y-range secara otomatis (default)\\
kuadrat : jika benar, coba pertahankan rentang x-y persegi\\
n : jumlah interval (default adaptif)\\
kisi : 0 = tidak ada kisi dan label,\\
1 = sumbu saja,\\
2 = grid normal (lihat di bawah untuk jumlah garis\\
grid)\\
3 = sumbu dalam\\
4 = tidak ada kisi-kisi\\
5 = kisi penuh termasuk margin\\
6 = kutu di bingkai\\
7 = sumbu saja\\
8 = sumbu saja, sub-centang\\
bingkai : 0 = tanpa bingkai\\
framecolor : warna bingkai dan kisi\\
margin : angka antara 0 dan 0,4 untuk margin di sekitar plot\\
warna : Warna kurva. Jika ini adalah vektor warna,\\
itu akan digunakan untuk setiap baris matriks plot. Dalam\\
kasus plot titik, itu harus berupa vektor kolom. Jika vektor baris
atau matriks penuh warna digunakan\\
untuk plot titik, itu akan digunakan untuk setiap titik data.\\
ketebalan: ketebalan garis untuk kurva\\
Nilai ini bisa lebih kecil dari 1 untuk garis yang sangat\\
tipis.\\
style : Plot style untuk garis, spidol, dan isian.\\
Untuk poin gunakan\\
"[]", "\textless{}\textgreater{}", ".", "..", "...",\\
"*", "+", "\textbar{}", "-", "o"\\
"[]#", "\textless{}\textgreater{}#", "o#" (bentuk terisi)\\
"[]w", "\textless{}\textgreater{}w", "ow" (tidak transparan)\\
Untuk penggunaan garis\\
"-", "--", "-.", ".", ".-.", "-.-", "-\textgreater{}"\\
Untuk poligon terisi atau plot batang gunakan\\
"#", "#O", "O", "/", "\textbackslash{}", "\textbackslash{}/",\\
"+", "\textbar{}", "-", "t"\\
poin : plot titik tunggal alih-alih segmen garis\\
addpoints : jika benar, plot segmen garis dan titik\\
add : menambahkan plot ke plot yang ada\\
pengguna : aktifkan interaksi pengguna untuk fungsi\\
delta : ukuran langkah untuk interaksi pengguna\\
bar : plot batang (x adalah batas interval, y nilai interval)\\
histogram : memplot frekuensi x dalam n subinterval\\
distribusi=n : memplot distribusi x dengan n subinterval\\
even : gunakan nilai antar untuk histogram otomatis.\\
langkah : memplot fungsi sebagai fungsi langkah (langkah=1,2)\\
adaptif : gunakan plot adaptif (n adalah jumlah langkah minimal)\\
level : plot garis level dari fungsi implisit dua variabel\\
outline : menggambar batas rentang level.\\
Jika nilai level adalah matriks 2xn, rentang level akan ditarik\\
dalam warna menggunakan gaya isian yang diberikan. Jika garis besar
benar, itu\\
akan digambar dalam warna kontur. Dengan menggunakan fitur ini,
wilayah\\
f(x,y) antara batas dapat ditandai.\\
hue : tambahkan warna hue ke plot level untuk menunjukkan fungsinya\\
nilai\\
kontur : Gunakan plot level dengan level otomatis\\
nc : jumlah garis level otomatis\\
judul : judul plot (default ””)\\
xl, yl : label untuk sumbu x dan y\\
lebih kecil : jika \textgreater{}0, akan ada lebih banyak ruang di sebelah kiri
untuk label.\\
vertikal :\\
Mengaktifkan atau menonaktifkan label vertikal. Ini mengubah\\
variabel global\\
verticallabels secara lokal untuk satu plot. Nilai 1 hanya set\\
vertikal\\
teks, nilai 2 menggunakan label numerik vertikal pada sumbu y.\\
terisi : mengisi plot kurva\\
fillcolor : mengisi warna untuk bar dan kurva yang terisi\\
outline : batas poligon yang terisi\\
logplot : mengatur plot logaritma\\
1 = logplot di y,\\
2 = plot log di xy,\\
3 = logplot dalam x\\
memiliki :\\
Sebuah string, yang menunjuk ke rutinitas plot sendiri. Dengan \textgreater{}\\
pengguna, Anda mendapatkan\\
interaksi pengguna yang sama seperti di plot2d. Rentang akan diatur\\
sebelum setiap panggilan ke fungsi Anda.\\
peta : ekspresi peta (0 lebih cepat), fungsi selalu dipetakan.\\
contourcolor : warna garis kontur\\
contourwidth : lebar garis kontur\\
clipping : mengaktifkan clipping (default adalah true)\\
judul :\\
Ini dapat digunakan untuk menggambarkan plot. Judul akan muncul di\\
atas\\
jalan cerita. Selain itu, label untuk sumbu x dan y dapat\\
ditambahkan dengan\\
xl="string" atau yl="string". Label lain dapat ditambahkan dengan\\
fungsi label() atau labelbox(). Judulnya bisa unicode\\
string atau gambar rumus Lateks.\\
jaringan :\\
Menentukan jumlah garis grid untuk plot grid yang kompleks.\\
Harus merupakan pembagi dari ukuran matriks dikurangi 1 (jumlah\\
subinterval). cgrid dapat berupa vektor [cx,cy].\\
Ringkasan\\
Fungsi dapat merencanakan\\
- ekspresi, koleksi panggilan atau fungsi dari satu variabel,\\
- kurva parametrik,\\
- data x terhadap data y,\\
- fungsi implisit,\\
- petak batang,\\
- jaringan kompleks,\\
- poligon.\\
Jika fungsi atau ekspresi untuk xv diberikan, plot2d() akan\\
menghitung\\
nilai dalam rentang yang diberikan menggunakan fungsi atau ekspresi.
Itu\\
ekspresi harus berupa ekspresi dalam variabel x. Rentang harus\\
didefinisikan dalam parameter a dan b kecuali rentang default\\
harus digunakan. Rentang y akan dihitung secara otomatis,\\
kecuali c dan d ditentukan, atau radius r, yang menghasilkan kisaran\\
r,r\\
untuk x dan y. Untuk plot fungsi, plot2d akan menggunakan\\
evaluasi adaptif fungsi secara default. Untuk mempercepat\\
plot untuk fungsi yang rumit, matikan ini dengan \textless{}adaptif, dan\\
opsional mengurangi jumlah interval n. Selain itu, plot2d()\\
akan secara default menggunakan pemetaan. Yaitu, itu akan menghitung
elemen plot\\
untuk elemen. Jika ekspresi atau fungsi Anda dapat menangani a\\
vector x, Anda dapat menonaktifkannya dengan \textless{}maps untuk evaluasi yang
lebih cepat.\\
Perhatikan bahwa plot adaptif selalu dihitung elemen untuk elemen.\\
Jika fungsi atau ekspresi untuk xv dan untuk yv ditentukan,\\
plot2d() akan menghitung kurva dengan nilai xv sebagai koordinat x\\
dan nilai yv sebagai koordinat y. Dalam hal ini, rentang harus\\
didefinisikan untuk parameter menggunakan xmin, xmax. Ekspresi yang
terkandung\\
dalam string harus selalu ekspresi dalam variabel parameter x.
\end{eulercomment}
\begin{eulercomment}


\begin{eulercomment}
\eulerheading{Menggambar Plot 3D dengan EMT}
\begin{eulercomment}
Ini adalah pengenalan plot 3D di Euler. Kita membutuhkan plot 3D untuk
memvisualisasikan fungsi dari dua variabel.

Euler menggambar fungsi tersebut menggunakan algoritma pengurutan
untuk menyembunyikan bagian di latar belakang. Secara umum, Euler
menggunakan proyeksi pusat. Standarnya adalah dari kuadran x-y positif
menuju titik asal x=y=z=0, tetapi sudut=0° terlihat dari arah sumbu y.
Sudut pandang dan tinggi dapat diubah.

Euler dapat merencanakan

- permukaan dengan bayangan dan garis level atau rentang level,\\
- awan poin,\\
- kurva parametrik,\\
- permukaan implisit.

Plot 3D dari suatu fungsi menggunakan plot3d. Cara termudah adalah
dengan memplot ekspresi dalam x dan y. Parameter r mengatur kisaran
plot di sekitar (0,0).
\end{eulercomment}
\begin{eulerprompt}
>aspect(1.5); plot3d("x^2+sin(y)",r=pi): 
\end{eulerprompt}
\eulersubheading{1) Menggambar Grafik Fungsi Dua Variabel dalam Bentuk}
\eulersubheading{Ekspresi Langsung}
\begin{eulercomment}
Grafik fungsi dua variabel dalam bentuk ekspresi langsung adalah
representasi visual dari hubungan matematis antara dua variabel
independen yang dinyatakan dalam bentuk persamaan atau ekspresi
matematis. Biasanya, grafik ini digunakan untuk menggambarkan hubungan
antara dua variabel dalam bidang dua dimensi dan tiga dimensi. Dalam
konteks ini, variabel independen (x dan y) adalah variabel input,
sedangkan variabel dependen (z) adalah variabel output yang dihasilkan
oleh ekspresi matematis.

Rumus umum untuk menggambar grafik fungsi dua variabel dalam bentuk
ekspresi langsung adalah:\\
\end{eulercomment}
\begin{eulerformula}
\[
z = f(x, y)
\]
\end{eulerformula}
\begin{eulercomment}
Dalam rumus ini:\\
- z adalah variabel dependen yang ingin kita gambar dalam grafik.\\
- f(x, y) adalah ekspresi matematis yang menghubungkan variabel z
dengan variabel independen x dan y. Ekspresi ini dapat berupa fungsi
linear, fungsi kuadrat, fungsi akar kuadrat, eksponensial, logaritma,
trigonometri, nilai mutlak, atau jenis fungsi matematis lainnya,
tergantung pada hubungan yang ingin diilustrasikan.

\end{eulercomment}
\eulersubheading{1. Grafik Fungsi Linear}
\begin{eulercomment}
Fungsi linear dua variabel biasanya dinyatakan dalam bentuk\\
\end{eulercomment}
\begin{eulerformula}
\[
f(x,y)=ax+by+c
\]
\end{eulerformula}
\begin{eulercomment}
dimana a,b, dan c adalah konstanta.  Grafik fungsi linear ini adalah
sebuah bidang datar, dan bentuknya akan bervariasi tergantung pada
nilai a dan b.

Contoh:\\
\end{eulercomment}
\begin{eulerformula}
\[
f(x,y)=2x+3y+5
\]
\end{eulerformula}
\begin{eulerprompt}
>plot3d("2*x+3*y+5"):
\end{eulerprompt}
\begin{eulerformula}
\[
f(x,y)=-2x-3y-5
\]
\end{eulerformula}
\begin{eulerprompt}
>plot3d("-2*x-3*y-5"):
\end{eulerprompt}
\begin{eulerformula}
\[
f(x,y)=5x-7y+9
\]
\end{eulerformula}
\begin{eulerprompt}
>plot3d("5*x-7*y+9"):
\end{eulerprompt}
\eulersubheading{2. Grafik Fungsi Kuadrat}
\begin{eulercomment}
Fungsi kuadrat dua variabel biasanya dinyatakan dalam bentuk\\
\end{eulercomment}
\begin{eulerformula}
\[
f(x,y)=ax^2+by^2+cxy+dx+ey+f
\]
\end{eulerformula}
\begin{eulercomment}
dimana a, b, c, d, e, dan f adalah konstanta. Grafik fungsi kuadrat
ini adalah sebuah permukaan yang dapat memiliki berbagai bentuk
tergantung pada nilai-nilai konstantanya.

Contoh:\\
\end{eulercomment}
\begin{eulerformula}
\[
f(x,y)= x^2+y^2+4xy+8x+3y+1
\]
\end{eulerformula}
\begin{eulerprompt}
>plot3d("x^2+y^2+4*x*y+8*x+3*y+1"):
\end{eulerprompt}
\begin{eulerformula}
\[
f(x,y)= 3x^2-y^2+7xy-5
\]
\end{eulerformula}
\begin{eulerprompt}
>plot3d("3*x^2-y^2+7*x*y-5"):
\end{eulerprompt}
\eulersubheading{3. Grafik Fungsi Akar Kuadrat}
\begin{eulercomment}
Grafik fungsi akar kuadrat dua variabel\\
\end{eulercomment}
\begin{eulerformula}
\[
f(x,y)=\sqrt{x^2+y^2}
\]
\end{eulerformula}
\begin{eulercomment}
adalah grafik permukaan yang menggambarkan jarak titik (x, y) dari
titik asal (0, 0) dalam ruang tiga dimensi.

Contoh:\\
\end{eulercomment}
\begin{eulerformula}
\[
f(x,y)=\sqrt{x^2+y^2}
\]
\end{eulerformula}
\begin{eulerprompt}
>plot3d("sqrt(x^2+y^2)"):
\end{eulerprompt}
\begin{eulerformula}
\[
f(x,y)=\sqrt{2x^2+7y^2}
\]
\end{eulerformula}
\begin{eulerprompt}
>plot3d("sqrt(2*x^2+7*y^2)"):
\end{eulerprompt}
\begin{eulerformula}
\[
f(x,y)=\sqrt{10x^2+y^2}
\]
\end{eulerformula}
\begin{eulerprompt}
>plot3d("sqrt(10*x^2+y^2)"):
\end{eulerprompt}
\eulersubheading{4. Grafik Fungsi Eksponensial}
\begin{eulercomment}
Fungsi eksponensial dua variabel bisa dinyatakan\\
\end{eulercomment}
\begin{eulerformula}
\[
f(x,y)=a.b^{xy}
\]
\end{eulerformula}
\begin{eulercomment}
dimana a dan b adalah konstanta, x dan y adalah variabel. Fungsi ini
menggambarkan pertumbuhan eksponensial yang bergantung pada nilai x
dan y.

Contoh:\\
\end{eulercomment}
\begin{eulerformula}
\[
f(x,y)= 2.3^{xy}
\]
\end{eulerformula}
\begin{eulerprompt}
>plot3d("2*3^(x*y)"):
\end{eulerprompt}
\begin{eulerformula}
\[
f(x,y)= -3.8^{xy}
\]
\end{eulerformula}
\begin{eulerprompt}
>plot3d("-3*8^(x*y)"):
\end{eulerprompt}
\eulersubheading{5. Grafik Fungsi Logaritma}
\begin{eulercomment}
Grafik fungsi logaritma dua variabel adalah grafik yang menggambarkan
nilai logaritma dari suatu ekspresi yang melibatkan dua variabel
(biasanya x dan y). Fungsi logaritma dua variabel ini dinyatakan
sebagai\\
\end{eulercomment}
\begin{eulerformula}
\[
f(x,y)=log_b(xy)
\]
\end{eulerformula}
\begin{eulercomment}
dimana b adalah basis logaritma. Basis logaritma ini dapat
berbeda-beda.\\
Contoh:\\
\end{eulercomment}
\begin{eulerformula}
\[
f(x,y)=log(xy), basis 10
\]
\end{eulerformula}
\begin{eulerprompt}
>plot3d("log(x*y)"):
\end{eulerprompt}
\begin{eulerformula}
\[
f(x,y)=log(2x.9y), basis 10
\]
\end{eulerformula}
\begin{eulerprompt}
>plot3d("log(2x*9y)"):
\end{eulerprompt}
\eulersubheading{6. Grafik Fungsi Trigonometri}
\begin{eulercomment}
Fungsi trigonometri dua variabel adalah fungsi matematika yang
melibatkan operasi trigonometri (seperti sin, cos, tan) pada kedua
variabel x dan y.

Contoh:\\
\end{eulercomment}
\begin{eulerformula}
\[
f(x,y)=sin(x).cos(y)
\]
\end{eulerformula}
\begin{eulerprompt}
>plot3d("sin(x)*cos(y)"):
\end{eulerprompt}
\begin{eulerformula}
\[
f(x,y)=sin(2x).cos(y)
\]
\end{eulerformula}
\begin{eulerprompt}
>plot3d("sin(2x)*cos(y)"):
\end{eulerprompt}
\begin{eulerformula}
\[
f(x,y)=sin(x).tan(y)
\]
\end{eulerformula}
\begin{eulerprompt}
>plot3d("sin(x)*tan(y)"):
\end{eulerprompt}
\begin{eulerformula}
\[
f(x,y)=cos(x).tan(y)
\]
\end{eulerformula}
\begin{eulerprompt}
>plot3d("cos(x)*tan(y)"):
\end{eulerprompt}
\begin{eulerformula}
\[
f(x,y)=cosec(x).sec(y)
\]
\end{eulerformula}
\begin{eulerprompt}
>plot3d("cosec(x)*sec(y)"):
\end{eulerprompt}
\begin{eulerformula}
\[
f(x,y)=cot(x).cosec(y)
\]
\end{eulerformula}
\begin{eulerprompt}
>plot3d("cot(x)*cosec(y)"):
\end{eulerprompt}
\eulersubheading{7. Grafik Fungsi Nilai Mutlak}
\begin{eulercomment}
Fungsi nilai mutlak dua variabel, juga dikenal sebagai fungsi modul
dua variabel, dinyatakan sebagai\\
\end{eulercomment}
\begin{eulerformula}
\[
f(x,y)=|g(x,y)|
\]
\end{eulerformula}
\begin{eulercomment}
dimana g(x,y) adalah fungsi dua variabel.

Contoh:\\
\end{eulercomment}
\begin{eulerformula}
\[
f(x,y)=|x^2 - y^2|
\]
\end{eulerformula}
\begin{eulerprompt}
>plot3d("abs(x^2 - y^2)"):
\end{eulerprompt}
\begin{eulerformula}
\[
f(x,y)=|x^2 + y^2|
\]
\end{eulerformula}
\begin{eulerprompt}
>plot3d("abs(x^2 + y^2)"):
\end{eulerprompt}
\begin{eulerformula}
\[
f(x,y)=|-2x^2 - 5y^2|
\]
\end{eulerformula}
\begin{eulerprompt}
>plot3d("abs(-2x^2 - 5y^2)"):
\end{eulerprompt}
\begin{eulerformula}
\[
f(x,y)=|x^2 - 8y^2|
\]
\end{eulerformula}
\begin{eulerprompt}
>plot3d("abs(x^2 - 8y^2)"):
\end{eulerprompt}
\eulersubheading{Latihan soal}
\begin{eulercomment}
Buatkan grafik dari fungsi berikut:\\
1.\\
\end{eulercomment}
\begin{eulerformula}
\[
f(x,y)=8x-3y+7
\]
\end{eulerformula}
\begin{eulerprompt}
>plot3d("8*x - 3*y +7"):
\end{eulerprompt}
\begin{eulercomment}
2.\\
\end{eulercomment}
\begin{eulerformula}
\[
f(x,y)=cos(5x).sin(9y)
\]
\end{eulerformula}
\begin{eulerprompt}
>plot3d("cos(5*x)*sin(9*y)"):
\end{eulerprompt}
\begin{eulercomment}
3.\\
\end{eulercomment}
\begin{eulerformula}
\[
f(x,y)=\sqrt{x^2+20y^2}
\]
\end{eulerformula}
\begin{eulerprompt}
>plot3d("sqrt(x^2+20*y^2)"):
\end{eulerprompt}
\eulersubheading{2) Menggambar Grafik Fungsi Dua Variabel yang Rumusnya Disimpan}
\eulersubheading{dalam Variabel Ekspresi}
\begin{eulercomment}
\end{eulercomment}
\eulersubheading{Apa yang dimaksud dengan Grafik Fungsi Dua Variabel?}
\begin{eulercomment}
Grafik fungsi dua variabel adalah representasi visual dari hubungan
antara sebuah fungsi matematika dengan dua variabel independen
(biasanya disebut sebagai "x" dan "y") dan variabel dependen (biasanya
disebut sebagai "z" atau "f(x, y)").

Dalam grafik ini, sumbu x dan sumbu y digunakan untuk menggambarkan
nilai-nilai dua variabel independen, sementara permukaan atau grafik
3D digunakan untuk menggambarkan nilai-nilai variabel dependen yang
dihasilkan oleh fungsi tersebut.

Grafik fungsi dua variabel membantu memvisualisasikan bagaimana nilai
variabel dependen (z) berubah seiring perubahan kedua variabel
independen (x dan y) sesuai dengan aturan fungsi tersebut.

\end{eulercomment}
\eulersubheading{Fungsi matematika yg terlibat dalam Menggambar Grafik}
\eulersubheading{Fungsi Dua Variabel}
\begin{eulercomment}
1. Fungsi Linear\\
Bentuk umum\\
f(x, y) = ax + by + c, di mana a, b, dan c adalah konstanta. Grafiknya
adalah bidang datar.

2. Fungsi Kuadratik\\
Bentuk umum f(x, y) = ax\textasciicircum{}2 + by\textasciicircum{}2 + cxy + dx + ey + f.\\
Grafiknya dapat berupa permukaan yang berbentuk paraboloid, baik
terbuka ke atas atau ke bawah.

3.Fungsi Trigonometri\\
Bentuk umum sinus dan cosinus\\
(x, y) = sin(x) + cos(y)\\
akan menghasilkan permukaan yang berulang-ulang naik dan turun.

4.Fungsi Pecahan\\
Bentuk umum f(x, y) = g(x, y) / h(x, y), di mana g(x, y) dan h(x, y)
adalah fungsi-fungsi lain.\\
grafiknya dapat menghasilkan berbagai pola yang tergantung pada sifat
fungsi g dan h.

\end{eulercomment}
\eulersubheading{Menggambar Grafik Fungsi}
\begin{eulercomment}
Perintah yang digunakan untuk menggambar grafik fungsi dalam EMT yaitu
dengan menggunakan plot3d.\\
Untuk menampilkan grafik, akhiri perintah plot3d dengan tanda (:).\\
Tanda (:) akan menampilkan grafik di layar yang berbeda.
\end{eulercomment}
\begin{eulerprompt}
>plot3d("x^2+y^2"):
>plot3d("x^3+x*sin(y)",-5,5,0,6*pi):
\end{eulerprompt}
\eulersubheading{Menyimpan Variabel Ekspresi}
\begin{eulercomment}
Untuk menyimpan sebuah fungsi, dapat dilakukan dengan menggunakan
perintah function. Lalu, ketika ingin memanggil atau membuat grafik
dari fungsi tersebut, cukup dengan memanggil nama fungsi tersebut.
\end{eulercomment}
\begin{eulerprompt}
>function a(x,y):= x^2+y^3
>plot3d("a"):
>function f(x,y):= x^3-y^2
>plot3d("f"):
\end{eulerprompt}
\eulersubheading{Contoh Latihan Soal}
\begin{eulercomment}
\end{eulercomment}
\begin{eulerformula}
\[
f(x,y)= x^3+y^4
\]
\end{eulerformula}
\begin{eulerprompt}
>function f(x,y):= x^3+y^4
>plot3d("f"):
>plot3d("a",>user, ...
>title="Turn with the vector keys (press return to finish)"):
\end{eulerprompt}
\begin{eulercomment}
Perintah ini mengizinkan pengguna untuk menggambar grafik 3D dari
fungsi yang mereka masukkan sendiri, serta memberikan petunjuk
interaktif tentang cara berinteraksi dengan grafik. Pengguna dapat
memutar tampilan grafik menggunakan tombol arah pada keyboard, dan
menekan "return" untuk menyelesaikan interaksi.
\end{eulercomment}
\begin{eulerprompt}
>plot3d("a",r=5,n=80,fscale=4,scale=1.2,frame=3):
\end{eulerprompt}
\begin{eulercomment}
perintah ini akan menggambar grafik tiga dimensi dari fungsi "a" dalam
rentang x dan y dari -5 hingga 5, dengan 80 titik untuk detail yang
lebih halus. Nilai fungsi akan diperbesar 4 kali, dan grafik akan
ditampilkan dengan skala 1.2 untuk tampilan yang lebih jelas. Bingkai
grafis akan ditentukan oleh parameter frame=3.
\end{eulercomment}
\begin{eulerprompt}
>view
\end{eulerprompt}
\begin{euleroutput}
  [5,  2.6,  2,  0.4]
\end{euleroutput}
\begin{eulerprompt}
>plot3d("a",distance=3,zoom=2,angle=0,height=0):
>plot3d("x^4+y^2",a=0,b=1,c=-1,d=1,angle=-20°,height=20°, ...
>center=[0.4,0,0],zoom=5):
>plot3d("a",r=2,<fscale,<scale,distance=13,height=50°, ...
> center=[0,0,-2],frame=3):
>plot3d("a",r=5,>polar, ...
>fscale=2,>hue,n=100,zoom=4,>contour,color=blue):
>plot3d("x","a","y",r=2,zoom=3.5,frame=3):
\end{eulerprompt}
\begin{eulercomment}
FUNGSI LINEAR\\
\end{eulercomment}
\eulersubheading{}
\begin{eulerprompt}
>function e(x,y):= 20x+10y-5
>plot3d("e"):
>plot3d("e",>user, ...
>title="Turn with the vector keys (press return to finish)"):
>plot3d("e",r=10,n=80,fscale=4,scale=1.2,frame=3):
>view
\end{eulerprompt}
\begin{euleroutput}
  [5,  2.6,  2,  0.4]
\end{euleroutput}
\begin{eulerprompt}
>plot3d("e",distance=3,zoom=2,angle=0,height=0):
>plot3d("e",a=0,b=1,c=-1,d=1,angle=-20°,height=20°, ...
>center=[0.4,0,0],zoom=5):
>plot3d("e",r=2,<fscale,<scale,distance=13,height=50°, ...
>center=[0,0,-2],frame=3):
>plot3d("e",r=5,>polar, ...
>fscale=2,>hue,n=100,zoom=4,>contour,color=gray):
>plot3d("x","e","y",r=2,zoom=3.5,frame=3):
\end{eulerprompt}
\begin{eulercomment}
FUNGSI TRIGONOMETRI\\
\end{eulercomment}
\eulersubheading{}
\begin{eulerprompt}
>function f(x,y):= sin(x+y)
>plot3d("f")
>plot3d("f",>user, ...
>title="Turn with the vector keys (press return to finish)"):
>plot3d("f",r=10,n=80,fscale=4,scale=1.2,frame=3):
>view
\end{eulerprompt}
\begin{euleroutput}
  [5,  2.6,  2,  0.4]
\end{euleroutput}
\begin{eulerprompt}
>plot3d("f",distance=3,zoom=2,angle=0,height=0):
>plot3d("f",a=0,b=1,c=-1,d=1,angle=-20°,height=20°, ...
>center=[0.4,0,0],zoom=5):
>plot3d("f",r=5,>polar, ...
>fscale=2,>hue,n=100,zoom=4,>contour,color=gray):
\end{eulerprompt}
\eulersubheading{3) Menggambar Grafik Fungsi Dua Variabel yang Fungsinya}
\eulersubheading{Didefinisikan sebagai Fungsi Numerik}
\begin{eulercomment}
Sebelum masuk ke cara memvisualisasikan grafik, perlu diketahui apa
itu fungsi dua variabel dan apa itu fungsi numerik.

\end{eulercomment}
\eulersubheading{Fungsi Dua Variabel}
\begin{eulercomment}
Fungsi dua variabel adalah jenis fungsi di mana ada dua variabel bebas
(biasanya dinotasikan sebagai x dan y) yang menentukan nilai dari
fungsi tersebut. Dengan kata lain, untuk setiap kombinasi nilai dari x
dan y, fungsi ini akan menghasilkan satu nilai output tertentu. Fungsi
dari dua variabel yang mana setiap kombinasi nilai dari kedua variabel
tersebut menghasilkan sebuah nilai tunggal.


\end{eulercomment}
\eulersubheading{Fungsi Numerik}
\begin{eulercomment}
Fungsi dimana setiap pasangan variabel independen adalah angka atau
bilangan nyata. Secara sederhana, ketika memasukkan angka atau
bilangan nyata ke variabel-variabel dalam fungsi maka hasil akhir yang
dihasilkan juga angka atau bilangan nyata, bukan simbol atau ekspresi
yang belum dihitung.\\
Contoh:\\
ada fungsi\\
\end{eulercomment}
\begin{eulerformula}
\[
f(x,y)=2x+y
\]
\end{eulerformula}
\begin{eulercomment}
Ketika kita memasukkan bilangan nyata ke x dan y maka akan dihasilkan
suatu bilangan nyata juga. Misal masukkan x=1 dan y=1. Akan diperoleh

\end{eulercomment}
\begin{eulerformula}
\[
2(1)+1=3
\]
\end{eulerformula}
\begin{eulercomment}
\end{eulercomment}
\eulersubheading{Visualisasi Grafik}
\begin{eulercomment}
Untuk memvisualisaikan fungsi dua variabel dengan fungsinya
didefinisikan sebagai fungsi numerik, akan dibuat grafik 3D dengan
sintaks plot3d. Untuk membedakan fungsi numerik dengan simbolik, pada
kali ini untuk setiap fungsi numerik dua variabel hanya akan memuat
variabel x dan y. Namun, dalam pemakaian secara umum, bisa digunakan
variabel apapun.

Penulisan Sintaks:\\
1) definisikan fungsi numerik\\
function f(x,y):= ax+by dengan a dan b adalah suatu konstanta dan
fungsi tidak selalu direpresentastikan dengan f tetapi bisa dengan
huruf apapun. Contoh : g(x,y)\\
2) sintaks plot3d\\
plot3d("f"):

Contoh Visualisasi Grafik:\\
1. Visualisasi grafik fungsi linear dua variabel

\end{eulercomment}
\begin{eulerformula}
\[
f(x,y) = 3x+7y
\]
\end{eulerformula}
\begin{eulerprompt}
>function f(x,y):= 2*x+5*y
>plot3d("f"):
\end{eulerprompt}
\begin{eulercomment}
2. Visualisasi grafik fungsi kuadrat dua variabel

\end{eulercomment}
\begin{eulerformula}
\[
f(x,y)=x^2+2xy+y^2
\]
\end{eulerformula}
\begin{eulerprompt}
>function f(x,y):= x^2+2*x*y+y^2 
>plot3d("f"):
\end{eulerprompt}
\begin{eulercomment}
3. Visualisasi Grafik Fungsi Eksponen Dua Variabel

\end{eulercomment}
\begin{eulerformula}
\[
g(x,y) = x^{2y+8}
\]
\end{eulerformula}
\begin{eulerprompt}
>function g(x,y):= x^(2y+8)
>plot3d("g"):
\end{eulerprompt}
\begin{eulercomment}
4. Grafik Fungsi Logaritma Dua Variabel

\end{eulercomment}
\begin{eulerformula}
\[
f(x,y)=\log(xy)
\]
\end{eulerformula}
\begin{eulerprompt}
>function f(x,y):= log(x*y)
>plot3d("f"):
\end{eulerprompt}
\begin{eulercomment}
5. Grafik Fungsi Trigonometri Dua Variabel

\end{eulercomment}
\begin{eulerformula}
\[
h(x,y)=sin(xy)cos(y)
\]
\end{eulerformula}
\begin{eulerprompt}
>function h(x,y):= sin(x*y)*cos(y)
>plot3d("h"):
\end{eulerprompt}
\begin{eulercomment}
6. Grafik Fungsi Nilai Mutlak Dua Variabel

\end{eulercomment}
\begin{eulerformula}
\[
i(x,y)=|(2x+y)|
\]
\end{eulerformula}
\begin{eulerprompt}
>function i(x,y):= abs(2x+y)
>plot3d("i"):
\end{eulerprompt}
\eulersubheading{Latihan Soal}
\begin{eulercomment}
Buatlah visualisasi grafik dari fungsi berikut ini!

\end{eulercomment}
\begin{eulerformula}
\[
f(x,y)=2^x+3y
\]
\end{eulerformula}
\begin{eulerprompt}
>function f(x,y):=2^x+3*y
>plot3d ("f"):
\end{eulerprompt}
\begin{eulerformula}
\[
g(x,y)=sin(x^2y+1)
\]
\end{eulerformula}
\begin{eulerprompt}
>function g(x,y):= sin(x^2*y+1)
>plot3d("g"):
\end{eulerprompt}
\begin{eulerformula}
\[
h(x,y)=|4x+sin(y^3+1)|
\]
\end{eulerformula}
\begin{eulerprompt}
>function h(x,y):=abs(4*x + sin(y^3+1))
>plot3d("h"):
\end{eulerprompt}
\eulersubheading{4) Menggambar Grafik Fungsi Dua Variabel yang Fungsinya}
\eulersubheading{Didefinisikan sebagai Fungsi Simbolik}
\begin{eulercomment}
Grafik Fungsi dua variabel yang fungsinya didefinisikan sebagai fungsi
simbolik adalah suatu grafik yang memvisualisasikan Persamaan Linear
Dua Variabel (PLDV) dalam koordinat kartesius dengan fungsinya
merupakan fungsi simbolik.

Proses visualisasi ini memungkinkan Kita untuk melihat dan memahami
bagaimana fungsi tersebut berperilaku dalam tiga dimensi.

Sedangkan yang dimaksud dengan fungsi simbolik yaitu fungsi yang
didefinisikan dalam bentuk matematika simbolik, artinya kita memiliki
ekspresi matematika yang menggambarkan hubungan antara dua variabel.
misalnya, jika kita memiliki fungsi\\
\end{eulercomment}
\begin{eulerformula}
\[
f(x, y) = x^2 + y^2
\]
\end{eulerformula}
\begin{eulercomment}
kita dapat menggambar grafiknya untuk melihat bentuk permukaan yang
dihasilkan oleh fungsi ini dalam tiga dimensi. Grafik ini mungkin akan
berupa tumpukan parabola yang membentuk kerucut.\\
Karakteristik fungsi simbolik adalah dengan mengganti tanda := menjadi
\&=

Perbedaan utama antara fungsi numerik dan fungsi simbolik adalah bahwa
fungsi numerik memberikan hasil numerik secara langsung (menghasilkan
angka), sementara fungsi simbolik memungkinkan kita untuk bekerja
dengan simbol matematika sebelum menghitung nilai numeriknya. Pilihan
antara keduanya tergantung pada kebutuhan analisis matematika yang
kita lakukan.

Tujuan menggambar grafik fungsi dua variabel adalah untuk memahami
pola, sifat, dan hubungan antara dua variabel tersebut secara visual,
yang dapat membantu dalam analisis dan pemahaman masalah matematika
atau ilmu pengetahuan yang melibatkan fungsi ini.

\end{eulercomment}
\eulersubheading{Langkah-langkah Membuat Grafik}
\begin{eulercomment}
1) Definisikan fungsinya terlebih dahulu. Tentukan fungsi dua variabel
yang akan divisualisasikan dalam bentuk simbolik.\\
Misal diambil fungsi\\
\end{eulercomment}
\begin{eulerformula}
\[
f(x,y)=x^2+y^2
\]
\end{eulerformula}
\begin{eulercomment}
Maka perintah yang dapat dituliskan yaitu:
\end{eulercomment}
\begin{eulerprompt}
>function f(x,y)&= x^2+y^2;
\end{eulerprompt}
\begin{eulercomment}
2) Panggil fungsinya
\end{eulercomment}
\begin{eulerprompt}
>plot3d("f(x,y)"):
\end{eulerprompt}
\begin{eulercomment}
3) Tentukan rentang variabelnya
\end{eulercomment}
\begin{eulerprompt}
>plot3d("f(x,y)",-5,5,0,6*pi):
\end{eulerprompt}
\eulersubheading{Membuat Grafik Fungsi Linear}
\begin{eulercomment}
\end{eulercomment}
\begin{eulerformula}
\[
g(x,y)=x+3y
\]
\end{eulerformula}
\begin{eulerprompt}
>function g(x,y) &= x+3*y;
>plot3d("g(x,y)"):
\end{eulerprompt}
\begin{eulercomment}
\end{eulercomment}
\begin{eulerformula}
\[
M(x,y)=-2x-4y
\]
\end{eulerformula}
\begin{eulerprompt}
>function M(x,y) &= -2*x-4*y;
>plot3d("M(x,y)"):
\end{eulerprompt}
\begin{eulercomment}
Rentang variabel:
\end{eulercomment}
\begin{eulerprompt}
>plot3d("M(x,y)",-2,2,0,6*pi):
\end{eulerprompt}
\begin{eulercomment}
\end{eulercomment}
\eulersubheading{Membuat Grafik Fungsi Kuadrat}
\begin{eulercomment}
\end{eulercomment}
\begin{eulerformula}
\[
Q(x,y)=x^2-y^2
\]
\end{eulerformula}
\begin{eulerprompt}
>function Q(x,y) &= x^2 - y^2;
>plot3d("Q(x,y)"):
\end{eulerprompt}
\begin{eulercomment}
\end{eulercomment}
\begin{eulerformula}
\[
P(x,y)=3x^2-4xy+2y^2
\]
\end{eulerformula}
\begin{eulerprompt}
>function P(x,y) &= 3*x^2-4*x*y+2*y^2;
>plot3d("P(x,y)"):
\end{eulerprompt}
\begin{eulercomment}
Rentang variabel:
\end{eulercomment}
\begin{eulerprompt}
>plot3d("P(x,y)",-10,10,0,9*pi):
\end{eulerprompt}
\eulersubheading{Membuat Grafik Fungsi Akar Kuadrat}
\begin{eulercomment}
\end{eulercomment}
\begin{eulerformula}
\[
A(x,y)= \sqrt{10x^2+2y^2}
\]
\end{eulerformula}
\begin{eulerprompt}
>function A(x,y) &= sqrt(10*x^2+2*y^2);
>plot3d("A"):
\end{eulerprompt}
\begin{eulercomment}
\end{eulercomment}
\begin{eulerformula}
\[
W(x,y)= \sqrt{x^2+2y^2}
\]
\end{eulerformula}
\begin{eulerprompt}
>function W(x,y) &= sqrt(x^2+2*y^2);
>plot3d("W"):
\end{eulerprompt}
\begin{eulercomment}
Rentang Variabel:
\end{eulercomment}
\begin{eulerprompt}
>plot3d("W(x,y)",-20,100,0,5*pi):
\end{eulerprompt}
\begin{eulercomment}
\end{eulercomment}
\eulersubheading{Membuat Grafik Fungsi Eksponensial}
\begin{eulercomment}
\end{eulercomment}
\begin{eulerformula}
\[
F(x,y)= 9.12^{xy}
\]
\end{eulerformula}
\begin{eulerprompt}
>function F(x,y) &= 9*12^(x*y);
>plot3d("F"):
\end{eulerprompt}
\begin{eulercomment}
\end{eulercomment}
\begin{eulerformula}
\[
H(x,y)=5.(-20)^{xy}
\]
\end{eulerformula}
\begin{eulerprompt}
>function H(x,y) &= 5*-12^(x*y);
>plot3d("H"):
\end{eulerprompt}
\begin{eulercomment}
\end{eulercomment}
\begin{eulerformula}
\[
T(x,y)=(-21).5^{xy}
\]
\end{eulerformula}
\begin{eulerprompt}
>function T(x,y) &= -21*-5^(x*y);
>plot3d("T"):
\end{eulerprompt}
\begin{eulercomment}
\end{eulercomment}
\eulersubheading{Membuat Grafik Fungsi Logaritma}
\begin{eulercomment}
\end{eulercomment}
\begin{eulerformula}
\[
B(x,y)=log(x.2y), Basis 10
\]
\end{eulerformula}
\begin{eulerprompt}
>function B(x,y) &= log(x*2*y);
>plot3d("B"):
\end{eulerprompt}
\begin{eulercomment}
\end{eulercomment}
\begin{eulerformula}
\[
C(x,y)=log(30x.5y), basis 10
\]
\end{eulerformula}
\begin{eulerprompt}
>function C(x,y) &= log(30*x*5*y);
>plot3d("C"):
\end{eulerprompt}
\eulersubheading{Membuat Grafik Fungsi Trigonometri}
\begin{eulercomment}
\end{eulercomment}
\begin{eulerformula}
\[
D(x,y)=sin(2x).cos(3y)
\]
\end{eulerformula}
\begin{eulerprompt}
>function D(x,y) &= sin(2*x)*cos(3*y);
>plot3d("D"):
\end{eulerprompt}
\begin{eulerformula}
\[
J(x,y)=sin(2x).tan(y)
\]
\end{eulerformula}
\begin{eulerprompt}
>function J(x,y) &= sin(2*x)*tan(y);
>plot3d("J"):
\end{eulerprompt}
\begin{eulerformula}
\[
G(x,y)=sec(2x).cot(5y)
\]
\end{eulerformula}
\begin{eulerprompt}
>function G(x,y) &= sec(2*x)*cot(y);
>plot3d("G"):
\end{eulerprompt}
\begin{eulercomment}
\end{eulercomment}
\eulersubheading{Membuat Grafik Fungsi Nilai Mutlak}
\begin{eulercomment}
\end{eulercomment}
\begin{eulerformula}
\[
T(x,y)=|2x^2-y^2|
\]
\end{eulerformula}
\begin{eulerprompt}
>function T(x,y) &= abs(2*x^2 - y^2);
>plot3d("T"):
\end{eulerprompt}
\begin{eulerformula}
\[
M(x,y)=|-10x^2-3y^2|
\]
\end{eulerformula}
\begin{eulerprompt}
>function M(x,y) &= abs(-10*x^2 - 3*y^2);
>plot3d("M"):
\end{eulerprompt}
\begin{eulercomment}
Rentang Variabel :
\end{eulercomment}
\begin{eulerprompt}
>plot3d("M(x,y)",-15,2,0,8*pi):
\end{eulerprompt}
\begin{eulercomment}
\end{eulercomment}
\eulersubheading{Latihan}
\begin{eulercomment}
Buatlah grafik dari fungsi berikut:

\end{eulercomment}
\begin{eulerformula}
\[
A(x,y)=x^2y+3y^2
\]
\end{eulerformula}
\begin{eulerprompt}
>function A(x,y) &= x^2*y+3*y^2;
>plot3d ("A"):
\end{eulerprompt}
\begin{eulercomment}
\end{eulercomment}
\begin{eulerformula}
\[
B(x,y)=y^2-2x^2y+4x^3+20x^2
\]
\end{eulerformula}
\begin{eulerprompt}
>function B(x,y)&= y^2-2*x^2*y+4*x^3+20*x^2;
>plot3d("B"):
\end{eulerprompt}
\begin{eulercomment}
\end{eulercomment}
\begin{eulerformula}
\[
C(x,y)=cosec(9x)-tan(2y)
\]
\end{eulerformula}
\begin{eulerprompt}
>function C(x,y)&= cosec(9*x)-tan(2*y);
>plot3d ("C"):
\end{eulerprompt}
\begin{eulercomment}
Beri rentang variabel untuk fungsi C(x,y):\\
-100,50,0,2*pi
\end{eulercomment}
\begin{eulerprompt}
>plot3d("C(x,y)",-100,50,0,2*pi): 
\end{eulerprompt}
\eulersubheading{5) Menggambar Data $x$, $y$, $z$ pada ruang Tiga Dimensi (3D)}
\begin{eulercomment}
\end{eulercomment}
\begin{eulerttcomment}
  Menggambar data pada ruang tiga dimensi (3D) adalah proses
\end{eulerttcomment}
\begin{eulercomment}
visualisasi data yang mengubah informasi dalam tiga dimensi, yaitu
panjang, lebar, dan tinggi, menjadi representasi visual yang dapat
dipahami dan dianalisis.

\end{eulercomment}
\begin{eulerttcomment}
  Tujuan dari menggambar data 3D adalah untuk membantu pemahaman dan
\end{eulerttcomment}
\begin{eulercomment}
interpretasi data yang lebih baik, terutama ketika data tersebut
memiliki komponen yang tidak dapat direpresentasikan dengan baik dalam
dua dimensi.

Sama seperti plot2d, plot3d menerima data. Untuk objek 3D, kita perlu
menyediakan matriks nilai x-, y- dan z, atau tiga fungsi atau ekspresi
fx(x,y), fy(x,y), fz(x,y).

\end{eulercomment}
\begin{eulerformula}
\[
\gamma(t,s) = (x(t,s),y(t,s),z(t,s))
\]
\end{eulerformula}
\begin{eulercomment}
Karena x,y,z adalah matriks, kita asumsikan bahwa (t,s) melalui sebuah
kotak persegi. Hasilnya, kita dapat memplot gambar persegi panjang di
ruang angkasa.

Kita dapat menggunakan bahasa matriks Euler untuk menghasilkan
koordinat secara efektif.

Dalam contoh berikut, kami menggunakan vektor nilai t dan vektor kolom
nilai s untuk membuat parameter permukaan bola. Dalam gambar kita
dapat menandai daerah, dalam kasus kita daerah kutub.

\end{eulercomment}
\eulersubheading{Contoh 1 grafik fungsi}
\begin{eulerprompt}
>t=-1:0.1:1; s=(-1:0.1:1)'; plot3d(t,s,t*s):
\end{eulerprompt}
\begin{eulercomment}
Penjelasan sintaks dari plot\\
- plot3d = membawa euler untuk mengetahui perintah apa yang harus
dilakukan\\
- (” ...”) = tempat kita untuk memasukkan perintah yang kita inginkan

\end{eulercomment}
\eulersubheading{Contoh 2}
\begin{eulercomment}
kita akan memebentuk plot dengan fungsi dibawah ini

\end{eulercomment}
\begin{eulerformula}
\[
x^2 + y^2
\]
\end{eulerformula}
\begin{eulerprompt}
>plot3d("x^2+y^2"):
\end{eulerprompt}
\begin{eulercomment}
Selanjutnya kita akan menggambar garis pada plot dengan menggunakan
grid
\end{eulercomment}
\begin{eulerprompt}
>plot3d("x^2+y^2",grid=2):
\end{eulerprompt}
\begin{eulercomment}
Jika kita ingin memodifikasi plot dengan menambahkan warna pada plot,
bisa menggunakan fillcolor.\\
Fillcolor dapat diisi dengan 1 warna yang sama atau 2 warna yang
berbeda
\end{eulercomment}
\begin{eulerprompt}
>plot3d("x^2+y^2",grid= 2,fillcolor=[blue,blue]):
\end{eulerprompt}
\eulersubheading{Contoh 3}
\begin{eulercomment}
Jika kita ingin membuat plot 3d pada fungsi

\end{eulercomment}
\begin{eulerformula}
\[
2x^2+y^3
\]
\end{eulerformula}
\begin{eulercomment}
kita bisa menggunakan perintah seperti dibawah ini
\end{eulercomment}
\begin{eulerprompt}
>plot3d("2x^2+y^3",grid=10,>hue, color=red);
>insimg()
\end{eulerprompt}
\begin{eulercomment}
Jika kita mau menebalkan warna pada gambar diatas makam bisa
menggunakan perintah
\end{eulercomment}
\begin{eulerprompt}
>plot3d("2x^2+y^3",grid=10,fillcolor=[red,red]);
>insimg()
\end{eulerprompt}
\eulersubheading{Contoh 4}
\begin{eulercomment}
Tentu saja, titik cloud juga dimungkinkan. Untuk memplot data titik
dalam ruang, kita membutuhkan tiga vektor untuk koordinat titik-titik
tersebut.

Gayanya sama seperti di plot2d dengan points=true;
\end{eulercomment}
\begin{eulerprompt}
>n=500;...
>plot3d(normal(1,n),normal(1,n),normal(1,n),points=true,style="."):
\end{eulerprompt}
\eulersubheading{Contoh Soal 5}
\begin{eulercomment}
\end{eulercomment}
\begin{eulerttcomment}
 Dalam contoh berikut, kita membuat tampilan bayangan dari bola yang
\end{eulerttcomment}
\begin{eulercomment}
terdistorsi. Koordinat biasa untuk bola adalah

\end{eulercomment}
\begin{eulerformula}
\[
\gamma(t,s) = (\cos(t)\cos(s),\sin(t)\sin(s),\cos(s))
\]
\end{eulerformula}
\begin{eulercomment}
dengan

\end{eulercomment}
\begin{eulerformula}
\[
0 \le t \le 2\pi, \quad \frac{-\pi}{2} \le s \le \frac{\pi}{2}.
\]
\end{eulerformula}
\begin{eulercomment}
Kami mendistorsi ini dengan sebuah faktor

\end{eulercomment}
\begin{eulerformula}
\[
d(t,s) = \frac{\cos(4t)+\cos(8s)}{4}
\]
\end{eulerformula}
\begin{eulerprompt}
>t=linspace(0,2pi,320); s=linspace(-pi/2,pi/2,160)';...
>d=1+0.2*(cos(4*t)+cos(8*s));...
>plot3d(cos(t)*cos(s)*d,sin(t)*cos(s)*d,sin(s)*d,hue=1,...
>light=[1,0,1],frame=0,zoom=5):
\end{eulerprompt}
\eulersubheading{6) Membuat Gambar Grafik Tiga Dimensi (3D) yang}
\eulersubheading{Bersifat Interaktif dan animasi grafik 3D}
\begin{eulercomment}
Membuat gambar grafik tiga dimensi (3D) yang bersifat interaktif
adalah proses menciptakan visualisasi tiga dimensi yang memungkinkan
pengguna berinteraksi dengan objek-objek 3D. Interaktivitas dalam
gambar 3D memungkinkan pengguna untuk melakukan tindakan seperti
mengubah sudut pandang, memindahkan objek, atau berinteraksi dengan
elemen-elemen dalam adegan 3D. Sedangkan animasi grafik 3D dapat
mencakup pergerakan, tetapi juga dapat berarti perubahan dalam
tampilan atau atribut objek tanpa pergerakan fisik yang mencolok.

Interaksi user dimungkinkan dengan parameter \textgreater{}user. dengan perintah
\textgreater{}user kita dapat menekan tombol berikut.

- kiri, kanan, atas, bawah: memutar sudut pandang\\
- +,-: memperbesar atau memperkecil\\
- a: menghasilkan anaglyph (lihat di bawah)\\
- l : tombol nyalakan sumber cahaya (lihat dibawah)\\
- spasi: reset ke default\\
- kembali: akhiri interaksi
\end{eulercomment}
\begin{eulerprompt}
>plot3d("exp(-x^2+y^2)",>user,...
>title="Coba gerakkan"): 
>plot3d("exp(x^2+y^2)",>user,...
>title="Coba gerakkan)"):
\end{eulerprompt}
\eulersubheading{Animasi 3D}
\begin{eulerprompt}
>function testplot () := plot3d("x^2+y^3");...
>rotate("testplot"); testplot():
\end{eulerprompt}
\begin{eulercomment}
Fungsi rotate yaitu untuk memutar plot.\\
Fungsi ini akan membuat animasi plot 3D dari fungsi\\
\end{eulercomment}
\begin{eulerformula}
\[
x^2 + y^3
\]
\end{eulerformula}
\begin{eulercomment}
yang berputar di sekitar sumbu z dari sudut 0 hingga 360 derajat
\end{eulercomment}
\begin{eulerprompt}
>plot3d("exp(-(x^2+y^2)/5)",r=10,n=80,fscale=8,scale=1.2,frame=3,>user):
\end{eulerprompt}
\begin{eulercomment}
Ada beberapa parameter untuk menskalakan fungsi atau mengubah tampilan
grafik.

fscale: menskalakan ke nilai fungsi (defaultnya adalah \textless{}fscale).\\
scale: angka atau vektor 1x2 untuk diskalakan ke arah x dan y.\\
frame: jenis bingkai (default 1).
\end{eulercomment}
\begin{eulerprompt}
>plot3d("x^2+y",distance=10,zoom=5,angle=0,height=5):
\end{eulerprompt}
\begin{eulercomment}
Tampilan dapat diubah dengan berbagai cara.

- distance: jarak pandang ke plot.\\
- zoom: nilai zoom.\\
- angle: sudut terhadap sumbu y negatif dalam radian.\\
- height: ketinggian tampilan dalam radian.
\end{eulercomment}
\begin{eulerprompt}
>plot3d("x^4+y^2",a=0,b=1,c=-1,d=1,angle=-20°,height=20°,...
>center=[0,0,1],zoom=5):
\end{eulerprompt}
\begin{eulercomment}
Plot selalu terlihat berada di tengah kubus plot. Kita dapat
memindahkan bagian tengah dengan center parameter.
\end{eulercomment}
\begin{eulerprompt}
>plot3d("x^2+1",a=-1,b=1,rotate=true,grid=5):
\end{eulerprompt}
\begin{eulercomment}
Parameter memutar memutar fungsi dalam x di sekitar sumbu x.

- rotate=1: Menggunakan sumbu x\\
- rotate=2: Menggunakan sumbu z
\end{eulercomment}
\eulersubheading{7) Menggambar Fungsi Parametrik Tiga Dimensi (3D)}
\begin{eulercomment}
\end{eulercomment}
\eulersubheading{Pengertian}
\begin{eulercomment}
Fungsi parametrik adalah jenis fungsi matematika yang menggambarkan
hubungan antara dua atau lebih variabel, di mana setiap variabel
dinyatakan sebagai fungsi dari satu atau lebih parameter. Fungsi
parametrik digunakan untuk menggambarkan kurva, lintasan, atau
hubungan antara berbagai variabel yang bergantung pada
parameter-parameter tertentu.

Fungsi parametrik merupakan salah satu cara mendefinisikan kurva atau
permukaan dalam ruang 2D atau 3D menggunakan satu atau lebih parameter
independen.

- Dalam 2D, kurva dinyatakan sebagai x(t) dan y(t), di mana adalah t
adalah parameter yang mengontrol posisi sepanjang kurva.\\
- Dalam 3D, kita menggunakan tiga persamaan parametrik untuk
mendeskripsikan posisi x,y,z sebagai fungsi dari parameter t.Fungsi
ini ditulis sebagai:\\
x=f(t),\\
y=g(t),\\
z=h(t),

\end{eulercomment}
\eulersubheading{Contoh Soal}
\begin{eulerprompt}
>plot3d("cos(x)*cos(y)","sin(x)*cos(y)","sin(y)", a=0,b=2*pi,c=pi/2,d=-pi/2,...
>>hue,color=red,light=[0,1,0],<frame,...
>n=90,grid=[25,50],wirecolor=black,zoom=4):
>aspect(16/9); allwindow; ...
>x:=linspace(0,2*pi,100); y:=(-1:0.1:1)'; ...
>plot3d(sin(x)*(y/2*sin(x/2)),cos(x)*(y/2*sin(x/2)),y/2*cos(x/2), ...
><frame,hue=2,max=0.5,scale=1.5):
>aspect(16/9); allwindow;...
>x:=linspace(0,2*pi,100); y:=(-1:0.1:1)';...
>plot3d(sin(x)*(y/2*sin(x/2)),cos(x)*(y/2*sin(x/2)),y/2*cos(x/2),...
>>lines,<frame,xmin=0,xmax=10,n=10,>user):
>reset;...
>S:=normal(10,1250); plot3d(S[3],S[6],S[9],>points,style="."):
>S:=normal(10,1250); T:=cumsum(normal(10,1250));...
>plot3d(T[2],T[5],T[8],>wire,...
>linewidth=2,>anaglyph,zoom=3):
>P=cumsum(normal(5,75));...
>plot3d(P[3],P[4],P[5],>anaglyph,>wire):
\end{eulerprompt}
\eulersubheading{8) Menggambar Fungsi Implisit Tiga Dimensi (3D)}
\begin{eulercomment}
Fungsi implisit (implicit function) adalah fungsi yang memuat lebih
dari satu variabel, berjenis variabel bebas dan variabel terikat yang
berada dalam satu ruas sehingga tidak bisa dipisahkan pada ruas yang
berbeda.

\end{eulercomment}
\begin{eulerformula}
\[
F(x,y,z)=0
\]
\end{eulerformula}
\begin{eulercomment}
(1 persamaan dan 3 variabel), terdiri dari 2 variabel bebas dan 1
terikat

Misalnya, F(x, y, z) = x\textasciicircum{}2 + y\textasciicircum{}2 + z\textasciicircum{}2 - 1 = 0 adalah persamaan
implisit yang menggambarkan bola dengan jari-jari 1 dan pusat di
(0,0,0).

Plot Implisit

Ada juga plot implisit dalam tiga dimensi. Euler menghasilkan
pemotongan melalui objek. Fitur plot3d mencakup plot implisit. Plot
ini menunjukkan himpunan nol suatu fungsi dalam tiga variabel.\\
Solusi dari



\end{eulercomment}
\begin{eulerformula}
\[
f(x,y,z) = 0
\]
\end{eulerformula}
\begin{eulercomment}
dapat divisualisasikan dalam potongan yang sejajar dengan bidang x-y,
x-z, z-x dan y-z.

- implisit=1: dipotong sejajar bidang y-z\\
- implisit=2: dipotong sejajar dengan bidang x-z\\
- implisit=3: dipotong sejajar dengan bidang z-x (yang berarti
pemotongan dilakukan dengan mempertahankan nilai y konstan)\\
- implisit=4: dipotong sejajar bidang x-y

Tambahkan nilai-nilai ini, jika Anda mau. Dalam contoh kita memplot

\end{eulercomment}
\begin{eulerformula}
\[
M = \{ (x,y,z) : x^2+y^3+zy=1 \}
\]
\end{eulerformula}
\begin{eulercomment}
\end{eulercomment}
\eulersubheading{}
\eulersubheading{Contoh Fungsi Implisit}
\begin{eulerprompt}
>plot3d("x^2+y^3+z*y-1",r=7,implicit=4):
>plot3d("2*x^2 + 3*y^2 + z^2 - 25",r=8,implicit=2):
>plot3d("4*x^3 + 3*y^4 + 6*z^2 - 10",r=3,implicit=3):
>plot3d("x^5 + 5*y^3 + 3*z^2 - 5*x - 7*y - 5*z + 10",r=5,implicit=2):
>plot3d("x^2 + y^2 - z^2",r=5,implicit=3):
>plot3d("x^3 + 2*y^2 +3*z^3-4",r=5,implicit=3):
>plot3d("x^2+y^2+z^2+2*x*y+4*y*z+8*z*x-20",r=5,implicit=3):
>c=1; d=1;
>plot3d("((x^2+y^2-c^2)^2+(z^2-1)^2)*((y^2+z^2-c^2)^2+(x^2-1)^2)*((z^2+x^2-c^2)^2+(y^2-1)^2)-d",r=2,<frame,>implicit,>user):
>c=1; d=1;
>plot3d("((x^2+y^2+c^2)^2+(z^2-1)^2)*((y^2+z^2-c^2)^2+(x^2-1)^2)*((z^2+x^2-c^2)^2+(y^2-1)^2)-d",r=2,<frame,>implicit,>user):
>plot3d("x^3+y^5+5*x*z+z^3",>implicit,r=3,zoom=2):
>plot3d("x^2+y^2+4*x*z+z^3-2",>implicit,r=2,zoom=2.5):
>plot3d("x^2*y^2+x^3+y^3*x",>implicit,r=5,zoom=2.5):
>plot3d("x*y-z^2+2*x*y*z-0",>implicit,r=5,zoom=2.5):
\end{eulerprompt}
\eulersubheading{Latihan soal}
\begin{eulercomment}
Gambarlah Fungsi implisit berikut dalam 3D

\end{eulercomment}
\begin{eulerformula}
\[
f(x,y,z)=x^2+y^2-z^2-1
\]
\end{eulerformula}
\begin{eulerprompt}
>plot3d("x^2+y^2-z^2-1",r=8,implicit=3):
\end{eulerprompt}
\begin{eulercomment}
Gambarlah fungsi 3D dari fungsi implisit berikut ini\\
\end{eulercomment}
\begin{eulerformula}
\[
f(x,y,z)=xy+x^3y^2+xz^3-9=0
\]
\end{eulerformula}
\begin{eulercomment}
dengan r=4
\end{eulercomment}
\begin{eulerprompt}
>plot3d("x*y+x^3*y^2+x*z^3-9", r=4, implicit=3):
\end{eulerprompt}
\eulersubheading{9) Mengatur tampilan, warna dan sudut pandang gambar permukaan}
\eulersubheading{Tiga Dimensi (3D) Dan Menampilkan kontur dan bidang kontur}
\eulersubheading{permukaan Tiga Dimensi(3D)}
\begin{eulercomment}
Untuk plot, Euler menambahkan garis grid. Sebagai gantinya
dimungkinkan untuk menggunakan garis level dan rona satu warna atau
rona berwarna spektral. Euler dapat menggambar tinggi fungsi pada plot
dengan bayangan. Di semua plot 3D, Euler dapat menghasilkan anaglyph
merah/sian.

-\textgreater{} hue: Menyalakan bayangan cahaya alih-alih kabel.\\
-\textgreater{} kontur: Memplot garis kontur otomatis pada plot.\\
- level=... (atau level): Sebuah vektor nilai untuk garis kontur.

Standarnya adalah level="auto", yang menghitung beberapa garis level
secara otomatis. Seperti yang Anda lihat di plot, level sebenarnya
adalah rentang level.

Gaya default dapat diubah. Untuk plot kontur berikut, kami menggunakan
grid yang lebih halus untuk 100x100 poin, skala fungsi dan plot, dan
menggunakan sudut pandang yang berbeda.
\end{eulercomment}
\begin{eulerprompt}
>plot3d("exp(-x^2-y^2)",r=2,n=100,level="thin", ...
> >contour,>spectral,fscale=1,scale=1.1,angle=45°,height=20°):
>plot3d("exp(x*y)",angle=100°,>contour,color=green):
\end{eulerprompt}
\begin{eulercomment}
Bayangan default menggunakan warna abu-abu. Tetapi rentang warna
spektral juga tersedia.

-\textgreater{} spektral: Menggunakan skema spektral default\\
- color=...: Menggunakan warna khusus atau skema spektral

Untuk plot berikut, kami menggunakan skema spektral default dan
menambah jumlah titik untuk mendapatkan tampilan yang sangat halus.
\end{eulercomment}
\begin{eulerprompt}
>plot3d("x^2+y^2",>spectral,>contour,n=100):
\end{eulerprompt}
\begin{eulercomment}
Alih-alih garis level otomatis, kita juga dapat mengatur nilai garis
level. Ini akan menghasilkan garis level tipis alih-alih rentang
level.
\end{eulercomment}
\begin{eulerprompt}
>plot3d("x^2-y^2",0,1,0,1,angle=220°,level=-1:0.2:1,color=redgreen):
\end{eulerprompt}
\begin{eulercomment}
Dalam plot berikut, kami menggunakan dua pita level yang sangat luas
dari -0,1 hingga 1, dan dari 0,9 hingga 1. Ini dimasukkan sebagai
matriks dengan batas level sebagai kolom.

Selain itu, kami melapisi kisi dengan 10 interval di setiap arah.
\end{eulercomment}
\begin{eulerprompt}
>plot3d("x^2+y^3",level=[-0.1,0.9;0,1], ...
>  >spectral,angle=30°,grid=10,contourcolor=gray):
\end{eulerprompt}
\begin{eulercomment}
Dalam contoh berikut, kami memplot himpunan, di mana

\end{eulercomment}
\begin{eulerformula}
\[
f(x,y) = x^y-y^x = 0
\]
\end{eulerformula}
\begin{eulercomment}
Kami menggunakan satu garis tipis untuk garis level.
\end{eulercomment}
\begin{eulerprompt}
>plot3d("x^y-y^x",level=0,a=0,b=6,c=0,d=6,contourcolor=red,n=100):
\end{eulerprompt}
\begin{eulercomment}
Dimungkinkan untuk menunjukkan bidang kontur di bawah plot. Warna dan
jarak ke plot dapat ditentukan.
\end{eulercomment}
\begin{eulerprompt}
>plot3d("x^2+y^4",>cp,cpcolor=green,cpdelta=0.2):
\end{eulerprompt}
\begin{eulercomment}
Berikut adalah beberapa gaya lagi. Kami selalu mematikan frame, dan
menggunakan berbagai skema warna untuk plot dan grid.
\end{eulercomment}
\begin{eulerprompt}
>figure(2,2); ...
>expr="y^3-x^2"; ...
>figure(1);  ...
>  plot3d(expr,<frame,>cp,cpcolor=spectral); ...
>figure(2);  ...
>  plot3d(expr,<frame,>spectral,grid=10,cp=2); ...
>figure(3);  ...
>  plot3d(expr,<frame,>contour,color=gray,nc=5,cp=3,cpcolor=greenred); ...
>figure(4);  ...
>  plot3d(expr,<frame,>hue,grid=10,>transparent,>cp,cpcolor=gray); ...
>figure(0):
\end{eulerprompt}
\begin{eulercomment}
Ada beberapa skema spektral lainnya, bernomor dari 1 hingga 9. Tetapi
Anda juga dapat menggunakan warna=nilai, di mana nilai

- spektral: untuk rentang dari biru ke merah\\
- putih: untuk rentang yang lebih redup\\
- kuningbiru,ungu hijau,birukuning,hijaumerah\\
- birukuning, hijau ungu, kuning biru, merah hijau
\end{eulercomment}
\begin{eulerprompt}
>figure(3,3); ...
>for i=1:9;  ...
>  figure(i); plot3d("x^2+y^2",spectral=i,>contour,>cp,<frame,zoom=4);  ...
>end; ...
>figure(0):
\end{eulerprompt}
\begin{eulercomment}
Sumber cahaya dapat diubah dengan l dan tombol kursor selama interaksi
pengguna. Itu juga dapat diatur dengan parameter.

- cahaya: arah untuk cahaya\\
- amb: cahaya sekitar antara 0 dan 1

Perhatikan bahwa program tidak membuat perbedaan antara sisi plot.
Tidak ada bayangan. Untuk ini, Anda perlu Povray.
\end{eulercomment}
\begin{eulerprompt}
>plot3d("-x^2-y^2", ...
>  hue=true,light=[0,1,1],amb=0,user=true, ...
>  title="Press l and cursor keys (return to exit)"):
\end{eulerprompt}
\begin{eulercomment}
Parameter warna mengubah warna permukaan. Warna garis level juga dapat
diubah.
\end{eulercomment}
\begin{eulerprompt}
>plot3d("-x^2-y^2",color=rgb(0.2,0.2,0),hue=true,frame=false, ...
>  zoom=3,contourcolor=red,level=-2:0.1:1,dl=0.01):
\end{eulerprompt}
\begin{eulercomment}
Warna 0 memberikan efek pelangi khusus.
\end{eulercomment}
\begin{eulerprompt}
>plot3d("x^2/(x^2+y^2+1)",color=0,hue=true,grid=10):
\end{eulerprompt}
\begin{eulercomment}
Permukaannya juga bisa transparan.
\end{eulercomment}
\begin{eulerprompt}
>plot3d("x^2+y^2",>transparent,grid=10,wirecolor=red):
\end{eulerprompt}
\eulersubheading{10) Menggambar Diagram Batang Tiga Dimensi}
\begin{eulercomment}
Bar plots/plot batang juga dimungkinkan. Untuk itu, kita harus
menyediakannya

- x: vektor baris dengan n+1 elemen\\
- y: vektor kolom dengan n+1 elemen\\
- z: matriks nilai nxn.

z bisa lebih besar, tetapi hanya nilai nxn yang akan digunakan.

Dalam contoh ini, pertama-tama kita menghitung nilainya. Kemudian kita
sesuaikan x dan y, sehingga vektor-vektornya berpusat pada nilai yang
digunakan.
\end{eulercomment}
\begin{eulerprompt}
>x=-1:0.1:1; y=x'; z=x^2+y^2; ...
>xa=(x|1.1)-0.05; ya=(y_1.1)-0.05; ...
>plot3d(xa,ya,z,bar=true):
>x=-0.01:0.1:1; y=x'; z=x+2/3*y; ...
>xa=(x|1.1)-0.05; ya=(y_1.1)-0.05; ...
>plot3d(xa,ya,z,bar=true):
>x=-0.01:0.1:1; y=x'; z=1/2*x+1/2*y; ...
>xa=(x|1.1); ya=(y_1.1); ...
>plot3d(xa,ya,z,bar=true):
\end{eulerprompt}
\begin{eulercomment}
Dimungkinkan untuk membagi plot suatu permukaan menjadi dua bagian
atau lebih.
\end{eulercomment}
\begin{eulerprompt}
>x=-1:0.1:1; y=x'; z=1/10*x+1/10*y; d=zeros(size(x)); ...
>plot3d(x,y,z,disconnect=2:2:5):
>x=-1:0.1:1; y=x'; z=x+y; d=zeros(size(x)); ...
>plot3d(x,y,z,disconnect=2:2:20):
\end{eulerprompt}
\begin{eulercomment}
Jika memuat atau menghasilkan matriks data M dari file dan perlu
memplotnya dalam 3D, Anda dapat menskalakan matriks ke [-1,1] dengan
skala(M), atau menskalakan matriks dengan \textgreater{}zscale. Hal ini dapat
dikombinasikan dengan faktor penskalaan individual yang diterapkan
sebagai tambahan.
\end{eulercomment}
\begin{eulerprompt}
>i=1:20; j=i'; ...
>plot3d(i*j^2+100*normal(20,20),>zscale,scale=[1,1,1.5],angle=-40°,zoom=1.8):
>Z=intrandom(5,100,6); v=zeros(5,6); ...
>loop 1 to 5; v[#]=getmultiplicities(1:6,Z[#]); end; ...
>columnsplot3d(v',scols=1:5,ccols=[1:5]):
>Z=intrandom(6,100,6); v=zeros(6,2); ...
>loop 1 to 6; v[#]=getmultiplicities(1:2,Z[#]); end; ...
>columnsplot3d(v',scols=1:6,ccols=[1:6]):
>Z=intrandom(7,1000,6); v=zeros(7,1); ...
>loop 1 to 7; v[#]=getmultiplicities(1:1,Z[#]); end; ...
>columnsplot3d(v',scols=1:7,ccols=[1:7]):
\end{eulerprompt}
\eulersubheading{11) Menggambar Permukaan Benda Putar}
\begin{eulerprompt}
>plot2d("(x^2+y^2-1)^3-x^2*y^3",r=1.3, ...
>style="#",color=red,<outline, ...
>level=[-2;0],n=100):
>ekspresi &= (x^2+y^2-1)^3-x^2*y^3; $ekspresi
\end{eulerprompt}
\begin{eulercomment}
Kami ingin memutar kurva jantung di sekitar sumbu y. Berikut adalah
ungkapan, yang mendefinisikan hati:

\end{eulercomment}
\begin{eulerformula}
\[
f(x,y)=(x^2+y^2-1)^3-x^2.y^3.
\]
\end{eulerformula}
\begin{eulercomment}
Selanjutnya kita atur

\end{eulercomment}
\begin{eulerformula}
\[
x=r.cos(a),\quad y=r.sin(a).
\]
\end{eulerformula}
\begin{eulerprompt}
>function fr(r,a) &= ekspresi with [x=r*cos(a),y=r*sin(a)] | trigreduce; $fr(r,a)
\end{eulerprompt}
\begin{eulercomment}
Hal ini memungkinkan untuk mendefinisikan fungsi numerik, yang
memecahkan r, jika a diberikan. Dengan fungsi itu kita dapat memplot
jantung yang diputar sebagai permukaan parametrik.
\end{eulercomment}
\begin{eulerprompt}
>function map f(a) := bisect("fr",0,2;a); ...
>t=linspace(-pi/2,pi/2,100); r=f(t);  ...
>s=linspace(pi,2pi,100)'; ...
>plot3d(r*cos(t)*sin(s),r*cos(t)*cos(s),r*sin(t), ...
>>hue,<frame,color=red,zoom=4,amb=0,max=0.7,grid=12,height=50°):
\end{eulerprompt}
\begin{eulercomment}
Berikut ini adalah plot 3D dari gambar di atas yang diputar di sekitar
sumbu z. Kami mendefinisikan fungsi, yang menggambarkan objek.
\end{eulercomment}
\begin{eulerprompt}
>function f(x,y,z) ...
\end{eulerprompt}
\begin{eulerudf}
  r=x^2+y^2;
  return (r+z^2-1)^3-r*z^3;
   endfunction
\end{eulerudf}
\begin{eulerprompt}
>plot3d("f(x,y,z)", ...
>xmin=0,xmax=1.2,ymin=-1.2,ymax=1.2,zmin=-1.2,zmax=1.4, ...
>implicit=1,angle=-30°,zoom=2.5,n=[10,60,60],>anaglyph):
\end{eulerprompt}
\eulersubheading{12) Menggambar Grafik 3D dengan Povray di EMT}
\begin{eulercomment}
Menggambar Povray Dengan bantuan file Euler povray.e, Euler dapat
menghasilkan file Povray. Hasilnya sangat bagus untuk dilihat.

Untuk dapat menjalankan sintaks dalam povray perlu menginstal Povray
(32bit atau 64bit) dari http://www.povray.org/, dan meletakkan
sub-direktori "bin" dari Povray ke pathway, atau mengatur variabel
"defaultpovray" dengan path lengkap yang menunjuk ke "pvengine.exe".

Interface Povray dari Euler menghasilkan file Povray di direktori home
pengguna, dan memanggil Povray untuk mengurai file-file ini. Nama file
default adalah current.pov, dan direktori default adalah eulerhome(),
biasanya c:\textbackslash{}Users\textbackslash{}Username\textbackslash{}Euler. Povray menghasilkan file PNG, yang
dapat dimuat oleh Euler ke dalam buku catatan. Untuk membersihkan
file-file ini, gunakan povclear().

Sintaks yang digunakan untuk menjalankan povray adalah pov3d. Fungsi
pov3d memiliki komponen yang sama dengan plot3d. Ini dapat
menghasilkan grafik fungsi f(x,y), atau permukaan dengan koordinat
X,Y,Z dalam matriks, termasuk garis level opsional. Fungsi ini memulai
raytracer secara otomatis, dan memuat gambar ke dalam notebook Euler.

Selain pov3d(), ada banyak fungsi yang menghasilkan objek Povray.
Fungsi-fungsi ini mengembalikan string, yang berisi kode Povray untuk
objek. Untuk menggunakan fungsi ini, mulai file Povray dengan
povstart(). Kemudian gunakan writeln(...) untuk menulis objek ke file
gambar. Terakhir, akhiri file dengan povend(). Secara default,
raytracer akan dimulai, dan PNG akan dimasukkan ke dalam notebook
Euler.

Fungsi objek memiliki parameter yang disebut "look", yang membutuhkan
string dengan kode Povray untuk tekstur dan hasil akhir objek. Fungsi
povlook() dapat digunakan untuk menghasilkan string ini. Ini memiliki
parameter untuk warna, transparansi, Phong Shading dll.

Lingkup Povray memiliki sistem koordinat lain. Interface ini
menerjemahkan semua koordinat ke sistem Povray. Jadi Anda dapat terus
berpikir dalam sistem koordinat Euler dengan z menunjuk vertikal ke
atas, dan x,y,z sumbu dalam arti tangan kanan.\\
Anda perlu memuat file povray.
\end{eulercomment}
\begin{eulerprompt}
>load povray;
\end{eulerprompt}
\begin{eulercomment}
Pastikan, direktori bin Povray ada di path. Jika tidak, edit variabel
berikut sehingga berisi path ke povray yang dapat dieksekusi.
\end{eulercomment}
\begin{eulerprompt}
>defaultpovray="C:\(\backslash\)Program Files\(\backslash\)POV-Ray\(\backslash\)v3.7\(\backslash\)bin\(\backslash\)pvengine.exe"
\end{eulerprompt}
\begin{euleroutput}
  C:\(\backslash\)Program Files\(\backslash\)POV-Ray\(\backslash\)v3.7\(\backslash\)bin\(\backslash\)pvengine.exe
\end{euleroutput}
\eulersubheading{Contoh Penggunaan}
\begin{eulercomment}
Akan diberikan contoh sederhana penggunaan povray pada EMT\\
Perintah berikut menghasilkan file povray di direktori pengguna dan
menjalankan Povray untuk ray tracing file ini.

Jika memulai perintah berikut, GUI Povray akan terbuka, menjalankan
file, dan menutup secara otomatis. Karena alasan keamanan, akan
ditanya, apakah ingin mengizinkan file exe untuk dijalankan. Agar
pertanyaan tersebut tidak muncul lagi bisa dipilih batal.
\end{eulercomment}
\begin{eulerprompt}
>pov3d("x^2+y^2",zoom=4);
\end{eulerprompt}
\begin{eulercomment}
hasil visualisasi fungsi dapat dibuat menjadi transparan dan
menambahkan hasil akhir lainnya.
\end{eulercomment}
\begin{eulerprompt}
> pov3d("(x^2+y^3)",axiscolor=green,angle=30°, ...
>  look=povlook(yellow,0.2),level=-1:0.5:1,zoom=3);
>pov3d("((x-1)^2+(y+1)^2)*((x+1)^2+y^2)/40",r=1.5, ...
>  angle=120°,level=1/40,dlevel=0.005,light=[-1,1,1],height=45°,n=50, ...
>  <fscale,zoom=3.8);
\end{eulerprompt}
\eulersubheading{Object Povray}
\begin{eulercomment}
Contoh-contoh di atas tadi merupakan visualisasi permukaan fungsi
dengan menggunakan sintaks pov3d. Untuk menghasilkan objek dalam
povray perlu ditulis menjadi file povray.

Untuk menghasilkan output dimulai dengan povstart()
\end{eulercomment}
\begin{eulerprompt}
>load povray; ...
>defaultpovray="C:\(\backslash\)Program Files\(\backslash\)POV-Ray\(\backslash\)v3.7\(\backslash\)bin\(\backslash\)pvengine.exe"
\end{eulerprompt}
\begin{euleroutput}
  C:\(\backslash\)Program Files\(\backslash\)POV-Ray\(\backslash\)v3.7\(\backslash\)bin\(\backslash\)pvengine.exe
\end{euleroutput}
\begin{eulerprompt}
>povstart(zoom=3.5)
>c1=povcylinder(-povx,povx,1,povlook(orange)); ...
>c2=povcylinder(-povy,povy,1,povlook(yellow)); ...
>c3=povcylinder(-povz,povz,1,povlook(lightblue));
\end{eulerprompt}
\begin{eulercomment}
Di atas telah didefinisikan tiga silinder yang disimpan dalam string\\
di Euler. Fungsi povx(), povy(), dll. hanya mengembalikan vektor\\
[1,0,0] yang dapat digunakan sebagai gantinya.
\end{eulercomment}
\begin{eulerprompt}
>c1
\end{eulerprompt}
\begin{euleroutput}
  cylinder \{ <-1,0,0>, <1,0,0>, 1
   texture \{ pigment \{ color rgb <0.941176,0.509804,0.392157> \}  \} 
   finish \{ ambient 0.2 \} 
   \}
\end{euleroutput}
\begin{eulercomment}
Akan ditambahkan tekstur ke objek dengan tiga warna berbeda yaitu
orange, yellow, dan lightblue.\\
Untuk menamahkan tekstur ini dapat digunakan sintaks povlook(), yang
mengembalikan string dengan kode Povray yang relevan. Selain
menambahkan warna, ditambahkan juga transparansi dan cahaya.
\end{eulercomment}
\begin{eulerprompt}
>povlook(rgb(0.1,0.2,0.3),0.1,0.5)
\end{eulerprompt}
\begin{euleroutput}
   texture \{ pigment \{ color rgbf <0.101961,0.2,0.301961,0.1> \}  \} 
   finish \{ ambient 0.5 \} 
  
\end{euleroutput}
\begin{eulerprompt}
>writeln(povintersection([c1,c2,c3]));
>povend;
\end{eulerprompt}
\eulersubheading{Contoh Lain}
\begin{eulercomment}
Akan ditampilkan fungsi untuk membuat sebuah donat
\end{eulercomment}
\begin{eulerprompt}
>povstart(angle=0,height=45°); //height untuk menampilkan fungsi dengan suatiu derajat tertentu 
>function povdonat (r1,r2,look="") := "torus \{"+r1+","+r2+look+"\}"; //fungsi untuk menampilkan sebuah donat
>writeln(povobject(povdonat(1,0.5),povlook(lightblue,>phong),xrotate(90°)));
>povend();
\end{eulerprompt}
\eulersubheading{13) Menggambar Grafik Tiga Dimensi alam modus anaglif}
\begin{eulercomment}
Untuk menghasilkan anaglyph untuk kacamata merah/sian, Povray harus
berjalan dua kali dari posisi kamera yang berbeda. Ini menghasilkan
dua file Povray dan dua file PNG, yang dimuat dengan fungsi
loadanaglyph().

Tentu saja, Anda memerlukan kacamata merah/sian untuk melihat contoh
berikut dengan benar.

Fungsi pov3d() memiliki sakelar sederhana untuk menghasilkan
anaglyphs.
\end{eulercomment}
\begin{eulerprompt}
>pov3d("-exp(-x^2-y^2)/2",r=2,height=45°,>anaglyph, ...
>  center=[0,0,0.5],zoom=3.5);
\end{eulerprompt}
\begin{eulercomment}
Jika Anda membuat adegan dengan objek, Anda perlu menempatkan generasi
adegan ke dalam fungsi, dan menjalankannya dua kali dengan nilai yang
berbeda untuk parameter anaglyph.
\end{eulercomment}
\begin{eulerprompt}
>function myscene ...
\end{eulerprompt}
\begin{eulerudf}
    s=povsphere(povc,1);
    cl=povcylinder(-povz,povz,0.5);
    clx=povobject(cl,rotate=xrotate(90°));
    cly=povobject(cl,rotate=yrotate(90°));
    c=povbox([-1,-1,0],1);
    un=povunion([cl,clx,cly,c]);
    obj=povdifference(s,un,povlook(red));
    writeln(obj);
    writeAxes();
  endfunction
\end{eulerudf}
\begin{eulercomment}
Fungsi povanaglyph() melakukan semua ini. Parameternya seperti di
povstart() dan povend() digabungkan.
\end{eulercomment}
\begin{eulerprompt}
>povanaglyph("myscene",zoom=4.5);
\end{eulerprompt}
\eulersubheading{14) Fungsi Implisit menggunakan Povray}
\begin{eulercomment}
Povray dapat memplot himpunan di mana f(x,y,z)=0, seperti parameter
implisit di plot3d. Namun hasilnya terlihat jauh lebih baik.

Sintaks untuk fungsinya sedikit berbeda. Anda tidak dapat menggunakan
keluaran ekspresi Maxima atau Euler.

\end{eulercomment}
\begin{eulerformula}
\[
((x^2+y^2-c^2)^2+(z^2-1)^2)*((y^2+z^2-c^2)^2+(x^2-1)^2)*((z^2+x^2-c^2)^2+(y^2-1)^2)=d
\]
\end{eulerformula}
\begin{eulerprompt}
>load povray;
>defaultpovray="C:\(\backslash\)Program Files\(\backslash\)POV-Ray\(\backslash\)v3.7\(\backslash\)bin\(\backslash\)pvengine.exe"
\end{eulerprompt}
\begin{euleroutput}
  C:\(\backslash\)Program Files\(\backslash\)POV-Ray\(\backslash\)v3.7\(\backslash\)bin\(\backslash\)pvengine.exe
\end{euleroutput}
\begin{eulerprompt}
>povstart(angle=25°,height=10°);
>writeln(povsurface("pow(x,2)+pow(y,2)*pow(z,2)-1",povlook(blue),povbox(-2,2,"")));
>povend();
>load povray;
>defaultpovray="C:\(\backslash\)Program Files\(\backslash\)POV-Ray\(\backslash\)v3.7\(\backslash\)bin\(\backslash\)pvengine.exe"
\end{eulerprompt}
\begin{euleroutput}
  C:\(\backslash\)Program Files\(\backslash\)POV-Ray\(\backslash\)v3.7\(\backslash\)bin\(\backslash\)pvengine.exe
\end{euleroutput}
\begin{eulerprompt}
>povstart(angle=70°,height=50°,zoom=4);
>writeln(povsurface("pow(x,2)*y-pow(y,3)-pow(z,2)",povlook(green))); ...
>writeAxes(); ...
>povend();
>load povray;
>defaultpovray="C:\(\backslash\)Program Files\(\backslash\)POV-Ray\(\backslash\)v3.7\(\backslash\)bin\(\backslash\)pvengine.exe"
\end{eulerprompt}
\begin{euleroutput}
  C:\(\backslash\)Program Files\(\backslash\)POV-Ray\(\backslash\)v3.7\(\backslash\)bin\(\backslash\)pvengine.exe
\end{euleroutput}
\begin{eulerprompt}
>povstart(angle=70°,height=30°);
>writeln(povsurface("pow(x,2)+pow(y,2)*pow(z,2)-1",povlook(red),povbox(-2,2,"")));
>povend();
\end{eulerprompt}
\eulersubheading{Contoh lain}
\begin{eulerprompt}
>povstart(angle=45, height=100);
>defaultpovray="C:\(\backslash\)Program Files\(\backslash\)POV-Ray\(\backslash\)v3.7\(\backslash\)bin\(\backslash\)pvengine.exe"
\end{eulerprompt}
\begin{euleroutput}
  C:\(\backslash\)Program Files\(\backslash\)POV-Ray\(\backslash\)v3.7\(\backslash\)bin\(\backslash\)pvengine.exe
\end{euleroutput}
\begin{eulerprompt}
>writeln(povsurface("pow(x,2)+pow(y,3)*pow(z,2)-1", povlook(blue),povbox(-25,4,""))); 
>povend(); 
\end{eulerprompt}
\eulersubheading{15) Menggambar Titik pada ruang Tiga Dimensi (3D)}
\begin{eulercomment}
Alih-alih fungsi, kita dapat memplot dengan koordinat. Seperti pada plot3d, kita membutuhkan tiga matriks untuk mendefinisikan objek.

Dalam contoh kita memutar fungsi di sekitar sumbu z.
\end{eulercomment}
\begin{eulerprompt}
>function f(x) := x^3-x+1; ...
>x=-1:0.01:1; t=linspace(0,2pi,8)'; ...
>Z=x; X=cos(t)*f(x); Y=sin(t)*f(x); ...
>pov3d(X,Y,Z,angle=40°,height=20°,axis=0,zoom=4,light=[10,-5,5]);
\end{eulerprompt}
\begin{eulercomment}
Dalam contoh berikut, kami memplot gelombang teredam. Kami
menghasilkan gelombang dengan bahasa matriks Euler.

Kami juga menunjukkan, bagaimana objek tambahan dapat ditambahkan ke
adegan pov3d. Untuk pembuatan objek, lihat contoh berikut. Perhatikan
bahwa plot3d menskalakan plot, sehingga cocok dengan kubus satuan.
\end{eulercomment}
\begin{eulerprompt}
>r=linspace(0,1,80); phi=linspace(0,2pi,80)'; ...
>x=r*cos(phi); y=r*sin(phi); z=exp(-5*r)*cos(8*pi*r)/3;  ...
>pov3d(x,y,z,zoom=5,axis=0,add=povsphere([0,0,0.5],0.1,povlook(green)), ...
>  w=500,h=300);
\end{eulerprompt}
\begin{eulercomment}
Dengan metode bayangan canggih dari Povray, sangat sedikit titik yang
dapat menghasilkan permukaan yang sangat halus. Hanya di perbatasan
dan dalam bayang-bayang triknya mungkin menjadi jelas.

Untuk ini, kita perlu menambahkan vektor normal di setiap titik
matriks.
\end{eulercomment}
\begin{eulerprompt}
>Z &= x^2*y^3
\end{eulerprompt}
\begin{euleroutput}
  
                                   2  3
                                  x  y
  
\end{euleroutput}
\begin{eulercomment}
Persamaan permukaannya adalah [x,y,Z]. Kami menghitung dua turunan ke
x dan y ini dan mengambil produk silang sebagai normal.
\end{eulercomment}
\begin{eulerprompt}
>dx &= diff([x,y,Z],x); dy &= diff([x,y,Z],y);
\end{eulerprompt}
\begin{eulercomment}
Kami mendefinisikan normal sebagai produk silang dari turunan ini, dan
mendefinisikan fungsi koordinat.
\end{eulercomment}
\begin{eulerprompt}
>N &= crossproduct(dx,dy); NX &= N[1]; NY &= N[2]; NZ &= N[3]; N,
\end{eulerprompt}
\begin{euleroutput}
  
                                 3       2  2
                         [- 2 x y , - 3 x  y , 1]
  
\end{euleroutput}
\begin{eulercomment}
Kami hanya menggunakan 25 poin.
\end{eulercomment}
\begin{eulerprompt}
>x=-1:0.5:1; y=x';
>pov3d(x,y,Z(x,y),angle=10°, ...
>  xv=NX(x,y),yv=NY(x,y),zv=NZ(x,y),<shadow);
\end{eulerprompt}
\begin{eulercomment}
Berikut ini adalah simpul Trefoil yang dilakukan oleh A. Busser di
Povray. Ada versi yang ditingkatkan dari ini dalam contoh.

See: Contoh\textbackslash{}Trefoil Simpul \textbar{} Simpul trefoil

Untuk tampilan yang bagus dengan tidak terlalu banyak titik, kami
menambahkan vektor normal di sini. Kami menggunakan Maxima untuk
menghitung normal bagi kami. Pertama, ketiga fungsi koordinat sebagai
ekspresi simbolik.
\end{eulercomment}
\begin{eulerprompt}
>X &= ((4+sin(3*y))+cos(x))*cos(2*y); ...
>Y &= ((4+sin(3*y))+cos(x))*sin(2*y); ...
>Z &= sin(x)+2*cos(3*y);
\end{eulerprompt}
\begin{eulercomment}
Kemudian kedua vektor turunan ke x dan y.
\end{eulercomment}
\begin{eulerprompt}
>dx &= diff([X,Y,Z],x); dy &= diff([X,Y,Z],y);
\end{eulerprompt}
\begin{eulercomment}
Sekarang normal, yang merupakan produk silang dari dua turunan.
\end{eulercomment}
\begin{eulerprompt}
>dn &= crossproduct(dx,dy);
\end{eulerprompt}
\begin{eulercomment}
Kami sekarang mengevaluasi semua ini secara numerik.
\end{eulercomment}
\begin{eulerprompt}
>x:=linspace(-%pi,%pi,40); y:=linspace(-%pi,%pi,100)';
\end{eulerprompt}
\begin{eulercomment}
Vektor normal adalah evaluasi dari ekspresi simbolik dn[i] untuk
i=1,2,3. Sintaks untuk ini adalah \&"expression"(parameters). Ini
adalah alternatif dari metode pada contoh sebelumnya, di mana kita
mendefinisikan ekspresi simbolik NX, NY, NZ terlebih dahulu.
\end{eulercomment}
\begin{eulerprompt}
>pov3d(X(x,y),Y(x,y),Z(x,y),axis=0,zoom=5,w=450,h=350, ...
>  <shadow,look=povlook(gray), ...
>  xv=&"dn[1]"(x,y), yv=&"dn[2]"(x,y), zv=&"dn[3]"(x,y));
\end{eulerprompt}
\begin{eulercomment}
Kami juga dapat menghasilkan grid dalam 3D.
\end{eulercomment}
\begin{eulerprompt}
>povstart(zoom=4); ...
>x=-1:0.5:1; r=1-(x+1)^2/6; ...
>t=(0°:30°:360°)'; y=r*cos(t); z=r*sin(t); ...
>writeln(povgrid(x,y,z,d=0.02,dballs=0.05)); ...
>povend();
\end{eulerprompt}
\begin{eulercomment}
With povgrid(), curves are possible.
\end{eulercomment}
\begin{eulerprompt}
>povstart(center=[0,0,1],zoom=3.6); ...
>t=linspace(0,2,1000); r=exp(-t); ...
>x=cos(2*pi*10*t)*r; y=sin(2*pi*10*t)*r; z=t; ...
>writeln(povgrid(x,y,z,povlook(red))); ...
>writeAxis(0,2,axis=3); ...
>povend();
\end{eulerprompt}
\eulerheading{Latihan Soal}
\begin{eulercomment}
1. Buatlah plot 3D dari fungsi\\
\end{eulercomment}
\begin{eulerformula}
\[
f(x,y)=x^3+3y^2
\]
\end{eulerformula}
\begin{eulercomment}
dengan zoom 3 dan angle 55 derajat menggunakan povray
\end{eulercomment}
\begin{eulerprompt}
>pov3d("x^3+3*y^2",zoom=3,angle=55°);
\end{eulerprompt}
\begin{eulercomment}
2. Buatlah gabungan 2 silinder dengan fungsi povx() berwarna merah dan
povz() berwarna kuning dan zoom 4
\end{eulercomment}
\begin{eulerprompt}
>povstart(zoom=4)
>c1 = povcylinder (-povx,povx,1,povlook(red));
>c2 = povcylinder (-povz,povz,1,povlook(yellow));
>writeln(povintersection([c1,c2]));
>povend();
> 
\end{eulerprompt}
\begin{eulercomment}
3. Buatlah grafik 3D dari fungsi kuadrat berikut ini dengan parameter
tambahan:\\
\end{eulercomment}
\begin{eulerformula}
\[
z=4x^2-2y^2
\]
\end{eulerformula}
\begin{eulercomment}
Tampilkan grafik tersebut dengan transparent, dan menggunakan grid
dengan resolusi 50, dengan warna biru pada garis di plot tersebut
\end{eulercomment}
\begin{eulerprompt}
>plot3d("4*x^2-2*y^2",>transparent,grid=50,wirecolor=blue):
\end{eulerprompt}
\begin{eulercomment}
\begin{eulercomment}
\eulerheading{Kalkulus dengan EMT}
\begin{eulercomment}
Materi Kalkulus mencakup di antaranya:

- Fungsi(fungsi aljabar, trigonometri, eksponensial, logaritma,
komposisi fungsi)\\
- Limit Fungsi,\\
- Turunan Fungsi,\\
- Integral Tak Tentu,\\
- Integral Tentu dan Aplikasinya,\\
- Barisan dan Deret (kekonvergenan barisan dan deret).

EMT (bersama Maxima) dapat digunakan untuk melakukan semua perhitungan
di dalam kalkulus, baik secara numerik maupun analitik (eksak).

\begin{eulercomment}
\eulerheading{Fungsi}
\begin{eulercomment}
Fungsi adalah suatu relasi yang menghubungkan setiap anggota dari
suatu himpunan (domain) dengan tepat satu anggota himpunan lain
(kodomain). Secara sederhana, fungsi dapat dipandang sebagai mesin
yang menerima input (nilai x) dan menghasilkan output (nilai y).

\end{eulercomment}
\begin{eulerformula}
\[
f: A \to B
\]
\end{eulerformula}
\begin{eulercomment}
\end{eulercomment}
\eulersubheading{Sifat-sifat Fungsi}
\begin{eulercomment}
1. Injektif (satu-satu)\\
Fungsi injektif adalah fungsi degan tiap elemen kondomain tidak
memiliki lebih dari satu elemen domain. Fungsi injektif disebut juga
dengan "fungsi satu-satu" karena tiap elemen kodomain hanya boleh
berelasi satu kali.

2. Surjektif\\
Fungsi surjektif adalah fungsi dengan semua elemen kodomain berelasi
dengan elemen domain. Fungsi surjektif juga disebut fungsi "on-to".

3. Bijektif\\
Fungsi bijektif adalah fungsi yang memenuhi sifat injektif dan
surjektif. Fungsi bijektif juga disebut fungsi korespondensi
satu-satu, karena elemen domain dan kodomain semuanya berelasi
satu-satu.

\end{eulercomment}
\eulersubheading{Jenis-jenis Fungsi yang akan Dielajari}
\begin{eulercomment}
1. Fungsi Aljabar\\
Fungsi aljabar adalah jenis fungsi matematika yang dibentuk dengan
menggunakan operasi-operasi aljabar seperti penjumlahan, pengurangan,
perkalian, pembagian, dan pangkat. Secara umum, fungsi aljabar
merupakan fungsi yang dapat dinyatakan dengan ekspresi aljabar yang
melibatkan variabel dan konstanta. Fungsi ini termasuk di dalam
kategori fungsi elementer karena bisa dinyatakan dalam bentuk dasar.

2. Fungsi Trigonometri\\
Fungsi trigonometri adalah fungsi yang menghubungkan sudut suatu
segitiga siku-siku dengan rasio panjang sisi-sisinya. Fungsi ini
sangat penting dalam matematika, khususnya dalam geometri, analisis,
dan aplikasi di bidang fisika, teknik, dan astronomi. Fungsi-fungsi
trigonometri yang paling umum adalah sinus, kosinus, tangen, serta
turunan dan kebalikannya seperti secan, kosekan, dan kotangen.

3. Fungsi Eksponensial\\
Fungsi eksponensial adalah fungsi matematika di mana variabel
independen (biasanya dilambangkan dengan x) berada di eksponen. Bentuk
umum fungsi eksponensial adalah:

\end{eulercomment}
\begin{eulerformula}
\[
f(x)=a^x
\]
\end{eulerformula}
\begin{eulercomment}
- a adalah bilangan positif (basis) dan tidak sama dengan 1\\
- x adalah variabel eksponen

Fungsi eksponensial yang paling umum adalah fungsi eksponensial alami
dengan basis bilangan euler (e = 2.718). Fungsi ini sering digunakan
dalam berbagai aplikasi ilmiah yang dinyatakan sebagai:

\end{eulercomment}
\begin{eulerformula}
\[
f(x)=e^x
\]
\end{eulerformula}
\begin{eulercomment}
4. Fungsi Logaritma\\
Fungsi logaritma adalah kebalikan atau invers dari fungsi
eksponensial. Fungsi logaritma dinyatakan sebagai:

\end{eulercomment}
\begin{eulerformula}
\[
f(x)= \log a(x)
\]
\end{eulerformula}
\begin{eulercomment}
5. Komposisi Fungsi\\
Komposisi fungsi adalah operasi matematika di mana dua fungsi
digabungkan sehingga output dari satu fungsi menjadi input dari fungsi
yang lain. Diketahui f gan g dua fungsi sebarang maka fungsi komposisi
f dan g ditulis\\
\end{eulercomment}
\begin{eulerformula}
\[
g o f
\]
\end{eulerformula}
\begin{eulercomment}
didefinisikan sebagai:

\end{eulercomment}
\begin{eulerformula}
\[
(g o f)(x) = g(f(x))
\]
\end{eulerformula}
\begin{eulercomment}
dengan f dikerjakan terlebih dahulu daripada g.

\end{eulercomment}
\eulersubheading{Mendefinisikan Fungsi}
\begin{eulercomment}
Terdapat beberapa cara mendefinisikan fungsi pada EMT, yakni:

- Menggunakan format nama\_fungsi := rumus fungsi (fungsi numerik),\\
- Menggunakan format nama\_fungsi \&= rumus fungsi (fungsi simbolik,
namun dapat dihitung secara numerik),\\
- Menggunakan format nama\_fungsi \&\&= rumus fungsi(fungsi simbolik
murni, tidak dapat dihitung langsung),\\
- Fungsi sebagai program EMT (Mendefinisikan fungsi yang lebih
kompleks dengan beberapa langkah atau percabangan)

Setiap format harus diawali dengan perintah function (bukan sebagai
ekspresi).


Berikut adalah adalah beberapa contoh cara mendefinisikan fungsi:

\end{eulercomment}
\begin{eulerformula}
\[
f(x)=2x^2+e^{\sin(x)}.
\]
\end{eulerformula}
\begin{eulerprompt}
>function f(x) := 2*x^2+exp(sin(x)) // fungsi numerik 
>f(0), f(1), f(pi)
\end{eulerprompt}
\begin{euleroutput}
  1
  4.31977682472
  20.7392088022
\end{euleroutput}
\begin{eulerprompt}
>f(a) // tidak dapat dihitung nilainya karena bukan fungsi numerik
\end{eulerprompt}
\begin{euleroutput}
  50.3833049952
\end{euleroutput}
\begin{eulerformula}
\[
f(x)=3x^3+e^{\cos(x)}.
\]
\end{eulerformula}
\begin{eulerprompt}
>function f(x) := 3*x^3+exp(cos(x))  
>f(1)
\end{eulerprompt}
\begin{euleroutput}
  4.71652569955
\end{euleroutput}
\begin{eulerprompt}
>f(20)
\end{eulerprompt}
\begin{euleroutput}
  24001.5039306
\end{euleroutput}
\begin{eulerprompt}
>f(-2:2)
\end{eulerprompt}
\begin{euleroutput}
  [-23.3404,  -1.28347,  2.71828,  4.71653,  24.6596]
\end{euleroutput}
\begin{eulerprompt}
>f(-2)+f(2)
\end{eulerprompt}
\begin{euleroutput}
  1.31916682485
\end{euleroutput}
\begin{eulerprompt}
>aspect(1.5); plot2d("f(x)",-2,2):
\end{eulerprompt}
\begin{eulercomment}
Berikutnya kita definisikan fungsi:

\end{eulercomment}
\begin{eulerformula}
\[
g(x)=\frac{\sqrt{x^2-3x}}{x+1}
\]
\end{eulerformula}
\begin{eulerprompt}
>function g(x) := sqrt(x^2-3*x)/(x+1)
>g(1)
\end{eulerprompt}
\begin{euleroutput}
  Floating point error!
  Error in sqrt
  Try "trace errors" to inspect local variables after errors.
  g:
      useglobal; return sqrt(x^2-3*x)/(x+1) 
  Error in:
  g(1) ...
      ^
\end{euleroutput}
\begin{eulercomment}
Floating point error karena untuk x=1, g(x) akan bernilai imajiner
yaitu:

\end{eulercomment}
\begin{eulerformula}
\[
\frac{\sqrt{-2}}{2}
\]
\end{eulerformula}
\begin{eulerprompt}
>g(5)
\end{eulerprompt}
\begin{euleroutput}
  0.527046276695
\end{euleroutput}
\begin{eulerprompt}
>g(20)
\end{eulerprompt}
\begin{euleroutput}
  0.878051853076
\end{euleroutput}
\begin{eulerprompt}
>g(21)
\end{eulerprompt}
\begin{euleroutput}
  0.883737367965
\end{euleroutput}
\begin{eulerprompt}
>g(20:25)
\end{eulerprompt}
\begin{euleroutput}
  [0.878052,  0.883737,  0.888915,  0.89365,  0.897998,  0.902003]
\end{euleroutput}
\begin{eulercomment}
Silakan Anda plot grafik fungsi di atas!
\end{eulercomment}
\begin{eulerprompt}
>aspect(1.5); plot2d("g(x)",20,25):
>f(g(5)) // komposisi fungsi
\end{eulerprompt}
\begin{euleroutput}
  2.81254111152
\end{euleroutput}
\begin{eulerprompt}
>g(f(5))
\end{eulerprompt}
\begin{euleroutput}
  0.993366510057
\end{euleroutput}
\begin{eulerprompt}
>function h(x) := f(g(x)) // definisi komposisi fungsi 
>h(5) // sama dengan f(g(5))
\end{eulerprompt}
\begin{euleroutput}
  2.81254111152
\end{euleroutput}
\begin{eulercomment}
Silakan Anda plot kurva fungsi komposisi fungsi f dan g:

\end{eulercomment}
\begin{eulerformula}
\[
h(x)=f(g(x))
\]
\end{eulerformula}
\begin{eulercomment}
dan

\end{eulercomment}
\begin{eulerformula}
\[
u(x)=g(f(x))
\]
\end{eulerformula}
\begin{eulercomment}
bersama-sama kurva fungsi f dan g dalam satu bidang koordinat.
\end{eulercomment}
\begin{eulerprompt}
>f(0:10) // nilai-nilai f(0), f(1), f(2), ..., f(10)
\end{eulerprompt}
\begin{euleroutput}
  [2.71828,  4.71653,  24.6596,  81.3716,  192.52,  376.328,  650.612,
  1031.13,  1536.86,  2187.4,  3000.43]
\end{euleroutput}
\begin{eulerprompt}
>fmap(0:10) // sama dengan f(0:10), berlaku untuk semua fungsi
\end{eulerprompt}
\begin{euleroutput}
  [2.71828,  4.71653,  24.6596,  81.3716,  192.52,  376.328,  650.612,
  1031.13,  1536.86,  2187.4,  3000.43]
\end{euleroutput}
\begin{eulerprompt}
>gmap(200:210)
\end{eulerprompt}
\begin{euleroutput}
  [0.987534,  0.987596,  0.987657,  0.987718,  0.987778,  0.987837,
  0.987896,  0.987954,  0.988012,  0.988069,  0.988126]
\end{euleroutput}
\begin{eulercomment}
Misalkan kita akan mendefinisikan fungsi

\end{eulercomment}
\begin{eulerformula}
\[
f(x) = \begin{cases} x^3 & x>0 \\ x^2 & x\le 0. \end{cases}
\]
\end{eulerformula}
\begin{eulercomment}
Fungsi tersebut tidak dapat didefinisikan sebagai fungsi numerik
secara "inline" menggunakan format :=, melainkan didefinisikan sebagai
program. Perhatikan, kata "map" digunakan agar fungsi dapat menerima
vektor sebagai input, dan hasilnya berupa vektor. Jika tanpa kata
"map" fungsinya hanya dapat menerima input satu nilai.
\end{eulercomment}
\begin{eulerprompt}
>function map f(x)
\end{eulerprompt}
\begin{eulerudf}
    if x>0 then return x^3
    else return x^2
    endif;
  endfunction
\end{eulerudf}
\begin{eulerprompt}
>f(1)
\end{eulerprompt}
\begin{euleroutput}
  1
\end{euleroutput}
\begin{eulerprompt}
>f(-2)
\end{eulerprompt}
\begin{euleroutput}
  4
\end{euleroutput}
\begin{eulerprompt}
>f(-5:5)
\end{eulerprompt}
\begin{euleroutput}
  [25,  16,  9,  4,  1,  0,  1,  8,  27,  64,  125]
\end{euleroutput}
\begin{eulerprompt}
>aspect(1.5); plot2d("f(x)",-5,5):
>function f(x) &= 2*E^x // fungsi simbolik
\end{eulerprompt}
\begin{euleroutput}
  
                                      x
                                   2 E
  
\end{euleroutput}
\begin{eulerprompt}
>$f(a) // nilai fungsi secara simbolik
>function a:=12
>f(a)
\end{eulerprompt}
\begin{euleroutput}
  296.826318205
\end{euleroutput}
\begin{eulerprompt}
>f(E) // nilai fungsi berupa bilangan desimal
\end{eulerprompt}
\begin{euleroutput}
  30.308524483
\end{euleroutput}
\begin{eulerprompt}
>$f(E), $float(%)
>function g(x) &= 3*x+1
\end{eulerprompt}
\begin{euleroutput}
  
                                 3 x + 1
  
\end{euleroutput}
\begin{eulerprompt}
>g(50)
\end{eulerprompt}
\begin{euleroutput}
  151
\end{euleroutput}
\begin{eulerprompt}
>function h(x) &= f(g(x)) // komposisi fungsi
\end{eulerprompt}
\begin{euleroutput}
  
                                   3 x + 1
                                2 E
  
\end{euleroutput}
\begin{eulerprompt}
>plot2d("h(x)",-1,1):
\end{eulerprompt}
\begin{eulercomment}
\end{eulercomment}
\eulersubheading{Contoh Fungsi Aljabar}
\begin{eulercomment}
1.Diberikan fungsi:\\
\end{eulercomment}
\begin{eulerformula}
\[
f(x) = 6x^3+12\sqrt x
\]
\end{eulerformula}
\begin{eulercomment}
Tentukan nilai f(1) dan f(4) kemudian cari nilai f(4)-2f(1)\\
Gambarkan juga grafiknya di f(1:4)!
\end{eulercomment}
\begin{eulerprompt}
>function f(x) := 6*x^3+12*sqrt(x)
>f(1)
\end{eulerprompt}
\begin{euleroutput}
  18
\end{euleroutput}
\begin{eulerprompt}
>f(4)
\end{eulerprompt}
\begin{euleroutput}
  408
\end{euleroutput}
\begin{eulerprompt}
>f(4)-2*f(1)
\end{eulerprompt}
\begin{euleroutput}
  372
\end{euleroutput}
\begin{eulerprompt}
>aspect(1.5); plot2d("f",1,4):
\end{eulerprompt}
\begin{eulercomment}
2. Diberikan fungsi:

\end{eulercomment}
\begin{eulerformula}
\[
k(x,y)= 12x^3+4y^2-3x+2
\]
\end{eulerformula}
\begin{eulercomment}
Tentukan nilai k(1,1)+k(2,3)!
\end{eulercomment}
\begin{eulerprompt}
>function k(x,y):= 12*x^3+4*y^2-3*x+2
>k(1,1)
\end{eulerprompt}
\begin{euleroutput}
  15
\end{euleroutput}
\begin{eulerprompt}
>k(2,3)
\end{eulerprompt}
\begin{euleroutput}
  128
\end{euleroutput}
\begin{eulerprompt}
>k(1,1)+k(2,3)
\end{eulerprompt}
\begin{euleroutput}
  143
\end{euleroutput}
\begin{eulercomment}
3. Diketahui suatu fungsi f(x,y,z) diddefinisikan sebagai betrikut

\end{eulercomment}
\begin{eulerformula}
\[
f(x,y,z)=(3x-1)+\sqrt{x^3-y^2-z+6}
\]
\end{eulerformula}
\begin{eulercomment}
Tentukan nilai f(3,4,5)!
\end{eulercomment}
\begin{eulerprompt}
>function f(x,y,z):= (3*x-1)+ sqrt(x^3-y^2-z+6)
>f(3,4,5)
\end{eulerprompt}
\begin{euleroutput}
  11.4641016151
\end{euleroutput}
\eulersubheading{Contoh Fungsi Trigonometri}
\begin{eulercomment}
1. Suatu fungsi trigonometri didefinisikan sebagai:

\end{eulercomment}
\begin{eulerformula}
\[
f(x)=\frac{tan(x)}{2x+3x^2}
\]
\end{eulerformula}
\begin{eulercomment}
Tentukan nilai dari f(6)-f(7)!
\end{eulercomment}
\begin{eulerprompt}
>function f(x):= tan(x)/2*x+3*x^2
>f(6)-f(7)
\end{eulerprompt}
\begin{euleroutput}
  -42.9230865137
\end{euleroutput}
\begin{eulercomment}
2. Diberikan fungsi:\\
\end{eulercomment}
\begin{eulerformula}
\[
f(x,y)=\frac{x^2+sin(y)}{xy}
\]
\end{eulerformula}
\begin{eulercomment}
Tentukan nilai f(4)!
\end{eulercomment}
\begin{eulerprompt}
>function f(x,y) := x^2+sin(x)/x*y
>f(4,2)
\end{eulerprompt}
\begin{euleroutput}
  15.6215987523
\end{euleroutput}
\begin{eulerprompt}
>plot3d("f",-4,4):
\end{eulerprompt}
\begin{eulercomment}
3. Diberikan suatu fungsi

\end{eulercomment}
\begin{eulerformula}
\[
f(x,y,z)=\frac{sin(x)}{cos(z)+sin(x)}+tan(y)
\]
\end{eulerformula}
\begin{eulercomment}
tentukan nilai f(x,y,z) bila x=1, y=2, dan z=-3!
\end{eulercomment}
\begin{eulerprompt}
>function f(x,y,z)&= sin(x)/(sin(x)+cos(z))+tan(y)
\end{eulerprompt}
\begin{euleroutput}
  
                             sin(x)
                         --------------- + tan(y)
                         cos(z) + sin(x)
  
\end{euleroutput}
\begin{eulerprompt}
>f(1,2,-3)
\end{eulerprompt}
\begin{euleroutput}
  -7.85069041214
\end{euleroutput}
\eulersubheading{Contoh Fungsi Eksponensial}
\begin{eulercomment}
1. Diberikan sebuah fungsi f(x) sebagai berikut:

\end{eulercomment}
\begin{eulerformula}
\[
f(x)=3^x
\]
\end{eulerformula}
\begin{eulercomment}
Tentukan nilai f(x+1)-f(x)!
\end{eulercomment}
\begin{eulerprompt}
>function f(x):= 3^x
>$f(x+1)
\end{eulerprompt}
\begin{euleroutput}
  Maxima said:
  Too few arguments supplied to f(x,y,z); found: [x+1]
   -- an error. To debug this try: debugmode(true);
  
  Error in:
   $f(x+1) ...
         ^
\end{euleroutput}
\begin{eulerprompt}
>$f(x)
\end{eulerprompt}
\begin{euleroutput}
  Maxima said:
  Too few arguments supplied to f(x,y,z); found: [x]
   -- an error. To debug this try: debugmode(true);
  
  Error in:
   $f(x) ...
       ^
\end{euleroutput}
\begin{eulerprompt}
>$f(x+1)-f(x)
\end{eulerprompt}
\begin{euleroutput}
  Maxima said:
  Too few arguments supplied to f(x,y,z); found: [x+1]
   -- an error. To debug this try: debugmode(true);
  
  Error in:
   $f(x+1)-f(x) ...
              ^
\end{euleroutput}
\begin{eulercomment}
2. Fungsi f(m) didefinisikan sebagai:

\end{eulercomment}
\begin{eulerformula}
\[
f(m)=27+8^m
\]
\end{eulerformula}
\begin{eulercomment}
Tentukan nilai dari f(3), 2f(5), dan f(3)/2f(5)
\end{eulercomment}
\begin{eulerprompt}
>function f(m)&=27+8^m;
>$f(3)
>$2*f(5)
>$f(3)/(2*f(5))
\end{eulerprompt}
\begin{eulercomment}
3. Diketahui nilai y dari fungsi di bawah ini adalah 1

\end{eulercomment}
\begin{eulerformula}
\[
f(x)=2^{4x-12y}
\]
\end{eulerformula}
\begin{eulercomment}
Tentukan nilai f(3), f(8), dan buat grafiknya di f(3:5)!
\end{eulercomment}
\begin{eulerprompt}
>function f(x):=2^(4*x-12*y)
>function y:=1
>f(8)
\end{eulerprompt}
\begin{euleroutput}
  [4.29497e+09,  1.51745e+09,  5.47254e+08,  2.04699e+08,  8.06415e+07,
  3.39465e+07,  1.54748e+07,  7.73187e+06,  4.27913e+06,  2.64671e+06,
  1.8426e+06,  1.45143e+06,  1.29785e+06,  1.31903e+06,  1.52251e+06,
  1.99039e+06,  2.93317e+06,  4.84062e+06,  8.87172e+06,  1.78786e+07,
  3.91673e+07,  9.21022e+07,  2.293e+08,  5.9565e+08,  1.59012e+09,
  4.29497e+09,  1.15549e+10,  3.04854e+10,  7.76848e+10,  1.88436e+11,
  4.29142e+11,  9.06006e+11,  1.75298e+12,  3.07733e+12,  4.86012e+12,
  6.85888e+12,  8.60662e+12,  9.57275e+12,  9.42643e+12,  8.22395e+12,
  6.37353e+12,  4.40751e+12,  2.73678e+12,  1.53803e+12,  7.89735e+11,
  3.74544e+11,  1.66061e+11,  6.97353e+10,  2.81245e+10,  1.10521e+10,
  4.29497e+09,  1.67538e+09,  6.6579e+08,  2.73468e+08,  1.17719e+08,
  5.38075e+07,  2.64326e+07,  1.41083e+07,  8.2603e+06,  5.34814e+06,
  3.85383e+06,  3.10543e+06,  2.80654e+06,  2.84796e+06,  3.24272e+06,
  4.13249e+06,  5.86929e+06,  9.23519e+06,  1.59777e+07,  3.01218e+07,
  6.12433e+07,  1.3276e+08,  3.03043e+08,  7.18849e+08,  1.74781e+09,
  4.29497e+09,  1.05164e+10,  2.52986e+10,  5.89768e+10,  1.31488e+11,
  2.76892e+11,  5.44449e+11,  9.89291e+11,  1.64613e+12,  2.48915e+12,
  3.39954e+12,  4.17463e+12,  4.59648e+12,  4.53286e+12,  4.00632e+12,
  3.18112e+12,  2.27844e+12,  1.4804e+12,  8.7886e+11,  4.80821e+11,
  2.44814e+11,  1.17278e+11,  5.34883e+10,  2.35191e+10,  1.01014e+10,
   ... ]
\end{euleroutput}
\begin{eulerprompt}
>f(3)
\end{eulerprompt}
\begin{euleroutput}
  [4096,  1447.16,  521.902,  195.216,  76.9058,  32.3739,  14.7579,
  7.37368,  4.0809,  2.5241,  1.75724,  1.38419,  1.23772,  1.25793,
  1.45198,  1.89818,  2.79729,  4.61637,  8.46073,  17.0504,  37.3528,
  87.8355,  218.678,  568.056,  1516.46,  4096,  11019.6,  29073.2,
  74086,  179706,  409262,  864034,  1.67177e+06,  2.93477e+06,
  4.63497e+06,  6.54114e+06,  8.20791e+06,  9.12928e+06,  8.98974e+06,
  7.84297e+06,  6.07827e+06,  4.20333e+06,  2.61e+06,  1.46678e+06,
  753150,  357193,  158368,  66504.7,  26821.6,  10540.1,  4096,
  1597.77,  634.947,  260.799,  112.266,  51.3148,  25.2081,  13.4547,
  7.87764,  5.10038,  3.6753,  2.96157,  2.67652,  2.71603,  3.0925,
  3.94105,  5.59739,  8.80736,  15.2375,  28.7264,  58.4062,  126.61,
  289.004,  685.547,  1666.85,  4096,  10029.2,  24126.6,  56244.6,
  125397,  264064,  519227,  943461,  1.56987e+06,  2.37383e+06,
  3.24205e+06,  3.98124e+06,  4.38354e+06,  4.32287e+06,  3.82072e+06,
  3.03375e+06,  2.17289e+06,  1.41181e+06,  838146,  458547,  233472,
  111845,  51010.4,  22429.6,  9633.44,  4096,  1747.52,  758.199,
  338.943,  158.089,  77.849,  40.9192,  23.1851,  14.2841,  9.63889,
  7.16572,  5.89402,  5.37828,  5.45006,  6.12932,  7.63297,  10.485,
  15.8013,  25.9482,  46.0541,  87.5224,  176.259,  371.946,  812.67,
  1815.74,  4096,  9209.93,  20380.2,  43834.2,  90548.7,  177635,
   ... ]
\end{euleroutput}
\begin{eulerprompt}
>f(3:5)
\end{eulerprompt}
\begin{euleroutput}
  Cannot combine a 1x3 and a 1x1001 matrix for -!
  Try "trace errors" to inspect local variables after errors.
  f:
      useglobal; return 2^(4*x-12*y) 
  Error in:
  f(3:5) ...
        ^
\end{euleroutput}
\begin{eulerprompt}
>aspect(1.3); plot2d("f",3,5):
\end{eulerprompt}
\begin{euleroutput}
  Error : f does not produce a real or column vector
  
  Error generated by error() command
  
   %ploteval:
      error(f$|" does not produce a real or column vector"); 
  adaptiveevalone:
      s=%ploteval(g$,t;args());
  Try "trace errors" to inspect local variables after errors.
  plot2d:
      dw/n,dw/n^2,dw/n,auto;args());
\end{euleroutput}
\eulersubheading{Contoh Fungsi Logaritma}
\begin{eulercomment}
1. Diberikan fungsi di bawah ini:

\end{eulercomment}
\begin{eulerformula}
\[
f(x)=e^2 \cdot \left( \frac{1}{3+4 \log(x)}+\frac{1}{7} \right)
\]
\end{eulerformula}
\begin{eulercomment}
Tentukan nilai f(0.6)!
\end{eulercomment}
\begin{eulerprompt}
>function f(x):= E^2*(1/(3+4*log(x))+1/7)
>f(0.6)
\end{eulerprompt}
\begin{euleroutput}
  8.77908249441
\end{euleroutput}
\begin{eulerprompt}
>plot2d("f",-0.6,0.6):
\end{eulerprompt}
\eulersubheading{Contoh Komposisi Fungsi}
\begin{eulercomment}
1. Untuk fungsi:

\end{eulercomment}
\begin{eulerformula}
\[
f(x) = x^2-3x+2
\]
\end{eulerformula}
\begin{eulercomment}
\end{eulercomment}
\begin{eulerformula}
\[
g(x)=3x+5
\]
\end{eulerformula}
\begin{eulercomment}
cari nilai\\
\end{eulercomment}
\begin{eulerformula}
\[
fog(-2), gof(0)
\]
\end{eulerformula}
\begin{eulercomment}
\end{eulercomment}
\begin{eulerprompt}
>function f(x) :=  x^2-3*x+2; $f(x)
>function g(x) := 3*x+5; $g(x)
>f(g(-2)), g(f(0))
\end{eulerprompt}
\begin{euleroutput}
  6
  11
\end{euleroutput}
\begin{eulerprompt}
>plot2d("(3*x+5)^2-3*(3*x+5)+2",-2,2):
\end{eulerprompt}
\eulerheading{Latihan}
\begin{eulercomment}
Bukalah buku Kalkulus. Cari dan pilih beberapa (paling sedikit 5
fungsi berbeda tipe/bentuk/jenis) fungsi dari buku tersebut, kemudian
definisikan fungsi-fungsi tersebut dan komposisinya di EMT pada
baris-baris perintah berikut (jika perlu tambahkan lagi). Untuk setiap
fungsi, hitung beberapa nilainya, baik untuk satu nilai maupun vektor.
Gambar grafik fungsi-fungsi tersebut dan komposisi-komposisi 2 fungsi.

Juga, carilah fungsi beberapa (dua) variabel. Lakukan hal sama seperti
di atas.

1. Diberikan fungsi:

\end{eulercomment}
\begin{eulerformula}
\[
b(x)= \frac{12x^3}{7x^2-3x}
\]
\end{eulerformula}
\begin{eulercomment}
Cari nilai b(1), vektor b(1:4), dan buatkan grafiknya!
\end{eulercomment}
\begin{eulerprompt}
> 
> 
\end{eulerprompt}
\begin{eulercomment}
2. Diketahui suatu fungsi f(x) sebagai berikut:

\end{eulercomment}
\begin{eulerformula}
\[
f(x,y)=\frac{3(x-1)(y-2)}{2}+\frac{(y-2)(x-3)}{2}-2(x-1)(y-3)
\]
\end{eulerformula}
\begin{eulercomment}
Tentukan nilai dari f(20,10) dan buat grafiknya di f(20,10:30,20)!
\end{eulercomment}
\begin{eulercomment}
3. Sketsakan grafik fungsi dibawah ini pada[-pi,2pi]

\end{eulercomment}
\begin{eulerformula}
\[
y(t)=sin(5t)
\]
\end{eulerformula}
\begin{eulercomment}
4. Diketahui suatu fungsi f(x):

\end{eulercomment}
\begin{eulerformula}
\[
f(x)=3^x
\]
\end{eulerformula}
\begin{eulercomment}
Tentukan nilai f(a+2\textasciicircum{}b-c) jika nilai a=1, b=2, dan c=5!
\end{eulercomment}
\begin{eulercomment}
5. Untuk fungsi:

\end{eulercomment}
\begin{eulerformula}
\[
f(x) = 2x^2-3x+20
\]
\end{eulerformula}
\begin{eulercomment}
\end{eulercomment}
\begin{eulerformula}
\[
g(x)= 3x-5
\]
\end{eulerformula}
\begin{eulercomment}
cari nilai\\
\end{eulercomment}
\begin{eulerformula}
\[
fog(5), gof(8)
\]
\end{eulerformula}
\begin{eulercomment}
\begin{eulercomment}
\eulerheading{Definisi limit}
\begin{eulercomment}
Pada dasarnya, limit digunakan untuk menyatakan sesuatu yang nilainya
mendekati nilai tertentu. Limit dapat diartikan sebagai menuju suatu
batas, sesuatu yang dekat namun tidak dapat dicapai. Mengapa nilainya
hanya mendekati? Karena suatu fungsi biasanya tidak terdefinisi pada
titik-titik tertentu. Walaupun suatu fungsi seringkali tidak
terdefinisi untuk titik tertentu, namun masih dapat dicari tahu berapa
nilai yang didekati oleh fungsi tersebut apabila titik tertentu
semakin didekati yaitu dengan limit.

Bentuk umum dari limit sendiri dinotasikan dengan:

maxima: 'limit(F(x),x,c)=L

Notasi tersebut menyatakan bahwa f(x) untuk niai x mendekati c sama
dengan L. F(x) disini dapat berupa bermacam-macam jenis fungsi. Dan L
dapat berupa konstanta, ataupun "und" (tak terdefinisi), "ind" (tak
tentu namun terbatas), "infinity" (kompleks tak hingga). Begitupun
dengan batas c, dapat berupa sebarang nilai atau pada tak hingga
(-inf, minf, dan inf).

Sebuah fungsi dapat dikatakan memiliki limit apabila limit kanan dan
limit kiri nya memiliki nilai yang sama. Dimana, limit dari fungsi
tersebut adalah nilai dari limit kanan dan limit kiri fungsi yang
bernilai sama tadi.

\begin{eulercomment}
\eulerheading{Sifat-sifat Limit Jika f(x) dan g(x) adalah fungsi yang memiliki}
\begin{eulercomment}
limit di c, n merupakan bilangan bulat positif, dan k adalah
konstanta.\\
1.\\
\end{eulercomment}
\begin{eulerformula}
\[
\lim_{x \to c}k=k
\]
\end{eulerformula}
\begin{eulercomment}
\end{eulercomment}
\begin{eulerprompt}
>$showev('limit(5,x,2))
\end{eulerprompt}
\begin{eulerformula}
\[
\lim_{x \to 2}5=5
\]
\end{eulerformula}
\begin{eulercomment}
2.\\
maxima: 'limit(x,x,c)=c
\end{eulercomment}
\begin{eulerprompt}
>$showev('limit(x,x,3))
\end{eulerprompt}
\begin{eulercomment}
3.\\
maxima: 'limit(kf(x),x,c)= k*maxima: 'limit(f(x),x,c)
\end{eulercomment}
\begin{eulerprompt}
>$showev('limit(4*x^2,x,1))
\end{eulerprompt}
\begin{eulercomment}
4.\\
maxima: 'limit([g(x)+f(x)],x,c)= maxima: 'limit(g(x),x,c) + maxima: 'limit(f(x),x,c)
\end{eulercomment}
\begin{eulerprompt}
>$showev('limit(x^3+2*x,x,1))
\end{eulerprompt}
\begin{eulercomment}
5.\\
maxima: 'limit([g(x)-f(x)],x,c)= maxima: 'limit(g(x),x,c) - maxima: 'limit(f(x),x,c)
\end{eulercomment}
\begin{eulerprompt}
>$showev('limit(x^2-4*x,x,5))
\end{eulerprompt}
\begin{eulercomment}
6.\\
maxima: 'limit([g(x)*f(x)],x,c)= maxima: 'limit(g(x),x,c) * maxima: 'limit(f(x),x,c)
\end{eulercomment}
\begin{eulerprompt}
>$showev('limit((x-1)*(x+4),x,3))
\end{eulerprompt}
\begin{eulercomment}
7.\\
maxima: 'limit(g(x)/f(x),x,c)= maxima: 'limit(g(x),x,c) / maxima: 'limit(f(x),x,c)
\end{eulercomment}
\begin{eulerprompt}
>$showev('limit((x+5)/(x+1),x,0))
\end{eulerprompt}
\begin{eulercomment}
8.\\
maxima: 'limit(f(x)\textasciicircum{}n,x,c)=maxima:'limit(f(x),x,c)\textasciicircum{}n
\end{eulercomment}
\begin{eulerprompt}
>$showev('limit((x+3)^2,x,1))
\end{eulerprompt}
\begin{eulercomment}
9.

maxima: 'limit(f(x)\textasciicircum{}(1/n),x,c)=maxima:'limit(f(x),x,c)\textasciicircum{}(1/n)
\end{eulercomment}
\begin{eulerprompt}
>$showev('limit((x^2+3*x+1)^(1/2),x,2))
\end{eulerprompt}
\eulerheading{Limit pada EMT}
\begin{eulercomment}
Pada EMT cara mendefinisikan limit yaitu dengan format :

\textdollar{}showev('limit(f(x),x,c))

Format tersebut akan menampilkan limit yang dimaksud dan hasilnya.
Jika kita ingin menampilkan hasilnya saja dari sebuah limit tanpa
menampilkan limitnya, kita bisa menggunakan format :

\textdollar{}limit(f(x),x,c)

Sedangkan, untuk limit kanan dan limit kiri seperti pada definisi
dapat ditampilkan di EMT dengan cara menambah opsi "plus" atau "minus"
:

\textdollar{}showev('limit(f(x),x,c, plus)) atau 'limit(f(x),x,c, minus)
\end{eulercomment}
\begin{eulerprompt}
>$showev('limit(x^2+3*x+4,x,2))
>$limit(x^2+3*x+4,x,2)
\end{eulerprompt}
\eulerheading{Menghitung dan Visualisasi Limit}
\begin{eulercomment}
Perhitungan limit pada EMT dapat dilakukan dengan menggunakan fungsi
Maxima, yakni "limit". Fungsi "limit" dapat digunakan untuk menghitung
limit fungsi dalam bentuk ekspresi maupun fungsi yang sudah
didefinisikan sebelumnya. Nilai limit dapat dihitung pada sebarang
nilai atau pada tak hingga (-inf, minf, dan inf). Limit kiri dan limit
kanan juga dapat dihitung, dengan cara memberi opsi "plus" atau
"minus". Hasil limit dapat berupa nilai, "und" (tak definisi), "ind"
(tak tentu namun terbatas), "infinity" (kompleks tak hingga).

Limit dapat divisualisasikan menggunakan plot 2 dimensi. Pada EMT
sendiri, format yang bisa digunakan untuk memvisualisasikan limit
adalah :

plot2d("f(x)",-c,c):

Dengan f(x) adalah fungsi pada limit yang dicari, dan c berupa
bilangan real menyesuaikan batas dari limit itu sendiri.

\end{eulercomment}
\eulersubheading{Limit Aljabar}
\begin{eulercomment}
1. Tunjukkan bahwa limit kiri dan kanan dari fungsi berikut bernilai
sama. Berapakah nilai limitnya?\\
maxima: 'limit(x\textasciicircum{}2+x-1,x,0)
\end{eulercomment}
\begin{eulerprompt}
>$showev('limit(x^2+x-1,x,0,minus))
>$showev('limit(x^2+x-1,x,0,plus))
\end{eulerprompt}
\begin{eulercomment}
Dapat terlihat bahwa nilai limit kiri = nilai limit kanan. Maka fungsi
tersebut memiliki nilai limit, yaitu
\end{eulercomment}
\begin{eulerprompt}
>$showev('limit(x^2+x-1,x,0))
\end{eulerprompt}
\begin{eulercomment}
2. Tunjukkan bahwa limit kiri dan kanan dari fungsi berikut bernilai
sama. Berapakah nilai limitnya?\\
maxima: 'limit(sqrt(x\textasciicircum{}2+x-1),x,3)
\end{eulercomment}
\begin{eulerprompt}
>$showev('limit(sqrt(x^2+x-1),x,3,minus))
>$showev('limit(sqrt(x^2+x-1),x,3,plus))
\end{eulerprompt}
\begin{eulercomment}
Dapat terlihat bahwa nilai limit kiri = nilai limit kanan. Maka fungsi
tersebut memiliki nilai limit, yaitu
\end{eulercomment}
\begin{eulerprompt}
>$showev('limit(sqrt(x^2+x-1),x,3))
\end{eulerprompt}
\begin{eulercomment}
3. Tunjukkan bahwa limit kiri dan kanan dari fungsi berikut bernilai
sama. Berapakah nilai limitnya?\\
maxima: 'limit(abs(x+3)/(x+3),x,-3)
\end{eulercomment}
\begin{eulerprompt}
>$showev('limit(abs(x+3)/(x+3),x,-3,minus))
>$showev('limit(abs(x+3)/(x+3),x,-3,plus))
\end{eulerprompt}
\begin{eulercomment}
Karena nilai limit kiri tidak sama dengan nilai limit kanan. Maka
fungsi diatas tidak memiliki limit di x=-3
\end{eulercomment}
\begin{eulerprompt}
>$showev('limit(abs(x+3)/(x+3),x,-3))
\end{eulerprompt}
\begin{eulercomment}
4. Berapakah nilai limit dari fungsi berikut? Tunjukkan dengan
menggunakan grafik!\\
maxima: 'limit(x\textasciicircum{}3+4*x+5,x,2)
\end{eulercomment}
\begin{eulerprompt}
>$limit(x^3+4*x+5,x,2)
>plot2d("x^3+4*x+5",1.8,2.2); plot2d(2,21,>points,style="ow",>add):
\end{eulerprompt}
\begin{eulercomment}
Jadi berdasarkan grafik, nilai limit tersebut adalah 21.

5.Berapakah nilai limit dari fungsi berikut? Tunjukkan dengan
menggunakan grafik!\\
maxima: 'limit((abs(x-5))/(x-5),x,5)
\end{eulercomment}
\begin{eulerprompt}
>plot2d("abs(x-5)/(x-5)",4.9,5.1):
\end{eulerprompt}
\begin{eulercomment}
Karena nilai limit kiri tidak sama dengan limit kanan. maka nili limit
fungsi tersebut tidak ada.

6. Berapakah nilai limit dari fungsi berikut?\\
maxima: 'limit((x\textasciicircum{}5-3*x\textasciicircum{}3+x\textasciicircum{}2-9)/(4*x\textasciicircum{}5-7*x\textasciicircum{}4+2*x-1),x,inf)
\end{eulercomment}
\begin{eulerprompt}
>$showev('limit((x^5-3*x^3+x^2-9)/(4*x^5-7*x^4+2*x-1),x,inf))
\end{eulerprompt}
\begin{eulercomment}
7. Tunjukkan nilai limit fungsi berikut dengan grafik!\\
maxima:  'limit(sqrt(x\textasciicircum{}2+x)-x,x,inf)
\end{eulercomment}
\begin{eulerprompt}
>plot2d("sqrt(x^2+x)-x",0,10):
\end{eulerprompt}
\begin{eulercomment}
Dengan menggunakan grafik diperlihatkan bahwa nilai limit mendekati
0.5. Mari kita buktikan di baris perintah
\end{eulercomment}
\begin{eulerprompt}
>$showev('limit(sqrt(x^2+x)-x,x,inf))
\end{eulerprompt}
\begin{eulercomment}
\end{eulercomment}
\eulersubheading{Limit Trigonometri}
\begin{eulercomment}
1. Tunjukkan bahwa limit kiri dan kanan dari fungsi berikut bernilai
sama. Berapakah nilai limitnya?

maxima: 'limit(cos(x)*sin(x),x,0)
\end{eulercomment}
\begin{eulerprompt}
>$showev('limit(cos(x)*sin(x),x,0,minus))
>$showev('limit(cos(x)*sin(x),x,0,plus))
\end{eulerprompt}
\begin{eulercomment}
Dapat terlihat bahwa nilai limit kiri = nilai limit kanan. Maka fungsi
tersebut memiliki nilai limit, yaitu
\end{eulercomment}
\begin{eulerprompt}
>$showev('limit(cos(x)*sin(x),x,0))
\end{eulerprompt}
\begin{eulercomment}
2. Berapakah nilai limit dari fungsi berikut? Tunjukkan dengan
menggunakan grafik\\
maxima: 'limit((1-cos(x))/x,x,0)
\end{eulercomment}
\begin{eulerprompt}
>plot2d("(1-cos(x))/x",-pi/4,pi/4); plot2d(0,0,>points,style="ow",>add):
\end{eulerprompt}
\begin{eulercomment}
3. Tunjukkan bahwa limit kiri dan kanan dari fungsi berikut bernilai
sama. Berapakah nilai limitnya?

maxima: 'limit(sin(x)/(1-cos(x)),x,0)
\end{eulercomment}
\begin{eulerprompt}
>$showev('limit(3*sin(x)^2/(1-cos(x)),x,0,minus))
>$showev('limit(3*sin(x)^2/(1-cos(x)),x,0,plus))
\end{eulerprompt}
\begin{eulercomment}
Dapat terlihat bahwa nilai limit kiri = nilai limit kanan. Maka fungsi
tersebut memiliki nilai limit, yaitu
\end{eulercomment}
\begin{eulerprompt}
>$showev('limit(3*sin(x)^2/(1-cos(x)),x,0))
\end{eulerprompt}
\begin{eulercomment}
4. Berapakah nilai limit dari fungsi berikut? Tunjukkan dengan
menggunakan grafik!\\
maxima: 'limit(abs(sin(2*x))-cos(x),x,0)
\end{eulercomment}
\begin{eulerprompt}
>$limit(abs(sin(2*x))-cos(x),x,0)
>plot2d("abs(sin(2x))-cos(x)",-pi/2,pi/2); plot2d(0,-1,>points,style="ow",>add):
\end{eulerprompt}
\begin{eulercomment}
Jadi berdasarkan grafik, nilai limit tersebut adalah -1.

5. Berapakah nilai limit dari fungsi berikut? Tunjukkan dengan
menggunakan grafik!\\
maxima: 'limit(2*sin(x)*cos(x)-tan(x),x,pi/3)
\end{eulercomment}
\begin{eulerprompt}
>$limit(2*cos(x)*sin(x)-tan(x),x,pi/3)
>plot2d("2*cos(x)*sin(x)-tan(x)",0,pi/2.9); plot2d(pi/3,-(3)^(1/2)/2,>points,style="ow",>add):
\end{eulerprompt}
\begin{eulercomment}
Jadi berdasarkan grafik, nilai limit tersebut adalah -0.866.

6. Berapakah nilai limit dari fungsi berikut? Tunjukkan dengan
menggunakan grafik!\\
maxima: 'limit(2*sec(x),x,0)
\end{eulercomment}
\begin{eulerprompt}
>$limit(2*sec(x),x,0)
>plot2d("2*sec(x)",-pi/3,pi/3); plot2d(0,2,>points,style="ow",>add):
\end{eulerprompt}
\begin{eulercomment}
Jadi berdasarkan grafik, nilai limit tersebut adalah 2.

\end{eulercomment}
\eulersubheading{Latihan}
\begin{eulercomment}
Tunjukkan bahwa limit kiri dan kanan dari fungsi berikut bernilai
sama.\\
maxima: 'limit(abs(x\textasciicircum{}2+5*x+8)/cos(x)*(x-3),x,3)
\end{eulercomment}
\begin{eulerprompt}
>$limit(abs(x^2+5*x+8)/cos(x)*(x-3),x,3)
\end{eulerprompt}
\eulerheading{3. Turunan Fungsi}
\begin{eulercomment}
\end{eulercomment}
\begin{eulerttcomment}
  Turunan adalah pengukuran terhadap bagaimana fungsi berubah seiring
\end{eulerttcomment}
\begin{eulercomment}
perubahan nilai yang dimasukkan, atau secara umum turunan menunjukkan
bagaimana suatu besaran berubah akibat perubahan besaran lainnya.
Proses dalam menemukan turunan disebut diferensiasi.

Definisi turunan:\\
\end{eulercomment}
\begin{eulerformula}
\[
f'(x) = \lim_{h\to 0}\frac{f(x+h)-f(x)}{h}
\]
\end{eulerformula}
\begin{eulercomment}
Berikut adalah contoh-contoh menentukan turunan fungsi dengan beberapa
cara.

\end{eulercomment}
\eulersubheading{Turunan fungsi Aljabar}
\begin{eulercomment}
Mencari turunan dari\\
\end{eulercomment}
\begin{eulerformula}
\[
f(x)=x^n
\]
\end{eulerformula}
\begin{eulerprompt}
>$showev('limit(((x+h)^n-x^n)/h,h,0)) // turunan x^n
\end{eulerprompt}
\begin{eulercomment}
Bukti:\\
\end{eulercomment}
\begin{eulerformula}
\[
(x+h)^n= \binom{n}{0}x^{n-0}h^0+ \binom{n}{1}x^{n-1}h^1+\binom{n}{2}x^{n-2}h^2+...+\binom{n}{n}x^{n-n}h^n
\]
\end{eulerformula}
\begin{eulercomment}
\end{eulercomment}
\begin{eulerformula}
\[
(x+h)^n = x^n+nx^{n-1}h+\binom{n}{2}x^{n-2}h^2+...+h^n
\]
\end{eulerformula}
\begin{eulerformula}
\[
(x+h)^n -x^n = nx^{n-1}h+\binom{n}{2}x^{n-2}h^2+...+h^n
\]
\end{eulerformula}
\begin{eulerformula}
\[
\frac{(x+h)^n -x^n}{h} = nx^{n-1}+\binom{n}{2}x^{n-2}h+...+h^{n-1}
\]
\end{eulerformula}
\begin{eulerformula}
\[
 \lim_{h\to 0}\frac{(x+h)^n -x^n}{h}=nx^{n-1}
\]
\end{eulerformula}
\begin{eulercomment}
Mencari turunan dari\\
\end{eulercomment}
\begin{eulerformula}
\[
f(x)=x^2
\]
\end{eulerformula}
\begin{eulerprompt}
>$showev('limit(((x+h)^2-x^2)/h,h,0)) // turunan x^2
>p &= expand((x+h)^2-x^2)|simplify; $p //pembilang dijabarkan dan disederhanakan
>q &=ratsimp(p/h); $q // ekspresi yang akan dihitung limitnya disederhanakan
>$limit(q,h,0) // nilai limit sebagai turunan
>plot2d(["x^2", "2x"], color=[black,red]):
\end{eulerprompt}
\begin{eulercomment}
Mencari turunan dari\\
\end{eulercomment}
\begin{eulerformula}
\[
f(x)=f(x)^x
\]
\end{eulerformula}
\begin{eulerprompt}
>$showev('limit((f(x+h)-f(x))/h,h,0)) // turunan f(x)=f(x)^x
\end{eulerprompt}
\begin{euleroutput}
  Answering "Is x an integer?" with "integer"
  Answering "Is x an integer?" with "integer"
  Answering "Is x an integer?" with "integer"
  Answering "Is x an integer?" with "integer"
  Answering "Is x an integer?" with "integer"
  Maxima is asking
  Acceptable answers are: yes, y, Y, no, n, N, unknown, uk
  Is x an integer?
  
  Use assume!
  Error in:
   $showev('limit((f(x+h)-f(x))/h,h,0)) // turunan f(x)=f(x)^x ...
                                       ^
\end{euleroutput}
\begin{eulercomment}
Di sini Maxima bermasalah terkait limit:

\end{eulercomment}
\begin{eulerformula}
\[
\lim_{h\to 0} \frac{(x+h)^{x+h}-x^x}{h}.
\]
\end{eulerformula}
\begin{eulercomment}
Dalam hal ini diperlukan asumsi nilai x.
\end{eulercomment}
\begin{eulerprompt}
>&assume(x>0); $showev('limit((f(x+h)-f(x))/h,h,0)) // turunan f(x)=f(x)^x
\end{eulerprompt}
\begin{euleroutput}
  Answering "Is x an integer?" with "integer"
  Answering "Is x an integer?" with "integer"
  Answering "Is x an integer?" with "integer"
  Answering "Is x an integer?" with "integer"
  Answering "Is x an integer?" with "integer"
  Maxima is asking
  Acceptable answers are: yes, y, Y, no, n, N, unknown, uk
  Is x an integer?
  
  Use assume!
  Error in:
  ... ssume(x>0); $showev('limit((f(x+h)-f(x))/h,h,0)) // turunan f( ...
                                                       ^
\end{euleroutput}
\begin{eulerprompt}
>&forget(x>0) // jangan lupa, lupakan asumsi untuk kembali ke semula
\end{eulerprompt}
\begin{euleroutput}
  
                                 [x > 0]
  
\end{euleroutput}
\begin{eulerprompt}
>&forget(x<0)
\end{eulerprompt}
\begin{euleroutput}
  
                                 [x < 0]
  
\end{euleroutput}
\begin{eulerprompt}
>&facts()
\end{eulerprompt}
\begin{euleroutput}
  
          [kind(sinh, one_to_one), kind(log, one_to_one), 
                          kind(tanh, one_to_one), kind(log, increasing)]
  
\end{euleroutput}
\begin{eulercomment}
Selain dengan showev('limit...) kita juga dapat mencari turunan dengan
menyederhanakan pembilang seperti di awal
\end{eulercomment}
\begin{eulerprompt}
>p &= expand(f(x+h)-f(x))|simplify;  $p
>q &=ratsimp(p/h); $q 
>$limit(q,h,0)
\end{eulerprompt}
\begin{euleroutput}
  Answering "Is x an integer?" with "integer"
  Answering "Is x an integer?" with "integer"
  Answering "Is x an integer?" with "integer"
  Answering "Is x an integer?" with "integer"
  Answering "Is x an integer?" with "integer"
  Maxima is asking
  Acceptable answers are: yes, y, Y, no, n, N, unknown, uk
  Is x an integer?
  
  Use assume!
  Error in:
   $limit(q,h,0) ...
               ^
\end{euleroutput}
\begin{eulercomment}
Mencari turunan dari\\
\end{eulercomment}
\begin{eulerformula}
\[
f(x)=\frac{1}{x}
\]
\end{eulerformula}
\begin{eulerprompt}
>$showev('limit((1/(x+h)-1/x)/h,h,0)) // turunan 1/x
>plot2d("-x^(-2)", color=red):
\end{eulerprompt}
\begin{eulercomment}
\end{eulercomment}
\eulersubheading{Turunan fungsi trigonometri}
\begin{eulercomment}
Mencari turunan dari\\
\end{eulercomment}
\begin{eulerformula}
\[
sin(x)
\]
\end{eulerformula}
\begin{eulerprompt}
>$showev('limit((sin(x+h)-sin(x))/h,h,0)) // turunan sin(x)
\end{eulerprompt}
\begin{eulercomment}
Bukti:\\
\end{eulercomment}
\begin{eulerformula}
\[
sin(x+h) = sin(x)cos(h)+cos(x)sin(h)
\]
\end{eulerformula}
\begin{eulerformula}
\[
sin(x+h)-sin(x) = sin(x)cos(h)+cos(x)sin(h)-sin(x)
\]
\end{eulerformula}
\begin{eulerformula}
\[
sin(x+h)-sin(x) = cos(x)sin(h) - sin(x)+sin(x)cos(h)
\]
\end{eulerformula}
\begin{eulerformula}
\[
sin(x+h)-sin(x) = cos(x)sin(h) - sin(x)(1-cos(h))
\]
\end{eulerformula}
\begin{eulerformula}
\[
\lim_{h\to 0}\frac{sin(x+h)-sin(x)}{h}= \lim_{h\to 0}\frac{cos(x)sin(h)}{h}-\lim_{h\to 0}\frac{sin(x)(1-cos(h)}{h}
\]
\end{eulerformula}
\begin{eulerformula}
\[
\lim_{h\to 0}\frac{sin(x+h)-sin(x)}{h}=cos(x)\lim_{h\to 0}\frac{sin(h)}{h}-sin(x)\lim_{h\to 0}\frac{(1-cos(h)}{h}
\]
\end{eulerformula}
\begin{eulerformula}
\[
=cos(x)\times 1 -sin(x) \times 0 =cos(x)
\]
\end{eulerformula}
\begin{eulerprompt}
>p &= expand((sin(x+h)-sin(x)))|simplify; $p
>q &=ratsimp(p/h); $q 
>$limit(q,h,0) // nilai limit sebagai turunan
>plot2d(["sin(x)", "cos(x)"],0, 2*pi, color=[blue,green]):
\end{eulerprompt}
\begin{eulercomment}
Mencari turunan dari\\
\end{eulercomment}
\begin{eulerformula}
\[
tan(x)
\]
\end{eulerformula}
\begin{eulerprompt}
>$showev('limit((tan(x+h)-tan(x))/h,h,0)) // turunan tan(x)
\end{eulerprompt}
\begin{eulercomment}
Bukti:\\
\end{eulercomment}
\begin{eulerformula}
\[
tan(x+h) = \frac {tan (x)+tan (h)}{1-tan(x)tan(h)}
\]
\end{eulerformula}
\begin{eulerformula}
\[
tan(x+h)-tan (x) ={\frac {tan (x)+tan (h)-tan (x)+tan^2(x)tan(h)}{1-tan(x)tan(h)}}
\]
\end{eulerformula}
\begin{eulerformula}
\[
\frac {tan(x+h)-tan (x)}{h} =\frac{ \frac {tan (x)+tan (h)-tan (x)+tan^2(x)tan(h)}{1-tan(x)tan(h)}}{h}
\]
\end{eulerformula}
\begin{eulerformula}
\[
= \frac {tan(h)+tan^2(x) tan(h)}{h(1-tan(x)tan(h)}
\]
\end{eulerformula}
\begin{eulerformula}
\[
= \lim_{h\to 0} \frac {tan(h)+tan^2(x) tan(h)}{h(1-tan(x)tan(h)}
\]
\end{eulerformula}
\begin{eulerformula}
\[
= \lim_{h\to 0} \frac {1 +tan^2(x)}{1-tan (x)tan(h)} \cdot \lim_{h\to 0} \frac{tan (h)}{h}
\]
\end{eulerformula}
\begin{eulerformula}
\[
= \frac {1 +tan^2(x)}{1-tan (x)tan(0)} \cdot 1
\]
\end{eulerformula}
\begin{eulerformula}
\[
= 1+tan^2(x)
\]
\end{eulerformula}
\begin{eulercomment}
\end{eulercomment}
\begin{eulerformula}
\[
1+tan^2(x)=1+ \frac{sin^2(x)}{cos^2(x)}
\]
\end{eulerformula}
\begin{eulerformula}
\[
= \frac{cos^2(x) + sin^2(x)}{cos^2(x)}
\]
\end{eulerformula}
\begin{eulerformula}
\[
= \frac{1}{cos^2(x)}
\]
\end{eulerformula}
\begin{eulerprompt}
>p &= expand((tan(x+h)-tan(x)))|simplify; $p
>q &=ratsimp(p/h); $q 
>$limit(q,h,0) // nilai limit sebagai turunan
\end{eulerprompt}
\begin{eulercomment}
Mencari turunan dari\\
\end{eulercomment}
\begin{eulerformula}
\[
arcsin(x)
\]
\end{eulerformula}
\begin{eulerprompt}
>$showev('limit((asin(x+h)-asin(x))/h,h,0)) // turunan arcsin(x)
\end{eulerprompt}
\begin{eulercomment}
Bukti:

\end{eulercomment}
\begin{eulerformula}
\[
f'(x)= \lim_{h\to 0} \frac {arcsin(x+h)-arcsin(x)}{h}
\]
\end{eulerformula}
\begin{eulercomment}
Misalkan arcsin(x+h)=A dan arcsin(x)=B

Sehingga sin A= x+h dan sin B=x\\
Kurangkan persamaan kedua dengan persamaan pertama

\end{eulercomment}
\begin{eulerformula}
\[
sin(A)-sin(B)=(x+h)-x
\]
\end{eulerformula}
\begin{eulerformula}
\[
sin(A)-sin(B) = h
\]
\end{eulerformula}
\begin{eulercomment}
karena\\
\end{eulercomment}
\begin{eulerformula}
\[
h \to 0, sin(A)-sin(B) \to 0
\]
\end{eulerformula}
\begin{eulercomment}
maka\\
\end{eulercomment}
\begin{eulerformula}
\[
sin(A)\to sin(B),  A \to B, A-B \to 0
\]
\end{eulerformula}
\begin{eulercomment}
sehingga f'(x) menjadi\\
\end{eulercomment}
\begin{eulerformula}
\[
f'(x) = \lim_{A-B \to 0} \frac{A-B}{sin A- sin B}
\]
\end{eulerformula}
\begin{eulercomment}
Identitas trigonometri\\
\end{eulercomment}
\begin{eulerformula}
\[
sin(A)-sin(B) = 2 sin[\frac{A-B}{2}]cos[\frac{A+B}{2}]
\]
\end{eulerformula}
\begin{eulercomment}
\end{eulercomment}
\begin{eulerformula}
\[
f'(x) = \lim_{A-B \to 0} \frac {A-B}{2 sin [\frac{A-B}{2}] cos [\frac{A+B}{2}]}
\]
\end{eulerformula}
\begin{eulerformula}
\[
f'(x) = \lim_{A-B \to 0} \frac {\frac{2(A-B)}{2}}{2 sin [\frac{A-B}{2}] cos [\frac{A+B}{2}]}
\]
\end{eulerformula}
\begin{eulercomment}
ingat bahwa\\
\end{eulercomment}
\begin{eulerformula}
\[
\lim_{x \to 0} \frac{x}{sin (x)}=1
\]
\end{eulerformula}
\begin{eulercomment}
\end{eulercomment}
\begin{eulerformula}
\[
f'(x) = \lim_{A-B \to 0}\frac{\frac{(A-B)}{2}}{sin [\frac{A-B}{2}]} \cdot \frac{2}{2cos [\frac{A+B}{2}]}
\]
\end{eulerformula}
\begin{eulerformula}
\[
f'(x) = 1 \cdot \frac{2}{2cos [\frac{B+B}{2}]}
\]
\end{eulerformula}
\begin{eulerformula}
\[
f'(x) = \frac {1}{cos (B)}
\]
\end{eulerformula}
\begin{eulercomment}
\end{eulercomment}
\begin{eulerformula}
\[
f'(x) = \frac {1}{\sqrt{(cos^2 (B)}}
\]
\end{eulerformula}
\begin{eulerformula}
\[
f'(x) = \frac {1}{\sqrt{(1- sin^2 (B)}}= \frac {1}{\sqrt{(1-x^2}}
\]
\end{eulerformula}
\begin{eulerprompt}
>p &= expand((asin(x+h)-asin(x)))|simplify; $p //pembilang dijabarkan dan disederhanakan
>q &=ratsimp(p/h); $q // ekspresi yang akan dihitung limitnya disederhanakan
>$limit(q,h,0) // nilai limit sebagai turunan
>plot2d(["asin(x)","1/(sqrt(1-x^2))"],-1,1,color=[blue,red]):
\end{eulerprompt}
\begin{eulercomment}
Mencari turunan dari\\
\end{eulercomment}
\begin{eulerformula}
\[
x^2+10
\]
\end{eulerformula}
\begin{eulercomment}
di titik x=5

\end{eulercomment}
\begin{eulerprompt}
>function f(x) &= x^2+10 // definisikan f(x)=x^2+10
\end{eulerprompt}
\begin{euleroutput}
  
                                  2
                                 x  + 10
  
\end{euleroutput}
\begin{eulerprompt}
>function df(x) &= showev('diff(f(x),x)); $df(x)
>$% with x=5
>diff(f,5), diffc(f,5)
\end{eulerprompt}
\begin{euleroutput}
  9.99999999997
  10
\end{euleroutput}
\begin{eulercomment}
diff: Diferensiasi numerik yang pada dasarnya kurang akurat untuk
fungsi umum.

diffc: Menghitung turunan pertama untuk fungsi analitik nyata dengan
menggunakan bagian imajiner dari f(x+ih)/h. Cara ini memungkinkan
untuk mengevaluasi f dengan lebih akurat.


Mencari turunan dari\\
\end{eulercomment}
\begin{eulerformula}
\[
sinh(x)
\]
\end{eulerformula}
\begin{eulerprompt}
>function f(x) &= sinh(x) // definisikan f(x)=sinh(x)
\end{eulerprompt}
\begin{euleroutput}
  
                                 sinh(x)
  
\end{euleroutput}
\begin{eulerprompt}
>function df(x) &= limit((f(x+h)-f(x))/h,h,0); $df(x) // df(x) = f'(x)
\end{eulerprompt}
\begin{eulercomment}
Hasilnya adalah cosh(x), karena

\end{eulercomment}
\begin{eulerformula}
\[
\frac{e^x+e^{-x}}{2}=\cosh(x).
\]
\end{eulerformula}
\begin{eulerprompt}
>plot2d(["f(x)","df(x)"],-pi,pi,color=[blue,red]):
>$showev('diff(f(x),x))
>$% with x=3
>$float(%)
>plot2d(f,0,3.1):
\end{eulerprompt}
\begin{eulercomment}
Mencari turunan dari\\
\end{eulercomment}
\begin{eulerformula}
\[
f(x)=sin(3x^5+7)^2
\]
\end{eulerformula}
\begin{eulerprompt}
>function f(x) &= sin(3*x^5+7)^2
\end{eulerprompt}
\begin{euleroutput}
  
                                 2    5
                              sin (3 x  + 7)
  
\end{euleroutput}
\begin{eulerprompt}
>$showev('diff(f(x),x))
\end{eulerprompt}
\begin{eulercomment}
Kita akan mencari nilai turunan tersebut ketika x=2
\end{eulercomment}
\begin{eulerprompt}
>$% with x = 2
>$float(%)
>diff(f,2), diffc(f,2)
\end{eulerprompt}
\begin{euleroutput}
  Variable or function f not found.
  Error in:
  diff(f,2), diffc(f,2) ...
        ^
\end{euleroutput}
\begin{eulercomment}
diff: Diferensiasi numerik yang pada dasarnya kurang akurat untuk
fungsi umum.

diffc: Menghitung turunan pertama untuk fungsi analitik nyata dengan
menggunakan bagian imajiner dari f(x+ih)/h. Cara ini memungkinkan
untuk mengevaluasi f dengan lebih akurat.

\end{eulercomment}
\begin{eulerprompt}
>plot2d(f,0,3.1) :
\end{eulerprompt}
\begin{euleroutput}
  Variable or function f not found.
  Error in:
  plot2d(f,0,3.1) : ...
          ^
\end{euleroutput}
\begin{eulercomment}
Mencari turunan dari\\
\end{eulercomment}
\begin{eulerformula}
\[
a(x)=5cos(2x)-2xsin(2x)
\]
\end{eulerformula}
\begin{eulerprompt}
>function a(x) &=5*cos(2*x)-2*x*sin(2*x) // mendifinisikan fungsi a
\end{eulerprompt}
\begin{euleroutput}
  
                        5 cos(2 x) - 2 x sin(2 x)
  
\end{euleroutput}
\begin{eulerprompt}
>function da(x) &=diff(a(x),x); $da(x) // da(x) = a'(x)
>$’a(1)=a(1), $float(a(1)), $’a(2)=a(2), $float(a(2)) // nilai a(1) dan a(2)
>xp=solve("da(x)",1,2,0) // solusi a'(x)=0 pada interval [1, 2]
\end{eulerprompt}
\begin{euleroutput}
  1.35822987384
\end{euleroutput}
\begin{eulerprompt}
>da(xp), a(xp) // cek bahwa a'(xp)=0 dan nilai ekstrim di titik tersebut
\end{eulerprompt}
\begin{euleroutput}
  0
  -5.67530133759
\end{euleroutput}
\begin{eulerprompt}
>plot2d(["a(x)","da(x)"],0,2*pi,color=[blue,red]): //grafik fungsi dan turunannya
\end{eulerprompt}
\begin{eulercomment}
Pada titik puncak grafik a(x), nilai turunan a'(x) akan selalu sama
dengan nol. Ini karena gradien garis singgung pada titik puncak adalah
nol.

\end{eulercomment}
\eulersubheading{Turunan fungsi logaritma}
\begin{eulercomment}
Mencari turunan dari\\
\end{eulercomment}
\begin{eulerformula}
\[
f(x)= log(x)
\]
\end{eulerformula}
\begin{eulerprompt}
>$showev('limit((log(x+h)- log(x))/h,h,0)) // turunan log(x)
\end{eulerprompt}
\begin{eulercomment}
Bukti:\\
\end{eulercomment}
\begin{eulerformula}
\[
f'(x) = \lim_{h\to 0} \frac{log(x+h)-log x}{h}
\]
\end{eulerformula}
\begin{eulercomment}
ingat bahwa\\
\end{eulercomment}
\begin{eulerformula}
\[
log(a)-log(b)= log (\frac{a}{b})
\]
\end{eulerformula}
\begin{eulercomment}
\end{eulercomment}
\begin{eulerformula}
\[
f'(x)=\lim_{h\to 0}\frac{ log (\frac{x+h}{x})}{h}
\]
\end{eulerformula}
\begin{eulerformula}
\[
f'(x)=\lim_{h\to 0}\frac{ log (1 + \frac{h}{x})}{h}
\]
\end{eulerformula}
\begin{eulercomment}
misalkan h/x = n sehingga h=nx

\end{eulercomment}
\begin{eulerformula}
\[
f'(x)=\lim_{n\to 0}\frac{ log (1 + n)}{nx}
\]
\end{eulerformula}
\begin{eulerformula}
\[
f'(x)=\lim_{n\to 0}\frac{1}{nx} \cdot log(1+n)
\]
\end{eulerformula}
\begin{eulercomment}
ingat bahwa\\
\end{eulercomment}
\begin{eulerformula}
\[
a log b = log b^a
\]
\end{eulerformula}
\begin{eulercomment}
\end{eulercomment}
\begin{eulerformula}
\[
f'(x)=\lim_{n\to 0} \frac{1}{x} \cdot log(1+n)^\frac{1}{n}
\]
\end{eulerformula}
\begin{eulerformula}
\[
f'(x)= \frac{1}{x}log \lim_{n\to 0}(1+n)^\frac{1}{n}
\]
\end{eulerformula}
\begin{eulercomment}
\end{eulercomment}
\begin{eulerformula}
\[
\lim_{n\to 0}(1+n)^\frac{1}{n} = e
\]
\end{eulerformula}
\begin{eulercomment}
\end{eulercomment}
\begin{eulerformula}
\[
f'(x)= \frac{1}{x}log e
\]
\end{eulerformula}
\begin{eulerformula}
\[
f'(x)=\frac{1}{x}
\]
\end{eulerformula}
\begin{eulercomment}
\end{eulercomment}
\eulersubheading{Turunan fungsi eksponensial}
\begin{eulercomment}
Mencari turunan dari\\
\end{eulercomment}
\begin{eulerformula}
\[
f(x)=e^x
\]
\end{eulerformula}
\begin{eulerprompt}
>$showev('limit((E^(x+h)-E^x)/h,h,0)) // turunan f(x)=e^x
\end{eulerprompt}
\begin{euleroutput}
  Answering "Is x an integer?" with "integer"
  Answering "Is x an integer?" with "integer"
  Answering "Is x an integer?" with "integer"
  Answering "Is x an integer?" with "integer"
  Answering "Is x an integer?" with "integer"
  Maxima is asking
  Acceptable answers are: yes, y, Y, no, n, N, unknown, uk
  Is x an integer?
  
  Use assume!
  Error in:
                                       ^
\end{euleroutput}
\begin{eulercomment}
Maxima bermasalah dengan limit:

\end{eulercomment}
\begin{eulerformula}
\[
\lim_{h\to 0}\frac{e^{x+h}-e^x}{h}.
\]
\end{eulerformula}
\begin{eulercomment}
Oleh karena itu diperlukan trik khusus agar hasilnya benar.
\end{eulercomment}
\begin{eulerprompt}
>$showev('factor(E^(x+h)-E^x))
>$showev('limit(factor((E^(x+h)-E^x)/h),h,0)) // turunan f(x)=e^x
>$showev('limit((E^h-1)/h,h,0))
\end{eulerprompt}
\eulersubheading{Sifat- sifat Turunan}
\begin{eulercomment}
1. Jika f(x)=k dengan k suatu konstanta maka untuk sebarang x, f'x = 0\\
\end{eulercomment}
\begin{eulerttcomment}
   Bukti:
\end{eulerttcomment}
\begin{eulerformula}
\[
f'(x)= \lim_{h\to 0}\frac{f(x+h)-f(x)}{h} = \lim_{h\to 0}\frac{k-k}{h} = \lim_{h\to 0}0=0
\]
\end{eulerformula}
\begin{eulercomment}
2. Jika f(x)=x, maka f'(x)=1\\
\end{eulercomment}
\begin{eulerttcomment}
   Bukti:
\end{eulerttcomment}
\begin{eulerformula}
\[
f'(x)= \lim_{h\to 0}\frac{f(x+h)-f(x)}{h} = \lim_{h\to 0}\frac{x+h-x}{h}=\lim_{h\to 0}\frac {h}{h}=1
\]
\end{eulerformula}
\begin{eulercomment}
3. Jika k suatu konstanta dan f suatu fungsi yang terdiferensialkan,\\
\end{eulercomment}
\begin{eulerttcomment}
   maka (kf)'(x)= kf'(x)
   Bukti:
   Andaikan F(x) = kf(x), maka
\end{eulerttcomment}
\begin{eulerformula}
\[
F'(x)=\lim_{h\to 0}\frac{F(x+h)-F(x)}{h}=\lim_{h\to 0}\frac{kf(x+h)-kf(x)}{h}
\]
\end{eulerformula}
\begin{eulerformula}
\[
    = \lim_{h\to 0}k \frac{f(x+h)-f(x)}{h}=k\lim_{h\to 0}\frac{f(x+h)-f(x)}{h}
\]
\end{eulerformula}
\begin{eulerformula}
\[
= kf'(x)
\]
\end{eulerformula}
\begin{eulercomment}
4. Jika f dan g adalah fungsi-fungsi yang terdiferensial,\\
\end{eulercomment}
\begin{eulerttcomment}
   maka (f+g)'(x)=f'(x)+g'(x)
   Bukti:
\end{eulerttcomment}
\begin{eulerformula}
\[
F'(x)=\lim_{h\to 0}\frac{f(x+h)+g(x+h)-f(x)+g(x)}{h}
\]
\end{eulerformula}
\begin{eulerformula}
\[
=\lim_{h\to 0}[\frac{f(x+h)-f(x)}{h}+\frac{g(x+h)-g(x)}{h}]
\]
\end{eulerformula}
\begin{eulerformula}
\[
=\lim_{h\to 0}\frac{f(x+h)-f(x)}{h}+\lim_{h\to 0}\frac{g(x+h)-g(x)}{h}=f'(x)+g'(x)
\]
\end{eulerformula}
\begin{eulercomment}
5. Andaikan f dan g adalah fungsi yang dapat didiferensialkan,\\
\end{eulercomment}
\begin{eulerttcomment}
   maka (fg)'(x)=f'(x)g(x)+ f(x)g'(x)
   Bukti:
   Andaikan F(x)=f(x)g(x)
\end{eulerttcomment}
\begin{eulerformula}
\[
F'(x)= \lim_{h\to 0}\frac{F(x+h)-F(x)}{h}
\]
\end{eulerformula}
\begin{eulerformula}
\[
= \lim_{h\to 0}\frac{f(x+h)g(x+h)-f(x)g(x)}{h}
\]
\end{eulerformula}
\begin{eulerformula}
\[
= \lim_{h\to 0}\frac{f(x+h)g(x+h)-f(x+h)g(x)+f(x+h)g(x)-f(x)g(x)}{h}
\]
\end{eulerformula}
\begin{eulerformula}
\[
=\lim_{h\to 0}[f(x+h)\frac{g(x+h)-g(x)+g(x)f(x+h)-f(x)}{h}]
\]
\end{eulerformula}
\begin{eulerformula}
\[
=\lim_{h\to 0} f(x+h)=\lim_{h \to 0}\frac{g(x+h)-g(x)}{h}+g(x)\lim_{h\to 0}\frac{f(x+h)-f(x)}{h}
\]
\end{eulerformula}
\begin{eulerformula}
\[
=f(x)g'(x)+g(x)f'(x)
\]
\end{eulerformula}
\begin{eulercomment}
6. Andaikan f dan g adalah fungsi yang dapat didiferensialkan\\
\end{eulercomment}
\begin{eulerttcomment}
   dengan g(x)tidak sama dengan 0, maka:
\end{eulerttcomment}
\begin{eulerformula}
\[
(\frac{f}{g})(x)=\frac{g(x)f'(x)-f(x)g'(x)}{g(x)^2}
\]
\end{eulerformula}
\begin{eulerttcomment}
   Bukti:
   Andaikan F(x)=f(x)/g(x)
\end{eulerttcomment}
\begin{eulerformula}
\[
F'(x)= \lim_{h\to 0}\frac{F(x+h)-F(x)}{h}
\]
\end{eulerformula}
\begin{eulerformula}
\[
= \lim_{h\to 0}\frac{\frac{f(x+h)}{g(x+h}-\frac{f(x)}{g(x)}}{h}
\]
\end{eulerformula}
\begin{eulerformula}
\[
= \lim_{h\to 0}\frac{g(x)f(x+h)-f(x)g(x+h)}{h}\frac{1}{g(x)g(x+h)}
\]
\end{eulerformula}
\begin{eulerformula}
\[
= \lim_{h\to 0}[\frac{g(x)f(x+h)-g(x)f(x)+g(x)f(x)-f(x)g(x+h)}{h}\frac{1}{g(x)g(x+h)}]
\]
\end{eulerformula}
\begin{eulerformula}
\[
=\lim_{h\to 0} [g(x)\frac{f(x+h)-f(x)}{h}-f(x)\frac{g(x+h)-g(x)}{h}]\frac{1}{g(x)g(x+h)}
\]
\end{eulerformula}
\begin{eulerformula}
\[
=[g(x)f'(x)-f(x)g'(x)]\frac{1}{g(x)g(x)}
\]
\end{eulerformula}
\begin{eulerformula}
\[
=\frac{g(x)f'(x)-f(x)g'(x)}{g(x)^2}
\]
\end{eulerformula}
\begin{eulercomment}
\end{eulercomment}
\eulersubheading{Aplikasi turunan}
\begin{eulercomment}
1. Persamaan gerak suatu partikel dinyatakan dengan rumus\\
\end{eulercomment}
\begin{eulerformula}
\[
s=f(t)=(3t+1)^{\frac{1}{2}}
\]
\end{eulerformula}
\begin{eulerttcomment}
   ( s dalam meter dan t dalam detik).
   Kecepatan partikel tersebut pada saat t=8 adalah ... m/detik.
\end{eulerttcomment}
\begin{eulercomment}
\end{eulercomment}
\begin{eulerttcomment}
   Penyelesaian:
\end{eulerttcomment}
\begin{eulerprompt}
>function f(t) &= (3*t+1)^(1/2); $f(t)
>function df(t) &= diff(f(t),t); $df(t)
>$df(8); fraction %
\end{eulerprompt}
\begin{eulerttcomment}
  Jadi, kecepatan partikel tersebut pada saat t=8 adalah 0,3 m/detik
\end{eulerttcomment}
\begin{eulercomment}
2. Sebuah pabrik baju memerlukan x meter kain untuk diproduksi yang
dinyatakan dengan fungsi:\\
\end{eulercomment}
\begin{eulerformula}
\[
P(x)={\frac{1}{3}}x^2-12x+150
\]
\end{eulerformula}
\begin{eulerttcomment}
   (dalam juta rupiah).
   Berapa biaya produksi minimum yang dikeluarkan oleh pabrik baju
\end{eulerttcomment}
\begin{eulercomment}
tersebut?

\end{eulercomment}
\begin{eulerttcomment}
   Penyelesaian:
\end{eulerttcomment}
\begin{eulerprompt}
>function P(x) &= (1/3)*x^2-12*x+150; $P(x)
>function dP(x) &= diff(P(x),x); $dP(x)
\end{eulerprompt}
\begin{eulerttcomment}
   P(x) akan bernilai minimum jika P'(x)=0, maka:
\end{eulerttcomment}
\begin{eulerprompt}
>$dP(x)=0
>$&solve(dP(x),x)
\end{eulerprompt}
\begin{eulerttcomment}
   Dengan demikian, biaya produksinya adalah:
\end{eulerttcomment}
\begin{eulerprompt}
>P(18)
\end{eulerprompt}
\begin{euleroutput}
  42
\end{euleroutput}
\begin{eulerttcomment}
   Jadi, biaya produksi minimum yang dikeluarkan oleh pabrik tersebut
\end{eulerttcomment}
\begin{eulercomment}
adalah 42 juta rupiah.


3. Sebuah peluru ditembakkan ke atas. Jika tinggi h meter setelah t\\
\end{eulercomment}
\begin{eulerttcomment}
   detik dirumuskan dengan
\end{eulerttcomment}
\begin{eulerformula}
\[
h(t)=120t-5t^2
\]
\end{eulerformula}
\begin{eulerttcomment}
   maka berapa tinggi maksimum yang dicapai peluru tersebut?
\end{eulerttcomment}
\begin{eulercomment}
\end{eulercomment}
\begin{eulerttcomment}
   Penyelesaian:
   Diketahui:
\end{eulerttcomment}
\begin{eulerformula}
\[
h(t)=120t-5t^2
\]
\end{eulerformula}
\begin{eulerttcomment}
   Turunan pertama fungsi h adalah
\end{eulerttcomment}
\begin{eulerprompt}
>function h(t) &= 120*t-5*t^2; $h(t)
>function dh(t) &= diff(h(t),t); $dh(t)
\end{eulerprompt}
\begin{eulerttcomment}
   Nilai t akan maksimum saat
\end{eulerttcomment}
\begin{eulerformula}
\[
h'(t)=0
\]
\end{eulerformula}
\begin{eulerprompt}
>$dh(t)=0
>$&solve(dh(t),t)
\end{eulerprompt}
\begin{eulerttcomment}
   Ketinggian maksimum yang dapat dicapai peluru saat t=12, yaitu
\end{eulerttcomment}
\begin{eulerprompt}
>$h(12)
\end{eulerprompt}
\begin{eulerttcomment}
 Jadi, ketinggian maksimum peluru adalah 720 meter.
\end{eulerttcomment}
\begin{eulercomment}
\end{eulercomment}
\eulersubheading{Soal-soal Latihan}
\begin{eulercomment}
1. Cari f'(4) dari fungsi\\
\end{eulercomment}
\begin{eulerformula}
\[
f(x) = \frac {2x-1}{x^2-x}
\]
\end{eulerformula}
\begin{eulerprompt}
>function f(x) &= (2*x -1)/(x^2-x); $f(x)
>function df(x) &= diff(f(x), x); $df(x)
>$df(4)
\end{eulerprompt}
\begin{eulercomment}
2. Cari turunan dari\\
\end{eulercomment}
\begin{eulerformula}
\[
g(x)=x^2.sin(x)
\]
\end{eulerformula}
\begin{eulerprompt}
>function g(x) &= x^2*sin(x)
\end{eulerprompt}
\begin{euleroutput}
  
                                 2
                                x  sin(x)
  
\end{euleroutput}
\begin{eulerprompt}
>function dg(x) &= diff (g(x), x)
\end{eulerprompt}
\begin{euleroutput}
  
                                        2
                          2 x sin(x) + x  cos(x)
  
\end{euleroutput}
\begin{eulercomment}
3. Cari turunan dari\\
\end{eulercomment}
\begin{eulerformula}
\[
(4x-7)^2(2x+3)
\]
\end{eulerformula}
\begin{eulerprompt}
>function f(x) &= (4*x-7)^2*(2*x+3); $f(x)
>function df(x) &= diff(f(x), x); $df(x)
>F &= expand(df(x))|simplify; $F
\end{eulerprompt}
\begin{eulercomment}
4. Temukan turunan dari\\
\end{eulercomment}
\begin{eulerformula}
\[
f(x)=ln(x^3+3x-4)
\]
\end{eulerformula}
\begin{eulercomment}
\begin{eulercomment}
\eulerheading{4. Integral Tak Tentu}
\begin{eulercomment}
EMT dapat digunakan untuk menghitung integral, baik integral tak tentu
maupun integral tentu.

Pada notebook ini akan ditunjukkan perhitungan integral tentu dengan
menggunakan Teorema Dasar Kalkulus:


Fungsi untuk menentukan integral adalah integrate. Fungsi ini dapat
digunakan untuk menentukan, baik integral tentu maupun tak tentu (jika
fungsinya memiliki antiderivatif).


\end{eulercomment}
\eulersubheading{CAKUPAN PEMBAHASAN}
\begin{eulercomment}
Definisi\\
Cara menulis integral pada EMT\\
Sifat-sifat\\
Rumus\\
Kurva

\end{eulercomment}
\eulersubheading{DEFINISI}
\begin{eulercomment}
Integral Tak Tentu adalah bentuk integral yang variabel integrasinya
tidak memiliki batas sehingga integrasi dari sebuah fungsi akan
menghasilkan banyak kemungkinan dan hanya dinyatakan sebagai
penyelesaian umum. Istilah tak tentu berarti bentuk fungsi F(x) memuat
konstanta real sebarang.

Misalkan diketahui suatu fungsi F(x) yang merupakan fungsi umum yang
memiliki sifat F'(x)=f(x), maka integral tak tentu merupakan himpunan
anti turunan F(x) dari f(x) pada interval negatif tak hingga sampai
positif tak hingga yang dinotasikan :\\
\end{eulercomment}
\begin{eulerformula}
\[
F(x) = \int f(x) dx+c
\]
\end{eulerformula}
\begin{eulercomment}
\end{eulercomment}
\eulersubheading{MENULIS INTEGRAL PADA LATEX}
\begin{eulercomment}
Untuk menulis dengan Latex menggunakan fungsi \textbackslash{}int


\end{eulercomment}
\begin{eulerformula}
\[
f(x)= \int x^3 dx
\]
\end{eulerformula}
\begin{eulerformula}
\[
\int 15x^7 dx
\]
\end{eulerformula}
\begin{eulerformula}
\[
\int x^4 dx
\]
\end{eulerformula}
\begin{eulerformula}
\[
\int 7x^3 dx
\]
\end{eulerformula}
\begin{eulerformula}
\[
\int 5x dx
\]
\end{eulerformula}
\begin{eulercomment}
\end{eulercomment}
\eulersubheading{SIFAT INTEGRAL TAK TENTU}
\begin{eulercomment}
1. Sifat Pangkat\\
\end{eulercomment}
\begin{eulerttcomment}
   Jika n adalah sembarang bilangan rasional kecuali -1, maka integral
\end{eulerttcomment}
\begin{eulercomment}
tak tentu dari x\textasciicircum{}n ditulis :\\
\end{eulercomment}
\begin{eulerformula}
\[
\int x^n dx+c=\frac{x^{n+1}}{n+1} +c
\]
\end{eulerformula}
\begin{eulercomment}
Contoh:\\
\end{eulercomment}
\begin{eulerformula}
\[
\int x^2 dx+c=\frac{x^{2+1}}{2+1} +c =\frac{x^3}{3} +c
\]
\end{eulerformula}
\begin{eulercomment}
2. Penjumlahan dan Pengurangan\\
\end{eulercomment}
\begin{eulerformula}
\[
\int[g(x)+f(x)]dx = \int g(x) dx + \int f(x)dx
\]
\end{eulerformula}
\begin{eulerformula}
\[
\int [x^2+4x] dx = \int x^2 dx + \int 4x dx
\]
\end{eulerformula}
\begin{eulerformula}
\[
\int[g(x)-f(x)]dx = \int g(x) dx - \int f(x)dx
\]
\end{eulerformula}
\begin{eulerformula}
\[
\int [x^7-3x] dx = \int x^7 dx - \int 3x dx
\]
\end{eulerformula}
\begin{eulercomment}
3. Perkalian\\
Bayangkan f(x) dan g(x) adalah dua fungsi yang bisa kita hitung
integral tak tentu (artinya kita mencari antiturunannya), dan anggap k
adalah suatu angka tetap (konstanta). Maka aturannya berlaku:

\end{eulercomment}
\begin{eulerformula}
\[
\int kf(x)dx=k \int f(x)dx
\]
\end{eulerformula}
\begin{eulerformula}
\[
\int 4x^2dx=4 \int x^2dx
\]
\end{eulerformula}
\begin{eulercomment}
\end{eulercomment}
\eulersubheading{RUMUS INTEGRAL TAK TENTU}
\begin{eulercomment}
Integral tak tentu berarti nilai atau batasannya belum pasti, sehingga
ada nilai konstanta(c) di dalamnya. Jadi rumus dasar integral tak
tentu adalah :
\end{eulercomment}
\begin{eulerprompt}
>$showev('integrate(x^n,x)+c)
\end{eulerprompt}
\begin{euleroutput}
  Answering "Is n equal to -1?" with "no"
\end{euleroutput}
\begin{eulerprompt}
>$showev('integrate(x^-1,x)+c)
\end{eulerprompt}
\begin{eulercomment}
Integral dari fungsi\\
\end{eulercomment}
\begin{eulerformula}
\[
f(x)=x^{-1}
\]
\end{eulerformula}
\begin{eulercomment}
menghasilkan fungsi ln karena berasal dari teorema kalkulus

\end{eulercomment}
\eulersubheading{KURVA FUNGSI INTEGRAL/ANTITURUNAN}
\begin{eulercomment}
Kurva fungsi antiturunan adalah kurva yang menggambarkan hubungan
antara suatu fungsi dan antiturunannya. Antiturunan, juga dikenal
sebagai integral.

Jangan lupa untuk menuliskan terhadap variabel apa suatu fungsi
tersebut diintegralkan

1. kurva antiturunan dari fungsi aljabar\\
\end{eulercomment}
\begin{eulerformula}
\[
 f(x)=4x^3 dx
\]
\end{eulerformula}
\begin{eulercomment}
2. kurva antiderivatif fungsi trigonometri dengan\\
\end{eulercomment}
\begin{eulerformula}
\[
f(x)= cos(x)
\]
\end{eulerformula}
\begin{eulercomment}
3. kurva antiderivatif dari fungsi eksponensial dengan\\
\end{eulercomment}
\begin{eulerformula}
\[
f(x)=x e^x
\]
\end{eulerformula}
\begin{eulerprompt}
>$showev('integrate(x*E^x,x)+c)
>plot2d(["x*E^x","(x-1)E^x+1","(x-1)E^x+2","(x-1)E^x+3","(x-1)E^x+4"]):
\end{eulerprompt}
\begin{eulercomment}
4. kurva antiderivatif dari fungsi logaritma dengan\\
\end{eulercomment}
\begin{eulerformula}
\[
f(x)=log(4x)
\]
\end{eulerformula}
\begin{eulerprompt}
>$showev('integrate(log(4*x),x)+c)
\end{eulerprompt}
\begin{eulercomment}
\begin{eulercomment}
\eulerheading{Integral Tentu}
\begin{eulercomment}
\end{eulercomment}
\eulersubheading{Pengertian Integral Tentu}
\begin{eulercomment}
Kata “tentu” di sini bermakna sudah pasti atau sudah ditentukan. Oleh
karena itu, Integral tentu adalah integral yang sudah ditentukan
batasan nilai awal dan akhirnya. Batas dari integral tentu adalah a
sampai b atau batas atas sampai batas bawah.


\end{eulercomment}
\eulersubheading{Rumus Integral Tentu}
\begin{eulercomment}
\end{eulercomment}
\begin{eulerformula}
\[
\int_a^b f(x)\ dx = F(b)-F(a), \quad \text{ dengan  } F'(x) = f(x).
\]
\end{eulerformula}
\begin{eulercomment}
Contoh:\\
carilah\\
\end{eulercomment}
\begin{eulerformula}
\[
\int_0^\pi sin(x)\ dx
\]
\end{eulerformula}
\begin{eulerprompt}
>$showev('integrate(sin(x),x,a,b))
>$showev('integrate(sin(x),x,0,pi))
>plot2d("sin(x)",0,pi):
\end{eulerprompt}
\begin{eulercomment}
Carilah\\
\end{eulercomment}
\begin{eulerformula}
\[
\int_0^2 x^2 \sqrt{2x+1} dx
\]
\end{eulerformula}
\begin{eulerprompt}
>$showev('integrate(x^2*sqrt(2*x+1),x))
>$showev('integrate(x^2*sqrt(2*x+1),x,0,2))
>$showev('integrate(x^2*sqrt(2*x+1),x,0,2))$ratsimp(%)
>$showev('integrate((sin(sqrt(x)+a)*E^sqrt(x))/sqrt(x),x,0,pi^2))
>$factor(%)
>function map f(x) &= E^(-x^2)
\end{eulerprompt}
\begin{euleroutput}
  
                                      2
                                   - x
                                  E
  
\end{euleroutput}
\begin{eulerprompt}
>$showev('integrate(f(x),x))
\end{eulerprompt}
\begin{eulercomment}
Fungsi f tidak memiliki antiturunan, integralnya masih memuat integral
lain.

\end{eulercomment}
\begin{eulerformula}
\[
erf(x) = \int \frac{e^{-x^2}}{\sqrt{\pi}} \ dx.
\]
\end{eulerformula}
\begin{eulercomment}
Kita tidak dapat menggunakan teorema Dasar kalkulus untuk menghitung
integral tentu fungsi tersebut jika semua batasnya berhingga. Dalam
hal ini dapat digunakan metode numerik (rumus kuadratur).

Misalkan kita akan menghitung:

maxima: 'integrate(f(x),x,0,pi)
\end{eulercomment}
\begin{eulerprompt}
>x=0:0.1:pi-0.1; plot2d(x,f(x+0.1),>bar); plot2d("f(x)",0,pi,>add):
\end{eulerprompt}
\begin{eulercomment}
Integral tentu

maxima: 'integrate(f(x),x,0,pi)

dapat dihampiri dengan jumlah luas persegi-persegi panjang di bawah
kurva y=f(x) tersebut. Langkah-langkahnya adalah sebagai berikut.
\end{eulercomment}
\begin{eulerprompt}
>t &= makelist(a,a,0,pi-0.1,0.1);
\end{eulerprompt}
\begin{eulerttcomment}
 mendefinisikan t sebagai list untuk menyimpan nilai-nilai x
\end{eulerttcomment}
\begin{eulerprompt}
>fx &= makelist(f(t[i]+0.1),i,1,length(t)); 
\end{eulerprompt}
\begin{eulercomment}
simpan nilai-nilai f(x) dan jangan menggunakan x sebagai list

Hasilnya adalah:

maxima: 'integrate(f(x),x,0,pi) = 0.1*sum(fx[i],i,1,length(f(x)))

Jumlah tersebut diperoleh dari hasil kali lebar sub-subinterval (=0.1)
dan jumlah nilai-nilai f(x) untuk x = 0.1, 0.2, 0.3, ..., 3.2.
\end{eulercomment}
\begin{eulerprompt}
>0.1*sum(f(x+0.1))
\end{eulerprompt}
\begin{euleroutput}
  0.836219610253
\end{euleroutput}
\begin{eulercomment}
Untuk mendapatkan nilai integral tentu yang mendekati nilai
sebenarnya, lebar sub-intervalnya dapat diperkecil lagi, sehingga
daerah di bawah kurva tertutup semuanya, misalnya dapat digunakan
lebar subinterval 0.001.

\end{eulercomment}
\begin{eulerprompt}
>x=0:0.001:pi-0.1; plot2d(x,f(x+0.1),>bar); plot2d("f(x)",0,pi,>add):
\end{eulerprompt}
\begin{eulercomment}
Berikut adalah contoh lain fungsi yang tidak memiliki antiderivatif,
sehingga integral tentunya hanya dapat dihitung dengan metode numerik.
\end{eulercomment}
\begin{eulerprompt}
>function f(x) &= x^x
\end{eulerprompt}
\begin{euleroutput}
  
                                     x
                                    x
  
\end{euleroutput}
\begin{eulerprompt}
>$showev('integrate(f(x),x,0,1))
>x=0:0.1:1-0.01; plot2d(x,f(x+0.01),>bar); plot2d("f(x)",0,1,>add):
\end{eulerprompt}
\begin{eulercomment}
Karena maxima gagal menghitung integral tentu tersebut secara langsung
menggunakan perintah integrate. Untuk mendapat hasil atau pendekatan
nilai integral tentu tersebut kita lakukan seperti contoh sebelumnya.
\end{eulercomment}
\begin{eulerprompt}
>t &= makelist(a,a,0,1-0.01,0.01);
>fx &= makelist(f(t[i]+0.01),i,1,length(t));
\end{eulerprompt}
\begin{eulercomment}
maxima: 'integrate(f(x),x,0,1) = 0.01*sum(fx[i],i,1,length(fx))
\end{eulercomment}
\begin{eulerprompt}
>function f(x) &= sin(3*x^5+7)^2; $f(x)
>integrate(f,0,1)
\end{eulerprompt}
\begin{euleroutput}
  0.542581176074
\end{euleroutput}
\eulersubheading{Sifat - Sifat Integral Tentu}
\begin{eulercomment}
1. Linearitas\\
sifat 1: Jika c adalah konstanta, maka:\\
\end{eulercomment}
\begin{eulerformula}
\[
\int_a^b c f(x) dx = c \int_a^b f(x) dx
\]
\end{eulerformula}
\begin{eulercomment}
Sifat 2: Jika f(x) dan g(x) adalah dua fungsi yang terintegralkan pada
interval [a, b], maka:\\
\end{eulercomment}
\begin{eulerformula}
\[
\int_a^b (f(x) + g(x)) dx = \int_a^b f(x) dx + \int_a^b g(x) dx
\]
\end{eulerformula}
\begin{eulercomment}
2. Interval

Sifat 3: Jika a \textless{} c \textless{} b, maka:\\
\end{eulercomment}
\begin{eulerformula}
\[
\int_a^b f(x) dx = \int_a^c f(x) dx + \int_c^b f(x) dx
\]
\end{eulerformula}
\begin{eulercomment}
3. Fungsi Genap dan Ganjil

Fungsi Genap: Jika f(x) adalah fungsi genap (yaitu, f(-x) = f(x)),
maka:\\
\end{eulercomment}
\begin{eulerformula}
\[
\int_{-a}^a f(x) dx = 2 \int_0^a f(x) dx
\]
\end{eulerformula}
\begin{eulercomment}
Fungsi Ganjil: Jika f(x) adalah fungsi ganjil (yaitu, f(-x) = -f(x)),
maka:\\
\end{eulercomment}
\begin{eulerformula}
\[
\int_{-a}^a f(x) dx = 0
\]
\end{eulerformula}
\begin{eulercomment}
4. Batas Integral Bertukar\\
\end{eulercomment}
\begin{eulerformula}
\[
\int_b^a f(x) dx = -\int_a^b f(x) dx
\]
\end{eulerformula}
\begin{eulercomment}
\end{eulercomment}
\eulersubheading{Aplikasi Integral Tentu pada Panjang Kurva}
\begin{eulercomment}
Sebuah jalan tol memiliki bentuk lengkungan yang dapat dimodelkan
dengan persamaan y = x\textasciicircum{}2 dari titik x = 0 hingga x = 2 (dalam
kilometer). Berapakah panjang jalan tol tersebut

Penyelesaian:\\
\end{eulercomment}
\begin{eulerformula}
\[
s = \int_0^2 \sqrt{1 + (2x)^2} dx
\]
\end{eulerformula}
\begin{eulerprompt}
>$showev('integrate(sqrt(1 + (2*x)^2),x,0,2))
>$float(%)
\end{eulerprompt}
\begin{eulercomment}
Hitunglah panjang kurva berikut ini dan luas daerah di dalam kurva
tersebut.

\end{eulercomment}
\begin{eulerformula}
\[
\gamma(t) = (r(t) \cos(t), r(t) \sin(t))
\]
\end{eulerformula}
\begin{eulercomment}
dengan

\end{eulercomment}
\begin{eulerformula}
\[
r(t) = 1 + \dfrac{\sin(3t)}{2},\quad 0\le t\le 2\pi.
\]
\end{eulerformula}
\begin{eulerprompt}
>t=linspace(0,2pi,1000); r=1+sin(3*t)/2; x=r*cos(t); y=r*sin(t); ...
>>plot2d(x,y,>filled,fillcolor=red,style="/",r=1.5): // Kita gambar kurvanya terlebih
\end{eulerprompt}
\begin{euleroutput}
  Syntax error in expression, or unfinished expression!
  Error in:
  ... i,1000); r=1+sin(3*t)/2; x=r*cos(t); y=r*sin(t); >plot2d(x,y,> ...
                                                       ^
\end{euleroutput}
\begin{eulerprompt}
>function r(t) &= 1+sin(3*t)/2; $'r(t)=r(t)
>function fx(t) &= r(t)*cos(t); $'fx(t)=fx(t)
>function fy(t) &= r(t)*sin(t); $'fy(t)=fy(t)
>function ds(t) &= trigreduce(radcan(sqrt(diff(fx(t),t)^2+diff(fy(t),t)^2))); $'ds(t)=ds(t)
\end{eulerprompt}
\begin{euleroutput}
  Maxima said:
  diff: second argument must be a variable; found errexp1
   -- an error. To debug this try: debugmode(true);
  
  Error in:
  ... e(radcan(sqrt(diff(fx(t),t)^2+diff(fy(t),t)^2))); $'ds(t)=ds(t ...
                                                       ^
\end{euleroutput}
\begin{eulerprompt}
>$integrate(ds(x),x,0,2*pi) //panjang (keliling) kurva
\end{eulerprompt}
\begin{eulercomment}
Maxima gagal melakukan perhitungan eksak integral tersebut.

Berikut kita hitung integralnya secara numerik dengan perintah EMT.
\end{eulercomment}
\begin{eulerprompt}
>integrate("ds(x)",0,2*pi)
\end{eulerprompt}
\begin{euleroutput}
  Function ds not found.
  Try list ... to find functions!
  Error in expression: ds(x)
   %mapexpression1:
      return expr(x,args());
  Error in map.
   %evalexpression:
      if maps then return %mapexpression1(x,f$;args());
  gauss:
      if maps then y=%evalexpression(f$,a+h-(h*xn)',maps;args());
  adaptivegauss:
      t1=gauss(f$,c,c+h;args(),=maps);
  Try "trace errors" to inspect local variables after errors.
  integrate:
      return adaptivegauss(f$,a,b,eps*1000;args(),=maps);
\end{euleroutput}
\eulerheading{Barisan dan Deret}
\begin{eulercomment}
Barisan adalah susunan bilangan yang memiliki pola atau karakteristik
tertentu, sedangkan deret adalah hasil penjumlahan dari
anggota-anggota dalam barian tertentu.

Barisan dapat didefinisikan dengan beberapa cara di dalam EMT, di
antaranya:\\
- menggunakan titik dua ":", sama seperti saat mendefinisikan vektor
dengan elemen-elemen beraturan\\
- menggunakan perintah "sequence" dan rumus barisan (suku ke -n);\\
- menggunakan perintah "iterate" atau "niterate";\\
- menggunakan fungsi Maxima "create\_list" atau "makelist" untuk
menghasilkan barisan simbolik;\\
- menggunakan fungsi biasa yang inputnya vektor atau barisan;\\
- menggunakan fungsi rekursif.

EMT menyediakan beberapa perintah (fungsi) terkait barisan, yakni:\\
- sum: menghitung jumlah semua elemen suatu barisan\\
- cumsum: jumlah kumulatif suatu barisan\\
- differences: selisih antar elemen-elemen berturutan

EMT juga dapat digunakan untuk menghitung jumlah deret berhingga
maupun deret tak hingga, dengan menggunakan perintah (fungsi) "sum".
Perhitungan dapat dilakukan secara numerik maupun simbolik dan eksak.

Contoh perhitungan barisan dan deret menggunakan EMT.

\end{eulercomment}
\eulersubheading{Menggunakan titik dua}
\begin{eulerprompt}
>1:15
\end{eulerprompt}
\begin{euleroutput}
  [1,  2,  3,  4,  5,  6,  7,  8,  9,  10,  11,  12,  13,  14,  15]
\end{euleroutput}
\begin{eulerprompt}
>sum(1:3:30)
\end{eulerprompt}
\begin{euleroutput}
  145
\end{euleroutput}
\begin{eulerprompt}
>cumsum(1:3:30)
\end{eulerprompt}
\begin{euleroutput}
  [1,  5,  12,  22,  35,  51,  70,  92,  117,  145]
\end{euleroutput}
\eulersubheading{Menggunakan perintah "sequence" dan rumus barisan (suku ke -n)}
\begin{eulercomment}
Untuk barisan yang lebih kompleks dapat digunakan fungsi "sequence()".
Fungsi ini menghitung nilai-nilai x[n] dari semua nilai sebelumnya,
x[1],...,x[n-1] yang diketahui.\\
Berikut adalah contoh barisan Fibonacci.

\end{eulercomment}
\begin{eulerformula}
\[
x_n = x_{n-1}+x_{n-2}, \quad x_1=1, \quad x_2 =1
\]
\end{eulerformula}
\begin{eulerprompt}
>sequence("x[n-1]+x[n-2]",[1,1],15) //suku ke satu :1, suku ke dua :1
\end{eulerprompt}
\begin{euleroutput}
  [1,  1,  2,  3,  5,  8,  13,  21,  34,  55,  89,  144,  233,  377,  610]
\end{euleroutput}
\begin{eulerprompt}
>plot2d(sequence("x[n-1]+x[n-2]",[1,1],15)):
>sequence("x[n-1]+x[n-2]",[1,1],15)^(1/(1:15))
\end{eulerprompt}
\begin{euleroutput}
  [1,  1,  1.25992,  1.31607,  1.37973,  1.41421,  1.44256,  1.46311,
  1.47967,  1.49292,  1.50389,  1.51309,  1.52091,  1.52765,  1.53352]
\end{euleroutput}
\begin{eulerprompt}
>plot2d(sequence("x[n-1]+x[n-2]",[1,1],15)^(1/(1:15))):
\end{eulerprompt}
\begin{eulercomment}
\end{eulercomment}
\eulersubheading{Menggunakan perintah "iterate" atau "niterate"}
\begin{eulercomment}
Dalam ilmu matematika, iterasi dapat diartikan sebagai proses atau
metode yang digunakan secara berulang-ulang (pengulangan) dalam
menyelesaikan permasalahan matematik.\\
EMT menyediakan fungsi iterate("g(x)", x0, n) untuk melakukan iterasi

\end{eulercomment}
\begin{eulerformula}
\[
x_{k+1}=g(x_k), \ x_0=x_0, k= 1, 2, 3, ..., n.
\]
\end{eulerformula}
\begin{eulercomment}
Berikut contoh penggunaan iterasi

Menunjukkan pertumbuhan dari nilai awal 1000 dengan laju pertambahan
5\%, selama 10 periode.
\end{eulercomment}
\begin{eulerprompt}
>q=1.05; iterate("x*q",1000,n=10)'
\end{eulerprompt}
\begin{euleroutput}
           1000 
           1050 
         1102.5 
        1157.63 
        1215.51 
        1276.28 
         1340.1 
         1407.1 
        1477.46 
        1551.33 
        1628.89 
\end{euleroutput}
\eulerheading{Spiral Theodorus}
\begin{eulercomment}
konstruksi segitiga siku-siku yang berkesinambungan menjadi spiral
atau dalam kata lain sebuah spiral yang dibangun dari segitiga
siku-siku yang berurutan. Setiap segitiga siku-siku yang baru dibuat
dengan menghubungkan sisi miring segitiga sebelumnya dengan sisi alas
yang panjangnya 1 satuan.

Spiral Theodorus (spiral segitiga siku-siku) dapat digambar secara
rekursif. Rumus rekursifnya adalah:

\end{eulercomment}
\begin{eulerformula}
\[
x_n = \left( 1 + \frac{i}{\sqrt{n-1}} \right) \, x_{n-1}, \quad x_1=1,
\]
\end{eulerformula}
\begin{eulercomment}
yang menghasilkan barisan bilangan kompleks.
\end{eulercomment}
\begin{eulerprompt}
>function g(n) := 1+I/sqrt(n)
>x=sequence("g(n-1)*x[n-1]",1,17); plot2d(x,r=3.5);...
>textbox(latex("Spiral\(\backslash\) Theodorus"),0.4):
\end{eulerprompt}
\begin{eulercomment}
Rekursinya dapat dijalankan sebanyak 17 untuk menghasilkan barisan 17
bilangan kompleks, kemudian digambar bilangan-bilangan kompleksnya.

\begin{eulercomment}
\eulerheading{Limit Barisan Limit barisan melambangkan nilai mutlak untuk bilangan}
\begin{eulercomment}
riil dan nilai modulus  untuk bilangan kompleks.\\
Definisi formal:\\
\end{eulercomment}
\begin{eulerformula}
\[
\forall \varepsilon > 0, \exists k \in \mathbb{N} : (\forall n \in \mathbb{N}, n \geq k \implies |x_n - L| <\varepsilon)
\]
\end{eulerformula}
\begin{eulercomment}
Dapat dinotasikan sebagai\\
\end{eulercomment}
\begin{eulerformula}
\[
\lim_{n \to \infty}x_n=L
\]
\end{eulerformula}
\begin{eulercomment}
Contoh:\\
1.\\
\end{eulercomment}
\begin{eulerformula}
\[
\lim_{n \to \infty}\frac{1}{n}
\]
\end{eulerformula}
\begin{eulerprompt}
>$showev('limit(1/n,n,inf))
\end{eulerprompt}
\begin{eulercomment}
\begin{eulercomment}
\eulerheading{Kekonvergenan}
\begin{eulercomment}
Iterasi sampai konvergen merupakan proses yang terus berulang sampai
mendapat nilai yang stabil atau tidak berubah secara signifikan.
Tetapi, apabila iterasinya tidak konvergen setelah ditunggu lama, Kita
dapat menghentikannya dengan menekan tombol [ESC].
\end{eulercomment}
\begin{eulerprompt}
>iterate("cos(x)",1) // iterasi x(n+1)=cos(x(n)), dengan x(0)=1.
\end{eulerprompt}
\begin{euleroutput}
  0.739085133216
\end{euleroutput}
\begin{eulercomment}
Iterasi tersebut konvergen ke penyelesaian persamaan

\end{eulercomment}
\begin{eulerformula}
\[
x = \cos(x).
\]
\end{eulerformula}
\begin{eulercomment}
Iterasi ini juga dapat dilakukan pada interval, hasilnya adalah
barisan interval yang memuat akar tersebut.

Jika interval hasil tersebut sedikit diperlebar, akan terlihat bahwa
interval tersebut memuat akar persamaan x=cos(x).
\end{eulercomment}
\begin{eulerprompt}
>h=expand(hasil,100), cos(h) << h
\end{eulerprompt}
\begin{euleroutput}
  Variable or function hasil not found.
  Error in:
  h=expand(hasil,100), cos(h) << h ...
                ^
\end{euleroutput}
\eulerheading{Deret Taylor}
\begin{eulercomment}
Deret Taylor suatu fungsi f yang diferensiabel sampai tak hingga di
sekitar x=a adalah:

\end{eulercomment}
\begin{eulerformula}
\[
f(x) = \sum_{k=0}^\infty \frac{(x-a)^k f^{(k)}(a)}{k!}.
\]
\end{eulerformula}
\begin{eulercomment}
Diberikan contoh fungsi exponensial yang didekati dengan Deret Taylor

Menghitung exp(x) untuk x=4 yang di potong pada k=10
\end{eulercomment}
\begin{eulerprompt}
>x := 4; k := 10; hasil := round(exp(x),k)
\end{eulerprompt}
\begin{euleroutput}
  54.5981500331
\end{euleroutput}
\begin{eulercomment}
Bukti :

\end{eulercomment}
\begin{eulerformula}
\[
e^x = \frac{(x)^{10}}{(3628800)} + \frac{(x)^{9}}{(362880)} + \frac{(x)^{8}}{(40320)} + \frac{(x)^{7}}{(5040)} + \frac{(x)^{6}}{(720)} + \frac{(x)^{5}}{(120)} + \frac{(x)^{4}}{(24)} + \frac{(x)^{3}}{(6)} + \frac{(x)^{2}}{(2)} + x + 1
\]
\end{eulerformula}
\begin{eulercomment}
untuk x=4

\end{eulercomment}
\begin{eulerformula}
\[
e^4 = \frac{(4)^{10}}{(3628800)} + \frac{(4)^{9}}{(362880)} + \frac{(4)^{8}}{(40320)} + \frac{(4)^{7}}{(5040)} + \frac{(4)^{6}}{(720)} + \frac{(4)^{5}}{(120)} + \frac{(4)^{4}}{(24)} + \frac{(4)^{3}}{(6)} + \frac{(4)^{2}}{(2)} + 4 + 1
\]
\end{eulerformula}
\begin{eulercomment}
\end{eulercomment}
\begin{eulerformula}
\[
e^4 = \frac{(4)^{10}}{(3628800)} + \frac{(4)^{9}}{(362880)} + \frac{(4)^{8}}{(40320)} + \frac{(4)^{7}}{(5040)} + \frac{(4)^{6}}{(720)} + \frac{(4)^{5}}{(120)} + \frac{(4)^{4}}{(24)} + \frac{(4)^{3}}{(6)} + \frac{(4)^{2}}{(2)} + 5
\]
\end{eulerformula}
\begin{eulercomment}
\end{eulercomment}
\begin{eulerformula}
\[
e^4 = 54,4431
\]
\end{eulerformula}
\begin{eulercomment}
\begin{eulercomment}
\eulerheading{Soal Latihan}
\begin{eulercomment}
1. Tentukan suku ke-5 dari deret berikut\\
\end{eulercomment}
\begin{eulerformula}
\[
[5, 25, 125]
\]
\end{eulerformula}
\begin{eulercomment}
2. Hitunglah jumlah suku ke-7 dari deret yang dihasilkan no 1

3. Hitunglah limit barisan dari\\
\end{eulercomment}
\begin{eulerformula}
\[
\lim_{x\to\infty}\sqrt{\frac{4x+1}{x}}
\]
\end{eulerformula}
\begin{eulercomment}
\begin{eulercomment}
\eulerheading{Fungsi Multivariabel}
\begin{eulercomment}
Fungsi multivariabel adalah sebuah konsep dalam matematika yang
menggambarkan hubungan antara satu variabel output dengan dua atau
lebih variabel input. Berbeda dengan fungsi satu variabel yang hanya
melibatkan satu variabel bebas, fungsi multivariabel melibatkan
beberapa variabel bebas yang secara bersama-sama menentukan nilai dari
variabel terikat. Fungsi multivariabel biasanya digunakan untuk
menggambarkan hubungan yang kompleks dalam bidang seperti fisika,
ekonomi, teknik, dan berbagai ilmu lainnya.

Secara umum, jika f adalah sebuah fungsi multivariabel, maka bentuk
umumnya dapat dinyatakan sebagai:\\
\end{eulercomment}
\begin{eulerformula}
\[
z=f(x_1,x_2,...,x_n)
\]
\end{eulerformula}
\begin{eulercomment}
dimana x1,x2,...,xn adalah variabel bebas (input) menunjukkan jumlah
variabel dan z adalah variable terikat (output).

contoh :\\
Luas Persegi Panjang: Luas persegi panjang (L) adalah fungsi dari
panjang (p) dan lebar (l). Kita dapat menuliskannya sebagai L(p, l) =
p * l. Di sini, L adalah variabel terikat, sedangkan p dan l adalah
variabel bebas.

Diberikan fungsi:\\
\end{eulercomment}
\begin{eulerformula}
\[
x^2+2xy+y^2
\]
\end{eulerformula}
\begin{eulerprompt}
>function f(x,y) &= x^2 + 2*x*y + y^2
\end{eulerprompt}
\begin{euleroutput}
  
                              2            2
                             y  + 2 x y + x
  
\end{euleroutput}
\begin{eulercomment}
diff(f,x): Bagian ini untuk menghitung turunan numerik.

f: Merupakan representasi dari fungsi yang ingin kita turunkan. Fungsi
ini bisa berupa fungsi anonim, fungsi yang telah didefinisikan
sebelumnya, atau bahkan persamaan matematika yang kompleks.

x: Menunjukkan variabel terhadap mana kita akan menurunan fungsi f.
Artinya, kita akan mencari laju perubahan fungsi f terhadap perubahan
nilai x.

\&=: mendefinikan fungsi, seperti yang sudah dijelaskan kemarin
\end{eulercomment}
\begin{eulerprompt}
>fx &= diff(f(x,y),x)
\end{eulerprompt}
\begin{euleroutput}
  
                                2 y + 2 x
  
\end{euleroutput}
\begin{eulercomment}
ini merupakan fungsi simbolik jadi harus didefinisikan dengan \&=



\begin{eulercomment}
\eulerheading{Turunan fungsi multivariabel}
\begin{eulercomment}
Diberikan fungsi:\\
\end{eulercomment}
\begin{eulerformula}
\[
\sqrt{x}+y^2+2xz
\]
\end{eulerformula}
\begin{eulerprompt}
>function f(x,y,z)&= sqrt(x)+y^2+2*x*z
\end{eulerprompt}
\begin{euleroutput}
  
                                    2
                           2 x z + y  + sqrt(x)
  
\end{euleroutput}
\begin{eulerprompt}
>fx &= diff(f(x,y,z),x)
\end{eulerprompt}
\begin{euleroutput}
  
                                       1
                             2 z + ---------
                                   2 sqrt(x)
  
\end{euleroutput}
\begin{eulerprompt}
>fy &= diff(f(x,y,z),y)
\end{eulerprompt}
\begin{euleroutput}
  
                                   2 y
  
\end{euleroutput}
\begin{eulerprompt}
>fz &= diff(f(x,y,z),z)
\end{eulerprompt}
\begin{euleroutput}
  
                                   2 x
  
\end{euleroutput}
\eulerheading{Grafik }
\begin{eulercomment}
Grafik fungsi multivariabel adalah representasi visual dari fungsi
yang memiliki dua atau lebih variabel input. Untuk fungsi dua variabel
\textdollar{}f(x,y)\textdollar{}, grafiknya berupa permukaan dalam ruang tiga dimensi.\\
Bidang z = 2x + 3y
\end{eulercomment}
\begin{eulerprompt}
>plot3d("2*x + 3*y", -2,2, -2,2):
>title("Kontur z = 2x + 3y"):
\end{eulerprompt}
\begin{eulercomment}
Paraboloid z = x\textasciicircum{}2 + y\textasciicircum{}2
\end{eulercomment}
\begin{eulerprompt}
>plot3d("x^2 + y^2", -8,2, -8,2):
\end{eulerprompt}
\begin{eulercomment}
Hyperbolic Paraboloid\\
\end{eulercomment}
\begin{eulerformula}
\[
z = x^2 - y^2
\]
\end{eulerformula}
\begin{eulerprompt}
>plot3d("x^2 - y^2", -2,2, -2,2):
\end{eulerprompt}
\begin{eulercomment}
z = sin(x)cos(y)
\end{eulercomment}
\begin{eulerprompt}
>plot3d("sin(x)*cos(y)", -pi,pi, -pi,pi):
\end{eulerprompt}
\begin{eulercomment}
Distribusi normal bivariat
\end{eulercomment}
\begin{eulerprompt}
>plot3d("(1/(2*pi))*exp(-(x^2+y^2)/2)", 1,3, 1,3):
>title("Distribusi Normal Bivariat")
\end{eulerprompt}
\eulerheading{Turunan }
\begin{eulercomment}
Turunan fungsi multivariabel merupakan perluasan konsep turunan dari
fungsi satu variabel ke fungsi dengan dua atau lebih variabel.\\
menghitung turunan parsial
\end{eulercomment}
\begin{eulerprompt}
>f &= x^4 + 6*x*y + y^3
\end{eulerprompt}
\begin{euleroutput}
  
                              3            4
                             y  + 6 x y + x
  
\end{euleroutput}
\eulerheading{Integral }
\begin{eulercomment}
Integral fungsi multivariabel adalah perluasan konsep integral dari
fungsi satu variabel ke fungsi dengan dua atau lebih variabel.\\
Contoh sederhana integral lipat dua
\end{eulercomment}
\begin{eulerprompt}
>$showev ('integrate(integrate(x^2 + y^2, x, 0, 1), y, 0, 2))
\end{eulerprompt}
\begin{eulercomment}
Hasil dari integral ganda ini akan memberikan nilai total area di
bawah permukaan fungsi\\
f(x,y)=x\textasciicircum{}2+y\textasciicircum{}2 di daerah persegi yang ditentukan.


\begin{eulercomment}
\eulerheading{Aplikasinya}
\begin{eulercomment}
1. Misalkan medan listrik di suatu daerah diberikan oleh\\
\end{eulercomment}
\begin{eulerformula}
\[
E(x,y)=x^2 - y
\]
\end{eulerformula}
\begin{eulercomment}
Hitung fluks medan listrik melalui permukaan persegi panjang
[0,1]×[0,2]!

\end{eulercomment}
\begin{eulerprompt}
>$showev('integrate(integrate(x^2 - y, x, 0, 1), y, 0, 2))
\end{eulerprompt}
\begin{eulercomment}
kita diberikan medan listrik dalam bentuk vektor\\
\end{eulercomment}
\begin{eulerformula}
\[
E(x,y)=x^2-y
\]
\end{eulerformula}
\begin{eulercomment}
dan diminta untuk menghitung fluks dari medan listrik melalui
permukaan tersebut.\\
\end{eulercomment}
\begin{eulerformula}
\[
x=0-1
\]
\end{eulerformula}
\begin{eulerformula}
\[
y=0-2
\]
\end{eulerformula}
\begin{eulercomment}
2. Misalkan sebuah perusahaan memproduksi dua jenis barang,x dan y.\\
Fungsi keuntungan perusahaan dinyatakan sebagai:


\end{eulercomment}
\begin{eulerformula}
\[
p(x,y)= 50x + 80y - 2x^2 - 3y^2 - 4xy
\]
\end{eulerformula}
\begin{eulercomment}
tentukan banyak barang x dan y yang harus diproduksi agar keuntungan
maksimal!
\end{eulercomment}
\begin{eulerprompt}
>function p(x,y) &= 50*x+80*y-2*x^2-3*y^2-4*x*y
\end{eulerprompt}
\begin{euleroutput}
  
                        2                     2
                   - 3 y  - 4 x y + 80 y - 2 x  + 50 x
  
\end{euleroutput}
\begin{eulerprompt}
>grad &= [diff(p(x,y),x), diff(p(x,y),y)]=0
\end{eulerprompt}
\begin{euleroutput}
  
                 [- 4 y - 4 x + 50, - 6 y - 4 x + 80] = 0
  
\end{euleroutput}
\begin{eulerprompt}
>n &= [diff(px,x), diff(py,y), diff(px,y)]
\end{eulerprompt}
\begin{euleroutput}
  
                                [0, 0, 0]
  
\end{euleroutput}
\begin{eulerprompt}
>&sol := 'solve(grad, n)
\end{eulerprompt}
\begin{euleroutput}
  
          solve([- 4 y - 4 x + 50, - 6 y - 4 x + 80] = 0, [0, 0, 0])
  
\end{euleroutput}
\begin{eulercomment}
\begin{eulercomment}
\eulerheading{Visualisasi dan Perhitungan Geometri dengan EMT}
\begin{eulercomment}
Euler menyediakan beberapa fungsi untuk melakukan visualisasi dan
perhitungan geometri, baik secara numerik maupun analitik (seperti
biasanya tentunya, menggunakan Maxima). Fungsi-fungsi untuk
visualisasi dan perhitungan geometeri tersebut disimpan di dalam file
program "geometry.e", sehingga file tersebut harus dipanggil sebelum
menggunakan fungsi-fungsi atau perintah-perintah untuk geometri.
\end{eulercomment}
\begin{eulerprompt}
>load geometry
\end{eulerprompt}
\begin{euleroutput}
  Numerical and symbolic geometry.
\end{euleroutput}
\eulersubheading{Fungsi-fungsi Geometri}
\begin{eulercomment}
Fungsi-fungsi untuk Menggambar Objek Geometri:

\end{eulercomment}
\begin{eulerttcomment}
  defaultd:=textheight()*1.5: nilai asli untuk parameter d
  setPlotrange(x1,x2,y1,y2): menentukan rentang x dan y pada bidang koordinat
  setPlotRange(r): pusat bidang koordinat (0,0) dan batas-batas sumbu-x dan y adalah -r sd r
  plotPoint (P, "P"): menggambar titik P dan diberi label "P"
  plotSegment (A,B, "AB", d): menggambar ruas garis AB, diberi label "AB" sejauh d
  plotLine (g, "g", d): menggambar garis g diberi label "g" sejauh d
  plotCircle (c,"c",v,d): Menggambar lingkaran c dan diberi label "c"
  plotLabel (label, P, V, d): menuliskan label pada posisi P
\end{eulerttcomment}
\begin{eulercomment}

Fungsi-fungsi Geometri Analitik (numerik maupun simbolik):

\end{eulercomment}
\begin{eulerttcomment}
  turn(v, phi): memutar vektor v sejauh phi
  turnLeft(v):   memutar vektor v ke kiri
  turnRight(v):  memutar vektor v ke kanan
  normalize(v): normal vektor v
  crossProduct(v, w): hasil kali silang vektorv dan w.
  lineThrough(A, B): garis melalui A dan B, hasilnya [a,b,c] sdh. ax+by=c.
  lineWithDirection(A,v): garis melalui A searah vektor v
  getLineDirection(g): vektor arah (gradien) garis g
  getNormal(g): vektor normal (tegak lurus) garis g
  getPointOnLine(g):  titik pada garis g
  perpendicular(A, g):  garis melalui A tegak lurus garis g
  parallel (A, g):  garis melalui A sejajar garis g
  lineIntersection(g, h):  titik potong garis g dan h
  projectToLine(A, g):   proyeksi titik A pada garis g
  distance(A, B):  jarak titik A dan B
  distanceSquared(A, B):  kuadrat jarak A dan B
  quadrance(A, B): kuadrat jarak A dan B
  areaTriangle(A, B, C):  luas segitiga ABC
  computeAngle(A, B, C):   besar sudut <ABC
  angleBisector(A, B, C): garis bagi sudut <ABC
  circleWithCenter (A, r): lingkaran dengan pusat A dan jari-jari r
  getCircleCenter(c):  pusat lingkaran c
  getCircleRadius(c):  jari-jari lingkaran c
  circleThrough(A,B,C):  lingkaran melalui A, B, C
  middlePerpendicular(A, B): titik tengah AB
  lineCircleIntersections(g, c): titik potong garis g dan lingkran c
  circleCircleIntersections (c1, c2):  titik potong lingkaran c1 dan c2
  planeThrough(A, B, C):  bidang melalui titik A, B, C
\end{eulerttcomment}
\begin{eulercomment}

Fungsi-fungsi Khusus Untuk Geometri Simbolik:

\end{eulercomment}
\begin{eulerttcomment}
  getLineEquation (g,x,y): persamaan garis g dinyatakan dalam x dan y
  getHesseForm (g,x,y,A): bentuk Hesse garis g dinyatakan dalam x dan y dengan titik A pada
  sisi positif (kanan/atas) garis
  quad(A,B): kuadrat jarak AB
  spread(a,b,c): Spread segitiga dengan panjang sisi-sisi a,b,c, yakni sin(alpha)^2 dengan
  alpha sudut yang menghadap sisi a.
  crosslaw(a,b,c,sa): persamaan 3 quads dan 1 spread pada segitiga dengan panjang sisi a, b, c.
  triplespread(sa,sb,sc): persamaan 3 spread sa,sb,sc yang memebntuk suatu segitiga
  doublespread(sa): Spread sudut rangkap Spread 2*phi, dengan sa=sin(phi)^2 spread a.
\end{eulerttcomment}
\begin{eulercomment}

\end{eulercomment}
\eulersubheading{Contoh 1: Luas, Lingkaran Luar, Lingkaran Dalam Segitiga}
\begin{eulercomment}
Untuk menggambar objek-objek geometri, langkah pertama adalah menentukan rentang sumbu-sumbu
koordinat. Semua objek geometri akan digambar pada satu bidang koordinat, sampai didefinisikan
bidang koordinat yang baru.
\end{eulercomment}
\begin{eulerprompt}
>setPlotRange(-0.5,2.5,-0.5,2.5); // mendefinisikan bidang koordinat baru 
\end{eulerprompt}
\begin{eulercomment}
Sekarang tetapkan tiga poin dan plot.
\end{eulercomment}
\begin{eulerprompt}
>A=[1,0]; plotPoint(A,"A"); // definisi dan gambar tiga titik
>B=[0,1]; plotPoint(B,"B");
>C=[2,2]; plotPoint(C,"C");
\end{eulerprompt}
\begin{eulercomment}
Kemudian untuk tiga segmen.
\end{eulercomment}
\begin{eulerprompt}
>plotSegment(A,B,"c"); // c=AB
>plotSegment(B,C,"a"); // a=BC
>plotSegment(A,C,"b"); // b=AC
\end{eulerprompt}
\begin{eulercomment}
Fungsi geometri mencakup fungsi untuk membuat garis dan lingkaran.
Format untuk garis adalah [a,b,c], yang mewakili garis dengan
persamaan ax+by=c.
\end{eulercomment}
\begin{eulerprompt}
>lineThrough(B,C) // garis yang melalui B dan C
\end{eulerprompt}
\begin{euleroutput}
  [-1,  2,  2]
\end{euleroutput}
\begin{eulercomment}
Hitung garis tegak lurus melalui A pada BC.
\end{eulercomment}
\begin{eulerprompt}
>h=perpendicular(A,lineThrough(B,C)); // garis h tegak lurus BC melalui A
\end{eulerprompt}
\begin{eulercomment}
Dan persimpangan dengan BC.
\end{eulercomment}
\begin{eulerprompt}
>D=lineIntersection(h,lineThrough(B,C)); // D adalah titik potong h dan BC
\end{eulerprompt}
\begin{eulercomment}
Plot that.
\end{eulercomment}
\begin{eulerprompt}
>plotPoint(D,value=1); // koordinat D ditampilkan
>aspect(1); plotSegment(A,D): // tampilkan semua gambar hasil plot...()
\end{eulerprompt}
\begin{eulercomment}
Hitung luas ABC:

\end{eulercomment}
\begin{eulerformula}
\[
L_{\triangle ABC}= \frac{1}{2}AD.BC.
\]
\end{eulerformula}
\begin{eulerprompt}
>norm(A-D)*norm(B-C)/2 // AD=norm(A-D), BC=norm(B-C)
\end{eulerprompt}
\begin{euleroutput}
  1.5
\end{euleroutput}
\begin{eulercomment}
Bandingkan dengan rumus determinan.
\end{eulercomment}
\begin{eulerprompt}
>areaTriangle(A,B,C) // hitung luas segitiga langusng dengan fungsi
\end{eulerprompt}
\begin{euleroutput}
  1.5
\end{euleroutput}
\begin{eulercomment}
Cara lain menghitung luas segitigas ABC:
\end{eulercomment}
\begin{eulerprompt}
>distance(A,D)*distance(B,C)/2
\end{eulerprompt}
\begin{euleroutput}
  1.5
\end{euleroutput}
\begin{eulercomment}
Sudut di C.
\end{eulercomment}
\begin{eulerprompt}
>degprint(computeAngle(B,C,A))
\end{eulerprompt}
\begin{euleroutput}
  36°52'11.63''
\end{euleroutput}
\begin{eulercomment}
Kemudian menggambarkan lingkaran luar segitiga.
\end{eulercomment}
\begin{eulerprompt}
>c=circleThrough(A,B,C); // lingkaran luar segitiga ABC
>R=getCircleRadius(c); // jari2 lingkaran luar 
>O=getCircleCenter(c); // titik pusat lingkaran c 
>plotPoint(O,"O"); // gambar titik "O"
>plotCircle(c,"Lingkaran luar segitiga ABC"):
\end{eulerprompt}
\begin{eulercomment}
Tampilkan koordinat titik pusat dan jari-jari lingkaran luar.
\end{eulercomment}
\begin{eulerprompt}
>O, R
\end{eulerprompt}
\begin{euleroutput}
  [1.16667,  1.16667]
  1.17851130198
\end{euleroutput}
\begin{eulercomment}
Sekarang akan digambar lingkaran dalam segitiga ABC. Titik pusat lingkaran dalam adalah
titik potong garis-garis bagi sudut.
\end{eulercomment}
\begin{eulerprompt}
>l=angleBisector(A,C,B); // garis bagi <ACB
>g=angleBisector(C,A,B); // garis bagi <CAB
>P=lineIntersection(l,g) // titik potong kedua garis bagi sudut
\end{eulerprompt}
\begin{euleroutput}
  [0.86038,  0.86038]
\end{euleroutput}
\begin{eulercomment}
Tambahkan semuanya ke plot.
\end{eulercomment}
\begin{eulerprompt}
>color(5); plotLine(l); plotLine(g); color(1); // gambar kedua garis bagi sudut
>plotPoint(P,"P"); // gambar titik potongnya
>r=norm(P-projectToLine(P,lineThrough(A,B))) // jari-jari lingkaran dalam
\end{eulerprompt}
\begin{euleroutput}
  0.509653732104
\end{euleroutput}
\begin{eulerprompt}
>plotCircle(circleWithCenter(P,r),"Lingkaran dalam segitiga ABC"): // gambar lingkaran dalam
\end{eulerprompt}
\eulersubheading{Latihan}
\begin{eulercomment}
1. Tentukan ketiga titik singgung lingkaran dalam dengan sisi-sisi
segitiga ABC.
\end{eulercomment}
\begin{eulerprompt}
>setPlotRange(-3.5,5.5,-3.5,5.5);
>A=[-3,1]; plotPoint(A,"A");
>B=[1,-3]; plotPoint(B,"B");
>C=[5,5]; plotPoint(C,"C");
\end{eulerprompt}
\begin{eulercomment}
2. Gambar segitiga dengan titik-titik sudut ketiga titik singgung
tersebut. Merupakan segitiga apakah itu?
\end{eulercomment}
\begin{eulerprompt}
>plotSegment(A,B,"c")
>plotSegment(B,C,"a")
>plotSegment(A,C,"b")
>aspect(1):
\end{eulerprompt}
\begin{eulercomment}
3. Hitung luas segitiga tersebut.
\end{eulercomment}
\begin{eulerprompt}
>lineThrough(A,B)
\end{eulerprompt}
\begin{euleroutput}
  [4,  4,  -8]
\end{euleroutput}
\begin{eulerprompt}
>h=perpendicular(C,lineThrough(A,B));
>D=lineIntersection(h,lineThrough(A,B));
>plotPoint(D,value=1);
>aspect(1); plotSegment(C,D):
>distance(C,D)*distance(A,B)/2
\end{eulerprompt}
\begin{euleroutput}
  24
\end{euleroutput}
\begin{eulercomment}
Jadi luas segitiga di atas adalah 24

4. Tunjukkan bahwa garis bagi sudut yang ke tiga juga melalui titik
pusat lingkaran dalam.
\end{eulercomment}
\begin{eulerprompt}
>l=angleBisector(A,C,B);
>g=angleBisector(C,A,B);
>P=lineIntersection(l,g)
\end{eulerprompt}
\begin{euleroutput}
  [0.441518,  0.441518]
\end{euleroutput}
\begin{eulerprompt}
>color(5); plotLine(l); plotLine(g); color(1);
>plotPoint(P,"P");
>r=norm(P-projectToLine(P,lineThrough(A,B)))
\end{eulerprompt}
\begin{euleroutput}
  2.03861492842
\end{euleroutput}
\begin{eulerprompt}
>plotCircle(circleWithCenter(P,r),"Lingkaran dalam segitiga ABC"):
\end{eulerprompt}
\begin{eulercomment}
Jadi, terbukti bahwa garis bagi sudut yang ketiga juga melalui titik
pusat lingkaran dalam.

5. Gambar jari-jari lingkaran dalam.
\end{eulercomment}
\begin{eulerprompt}
>r=norm(P-projectToLine(P,lineThrough(A,B)))
\end{eulerprompt}
\begin{euleroutput}
  2.03861492842
\end{euleroutput}
\begin{eulerprompt}
>plotCircle(circleWithCenter(P,r),"Lingkaran dalam segitiga ABC"):
\end{eulerprompt}
\begin{eulercomment}
6. Hitung luas lingkaran luar dan luas lingkaran dalam segitiga ABC.
Adakah hubungan antara luas kedua lingkaran tersebut dengan luas
segitiga ABC?
\end{eulercomment}
\begin{eulerprompt}
>c=circleThrough(A,B,C);
>R=getCircleRadius(c);
>O=getCircleCenter(c);
>plotPoint(O,"O");
>plotCircle(c,"Lingkaran luar segitiga ABC"):
\end{eulerprompt}
\eulersubheading{Contoh 2: Geometri Simbolik}
\begin{eulercomment}
Kita dapat menghitung geometri eksak dan simbolik menggunakan Maxima.

File geometri.e menyediakan fungsi yang sama (dan lebih banyak lagi)
di Maxima. Namun, kita dapat menggunakan perhitungan simbolis
sekarang.
\end{eulercomment}
\begin{eulerprompt}
>A &= [1,0]; B &= [0,1]; C &= [2,2]; // menentukan tiga titik A, B, C
\end{eulerprompt}
\begin{eulercomment}
Fungsi untuk garis dan lingkaran bekerja seperti fungsi Euler, tetapi
memberikan perhitungan simbolis.
\end{eulercomment}
\begin{eulerprompt}
>c &= lineThrough(B,C) // c=BC
\end{eulerprompt}
\begin{euleroutput}
  
                               [- 1, 2, 2]
  
\end{euleroutput}
\begin{eulercomment}
Kita bisa mendapatkan persamaan garis dengan mudah.
\end{eulercomment}
\begin{eulerprompt}
>$getLineEquation(c,x,y), $solve(%,y) | expand // persamaan garis c
>$getLineEquation(lineThrough([x1,y1],[x2,y2]),x,y), $solve(%,y) // persamaan garis melalui(x1, y1) dan (x2, y2)
>$getLineEquation(lineThrough(A,[x1,y1]),x,y) // persamaan garis melalui A dan (x1, y1)
>h &= perpendicular(A,lineThrough(B,C)) // h melalui A tegak lurus BC
\end{eulerprompt}
\begin{euleroutput}
  
                                [2, 1, 2]
  
\end{euleroutput}
\begin{eulerprompt}
>Q &= lineIntersection(c,h) // Q titik potong garis c=BC dan h
\end{eulerprompt}
\begin{euleroutput}
  
                                   2  6
                                  [-, -]
                                   5  5
  
\end{euleroutput}
\begin{eulerprompt}
>$projectToLine(A,lineThrough(B,C)) // proyeksi A pada BC
>$distance(A,Q) // jarak AQ
>cc &= circleThrough(A,B,C); $cc // (titik pusat dan jari-jari) lingkaran melalui A, B, C
>r&=getCircleRadius(cc); $r , $float(r) // tampilkan nilai jari-jari
>$computeAngle(A,C,B) // nilai <ACB
>$solve(getLineEquation(angleBisector(A,C,B),x,y),y)[1] // persamaan garis bagi <ACB
>P &= lineIntersection(angleBisector(A,C,B),angleBisector(C,B,A)); $P // titik potong 2 garis bagi sudut
>P() // hasilnya sama dengan perhitungan sebelumnya
\end{eulerprompt}
\begin{euleroutput}
  [0.86038,  0.86038]
\end{euleroutput}
\eulersubheading{Garis dan Lingkaran yang Berpotongan}
\begin{eulercomment}
Tentu saja, kita juga bisa memotong garis dengan lingkaran, dan
lingkaran dengan lingkaran.
\end{eulercomment}
\begin{eulerprompt}
>A &:= [1,0]; c=circleWithCenter(A,4);
>B &:= [1,2]; C &:= [2,1]; l=lineThrough(B,C);
>setPlotRange(5); plotCircle(c); plotLine(l);
\end{eulerprompt}
\begin{eulercomment}
Persimpangan garis dengan lingkaran mengembalikan dua titik dan jumlah
titik persimpangan.
\end{eulercomment}
\begin{eulerprompt}
>\{P1,P2,f\}=lineCircleIntersections(l,c);
>P1, P2, f
\end{eulerprompt}
\begin{euleroutput}
  [4.64575,  -1.64575]
  [-0.645751,  3.64575]
  2
\end{euleroutput}
\begin{eulerprompt}
>plotPoint(P1); plotPoint(P2):
\end{eulerprompt}
\begin{eulercomment}
Begitu pula di Maxima.
\end{eulercomment}
\begin{eulerprompt}
>c &= circleWithCenter(A,4) // lingkaran dengan pusat A jari-jari 4
\end{eulerprompt}
\begin{euleroutput}
  
                                [1, 0, 4]
  
\end{euleroutput}
\begin{eulerprompt}
>l &= lineThrough(B,C) // garis l melalui B dan C
\end{eulerprompt}
\begin{euleroutput}
  
                                [1, 1, 3]
  
\end{euleroutput}
\begin{eulerprompt}
>$lineCircleIntersections(l,c) | radcan, // titik potong lingkaran c dan garis l
\end{eulerprompt}
\begin{eulercomment}
Akan ditunjukkan bahwa sudut-sudut yang menghadap busur yang sama
adalah sama besar.
\end{eulercomment}
\begin{eulerprompt}
>C=A+normalize([-2,-3])*4; plotPoint(C); plotSegment(P1,C); plotSegment(P2,C);
>degprint(computeAngle(P1,C,P2))
\end{eulerprompt}
\begin{euleroutput}
  69°17'42.68''
\end{euleroutput}
\begin{eulerprompt}
>C=A+normalize([-4,-3])*4; plotPoint(C); plotSegment(P1,C); plotSegment(P2,C);
>degprint(computeAngle(P1,C,P2))
\end{eulerprompt}
\begin{euleroutput}
  69°17'42.68''
\end{euleroutput}
\begin{eulerprompt}
>insimg;
\end{eulerprompt}
\eulersubheading{Garis Sumbu}
\begin{eulercomment}
Berikut adalah langkah-langkah menggambar garis sumbu ruas garis AB:

1. Gambar lingkaran dengan pusat A melalui B.\\
2. Gambar lingkaran dengan pusat B melalui A.\\
3. Tarik garis melallui kedua titik potong kedua lingkaran tersebut. Garis ini merupakan
garis sumbu (melalui titik tengah dan tegak lurus) AB.
\end{eulercomment}
\begin{eulerprompt}
>A=[2,2]; B=[-1,-2];
>c1=circleWithCenter(A,distance(A,B));
>c2=circleWithCenter(B,distance(A,B));
>\{P1,P2,f\}=circleCircleIntersections(c1,c2);
>l=lineThrough(P1,P2);
>setPlotRange(5); plotCircle(c1); plotCircle(c2);
>plotPoint(A); plotPoint(B); plotSegment(A,B); plotLine(l):
\end{eulerprompt}
\begin{eulercomment}
Selanjutnya, kami melakukan hal yang sama di Maxima dengan koordinat
umum.
\end{eulercomment}
\begin{eulerprompt}
>A &= [a1,a2]; B &= [b1,b2];
>c1 &= circleWithCenter(A,distance(A,B));
>c2 &= circleWithCenter(B,distance(A,B));
>P &= circleCircleIntersections(c1,c2); P1 &= P[1]; P2 &= P[2];
\end{eulerprompt}
\begin{eulercomment}
Persamaan untuk persimpangan cukup terlibat. Tetapi kita dapat
menyederhanakannya, jika kita memecahkan y.
\end{eulercomment}
\begin{eulerprompt}
>g &= getLineEquation(lineThrough(P1,P2),x,y);
>$solve(g,y)
\end{eulerprompt}
\begin{eulercomment}
Ini memang sama dengan tegak lurus tengah, yang dihitung dengan cara
yang sama sekali berbeda.
\end{eulercomment}
\begin{eulerprompt}
>$solve(getLineEquation(middlePerpendicular(A,B),x,y),y)
>h &=getLineEquation(lineThrough(A,B),x,y);
>$solve(h,y)
\end{eulerprompt}
\begin{eulercomment}
Perhatikan hasil kali gradien garis g dan h adalah:

\end{eulercomment}
\begin{eulerformula}
\[
\frac{-(b_1-a_1)}{(b_2-a_2)}\times \frac{(b_2-a_2)}{(b_1-a_1)} = -1.
\]
\end{eulerformula}
\begin{eulercomment}
Artinya kedua garis tegak lurus.
\end{eulercomment}
\eulerheading{Contoh 3: Rumus Heron}
\begin{eulercomment}
Rumus Heron menyatakan bahwa luas segitiga dengan panjang sisi-sisi a,
b dan c adalah:

\end{eulercomment}
\begin{eulerformula}
\[
L = \sqrt{s(s-a)(s-b)(s-c)}\quad \text{ dengan } s=(a+b+c)/2,
\]
\end{eulerformula}
\begin{eulercomment}
atau bisa ditulis dalam bentuk lain:

\end{eulercomment}
\begin{eulerformula}
\[
L = \frac{1}{4}\sqrt{(a+b+c)(b+c-a)(a+c-b)(a+b-c)}
\]
\end{eulerformula}
\begin{eulercomment}
Untuk membuktikan hal ini kita misalkan C(0,0), B(a,0) dan A(x,y),
b=AC, c=AB. Luas segitiga ABC adalah

\end{eulercomment}
\begin{eulerformula}
\[
L_{\triangle ABC}=\frac{1}{2}a\times y.
\]
\end{eulerformula}
\begin{eulercomment}
Nilai y didapat dengan menyelesaikan sistem persamaan:

\end{eulercomment}
\begin{eulerformula}
\[
x^2+y^2=b^2, \quad (x-a)^2+y^2=c^2.
\]
\end{eulerformula}
\begin{eulerprompt}
>setPlotRange(-1,10,-1,8); plotPoint([0,0], "C(0,0)"); plotPoint([5.5,0], "B(a,0)");  ...
> plotPoint([7.5,6], "A(x,y)");
>plotSegment([0,0],[5.5,0], "a",25); plotSegment([5.5,0],[7.5,6],"c",15);  ...
>plotSegment([0,0],[7.5,6],"b",25); 
>plotSegment([7.5,6],[7.5,0],"t=y",25):
>sol &= solve([x^2+y^2=b^2,(x-a)^2+y^2=c^2],[x,y])
\end{eulerprompt}
\begin{euleroutput}
  
                                    []
  
\end{euleroutput}
\begin{eulerprompt}
>ysol &= y with sol[2][2]; $'y=sqrt(factor(ysol^2))
\end{eulerprompt}
\begin{euleroutput}
  Maxima said:
  part: invalid index of list or matrix.
   -- an error. To debug this try: debugmode(true);
  
  Error in:
  ysol &= y with sol[2][2]; $'y=sqrt(factor(ysol^2)) ...
                          ^
\end{euleroutput}
\begin{eulercomment}
Kami mendapatkan formula Heron.
\end{eulercomment}
\begin{eulerprompt}
>function H(a,b,c) &= sqrt(factor((ysol*a/2)^2)); $'H(a,b,c)=H(a,b,c)
>$'Luas=H(2,5,6) // luas segitiga dengan panjang sisi-sisi 2, 5, 6
\end{eulerprompt}
\begin{eulercomment}
Tentu saja, setiap segitiga persegi panjang adalah kasus yang
terkenal.
\end{eulercomment}
\begin{eulerprompt}
>$H(3,4,5) //luas segitiga siku-siku dengan panjang sisi 3, 4, 5
\end{eulerprompt}
\begin{eulercomment}
Dan juga jelas, bahwa ini adalah segitiga dengan luas maksimal dan dua
sisi 3 dan 4.
\end{eulercomment}
\begin{eulerprompt}
>aspect (1.5); plot2d($H(3,4,x),1,7): // Kurva luas segitiga sengan panjang sisi 3, 4, x (1<= x <=7)
\end{eulerprompt}
\begin{euleroutput}
  Variable or function ysol not found.
  Error in expression: 3*abs(ysol)/2
   %ploteval:
      y0=f$(x[1],args());
  adaptiveevalone:
      s=%ploteval(g$,t;args());
  Try "trace errors" to inspect local variables after errors.
  plot2d:
      dw/n,dw/n^2,dw/n,auto;args());
\end{euleroutput}
\begin{eulercomment}
Kasus umum juga berhasil.
\end{eulercomment}
\begin{eulerprompt}
>$solve(diff(H(a,b,c)^2,c)=0,c)
\end{eulerprompt}
\begin{euleroutput}
  Maxima said:
  diff: second argument must be a variable; found [1,0,4]
   -- an error. To debug this try: debugmode(true);
  
  Error in:
   $solve(diff(H(a,b,c)^2,c)=0,c) ...
                                ^
\end{euleroutput}
\begin{eulercomment}
Sekarang mari kita temukan himpunan semua titik di mana b+c=d untuk
beberapa konstanta d. Diketahui bahwa ini adalah elips.
\end{eulercomment}
\begin{eulerprompt}
>s1 &= subst(d-c,b,sol[2]); $s1
\end{eulerprompt}
\begin{euleroutput}
  Maxima said:
  part: invalid index of list or matrix.
   -- an error. To debug this try: debugmode(true);
  
  Error in:
  s1 &= subst(d-c,b,sol[2]); $s1 ...
                           ^
\end{euleroutput}
\begin{eulercomment}
And make functions of this.
\end{eulercomment}
\begin{eulerprompt}
>function fx(a,c,d) &= rhs(s1[1]); $fx(a,c,d), function fy(a,c,d) &= rhs(s1[2]); $fy(a,c,d)
\end{eulerprompt}
\begin{eulercomment}
Sekarang kita bisa menggambar set. Sisi b bervariasi dari 1 hingga 4.
Sudah diketahui bahwa kita mendapatkan elips.
\end{eulercomment}
\begin{eulercomment}
Kita dapat memeriksa persamaan umum untuk elips ini, i.e.

\end{eulercomment}
\begin{eulerformula}
\[
\frac{(x-x_m)^2}{u^2}+\frac{(y-y_m)}{v^2}=1,
\]
\end{eulerformula}
\begin{eulercomment}
dimana (xm,ym) adalah pusatnya, dan u dan v adalah setengah sumbu.
\end{eulercomment}
\begin{eulerprompt}
>$ratsimp((fx(a,c,d)-a/2)^2/u^2+fy(a,c,d)^2/v^2 with [u=d/2,v=sqrt(d^2-a^2)/2])
\end{eulerprompt}
\begin{eulercomment}
Kita melihat bahwa tinggi dan dengan demikian luas segitiga adalah
maksimum untuk x=0. Dengan demikian luas segitiga dengan a+b+c=d
adalah maksimal, jika sama sisi. Kami ingin menurunkan ini secara
analitis.
\end{eulercomment}
\begin{eulerprompt}
>eqns &= [diff(H(a,b,d-(a+b))^2,a)=0,diff(H(a,b,d-(a+b))^2,b)=0]; $eqns
\end{eulerprompt}
\begin{eulercomment}
Kita mendapatkan beberapa minimum, yang termasuk segitiga dengan satu
sisi 0, dan solusi a=b=c=d/3.
\end{eulercomment}
\begin{eulerprompt}
>$solve(eqns,[a,b])
\end{eulerprompt}
\begin{eulercomment}
Ada juga metode Lagrange, memaksimalkan H(a,b,c)\textasciicircum{}2 sehubungan dengan
a+b+d=d.
\end{eulercomment}
\begin{eulerprompt}
>&solve([diff(H(a,b,c)^2,a)=la,diff(H(a,b,c)^2,b)=la, ...
>   diff(H(a,b,c)^2,c)=la,a+b+c=d],[a,b,c,la])
\end{eulerprompt}
\begin{euleroutput}
  Maxima said:
  diff: second argument must be a variable; found [1,0,4]
   -- an error. To debug this try: debugmode(true);
  
  Error in:
  ... la,    diff(H(a,b,c)^2,c)=la,a+b+c=d],[a,b,c,la]) ...
                                                       ^
\end{euleroutput}
\begin{eulercomment}
Kita bisa membuat plot situasinya
\end{eulercomment}
\begin{eulercomment}
First set the points in Maxima.
\end{eulercomment}
\begin{eulerprompt}
>A &= at([x,y],sol[2]); $A
\end{eulerprompt}
\begin{euleroutput}
  Maxima said:
  part: invalid index of list or matrix.
   -- an error. To debug this try: debugmode(true);
  
  Error in:
  A &= at([x,y],sol[2]); $A ...
                       ^
\end{euleroutput}
\begin{eulerprompt}
>B &= [0,0]; $B, C &= [a,0]; $C
\end{eulerprompt}
\begin{eulercomment}
Kemudian atur rentang plot, dan plot poinnya.
\end{eulercomment}
\begin{eulerprompt}
>setPlotRange(0,5,-2,3); ...
>a=4; b=3; c=2; ...
>plotPoint(mxmeval("B"),"B"); plotPoint(mxmeval("C"),"C"); ...
>plotPoint(mxmeval("A"),"A"):
\end{eulerprompt}
\begin{euleroutput}
  Variable a1 not found!
  Use global variables or parameters for string evaluation.
  Error in Evaluate, superfluous characters found.
  Try "trace errors" to inspect local variables after errors.
  mxmeval:
      return evaluate(mxm(s));
  Error in:
  ... otPoint(mxmeval("C"),"C"); plotPoint(mxmeval("A"),"A"): ...
                                                       ^
\end{euleroutput}
\begin{eulercomment}
Plot the segments.
\end{eulercomment}
\begin{eulerprompt}
>plotSegment(mxmeval("A"),mxmeval("C")); ...
>plotSegment(mxmeval("B"),mxmeval("C")); ...
>plotSegment(mxmeval("B"),mxmeval("A")):
\end{eulerprompt}
\begin{euleroutput}
  Variable a1 not found!
  Use global variables or parameters for string evaluation.
  Error in Evaluate, superfluous characters found.
  Try "trace errors" to inspect local variables after errors.
  mxmeval:
      return evaluate(mxm(s));
  Error in:
  plotSegment(mxmeval("A"),mxmeval("C")); plotSegment(mxmeval("B ...
                          ^
\end{euleroutput}
\begin{eulercomment}
CompHitung tengah tegak lurus di Maxima.
\end{eulercomment}
\begin{eulerprompt}
>h &= middlePerpendicular(A,B); g &= middlePerpendicular(B,C);
\end{eulerprompt}
\begin{eulercomment}
Dan pusat keliling.
\end{eulercomment}
\begin{eulerprompt}
>U &= lineIntersection(h,g);
\end{eulerprompt}
\begin{eulercomment}
Kita mendapatkan rumus untuk jari-jari lingkaran melingkar.
\end{eulercomment}
\begin{eulerprompt}
>&assume(a>0,b>0,c>0); $distance(U,B) | radcan
\end{eulerprompt}
\begin{eulercomment}
Mari kita tambahkan ini ke plot.
\end{eulercomment}
\begin{eulerprompt}
>plotPoint(U()); ...
>plotCircle(circleWithCenter(mxmeval("U"),mxmeval("distance(U,C)"))):
\end{eulerprompt}
\begin{euleroutput}
  Variable a2 not found!
  Use global variables or parameters for string evaluation.
  Error in ^
  Error in expression: [a/2,(a2^2+a1^2-a*a1)/(2*a2)]
  Error in:
  plotPoint(U()); plotCircle(circleWithCenter(mxmeval("U"),mxmev ...
               ^
\end{euleroutput}
\begin{eulercomment}
Dengan menggunakan geometri, kita mendapatkan rumus sederhana

\end{eulercomment}
\begin{eulerformula}
\[
\frac{a}{\sin(\alpha)}=2r
\]
\end{eulerformula}
\begin{eulercomment}
untuk radius. Kita dapat memeriksa, apakah ini benar-benar benar
dengan Maxima. Maxima akan memperhitungkan ini hanya jika kita
mengkutududukkannya.
\end{eulercomment}
\begin{eulerprompt}
>$c^2/sin(computeAngle(A,B,C))^2  | factor
\end{eulerprompt}
\eulerheading{Contoh 4: Garis Euler dan Parabola}
\begin{eulercomment}
Garis Euler adalah garis yang ditentukan dari segitiga apa pun yang
tidak sama sisi. Ini adalah garis tengah segitiga, dan melewati
beberapa titik penting yang ditentukan dari segitiga, termasuk
ortosentrum, pusat lilitan, centroid, titik Exeter dan pusat lingkaran
sembilan titik segitiga.

Untuk demonstrasi, kita menghitung dan memplot garis Euler dalam
segitiga.

Pertama, kita mendefinisikan sudut segitiga di Euler. Kami menggunakan
definisi, yang terlihat dalam ekspresi simbolis.
\end{eulercomment}
\begin{eulerprompt}
>A::=[-1,-1]; B::=[2,0]; C::=[1,2];
\end{eulerprompt}
\begin{eulercomment}
Untuk memplot objek geometris, kita mengatur area plot, dan
menambahkan titik-titik ke dalamnya. Semua plot objek geometris
ditambahkan ke plot saat ini.
\end{eulercomment}
\begin{eulerprompt}
>setPlotRange(3); plotPoint(A,"A"); plotPoint(B,"B"); plotPoint(C,"C");
\end{eulerprompt}
\begin{eulercomment}
Kita juga bisa menambahkan sisi segitiga.
\end{eulercomment}
\begin{eulerprompt}
>plotSegment(A,B,""); plotSegment(B,C,""); plotSegment(C,A,""):
\end{eulerprompt}
\begin{eulercomment}
Berikut adalah luas segitiga, menggunakan rumus penentu. Tentu saja,
kita harus mengambil nilai absolut dari hasil ini.
\end{eulercomment}
\begin{eulerprompt}
>$areaTriangle(A,B,C)
\end{eulerprompt}
\begin{eulercomment}
Kita dapat menghitung koefisien sisi c.
\end{eulercomment}
\begin{eulerprompt}
>c &= lineThrough(A,B)
\end{eulerprompt}
\begin{euleroutput}
  
                              [- 1, 3, - 2]
  
\end{euleroutput}
\begin{eulercomment}
Dan juga dapatkan rumus untuk baris ini.
\end{eulercomment}
\begin{eulerprompt}
>$getLineEquation(c,x,y)
\end{eulerprompt}
\begin{eulercomment}
Untuk bentuk Hesse, kita perlu menentukan titik, sehingga titik
tersebut berada di sisi positif dari Hesseform. Memasukkan titik
menghasilkan jarak positif ke garis.
\end{eulercomment}
\begin{eulerprompt}
>$getHesseForm(c,x,y,C), $at(%,[x=C[1],y=C[2]])
\end{eulerprompt}
\begin{eulercomment}
Sekarang kita menghitung lingkaran lingkaran ABC.
\end{eulercomment}
\begin{eulerprompt}
>LL &= circleThrough(A,B,C); $getCircleEquation(LL,x,y)
>O &= getCircleCenter(LL); $O
\end{eulerprompt}
\begin{eulercomment}
Plot lingkaran dan pusatnya. Cu dan Anda simbolis. Kami mengevaluasi
ekspresi ini untuk Euler.
\end{eulercomment}
\begin{eulerprompt}
>plotCircle(LL()); plotPoint(O(),"O"):
\end{eulerprompt}
\begin{eulercomment}
Kita dapat menghitung persimpangan ketinggian di ABC (ortosent) secara
numerik dengan perintah berikut.
\end{eulercomment}
\begin{eulerprompt}
>H &= lineIntersection(perpendicular(A,lineThrough(C,B)),...
>  perpendicular(B,lineThrough(A,C))); $H
\end{eulerprompt}
\begin{eulercomment}
Sekarang kita dapat menghitung garis Euler dari segitiga.
\end{eulercomment}
\begin{eulerprompt}
>el &= lineThrough(H,O); $getLineEquation(el,x,y)
\end{eulerprompt}
\begin{eulercomment}
Tambahkan ke plot kita.
\end{eulercomment}
\begin{eulerprompt}
>plotPoint(H(),"H"); plotLine(el(),"Garis Euler"):
\end{eulerprompt}
\begin{eulercomment}
Pusat gravitasi harus berada di garis ini.
\end{eulercomment}
\begin{eulerprompt}
>M &= (A+B+C)/3; $getLineEquation(el,x,y) with [x=M[1],y=M[2]]
>plotPoint(M(),"M"): // titik berat
\end{eulerprompt}
\begin{eulercomment}
Teorinya memberitahu kita MH=2*MO. Kita perlu menyederhanakan dengan
radcan untuk mencapai ini.
\end{eulercomment}
\begin{eulerprompt}
>$distance(M,H)/distance(M,O)|radcan
\end{eulerprompt}
\begin{eulercomment}
Fungsi termasuk fungsi untuk sudut juga.
\end{eulercomment}
\begin{eulerprompt}
>$computeAngle(A,C,B), degprint(%())
\end{eulerprompt}
\begin{euleroutput}
  60°15'18.43''
\end{euleroutput}
\begin{eulercomment}
Persamaan untuk pusat incircle tidak terlalu bagus.
\end{eulercomment}
\begin{eulerprompt}
>Q &= lineIntersection(angleBisector(A,C,B),angleBisector(C,B,A))|radcan; $Q
\end{eulerprompt}
\begin{eulercomment}
Mari kita hitung juga ekspresi untuk jari-jari lingkaran yang
tertulis.
\end{eulercomment}
\begin{eulerprompt}
>r &= distance(Q,projectToLine(Q,lineThrough(A,B)))|ratsimp; $r
>LD &=  circleWithCenter(Q,r); // Lingkaran dalam
\end{eulerprompt}
\begin{eulercomment}
Mari kita tambahkan ini ke plot.
\end{eulercomment}
\begin{eulerprompt}
>color(5); plotCircle(LD()):
\end{eulerprompt}
\eulersubheading{Parabola}
\begin{eulercomment}
Selanjutnya akan dicari persamaan tempat kedudukan titik-titik yang berjarak sama ke titik C
dan ke garis AB.
\end{eulercomment}
\begin{eulerprompt}
>p &= getHesseForm(lineThrough(A,B),x,y,C)-distance([x,y],C); $p='0
\end{eulerprompt}
\begin{eulercomment}
Persamaan tersebut dapat digambar menjadi satu dengan gambar sebelumnya.
\end{eulercomment}
\begin{eulerprompt}
>plot2d(p,level=0,add=1,contourcolor=6):
\end{eulerprompt}
\begin{eulercomment}
Ini seharusnya menjadi beberapa fungsi, tetapi pemecah default Maxima
hanya dapat menemukan solusinya, jika kita kuadratkan persamaannya.
Akibatnya, kami mendapatkan solusi palsu.
\end{eulercomment}
\begin{eulerprompt}
>akar &= solve(getHesseForm(lineThrough(A,B),x,y,C)^2-distance([x,y],C)^2,y)
\end{eulerprompt}
\begin{euleroutput}
  
          [y = - 3 x - sqrt(70) sqrt(9 - 2 x) + 26, 
                                y = - 3 x + sqrt(70) sqrt(9 - 2 x) + 26]
  
\end{euleroutput}
\begin{eulercomment}
Solusi pertama adalah

maxima: akar[1]

Menambahkan solusi pertama ke plot menunjukkan, bahwa itu memang jalan
yang kita cari. Teorinya memberi tahu kita bahwa itu adalah parabola
yang diputar.
\end{eulercomment}
\begin{eulerprompt}
>plot2d(&rhs(akar[1]),add=1):
>function g(x) &= rhs(akar[1]); $'g(x)= g(x)// fungsi yang mendefinisikan kurva di atas
>T &=[-1, g(-1)]; // ambil sebarang titik pada kurva tersebut
>dTC &= distance(T,C); $fullratsimp(dTC), $float(%) // jarak T ke C
>U &= projectToLine(T,lineThrough(A,B)); $U // proyeksi T pada garis AB 
>dU2AB &= distance(T,U); $fullratsimp(dU2AB), $float(%) // jatak T ke AB
\end{eulerprompt}
\begin{eulercomment}
Ternyata jarak T ke C sama dengan jarak T ke AB. Coba Anda pilih titik T yang lain dan
ulangi perhitungan-perhitungan di atas untuk menunjukkan bahwa hasilnya juga sama.
\end{eulercomment}
\begin{eulercomment}

\begin{eulercomment}
\eulerheading{Contoh 5: Trigonometri Rasional}
\begin{eulercomment}
Ini terinspirasi oleh sebuah pembicaraan N.J.Wildberger. Dalam bukunya
"Divine Proportions", Wildberger mengusulkan untuk mengganti gagasan
klasik tentang jarak dan sudut dengan segi empat dan penyebaran.
Dengan menggunakan ini, memang mungkin untuk menghindari fungsi
trigonometri dalam banyak contoh, dan tetap "rasional".

Berikut ini, saya memperkenalkan konsep, dan memecahkan beberapa
masalah. Saya menggunakan perhitungan simbolik Maxima di sini, yang
menyembunyikan keuntungan utama dari trigonometri rasional bahwa
perhitungan dapat dilakukan hanya dengan kertas dan pensil. Anda
diundang untuk memeriksa hasilnya tanpa komputer.

Intinya adalah bahwa perhitungan rasional simbolik sering kali
menghasilkan hasil yang sederhana. Sebaliknya, trigonometri klasik
menghasilkan hasil trigonometri yang rumit, yang mengevaluasi
perkiraan numerik saja.
\end{eulercomment}
\begin{eulerprompt}
>load geometry;
\end{eulerprompt}
\begin{eulercomment}
Untuk perkenalan pertama, kami menggunakan segitiga persegi panjang
dengan proporsi Mesir yang terkenal 3, 4 dan 5. Perintah berikut
adalah perintah Euler untuk memplot geometri bidang yang terkandung
dalam file Euler "geometry.e".
\end{eulercomment}
\begin{eulerprompt}
>C&:=[0,0]; A&:=[4,0]; B&:=[0,3]; ...
>setPlotRange(-1,5,-1,5); ...
>plotPoint(A,"A"); plotPoint(B,"B"); plotPoint(C,"C"); ...
>plotSegment(B,A,"c"); plotSegment(A,C,"b"); plotSegment(C,B,"a"); ...
>insimg(30);
\end{eulerprompt}
\begin{eulercomment}
Tentu,

\end{eulercomment}
\begin{eulerformula}
\[
\sin(w_a)=\frac{a}{c},
\]
\end{eulerformula}
\begin{eulercomment}
di mana wa adalah sudut pada A. Cara biasa untuk menghitung sudut ini,
adalah dengan mengambil kebalikan dari fungsi sinus. Hasilnya adalah
sudut yang tidak dapat dicerna, yang hanya dapat dicetak kira-kira.
\end{eulercomment}
\begin{eulerprompt}
>wa := arcsin(3/5); degprint(wa)
\end{eulerprompt}
\begin{euleroutput}
  36°52'11.63''
\end{euleroutput}
\begin{eulercomment}
Trigonometri rasional mencoba menghindari hal ini.

Gagasan pertama dari trigonometri rasional adalah segi empat, yang
menggantikan jarak. Faktanya, itu hanya jarak kuadrat. Berikut ini, a,
b, dan c menunjukkan segi empat sisi.

Teorema Pythogoras hanya menjadi a+b=c kemudian.
\end{eulercomment}
\begin{eulerprompt}
>a &= 3^2; b &= 4^2; c &= 5^2; &a+b=c
\end{eulerprompt}
\begin{euleroutput}
  
                                 25 = 25
  
\end{euleroutput}
\begin{eulercomment}
Gagasan kedua dari trigonometri rasional adalah penyebaran. Spread
mengukur pembukaan antar garis. Ini adalah 0, jika garisnya sejajar,
dan 1, jika garisnya persegi panjang. Ini adalah kuadrat sinus dari
sudut antara dua garis.

Penyebaran garis AB dan AC pada gambar di atas didefinisikan
sebagaithe lines AB and AC in the image above is defined as

\end{eulercomment}
\begin{eulerformula}
\[
s_a = \sin(\alpha)^2 = \frac{a}{c},
\]
\end{eulerformula}
\begin{eulercomment}
di mana a dan c adalah kuadrat dari segitiga persegi panjang dengan
satu sudut di A.
\end{eulercomment}
\begin{eulerprompt}
>sa &= a/c; $sa
\end{eulerprompt}
\begin{eulercomment}
Ini lebih mudah dihitung daripada sudut, tentu saja. Tetapi Anda
kehilangan properti bahwa sudut dapat ditambahkan dengan mudah.

Tentu saja, kita dapat mengonversi nilai perkiraan kita untuk sudut wa
menjadi sprad, dan mencetaknya sebagai pecahan.
\end{eulercomment}
\begin{eulerprompt}
>fracprint(sin(wa)^2)
\end{eulerprompt}
\begin{euleroutput}
  9/25
\end{euleroutput}
\begin{eulercomment}
Hukum kosinus trgonometri klasik diterjemahkan menjadi berikut "cross
law".

\end{eulercomment}
\begin{eulerformula}
\[
(c+b-a)^2 = 4 b c \, (1-s_a)
\]
\end{eulerformula}
\begin{eulercomment}
Di sini a, b, dan c adalah segi empat dari sisi-sisi segitiga, dan sa
adalah penyebaran di sudut A. Sisi a, seperti biasa, berlawanan dengan
sudut A.

Hukum ini diimplementasikan dalam file geometri.e yang kita muat ke
Euler.
\end{eulercomment}
\begin{eulerprompt}
>$crosslaw(aa,bb,cc,saa)
\end{eulerprompt}
\begin{eulercomment}
Dalam kasus kami, kami mendapatkan
\end{eulercomment}
\begin{eulerprompt}
>$crosslaw(a,b,c,sa)
\end{eulerprompt}
\begin{eulercomment}
Mari kita gunakan crosslaw ini untuk menemukan penyebaran di A. Untuk
melakukan ini, kami menghasilkan crosslaw untuk segi empat a, b, dan
c, dan menyelesaikannya untuk spread sa yang tidak diketahui.

Anda dapat melakukan ini dengan tangan dengan mudah, tetapi saya
menggunakan Maxima. Tentu saja, kami mendapatkan hasilnya, kami sudah
memilikinya.
\end{eulercomment}
\begin{eulerprompt}
>$crosslaw(a,b,c,x), $solve(%,x)
\end{eulerprompt}
\begin{eulercomment}
Kami sudah tahu ini. Definisi spread adalah kasus khusus dari
crosslaw.

Kita juga dapat menyelesaikan ini untuk umum a, b, c. Hasilnya adalah
rumus yang menghitung penyebaran sudut segitiga yang diberikan segi
empat dari tiga sisi.
\end{eulercomment}
\begin{eulerprompt}
>$solve(crosslaw(aa,bb,cc,x),x)
\end{eulerprompt}
\begin{eulercomment}
Kita bisa membuat fungsi dari hasilnya. Fungsi seperti itu sudah
didefinisikan dalam file geometri.e Euler.
\end{eulercomment}
\begin{eulerprompt}
>$spread(a,b,c)
\end{eulerprompt}
\begin{eulercomment}
Sebagai contoh, kita dapat menggunakannya untuk menghitung sudut
segitiga dengan sisi

\end{eulercomment}
\begin{eulerformula}
\[
a, \quad a, \quad \frac{4a}{7}
\]
\end{eulerformula}
\begin{eulercomment}
Hasilnya rasional, yang tidak mudah didapatkan jika kita menggunakan
trigonometri klasik.
\end{eulercomment}
\begin{eulerprompt}
>$spread(a,a,4*a/7)
\end{eulerprompt}
\begin{eulercomment}
Ini adalah sudut dalam derajat.
\end{eulercomment}
\begin{eulerprompt}
>degprint(arcsin(sqrt(6/7)))
\end{eulerprompt}
\begin{euleroutput}
  67°47'32.44''
\end{euleroutput}
\eulersubheading{Contoh lain}
\begin{eulercomment}
Sekarang, mari kita coba contoh yang lebih lanjut.

Kami menetapkan tiga sudut segitiga sebagai berikut.
\end{eulercomment}
\begin{eulerprompt}
>A&:=[1,2]; B&:=[4,3]; C&:=[0,4]; ...
>setPlotRange(-1,5,1,7); ...
>plotPoint(A,"A"); plotPoint(B,"B"); plotPoint(C,"C"); ...
>plotSegment(B,A,"c"); plotSegment(A,C,"b"); plotSegment(C,B,"a"); ...
>insimg;
\end{eulerprompt}
\begin{eulercomment}
Dengan menggunakan Pythogoras, mudah untuk menghitung jarak antara dua
titik. Pertama-tama saya menggunakan jarak fungsi file Euler untuk
geometri. Fungsi jarak menggunakan geometri klasik.
\end{eulercomment}
\begin{eulerprompt}
>$distance(A,B)
\end{eulerprompt}
\begin{eulercomment}
Euler juga berisi fungsi untuk kuadran antara dua titik.

Dalam contoh berikut, karena c+b bukan a, segitiga tidak persegi
panjang.
\end{eulercomment}
\begin{eulerprompt}
>c &= quad(A,B); $c, b &= quad(A,C); $b, a &= quad(B,C); $a,
\end{eulerprompt}
\begin{eulercomment}
Pertama, mari kita hitung sudut tradisional. Fungsi computeAngle
menggunakan metode biasa berdasarkan produk titik dari dua vektor.
Hasilnya adalah beberapa perkiraan floating point.

\end{eulercomment}
\begin{eulerformula}
\[
A=<1,2>\quad B=<4,3>,\quad C=<0,4>
\]
\end{eulerformula}
\begin{eulerformula}
\[
\mathbf{a}=C-B=<-4,1>,\quad \mathbf{c}=A-B=<-3,-1>,\quad \beta=\angle ABC
\]
\end{eulerformula}
\begin{eulerformula}
\[
\mathbf{a}.\mathbf{c}=|\mathbf{a}|.|\mathbf{c}|\cos \beta
\]
\end{eulerformula}
\begin{eulerformula}
\[
\cos \angle ABC =\cos\beta=\frac{\mathbf{a}.\mathbf{c}}{|\mathbf{a}|.|\mathbf{c}|}=\frac{12-1}{\sqrt{17}\sqrt{10}}=\frac{11}{\sqrt{17}\sqrt{10}}
\]
\end{eulerformula}
\begin{eulerprompt}
>wb &= computeAngle(A,B,C); $wb, $(wb/pi*180)()
\end{eulerprompt}
\begin{euleroutput}
  32.4711922908
\end{euleroutput}
\begin{eulercomment}
Dengan menggunakan pensil dan kertas, kita dapat melakukan hal yang
sama dengan hukum silang. Kami memasukkan kuadrat a, b, dan c ke dalam
hukum silang dan menyelesaikan x.
\end{eulercomment}
\begin{eulerprompt}
>$crosslaw(a,b,c,x), $solve(%,x), //(b+c-a)^=4b.c(1-x)
\end{eulerprompt}
\begin{eulercomment}
Artinya, apa yang dilakukan penyebaran fungsi yang didefinisikan dalam
"geometri.e".
\end{eulercomment}
\begin{eulerprompt}
>sb &= spread(b,a,c); $sb
\end{eulerprompt}
\begin{eulercomment}
Maxima mendapatkan hasil yang sama dengan menggunakan trigonometri
biasa, jika kita memaksanya. Itu menyelesaikan suku sin(arccos(...))
menjadi hasil pecahan. Sebagian besar siswa tidak dapat melakukan ini.
\end{eulercomment}
\begin{eulerprompt}
>$sin(computeAngle(A,B,C))^2
\end{eulerprompt}
\begin{eulercomment}
Setelah kita memiliki penyebaran di B, kita dapat menghitung tinggi ha
di sisi a. Ingatlah bahwa

\end{eulercomment}
\begin{eulerformula}
\[
s_b=\frac{h_a}{c}
\]
\end{eulerformula}
\begin{eulercomment}
by definition.
\end{eulercomment}
\begin{eulerprompt}
>ha &= c*sb; $ha
\end{eulerprompt}
\begin{eulercomment}
Gambar berikut telah diproduksi dengan program geometri C.a.R., yang
dapat menggambar segi empat dan penyebaran.

gambar: (20) Rational\_Geometry\_CaR.png

Menurut definisi panjang ha adalah akar kuadrat dari segi empatnya.
\end{eulercomment}
\begin{eulerprompt}
>$sqrt(ha)
\end{eulerprompt}
\begin{eulercomment}
Sekarang kita dapat menghitung luas segitiga. Jangan lupa, bahwa kita
berurusan dengan segi empat!
\end{eulercomment}
\begin{eulerprompt}
>$sqrt(ha)*sqrt(a)/2
\end{eulerprompt}
\begin{eulercomment}
Rumus penentu yang biasa menghasilkan hasil yang sama.
\end{eulercomment}
\begin{eulerprompt}
>$areaTriangle(B,A,C)
\end{eulerprompt}
\eulersubheading{Rumus Heron}
\begin{eulercomment}
Sekarang, mari kita selesaikan masalah ini secara umum!
\end{eulercomment}
\begin{eulerprompt}
>&remvalue(a,b,c,sb,ha);
\end{eulerprompt}
\begin{eulercomment}
Pertama-tama kita menghitung spread di B untuk segitiga dengan sisi a,
b, dan c. Kemudian kita menghitung luas kuadrat ("quadrea"?),
faktorkan dengan Maxima, dan kita mendapatkan rumus Heron yang
terkenal.

Harus diakui, ini sulit dilakukan dengan pensil dan kertas.
\end{eulercomment}
\begin{eulerprompt}
>$spread(b^2,c^2,a^2), $factor(%*c^2*a^2/4)
\end{eulerprompt}
\eulersubheading{Aturan Triple Spread}
\begin{eulercomment}
Kerugian spread adalah tidak lagi sekadar menambahkan sudut yang sama.

Namun, tiga spread segitiga memenuhi aturan "triple spread" berikut.
\end{eulercomment}
\begin{eulerprompt}
>&remvalue(sa,sb,sc); $triplespread(sa,sb,sc)
\end{eulerprompt}
\begin{eulercomment}
Aturan ini berlaku untuk tiga sudut mana pun yang jumlahnya mencapai
180°.\\
\end{eulercomment}
\begin{eulerformula}
\[
\alpha+\beta+\gamma=\pi
\]
\end{eulerformula}
\begin{eulercomment}
Sejak penyebaran

\end{eulercomment}
\begin{eulerformula}
\[
\alpha, \pi-\alpha
\]
\end{eulerformula}
\begin{eulercomment}
sama, aturan penyebaran tiga kali lipat juga benar, jika

\end{eulercomment}
\begin{eulerformula}
\[
\alpha+\beta=\gamma
\]
\end{eulerformula}
\begin{eulercomment}
Karena penyebaran sudut negatifnya sama, maka aturan penyebaran
rangkap tiga juga berlaku, jika

\end{eulercomment}
\begin{eulerformula}
\[
\alpha+\beta+\gamma=0
\]
\end{eulerformula}
\begin{eulercomment}
Misalnya, kita dapat menghitung sebaran sudut 60°. Yaitu 3/4.
Persamaan tersebut memiliki solusi kedua, di mana semua sebarannya
adalah 0.
\end{eulercomment}
\begin{eulerprompt}
>$solve(triplespread(x,x,x),x)
\end{eulerprompt}
\begin{eulercomment}
Sebaran 90° jelas adalah 1. Jika dua sudut dijumlahkan menjadi 90°,
sebarannya memecahkan persamaan sebaran rangkap tiga dengan a,b,1.
Dengan perhitungan berikut kita memperoleh a+b=1.
\end{eulercomment}
\begin{eulerprompt}
>$triplespread(x,y,1), $solve(%,x)
\end{eulerprompt}
\begin{eulercomment}
Karena sebaran 180°-t sama dengan sebaran t, rumus sebaran rangkap
tiga juga berlaku, jika satu sudut adalah jumlah atau selisih dari dua
sudut lainnya.

Jadi kita dapat menemukan sebaran sudut yang digandakan. Perhatikan
bahwa ada dua solusi lagi. Kita buat ini menjadi fungsi.
\end{eulercomment}
\begin{eulerprompt}
>$solve(triplespread(a,a,x),x), function doublespread(a) &= factor(rhs(%[1]))
\end{eulerprompt}
\begin{euleroutput}
  
                              - 4 (a - 1) a
  
\end{euleroutput}
\eulersubheading{Garis Bagi Sudut}
\begin{eulercomment}
Kita sudah tahu situasinya seperti ini.
\end{eulercomment}
\begin{eulerprompt}
>C&:=[0,0]; A&:=[4,0]; B&:=[0,3]; ...
>setPlotRange(-1,5,-1,5); ...
>plotPoint(A,"A"); plotPoint(B,"B"); plotPoint(C,"C"); ...
>plotSegment(B,A,"c"); plotSegment(A,C,"b"); plotSegment(C,B,"a"); ...
>insimg;
\end{eulerprompt}
\begin{eulercomment}
Mari kita hitung panjang garis bagi sudut di A. Namun, kita ingin
menyelesaikannya untuk a,b,c umum.
\end{eulercomment}
\begin{eulerprompt}
>&remvalue(a,b,c);
\end{eulerprompt}
\begin{eulercomment}
Jadi pertama-tama kita hitung sebaran sudut yang dibagi dua di A,
menggunakan rumus sebaran rangkap tiga.

Masalah dengan rumus ini muncul lagi. Rumus ini memiliki dua solusi.
Kita harus memilih yang benar. Solusi lainnya mengacu pada sudut yang
dibagi dua 180°-wa.
\end{eulercomment}
\begin{eulerprompt}
>$triplespread(x,x,a/(a+b)), $solve(%,x), sa2 &= rhs(%[1]); $sa2
\end{eulerprompt}
\begin{eulercomment}
Mari kita periksa persegi panjang Mesir.
\end{eulercomment}
\begin{eulerprompt}
>$sa2 with [a=3^2,b=4^2]
\end{eulerprompt}
\begin{eulercomment}
Kita dapat mencetak sudut dalam Euler, setelah mentransfer sebaran ke
radian. 
\end{eulercomment}
\begin{eulerprompt}
>wa2 := arcsin(sqrt(1/10)); degprint(wa2)
\end{eulerprompt}
\begin{euleroutput}
  18°26'5.82''
\end{euleroutput}
\begin{eulercomment}
Titik P merupakan perpotongan garis bagi sudut dengan sumbu y.
\end{eulercomment}
\begin{eulerprompt}
>P := [0,tan(wa2)*4]
\end{eulerprompt}
\begin{euleroutput}
  [0,  1.33333]
\end{euleroutput}
\begin{eulerprompt}
>plotPoint(P,"P"); plotSegment(A,P):
\end{eulerprompt}
\begin{eulercomment}
Mari kita periksa sudut-sudut pada contoh spesifik kita.
\end{eulercomment}
\begin{eulerprompt}
>computeAngle(C,A,P), computeAngle(P,A,B)
\end{eulerprompt}
\begin{euleroutput}
  0.321750554397
  0.321750554397
\end{euleroutput}
\begin{eulercomment}
Sekarang kita hitung panjang garis bagi AP.

Kita gunakan teorema sinus pada segitiga APC. Teorema ini menyatakan
bahwa

\end{eulercomment}
\begin{eulerformula}
\[
\frac{BC}{\sin(w_a)} = \frac{AC}{\sin(w_b)} = \frac{AB}{\sin(w_c)}
\]
\end{eulerformula}
\begin{eulercomment}
berlaku di sembarang segitiga. Kuadratkan, itu diterjemahkan menjadi
apa yang disebut "hukum sebaran"

\end{eulercomment}
\begin{eulerformula}
\[
\frac{a}{s_a} = \frac{b}{s_b} = \frac{c}{s_b}
\]
\end{eulerformula}
\begin{eulercomment}
di mana a, b, c menunjukkan kuadran.

Karena CPA sebarannya adalah 1-sa2, kita memperoleh bisa/1=b/(1-sa2)
dan dapat menghitung bisa (kuadran garis bagi sudut).
\end{eulercomment}
\begin{eulerprompt}
>&factor(ratsimp(b/(1-sa2))); bisa &= %; $bisa
\end{eulerprompt}
\begin{eulercomment}
Mari kita periksa rumus ini untuk nilai-nilai Mesir kita.
\end{eulercomment}
\begin{eulerprompt}
>sqrt(mxmeval("at(bisa,[a=3^2,b=4^2])")), distance(A,P)
\end{eulerprompt}
\begin{euleroutput}
  4.21637021356
  4.21637021356
\end{euleroutput}
\begin{eulercomment}
Kita juga dapat menghitung P menggunakan rumus spread.
\end{eulercomment}
\begin{eulerprompt}
>py&=factor(ratsimp(sa2*bisa)); $py
\end{eulerprompt}
\begin{eulercomment}
Nilainya sama dengan yang kita dapatkan dengan rumus trigonometri.
\end{eulercomment}
\begin{eulerprompt}
>sqrt(mxmeval("at(py,[a=3^2,b=4^2])"))
\end{eulerprompt}
\begin{euleroutput}
  1.33333333333
\end{euleroutput}
\eulersubheading{Sudut Tali Busur}
\begin{eulercomment}
Perhatikan situasi berikut.
\end{eulercomment}
\begin{eulerprompt}
>setPlotRange(1.2); ...
>color(1); plotCircle(circleWithCenter([0,0],1)); ...
>A:=[cos(1),sin(1)]; B:=[cos(2),sin(2)]; C:=[cos(6),sin(6)]; ...
>plotPoint(A,"A"); plotPoint(B,"B"); plotPoint(C,"C"); ...
>color(3); plotSegment(A,B,"c"); plotSegment(A,C,"b"); plotSegment(C,B,"a"); ...
>color(1); O:=[0,0];  plotPoint(O,"0"); ...
>plotSegment(A,O); plotSegment(B,O); plotSegment(C,O,"r"); ...
>insimg;
\end{eulerprompt}
\begin{eulercomment}
Kita dapat menggunakan Maxima untuk memecahkan rumus penyebaran
rangkap tiga untuk sudut-sudut di pusat O untuk r. Dengan demikian,
kita memperoleh rumus untuk jari-jari kuadrat pericircle dalam bentuk
kuadran sisi-sisinya.

Kali ini, Maxima menghasilkan beberapa nol kompleks, yang kita
abaikan.
\end{eulercomment}
\begin{eulerprompt}
>&remvalue(a,b,c,r); // hapus nilai-nilai sebelumnya untuk perhitungan baru
>rabc &= rhs(solve(triplespread(spread(b,r,r),spread(a,r,r),spread(c,r,r)),r)[4]); $rabc
\end{eulerprompt}
\begin{eulercomment}
Kita dapat menjadikannya fungsi Euler.
\end{eulercomment}
\begin{eulerprompt}
>function periradius(a,b,c) &= rabc;
\end{eulerprompt}
\begin{eulercomment}
Mari kita periksa hasilnya untuk titik A, B, C.
\end{eulercomment}
\begin{eulerprompt}
>a:=quadrance(B,C); b:=quadrance(A,C); c:=quadrance(A,B);
\end{eulerprompt}
\begin{eulercomment}
Radiusnya memang 1.
\end{eulercomment}
\begin{eulerprompt}
>periradius(a,b,c)
\end{eulerprompt}
\begin{euleroutput}
  1
\end{euleroutput}
\begin{eulercomment}
Faktanya, sebaran CBA hanya bergantung pada b dan c. Ini adalah
teorema sudut tali busur.
\end{eulercomment}
\begin{eulerprompt}
>$spread(b,a,c)*rabc | ratsimp
\end{eulerprompt}
\begin{eulercomment}
Faktanya, sebarannya adalah b/(4r), dan kita melihat bahwa sudut tali
busur b adalah setengah sudut pusat.
\end{eulercomment}
\begin{eulerprompt}
>$doublespread(b/(4*r))-spread(b,r,r) | ratsimp
\end{eulerprompt}
\begin{eulercomment}
\begin{eulercomment}
\eulerheading{Contoh 6: Jarak Minimal pada Bidang}
\begin{eulercomment}
\end{eulercomment}
\eulersubheading{Catatan awal}
\begin{eulercomment}
Fungsi yang, pada titik M di bidang, menetapkan jarak AM antara titik
tetap A dan M, memiliki garis datar yang agak sederhana: lingkaran
yang berpusat di A.
\end{eulercomment}
\begin{eulerprompt}
>&remvalue();
>A=[-1,-1];
>function d1(x,y):=sqrt((x-A[1])^2+(y-A[2])^2)
>fcontour("d1",xmin=-2,xmax=0,ymin=-2,ymax=0,hue=1, ...
>title="If you see ellipses, please set your window square"):
\end{eulerprompt}
\begin{eulercomment}
dan grafiknya cukup sederhana: bagian atas kerucut:
\end{eulercomment}
\begin{eulerprompt}
>plot3d("d1",xmin=-2,xmax=0,ymin=-2,ymax=0):
\end{eulerprompt}
\begin{eulercomment}
Tentu saja minimum 0 dicapai di A.

\end{eulercomment}
\eulersubheading{Dua titik}
\begin{eulercomment}
Sekarang kita lihat fungsi MA+MB di mana A dan B adalah dua titik
(tetap). Merupakan "fakta yang diketahui" bahwa kurva level adalah
elips, titik fokusnya adalah A dan B; kecuali untuk minimum AB yang
konstan pada segmen [AB]:
\end{eulercomment}
\begin{eulerprompt}
>B=[1,-1];
>function d2(x,y):=d1(x,y)+sqrt((x-B[1])^2+(y-B[2])^2)
>fcontour("d2",xmin=-2,xmax=2,ymin=-3,ymax=1,hue=1):
\end{eulerprompt}
\begin{eulercomment}
Grafiknya lebih menarik:
\end{eulercomment}
\begin{eulerprompt}
>plot3d("d2",xmin=-2,xmax=2,ymin=-3,ymax=1):
\end{eulerprompt}
\begin{eulercomment}
Pembatasan pada garis (AB) lebih terkenal:
\end{eulercomment}
\begin{eulerprompt}
>plot2d("abs(x+1)+abs(x-1)",xmin=-3,xmax=3):
\end{eulerprompt}
\eulersubheading{Tiga poin}
\begin{eulercomment}
Sekarang semuanya menjadi kurang sederhana: Tidak banyak yang tahu
bahwa MA+MB+MC mencapai nilai minimumnya di satu titik bidang, tetapi
menentukannya tidaklah sesederhana itu:

1) Jika salah satu sudut segitiga ABC lebih dari 120° (misalkan di A),
maka nilai minimumnya tercapai di titik ini (misalkan AB+AC).

Contoh:
\end{eulercomment}
\begin{eulerprompt}
>C=[-4,1];
>function d3(x,y):=d2(x,y)+sqrt((x-C[1])^2+(y-C[2])^2)
>plot3d("d3",xmin=-5,xmax=3,ymin=-4,ymax=4);
>insimg;
>fcontour("d3",xmin=-4,xmax=1,ymin=-2,ymax=2,hue=1,title="The minimum is on A");
>P=(A_B_C_A)'; plot2d(P[1],P[2],add=1,color=12);
>insimg;
\end{eulerprompt}
\begin{eulercomment}
2) Namun jika semua sudut segitiga ABC kurang dari 120°, maka nilai
minimumnya berada di titik F di bagian dalam segitiga, yang merupakan
satu-satunya titik yang sudut-sudut sisi ABC-nya sama (masing-masing
sudutnya 120°):
\end{eulercomment}
\begin{eulerprompt}
>C=[-0.5,1];
>plot3d("d3",xmin=-2,xmax=2,ymin=-2,ymax=2):
>fcontour("d3",xmin=-2,xmax=2,ymin=-2,ymax=2,hue=1,title="The Fermat point");
>P=(A_B_C_A)'; plot2d(P[1],P[2],add=1,color=12);
>insimg;
\end{eulerprompt}
\begin{eulercomment}
Merupakan aktivitas yang menarik untuk mewujudkan gambar di atas
dengan perangkat lunak geometri; misalnya, saya mengetahui perangkat
lunak yang ditulis dalam Java yang memiliki instruksi "garis
kontur"...

Semua ini ditemukan oleh seorang hakim Prancis bernama Pierre de
Fermat; ia menulis surat kepada para dilettan lain seperti pendeta
Marin Mersenne dan Blaise Pascal yang bekerja di pajak penghasilan.
Jadi titik unik F sehingga FA+FB+FC minimal, disebut titik Fermat dari
segitiga tersebut. Namun tampaknya beberapa tahun sebelumnya,
Torriccelli dari Italia telah menemukan titik ini sebelum Fermat
menemukannya! Bagaimanapun tradisinya adalah mencatat titik F ini...

\end{eulercomment}
\eulersubheading{Empat titik}
\begin{eulercomment}
Langkah berikutnya adalah menambahkan titik ke-4 D dan mencoba
meminimalkan MA+MB+MC+MD; katakanlah Anda adalah operator TV kabel dan
ingin mencari di bidang mana Anda harus meletakkan antena Anda
sehingga Anda dapat menyalurkan sinyal ke empat desa dan menggunakan
panjang kabel sesedikit mungkin!
\end{eulercomment}
\begin{eulerprompt}
>D=[1,1];
>function d4(x,y):=d3(x,y)+sqrt((x-D[1])^2+(y-D[2])^2)
>plot3d("d4",xmin=-1.5,xmax=1.5,ymin=-1.5,ymax=1.5):
>fcontour("d4",xmin=-1.5,xmax=1.5,ymin=-1.5,ymax=1.5,hue=1);
>P=(A_B_C_D)'; plot2d(P[1],P[2],points=1,add=1,color=12);
>insimg;
\end{eulerprompt}
\begin{eulercomment}
Masih terdapat nilai minimum dan tidak tercapai di titik A, B, C,
maupun D:
\end{eulercomment}
\begin{eulerprompt}
>function f(x):=d4(x[1],x[2])
>neldermin("f",[0.2,0.2])
\end{eulerprompt}
\begin{euleroutput}
  [0.142858,  0.142857]
\end{euleroutput}
\begin{eulercomment}
Tampaknya dalam kasus ini, koordinat titik optimal bersifat rasional
atau mendekati rasional...

Sekarang ABCD adalah persegi, kita mengharapkan bahwa titik optimal
akan menjadi pusat ABCD:
\end{eulercomment}
\begin{eulerprompt}
>C=[-1,1];
>plot3d("d4",xmin=-1,xmax=1,ymin=-1,ymax=1):
>fcontour("d4",xmin=-1.5,xmax=1.5,ymin=-1.5,ymax=1.5,hue=1);
>P=(A_B_C_D)'; plot2d(P[1],P[2],add=1,color=12,points=1);
>insimg;
\end{eulerprompt}
\eulerheading{Contoh 7: Bola Dandelin dengan Povray}
\begin{eulercomment}
Anda dapat menjalankan demonstrasi ini, jika Anda telah menginstal
Povray, dan pvengine.exe di jalur program.

Pertama, kita hitung jari-jari bola.

Jika Anda melihat gambar di bawah, Anda melihat bahwa kita memerlukan
dua lingkaran yang menyentuh dua garis yang membentuk kerucut, dan
satu garis yang membentuk bidang yang memotong kerucut.

Kami menggunakan file geometry.e milik Euler untuk ini.
\end{eulercomment}
\begin{eulerprompt}
>load geometry;
\end{eulerprompt}
\begin{eulercomment}
Pertama dua garis membentuk kerucut.
\end{eulercomment}
\begin{eulerprompt}
>g1 &= lineThrough([0,0],[1,a])
\end{eulerprompt}
\begin{euleroutput}
  
                               [- a, 1, 0]
  
\end{euleroutput}
\begin{eulerprompt}
>g2 &= lineThrough([0,0],[-1,a])
\end{eulerprompt}
\begin{euleroutput}
  
                              [- a, - 1, 0]
  
\end{euleroutput}
\begin{eulercomment}
Lalu baris ketiga.
\end{eulercomment}
\begin{eulerprompt}
>g &= lineThrough([-1,0],[1,1])
\end{eulerprompt}
\begin{euleroutput}
  
                               [- 1, 2, 1]
  
\end{euleroutput}
\begin{eulercomment}
Kita merencanakan segalanya sejauh ini.
\end{eulercomment}
\begin{eulerprompt}
>setPlotRange(-1,1,0,2);
>color(black); plotLine(g(),"")
>a:=2; color(blue); plotLine(g1(),""), plotLine(g2(),""):
\end{eulerprompt}
\begin{eulercomment}
Sekarang kita ambil titik umum pada sumbu y.
\end{eulercomment}
\begin{eulerprompt}
>P &= [0,u]
\end{eulerprompt}
\begin{euleroutput}
  
                                  [0, u]
  
\end{euleroutput}
\begin{eulercomment}
Hitunglah jarak ke g1.
\end{eulercomment}
\begin{eulerprompt}
>d1 &= distance(P,projectToLine(P,g1)); $d1
\end{eulerprompt}
\begin{eulercomment}
Hitunglah jarak ke g.
\end{eulercomment}
\begin{eulerprompt}
>d &= distance(P,projectToLine(P,g)); $d
\end{eulerprompt}
\begin{eulercomment}
Dan temukan pusat kedua lingkaran, yang jaraknya sama.
\end{eulercomment}
\begin{eulerprompt}
>sol &= solve(d1^2=d^2,u); $sol
\end{eulerprompt}
\begin{eulercomment}
Ada dua solusi.

Kita mengevaluasi solusi simbolik, dan menemukan kedua pusat, dan
kedua jarak.
\end{eulercomment}
\begin{eulerprompt}
>u := sol()
\end{eulerprompt}
\begin{euleroutput}
  [0.333333,  1]
\end{euleroutput}
\begin{eulerprompt}
>dd := d()
\end{eulerprompt}
\begin{euleroutput}
  [0.149071,  0.447214]
\end{euleroutput}
\begin{eulercomment}
Gambarkan lingkaran-lingkaran tersebut ke dalam gambar.
\end{eulercomment}
\begin{eulerprompt}
>color(red);
>plotCircle(circleWithCenter([0,u[1]],dd[1]),"");
>plotCircle(circleWithCenter([0,u[2]],dd[2]),"");
>insimg;
\end{eulerprompt}
\eulersubheading{Plot dengan Povray}
\begin{eulercomment}
Selanjutnya kita plot semuanya dengan Povray. Perhatikan bahwa Anda
mengubah perintah apa pun dalam urutan perintah Povray berikut, dan
menjalankan kembali semua perintah dengan Shift-Return.

Pertama-tama kita memuat fungsi povray.
\end{eulercomment}
\begin{eulerprompt}
>load povray;
>defaultpovray="C:\(\backslash\)Program Files\(\backslash\)POV-Ray\(\backslash\)v3.7\(\backslash\)bin\(\backslash\)pvengine.exe"
\end{eulerprompt}
\begin{euleroutput}
  C:\(\backslash\)Program Files\(\backslash\)POV-Ray\(\backslash\)v3.7\(\backslash\)bin\(\backslash\)pvengine.exe
\end{euleroutput}
\begin{eulercomment}
Kami menyiapkan suasananya dengan tepat.
\end{eulercomment}
\begin{eulerprompt}
>povstart(zoom=11,center=[0,0,0.5],height=10°,angle=140°);
\end{eulerprompt}
\begin{eulercomment}
Berikutnya kita menulis kedua bola itu ke dalam file Povray.
\end{eulercomment}
\begin{eulerprompt}
>writeln(povsphere([0,0,u[1]],dd[1],povlook(red)));
>writeln(povsphere([0,0,u[2]],dd[2],povlook(red)));
\end{eulerprompt}
\begin{eulercomment}
Dan kerucutnya, transparan.
\end{eulercomment}
\begin{eulerprompt}
>writeln(povcone([0,0,0],0,[0,0,a],1,povlook(lightgray,1)));
\end{eulerprompt}
\begin{eulercomment}
Kita buat bidang yang dibatasi pada kerucut.
\end{eulercomment}
\begin{eulerprompt}
>gp=g();
>pc=povcone([0,0,0],0,[0,0,a],1,"");
>vp=[gp[1],0,gp[2]]; dp=gp[3];
>writeln(povplane(vp,dp,povlook(blue,0.5),pc));
\end{eulerprompt}
\begin{eulercomment}
Sekarang kita buat dua titik pada lingkaran, di mana bola menyentuh
kerucut.
\end{eulercomment}
\begin{eulerprompt}
>function turnz(v) := return [-v[2],v[1],v[3]]
>P1=projectToLine([0,u[1]],g1()); P1=turnz([P1[1],0,P1[2]]);
>writeln(povpoint(P1,povlook(yellow)));
>P2=projectToLine([0,u[2]],g1()); P2=turnz([P2[1],0,P2[2]]);
>writeln(povpoint(P2,povlook(yellow)));
\end{eulerprompt}
\begin{eulercomment}
Kemudian kita buat dua titik tempat bola-bola tersebut menyentuh
bidang. Titik-titik ini adalah fokus elips.
\end{eulercomment}
\begin{eulerprompt}
>P3=projectToLine([0,u[1]],g()); P3=[P3[1],0,P3[2]];
>writeln(povpoint(P3,povlook(yellow)));
>P4=projectToLine([0,u[2]],g()); P4=[P4[1],0,P4[2]];
>writeln(povpoint(P4,povlook(yellow)));
\end{eulerprompt}
\begin{eulercomment}
Berikutnya kita hitung perpotongan P1P2 dengan bidang.
\end{eulercomment}
\begin{eulerprompt}
>t1=scalp(vp,P1)-dp; t2=scalp(vp,P2)-dp; P5=P1+t1/(t1-t2)*(P2-P1);
>writeln(povpoint(P5,povlook(yellow)));
\end{eulerprompt}
\begin{eulercomment}
Kita menghubungkan titik-titik dengan segmen garis.
\end{eulercomment}
\begin{eulerprompt}
>writeln(povsegment(P1,P2,povlook(yellow)));
>writeln(povsegment(P5,P3,povlook(yellow)));
>writeln(povsegment(P5,P4,povlook(yellow)));
\end{eulerprompt}
\begin{eulercomment}
Sekarang kita buat pita abu-abu, di mana bola-bola menyentuh kerucut.
\end{eulercomment}
\begin{eulerprompt}
>pcw=povcone([0,0,0],0,[0,0,a],1.01);
>pc1=povcylinder([0,0,P1[3]-defaultpointsize/2],[0,0,P1[3]+defaultpointsize/2],1);
>writeln(povintersection([pcw,pc1],povlook(gray)));
>pc2=povcylinder([0,0,P2[3]-defaultpointsize/2],[0,0,P2[3]+defaultpointsize/2],1);
>writeln(povintersection([pcw,pc2],povlook(gray)));
\end{eulerprompt}
\begin{eulercomment}
Mulai program Povray.
\end{eulercomment}
\begin{eulerprompt}
>povend();
\end{eulerprompt}
\begin{eulercomment}
Untuk mendapatkan Anaglyph ini, kita perlu memasukkan semuanya ke
dalam fungsi scene. Fungsi ini akan digunakan dua kali nanti.
\end{eulercomment}
\begin{eulerprompt}
>function scene () ...
\end{eulerprompt}
\begin{eulerudf}
  global a,u,dd,g,g1,defaultpointsize;
  writeln(povsphere([0,0,u[1]],dd[1],povlook(red)));
  writeln(povsphere([0,0,u[2]],dd[2],povlook(red)));
  writeln(povcone([0,0,0],0,[0,0,a],1,povlook(lightgray,1)));
  gp=g();
  pc=povcone([0,0,0],0,[0,0,a],1,"");
  vp=[gp[1],0,gp[2]]; dp=gp[3];
  writeln(povplane(vp,dp,povlook(blue,0.5),pc));
  P1=projectToLine([0,u[1]],g1()); P1=turnz([P1[1],0,P1[2]]);
  writeln(povpoint(P1,povlook(yellow)));
  P2=projectToLine([0,u[2]],g1()); P2=turnz([P2[1],0,P2[2]]);
  writeln(povpoint(P2,povlook(yellow)));
  P3=projectToLine([0,u[1]],g()); P3=[P3[1],0,P3[2]];
  writeln(povpoint(P3,povlook(yellow)));
  P4=projectToLine([0,u[2]],g()); P4=[P4[1],0,P4[2]];
  writeln(povpoint(P4,povlook(yellow)));
  t1=scalp(vp,P1)-dp; t2=scalp(vp,P2)-dp; P5=P1+t1/(t1-t2)*(P2-P1);
  writeln(povpoint(P5,povlook(yellow)));
  writeln(povsegment(P1,P2,povlook(yellow)));
  writeln(povsegment(P5,P3,povlook(yellow)));
  writeln(povsegment(P5,P4,povlook(yellow)));
  pcw=povcone([0,0,0],0,[0,0,a],1.01);
  pc1=povcylinder([0,0,P1[3]-defaultpointsize/2],[0,0,P1[3]+defaultpointsize/2],1);
  writeln(povintersection([pcw,pc1],povlook(gray)));
  pc2=povcylinder([0,0,P2[3]-defaultpointsize/2],[0,0,P2[3]+defaultpointsize/2],1);
  writeln(povintersection([pcw,pc2],povlook(gray)));
  endfunction
\end{eulerudf}
\begin{eulercomment}
Anda memerlukan kacamata merah/cyan untuk menghargai efek berikut.
\end{eulercomment}
\begin{eulerprompt}
>povanaglyph("scene",zoom=11,center=[0,0,0.5],height=10°,angle=140°);
\end{eulerprompt}
\eulerheading{Contoh 8: Geometri Bumi}
\begin{eulercomment}
Dalam buku catatan ini, kami ingin melakukan beberapa perhitungan
sferis. Fungsi-fungsi tersebut terdapat dalam berkas "spherical.e" di
folder contoh. Kami perlu memuat berkas tersebut terlebih dahulu.
\end{eulercomment}
\begin{eulerprompt}
>load "spherical.e";
\end{eulerprompt}
\begin{eulercomment}
Untuk memasukkan posisi geografis, kami menggunakan vektor dengan dua
koordinat dalam radian (utara dan timur, nilai negatif untuk selatan
dan barat). Berikut ini adalah koordinat untuk Kampus FMIPA UNY.
\end{eulercomment}
\begin{eulerprompt}
>FMIPA=[rad(-7,-46.467),rad(110,23.05)]
\end{eulerprompt}
\begin{euleroutput}
  [-0.13569,  1.92657]
\end{euleroutput}
\begin{eulercomment}
Anda dapat mencetak posisi ini dengan sposprint (cetak posisi bulat).
\end{eulercomment}
\begin{eulerprompt}
>sposprint(FMIPA) // posisi garis lintang dan garis bujur FMIPA UNY
\end{eulerprompt}
\begin{euleroutput}
  S 7°46.467' E 110°23.050'
\end{euleroutput}
\begin{eulercomment}
Mari kita tambahkan dua kota lagi, Solo dan Semarang.
\end{eulercomment}
\begin{eulerprompt}
>Solo=[rad(-7,-34.333),rad(110,49.683)]; Semarang=[rad(-6,-59.05),rad(110,24.533)];
>sposprint(Solo), sposprint(Semarang),
\end{eulerprompt}
\begin{euleroutput}
  S 7°34.333' E 110°49.683'
  S 6°59.050' E 110°24.533'
\end{euleroutput}
\begin{eulercomment}
Pertama, kita hitung vektor dari satu ke yang lain pada bola ideal.
Vektor ini adalah [arah, jarak] dalam radian. Untuk menghitung jarak
di bumi, kita kalikan dengan jari-jari bumi pada garis lintang 7°.
\end{eulercomment}
\begin{eulerprompt}
>br=svector(FMIPA,Solo); degprint(br[1]), br[2]*rearth(7°)->km // perkiraan jarak FMIPA-Solo
\end{eulerprompt}
\begin{euleroutput}
  65°20'26.60''
  53.8945384608
\end{euleroutput}
\begin{eulercomment}
Ini adalah perkiraan yang bagus. Rutin berikut menggunakan perkiraan
yang lebih baik lagi. Pada jarak yang pendek, hasilnya hampir sama.
\end{eulercomment}
\begin{eulerprompt}
>esdist(FMIPA,Semarang)->" km" // perkiraan jarak FMIPA-Semarang
\end{eulerprompt}
\begin{euleroutput}
  Commands must be separated by semicolon or comma!
  Found:  // perkiraan jarak FMIPA-Semarang (character 32)
  You can disable this in the Options menu.
  Error in:
  esdist(FMIPA,Semarang)->" km" // perkiraan jarak FMIPA-Semaran ...
                               ^
\end{euleroutput}
\begin{eulercomment}
Ada fungsi untuk judul, yang memperhitungkan bentuk elips bumi. Sekali
lagi, kami mencetak dengan cara yang canggih.
\end{eulercomment}
\begin{eulerprompt}
>sdegprint(esdir(FMIPA,Solo))
\end{eulerprompt}
\begin{euleroutput}
       65.34°
\end{euleroutput}
\begin{eulercomment}
Sudut suatu segitiga melebihi 180° pada bola.
\end{eulercomment}
\begin{eulerprompt}
>asum=sangle(Solo,FMIPA,Semarang)+sangle(FMIPA,Solo,Semarang)+sangle(FMIPA,Semarang,Solo); degprint(asum)
\end{eulerprompt}
\begin{euleroutput}
  180°0'10.77''
\end{euleroutput}
\begin{eulercomment}
Ini dapat digunakan untuk menghitung luas segitiga. Catatan: Untuk
segitiga kecil, ini tidak akurat karena kesalahan pengurangan dalam
asum-pi.
\end{eulercomment}
\begin{eulerprompt}
>(asum-pi)*rearth(48°)^2->" km^2" // perkiraan luas segitiga FMIPA-Solo-Semarang
\end{eulerprompt}
\begin{euleroutput}
  Commands must be separated by semicolon or comma!
  Found:  // perkiraan luas segitiga FMIPA-Solo-Semarang (character 32)
  You can disable this in the Options menu.
  Error in:
  (asum-pi)*rearth(48°)^2->" km^2" // perkiraan luas segitiga FM ...
                                  ^
\end{euleroutput}
\begin{eulercomment}
Ada fungsi untuk ini, yang menggunakan lintang rata-rata segitiga
untuk menghitung jari-jari bumi, dan menangani kesalahan pembulatan
untuk segitiga yang sangat kecil.
\end{eulercomment}
\begin{eulerprompt}
>esarea(Solo,FMIPA,Semarang)->" km^2", //perkiraan yang sama dengan fungsi esarea()
\end{eulerprompt}
\begin{euleroutput}
  2123.64310526 km^2
\end{euleroutput}
\begin{eulercomment}
Kita juga dapat menambahkan vektor ke posisi. Vektor berisi arah dan
jarak, keduanya dalam radian. Untuk mendapatkan vektor, kita
menggunakan svector. Untuk menambahkan vektor ke posisi, kita
menggunakan saddvector.
\end{eulercomment}
\begin{eulerprompt}
>v=svector(FMIPA,Solo); sposprint(saddvector(FMIPA,v)), sposprint(Solo),
\end{eulerprompt}
\begin{euleroutput}
  S 7°34.333' E 110°49.683'
  S 7°34.333' E 110°49.683'
\end{euleroutput}
\begin{eulercomment}
Fungsi-fungsi ini mengasumsikan bentuk bola yang ideal. Sama halnya di
bumi.
\end{eulercomment}
\begin{eulerprompt}
>sposprint(esadd(FMIPA,esdir(FMIPA,Solo),esdist(FMIPA,Solo))), sposprint(Solo),
\end{eulerprompt}
\begin{euleroutput}
  S 7°34.333' E 110°49.683'
  S 7°34.333' E 110°49.683'
\end{euleroutput}
\begin{eulercomment}
Mari kita lihat contoh yang lebih besar, Tugu Jogja dan Monas Jakarta
(menggunakan Google Earth untuk mencari koordinatnya).
\end{eulercomment}
\begin{eulerprompt}
>Tugu=[-7.7833°,110.3661°]; Monas=[-6.175°,106.811944°];
>sposprint(Tugu), sposprint(Monas)
\end{eulerprompt}
\begin{euleroutput}
  S 7°46.998' E 110°21.966'
  S 6°10.500' E 106°48.717'
\end{euleroutput}
\begin{eulercomment}
Menurut Google Earth, jaraknya adalah 429,66 km. Kami memperoleh
perkiraan yang baik.
\end{eulercomment}
\begin{eulerprompt}
>esdist(Tugu,Monas)->" km" // perkiraan jarak Tugu Jogja - Monas Jakarta
\end{eulerprompt}
\begin{euleroutput}
  Commands must be separated by semicolon or comma!
  Found:  // perkiraan jarak Tugu Jogja - Monas Jakarta (character 32)
  You can disable this in the Options menu.
  Error in:
  esdist(Tugu,Monas)->" km" // perkiraan jarak Tugu Jogja - Mona ...
                           ^
\end{euleroutput}
\begin{eulercomment}
Judulnya sama dengan yang dihitung di Google Earth.
\end{eulercomment}
\begin{eulerprompt}
>degprint(esdir(Tugu,Monas))
\end{eulerprompt}
\begin{euleroutput}
  294°17'2.85''
\end{euleroutput}
\begin{eulercomment}
Akan tetapi, kita tidak lagi memperoleh posisi target yang tepat, jika
kita menambahkan arah dan jarak ke posisi awal. Hal ini terjadi karena
kita tidak menghitung fungsi invers secara tepat, tetapi mengambil
perkiraan radius bumi di sepanjang lintasan.
\end{eulercomment}
\begin{eulerprompt}
>sposprint(esadd(Tugu,esdir(Tugu,Monas),esdist(Tugu,Monas)))
\end{eulerprompt}
\begin{euleroutput}
  S 6°10.500' E 106°48.717'
\end{euleroutput}
\begin{eulercomment}
Namun, kesalahannya tidak besar.
\end{eulercomment}
\begin{eulerprompt}
>sposprint(Monas),
\end{eulerprompt}
\begin{euleroutput}
  S 6°10.500' E 106°48.717'
\end{euleroutput}
\begin{eulercomment}
Tentu saja, kita tidak dapat berlayar dengan arah yang sama dari satu
tujuan ke tujuan lain, jika kita ingin mengambil jalur terpendek.
Bayangkan, Anda terbang ke arah timur laut mulai dari titik mana pun
di bumi. Kemudian Anda akan berputar ke kutub utara. Lingkaran besar
tidak mengikuti arah yang konstan!

Perhitungan berikut menunjukkan bahwa kita jauh dari tujuan yang
benar, jika kita menggunakan arah yang sama selama perjalanan kita.
\end{eulercomment}
\begin{eulerprompt}
>dist=esdist(Tugu,Monas); hd=esdir(Tugu,Monas);
\end{eulerprompt}
\begin{eulercomment}
Sekarang kita tambahkan 10 dikalikan sepersepuluh jaraknya, dengan
memakai arah ke Monas, kita sampai di Tugu.
\end{eulercomment}
\begin{eulerprompt}
>p=Tugu; loop 1 to 10; p=esadd(p,hd,dist/10); end;
\end{eulerprompt}
\begin{eulercomment}
Hasilnya sangat jauh.
\end{eulercomment}
\begin{eulerprompt}
>sposprint(p), skmprint(esdist(p,Monas))
\end{eulerprompt}
\begin{euleroutput}
  S 6°11.250' E 106°48.372'
       1.529km
\end{euleroutput}
\begin{eulercomment}
Sebagai contoh lain, mari kita ambil dua titik di bumi pada garis
lintang yang sama.
\end{eulercomment}
\begin{eulerprompt}
>P1=[30°,10°]; P2=[30°,50°];
\end{eulerprompt}
\begin{eulercomment}
Lintasan terpendek dari P1 ke P2 bukanlah lingkaran lintang 30°,
tetapi lintasan yang lebih pendek yang dimulai 10° lebih jauh ke utara
di P1.
\end{eulercomment}
\begin{eulerprompt}
>sdegprint(esdir(P1,P2))
\end{eulerprompt}
\begin{euleroutput}
       79.69°
\end{euleroutput}
\begin{eulercomment}
Namun, jika kita mengikuti pembacaan kompas ini, kita akan berputar ke
kutub utara! Jadi kita harus menyesuaikan arah kita di sepanjang
jalan. Untuk tujuan kasar, kita menyesuaikannya pada 1/10 dari total
jarak.
\end{eulercomment}
\begin{eulerprompt}
>p=P1;  dist=esdist(P1,P2); ...
>  loop 1 to 10; dir=esdir(p,P2); sdegprint(dir), p=esadd(p,dir,dist/10); end;
\end{eulerprompt}
\begin{euleroutput}
       79.69°
       81.67°
       83.71°
       85.78°
       87.89°
       90.00°
       92.12°
       94.22°
       96.29°
       98.33°
\end{euleroutput}
\begin{eulercomment}
Jaraknya tidak tepat, karena kita akan menambahkan sedikit kesalahan,
jika kita mengikuti arah yang sama terlalu lama.
\end{eulercomment}
\begin{eulerprompt}
>skmprint(esdist(p,P2))
\end{eulerprompt}
\begin{euleroutput}
       0.203km
\end{euleroutput}
\begin{eulercomment}
Kita memperoleh perkiraan yang baik, jika kita menyesuaikan arah
setelah setiap 1/100 jarak total dari Tugu ke Monas.
\end{eulercomment}
\begin{eulerprompt}
>p=Tugu; dist=esdist(Tugu,Monas); ...
>  loop 1 to 100; p=esadd(p,esdir(p,Monas),dist/100); end;
>skmprint(esdist(p,Monas))
\end{eulerprompt}
\begin{euleroutput}
       0.000km
\end{euleroutput}
\begin{eulercomment}
Untuk keperluan navigasi, kita bisa mendapatkan urutan posisi GPS
sepanjang lingkaran besar menuju Monas dengan fungsi navigasi.
\end{eulercomment}
\begin{eulerprompt}
>load spherical; v=navigate(Tugu,Monas,10); ...
>  loop 1 to rows(v); sposprint(v[#]), end;
\end{eulerprompt}
\begin{euleroutput}
  S 7°46.998' E 110°21.966'
  S 7°37.422' E 110°0.573'
  S 7°27.829' E 109°39.196'
  S 7°18.219' E 109°17.834'
  S 7°8.592' E 108°56.488'
  S 6°58.948' E 108°35.157'
  S 6°49.289' E 108°13.841'
  S 6°39.614' E 107°52.539'
  S 6°29.924' E 107°31.251'
  S 6°20.219' E 107°9.977'
  S 6°10.500' E 106°48.717'
\end{euleroutput}
\begin{eulercomment}
Kita menulis suatu fungsi yang memplot bumi, dua posisi, dan posisi di
antaranya.
\end{eulercomment}
\begin{eulerprompt}
>function testplot ...
\end{eulerprompt}
\begin{eulerudf}
  useglobal;
  plotearth;
  plotpos(Tugu,"Tugu Jogja"); plotpos(Monas,"Tugu Monas");
  plotposline(v);
  endfunction
\end{eulerudf}
\begin{eulercomment}
Sekarang rencanakan semuanya.
\end{eulercomment}
\begin{eulerprompt}
>plot3d("testplot",angle=25, height=6,>own,>user,zoom=4):
\end{eulerprompt}
\begin{eulercomment}
Atau gunakan plot3d untuk mendapatkan tampilan anaglifnya. Ini tampak
sangat bagus dengan kaca mata merah/biru kehijauan.
\end{eulercomment}
\begin{eulerprompt}
>plot3d("testplot",angle=25,height=6,distance=5,own=1,anaglyph=1,zoom=4):
\end{eulerprompt}
\eulerheading{Latihan}
\begin{eulercomment}
1. Gambarlah segi-n beraturan jika diketahui titik pusat O, n, dan
jarak titik pusat ke titik-titik sudut segi-n tersebut (jari-jari
lingkaran luar segi-n), r.

Petunjuk:

- Besar sudut pusat yang menghadap masing-masing sisi segi-n adalah
(360/n).\\
- Titik-titik sudut segi-n merupakan perpotongan lingkaran luar segi-n
dan garis-garis yang melalui pusat dan saling membentuk sudut sebesar
kelipatan (360/n).\\
- Untuk n ganjil, pilih salah satu titik sudut adalah di atas.\\
- Untuk n genap, pilih 2 titik di kanan dan kiri lurus dengan titik
pusat.\\
- Anda dapat menggambar segi-3, 4, 5, 6, 7, dst beraturan.
\end{eulercomment}
\begin{eulerprompt}
>load geometry
\end{eulerprompt}
\begin{euleroutput}
  Numerical and symbolic geometry.
\end{euleroutput}
\begin{eulerprompt}
>setPlotRange(-3.5,3.5,-3.5,3.5);
>A=[-2,-2]; plotPoint(A,"A");
>B=[2,-2]; plotPoint(B,"B");
>C=[0,3]; plotPoint(C,"C");
>plotSegment(A,B,"c");
>plotSegment(B,C,"a");
>plotSegment(A,C,"b");
>aspect(1):
>c=circleThrough(A,B,C);
>R=getCircleRadius(c);
>O=getCircleCenter(c);
>plotPoint(O,"O");
>l=angleBisector(A,C,B);
>color(2); plotLine(l); color(1);
>plotCircle(c,"Lingkaran luar segitiga ABC"):
\end{eulerprompt}
\begin{eulercomment}
2. Gambarlah suatu parabola yang melalui 3 titik yang diketahui.

Petunjuk:\\
- Misalkan persamaan parabolanya y= ax\textasciicircum{}2+bx+c.\\
- Substitusikan koordinat titik-titik yang diketahui ke persamaan
tersebut.\\
- Selesaikan SPL yang terbentuk untuk mendapatkan nilai-nilai a, b, c.
\end{eulercomment}
\begin{eulerprompt}
>load geometry;
>setPlotRange(5); P=[2,0]; Q=[4,0]; R=[0,-4];
>plotPoint(P,"P"); plotPoint(Q,"Q"); plotPoint(R,"R"):
>sol &= solve([a+b=-c,16*a+4*b=-c,c=-4],[a,b,c])
\end{eulerprompt}
\begin{euleroutput}
  
                       [[a = - 1, b = 5, c = - 4]]
  
\end{euleroutput}
\begin{eulercomment}
Sehingga dapat ditentukan nilai a = -1, b = 5 dan c = -4
\end{eulercomment}
\begin{eulerprompt}
>function y&=4*x^2+5*x-12
\end{eulerprompt}
\begin{euleroutput}
  
                                2
                             4 x  + 5 x - 12
  
\end{euleroutput}
\begin{eulercomment}
\end{eulercomment}
\begin{eulerprompt}
>plot2d("4*x^2+5*x-12",-13,13,-13,13):
\end{eulerprompt}
\begin{eulercomment}
3. Gambarlah suatu segi-4 yang diketahui keempat titik sudutnya,
misalnya A, B, C, D.\\
\end{eulercomment}
\begin{eulerttcomment}
   - Tentukan apakah segi-4 tersebut merupakan segi-4 garis singgung
\end{eulerttcomment}
\begin{eulercomment}
(sisinya-sisintya merupakan garis singgung lingkaran yang sama yakni
lingkaran dalam segi-4 tersebut).\\
\end{eulercomment}
\begin{eulerttcomment}
   - Suatu segi-4 merupakan segi-4 garis singgung apabila keempat
\end{eulerttcomment}
\begin{eulercomment}
garis bagi sudutnya bertemu di satu titik.\\
\end{eulercomment}
\begin{eulerttcomment}
   - Jika segi-4 tersebut merupakan segi-4 garis singgung, gambar
\end{eulerttcomment}
\begin{eulercomment}
lingkaran dalamnya.\\
\end{eulercomment}
\begin{eulerttcomment}
   - Tunjukkan bahwa syarat suatu segi-4 merupakan segi-4 garis
\end{eulerttcomment}
\begin{eulercomment}
singgung apabila hasil kali panjang sisi-sisi yang berhadapan sama.
\end{eulercomment}
\begin{eulerprompt}
>load geometry
\end{eulerprompt}
\begin{euleroutput}
  Numerical and symbolic geometry.
\end{euleroutput}
\begin{eulerprompt}
>setPlotRange(-5,5,-5,5);
>A=[-4,-4]; plotPoint(A,"A");
>B=[4,-4]; plotPoint(B,"B");
>C=[4,4]; plotPoint(C,"C");
>D=[-4,4]; plotPoint(D,"D");
>plotSegment(A,B,"");
>plotSegment(B,C,"");
>plotSegment(C,D,"");
>plotSegment(A,D,"");
>aspect(1):
>l=angleBisector(A,B,C);
>m=angleBisector(B,C,D);
>P=lineIntersection(l,m);
>color(5); plotLine(l); plotLine(m); color(1);
>plotPoint(P,"P"):
\end{eulerprompt}
\begin{eulercomment}
Dapat dilihat bahwa keempat garis bagi sudutnya bertemu di satu titik
yaitu titik P.
\end{eulercomment}
\begin{eulerprompt}
>r=norm(P-projectToLine(P,lineThrough(A,B)));
>plotCircle(circleWithCenter(P,r),"Lingkaran dalam segiempat ABCD"):
\end{eulerprompt}
\begin{eulercomment}
Dapat dilihat bahwa sisi-sisinya merupakan garis singgung lingkaran
yang sama.

Akan ditunjukkan bahwa hasil kali panjang sisi-sisi yang berhadapan
sama.
\end{eulercomment}
\begin{eulerprompt}
>AB=norm(A-B)
\end{eulerprompt}
\begin{euleroutput}
  8
\end{euleroutput}
\begin{eulerprompt}
>CD=norm(C-D)
\end{eulerprompt}
\begin{euleroutput}
  8
\end{euleroutput}
\begin{eulerprompt}
>AD=norm(A-D)
\end{eulerprompt}
\begin{euleroutput}
  8
\end{euleroutput}
\begin{eulerprompt}
>BC=norm(B-C)
\end{eulerprompt}
\begin{euleroutput}
  8
\end{euleroutput}
\begin{eulerprompt}
>AB.CD
\end{eulerprompt}
\begin{euleroutput}
  64
\end{euleroutput}
\begin{eulerprompt}
>AD.BC
\end{eulerprompt}
\begin{euleroutput}
  64
\end{euleroutput}
\begin{eulercomment}
Terbukti bahwa hasil kali panjang sisi-sisi yang berhadapan sama yaitu
64. Jadi dapat dipastikan bahwa segiempat tersebut merupakan segiempat
garis singgung.


4. Gambarlah suatu ellips jika diketahui kedua titik fokusnya,
misalnya P dan Q. Ingat ellips dengan fokus P dan Q adalah tempat
kedudukan titik-titik yang jumlah jarak ke P dan ke Q selalu sama
(konstan).


Penyelesaian :\\
Diketahui kedua titik fokus P = [-2,-2] dan Q = [2,-2]
\end{eulercomment}
\begin{eulerprompt}
>P=[-2,-2]; Q=[2,-2];
>function d1(x,y):=sqrt((x-P[1])^2+(y-P[2])^2)
>Q=[2,-2]; function d2(x,y):=sqrt((x-P[1])^2+(y-P[2])^2)+sqrt((x-Q[1])^2+(y-Q[2])^2)
>fcontour("d2",xmin=-2,xmax=2,ymin=-3,ymax=1,hue=1):
>plot3d("d2",xmin=-2,xmax=2,ymin=-3,ymax=1):
>plot2d("abs(x+1)+abs(x-1)",xmin=-3,xmax=3):
\end{eulerprompt}
\begin{eulercomment}
5. Gambarlah suatu hiperbola jika diketahui kedua titik fokusnya,
misalnya P dan Q. Ingat ellips dengan fokus P dan Q adalah tempat
kedudukan titik-titik yang selisih jarak ke P dan ke Q selalu sama
(konstan).
\end{eulercomment}
\begin{eulerprompt}
>P=[-2,-2]; Q=[2,-2];
>function d1(x,y):=sqrt((x-p[1])^2+(y-p[2])^2)
>Q=[2,-2]; function d2(x,y):=sqrt((x-P[1])^2+(y-P[2])^2)+sqrt((x+Q[1])^2+(y+Q[2])^2)
>fcontour("d2",xmin=-2,xmax=2,ymin=-3,ymax=1,hue=1):
>plot3d("d2",xmin=-2,xmax=2,ymin=-3,ymax=1):
>plot2d("abs(x+1)+abs(x-1)",xmin=-3,xmax=3):
>  
\end{eulerprompt}
\begin{euleroutput}
  
\end{euleroutput}
\begin{eulercomment}
\begin{eulercomment}
\eulerheading{EMT untuk Statistika}
\begin{eulercomment}
Dalam buku catatan ini, kami mendemonstrasikan plot statistik utama,
tes dan distribusi dalam Euler.

Mari kita mulai dengan beberapa statistik deskriptif. Ini bukanlah
sebuah pengantar statistik. Jadi, Anda mungkin memerlukan beberapa
latar belakang untuk memahami detailnya.

Asumsikan pengukuran berikut ini. Kita ingin menghitung nilai
rata-rata dan deviasi standar yang diukur.

\end{eulercomment}
\begin{eulerprompt}
>M=[1000,1004,998,997,1002,1001,998,1004,998,997]; ...
>median(M), mean(M), dev(M),
\end{eulerprompt}
\begin{euleroutput}
  999
  999.9
  2.72641400622
\end{euleroutput}
\begin{eulercomment}
Kita dapat memplot plot kotak dan kumis untuk data tersebut. Dalam
kasus kami, tidak ada pencilan.
\end{eulercomment}
\begin{eulerprompt}
>aspect(1.75); boxplot(M):
\end{eulerprompt}
\begin{eulercomment}
Kami menghitung probabilitas bahwa suatu nilai lebih besar dari 1005,
dengan mengasumsikan nilai yang diukur dari distribusi normal.

Semua fungsi untuk distribusi dalam Euler diakhiri dengan ...dis dan
menghitung distribusi probabilitas kumulatif (CPF).

\end{eulercomment}
\begin{eulerformula}
\[
\text{normaldis(x,m,d)}=\int_{-\infty}^x \frac{1}{d\sqrt{2\pi}}e^{-\frac{1}{2}(\frac{t-m}{d})^2}\ dt.
\]
\end{eulerformula}
\begin{eulercomment}
Kami mencetak hasilnya dalam \% dengan akurasi 2 digit menggunakan
fungsi cetak.

\end{eulercomment}
\begin{eulerprompt}
>print((1-normaldis(1005,mean(M),dev(M)))*100,2,unit=" %")
\end{eulerprompt}
\begin{euleroutput}
        3.07 %
\end{euleroutput}
\begin{eulercomment}
Untuk contoh berikutnya, kami mengasumsikan jumlah pria berikut ini
dalam rentang ukuran tertentu.
\end{eulercomment}
\begin{eulerprompt}
>r=155.5:4:187.5; v=[22,71,136,169,139,71,32,8];
\end{eulerprompt}
\begin{eulercomment}
Berikut ini adalah plot distribusinya.
\end{eulercomment}
\begin{eulerprompt}
>plot2d(r,v,a=150,b=200,c=0,d=190,bar=1,style="\(\backslash\)/"):
\end{eulerprompt}
\begin{eulercomment}
Kita dapat memasukkan data mentah tersebut ke dalam tabel.

Tabel adalah sebuah metode untuk menyimpan data statistik. Tabel kita
harus berisi tiga kolom: Awal rentang, akhir rentang, jumlah orang
dalam rentang.

Tabel dapat dicetak dengan header. Kami menggunakan vektor string
untuk mengatur header.
\end{eulercomment}
\begin{eulerprompt}
>T:=r[1:8]' | r[2:9]' | v'; writetable(T,labc=["BB","BA","Frek"])
\end{eulerprompt}
\begin{euleroutput}
          BB        BA      Frek
       155.5     159.5        22
       159.5     163.5        71
       163.5     167.5       136
       167.5     171.5       169
       171.5     175.5       139
       175.5     179.5        71
       179.5     183.5        32
       183.5     187.5         8
\end{euleroutput}
\begin{eulercomment}
Jika kita membutuhkan nilai rata-rata dan statistik lain dari ukuran,
kita perlu menghitung titik tengah rentang. Kita dapat menggunakan dua
kolom pertama dari tabel kita untuk hal ini.

Sumbol “\textbar{}” digunakan untuk memisahkan kolom, fungsi “writetable”
digunakan untuk menulis tabel, dengan opsi “labc” untuk menentukan
judul kolom.
\end{eulercomment}
\begin{eulerprompt}
>(T[,1]+T[,2])/2 // the midpoint of each interval
\end{eulerprompt}
\begin{euleroutput}
          157.5 
          161.5 
          165.5 
          169.5 
          173.5 
          177.5 
          181.5 
          185.5 
\end{euleroutput}
\begin{eulercomment}
Tetapi akan lebih mudah, untuk melipat rentang dengan vektor
[1/2,1/2].
\end{eulercomment}
\begin{eulerprompt}
>M=fold(r,[0.5,0.5])
\end{eulerprompt}
\begin{euleroutput}
  [157.5,  161.5,  165.5,  169.5,  173.5,  177.5,  181.5,  185.5]
\end{euleroutput}
\begin{eulercomment}
Sekarang kita dapat menghitung rata-rata dan deviasi sampel dengan
frekuensi yang diberikan.
\end{eulercomment}
\begin{eulerprompt}
>\{m,d\}=meandev(M,v); m, d,
\end{eulerprompt}
\begin{euleroutput}
  169.901234568
  5.98912964449
\end{euleroutput}
\begin{eulercomment}
Mari kita tambahkan distribusi normal dari nilai-nilai tersebut ke
dalam diagram batang di atas. Rumus untuk distribusi normal dengan
rata-rata m dan deviasi standar d adalah:

\end{eulercomment}
\begin{eulerformula}
\[
y=\frac{1}{d\sqrt{2\pi}}e^{\frac{-(x-m)^2}{2d^2}}.
\]
\end{eulerformula}
\begin{eulercomment}
Karena nilainya antara 0 dan 1, untuk memplotnya pada diagram batang,
nilai tersebut harus dikalikan dengan 4 kali jumlah data.
\end{eulercomment}
\begin{eulerprompt}
>plot2d("qnormal(x,m,d)*sum(v)*4", ...
>  xmin=min(r),xmax=max(r),thickness=3,add=1):
\end{eulerprompt}
\eulerheading{Tabel}
\begin{eulercomment}
Dalam direktori buku catatan ini, Anda dapat menemukan file dengan
tabel. Data tersebut mewakili hasil survei. Berikut adalah empat baris
pertama dari file tersebut. Data berasal dari buku online berbahasa
Jerman “Einführung in die Statistik mit R” oleh A. Handl.
\end{eulercomment}
\begin{eulerprompt}
>printfile("table.dat",4);
\end{eulerprompt}
\begin{euleroutput}
  Person Sex Age Titanic Evaluation Tip Problem
  1 m 30 n . 1.80 n
  2 f 23 y g 1.80 n
  3 f 26 y g 1.80 y
\end{euleroutput}
\begin{eulercomment}
Tabel berisi 7 kolom angka atau token (string). Kita ingin membaca
tabel tersebut dari file. Pertama, kita menggunakan terjemahan kita
sendiri untuk token-token tersebut.

Untuk itu, kita mendefinisikan set token. Fungsi strtokens()
mendapatkan vektor string token dari string yang diberikan.
\end{eulercomment}
\begin{eulerprompt}
>mf:=["m","f"]; yn:=["y","n"]; ev:=strtokens("g vg m b vb");
\end{eulerprompt}
\begin{eulercomment}
Sekarang kita membaca tabel dengan terjemahan ini.

Argumen tok2, tok4, dan lain-lain adalah terjemahan dari kolom-kolom
tabel. Argumen-argumen ini tidak ada dalam daftar parameter
readtable(), jadi Anda harus menyediakannya dengan “:=”.
\end{eulercomment}
\begin{eulerprompt}
>\{MT,hd\}=readtable("table.dat",tok2:=mf,tok4:=yn,tok5:=ev,tok7:=yn);
>load over statistics;
\end{eulerprompt}
\begin{eulercomment}
Untuk mencetak, kita perlu menentukan set token yang sama. Kami
mencetak empat baris pertama saja.
\end{eulercomment}
\begin{eulerprompt}
>writetable(MT[1:10],labc=hd,wc=5,tok2:=mf,tok4:=yn,tok5:=ev,tok7:=yn);
\end{eulerprompt}
\begin{euleroutput}
   Person  Sex  Age Titanic Evaluation  Tip Problem
        1    m   30       n          .  1.8       n
        2    f   23       y          g  1.8       n
        3    f   26       y          g  1.8       y
        4    m   33       n          .  2.8       n
        5    m   37       n          .  1.8       n
        6    m   28       y          g  2.8       y
        7    f   31       y         vg  2.8       n
        8    m   23       n          .  0.8       n
        9    f   24       y         vg  1.8       y
       10    m   26       n          .  1.8       n
\end{euleroutput}
\begin{eulercomment}
Tanda titik “.” mewakili nilai yang tidak tersedia.

Jika kita tidak ingin menentukan token untuk terjemahan sebelumnya,
kita hanya perlu menentukan kolom mana yang berisi token dan bukan
angka.
\end{eulercomment}
\begin{eulerprompt}
>ctok=[2,4,5,7]; \{MT,hd,tok\}=readtable("table.dat",ctok=ctok);
\end{eulerprompt}
\begin{eulercomment}
Fungsi readtable() sekarang mengembalikan satu set token.
\end{eulercomment}
\begin{eulerprompt}
>tok
\end{eulerprompt}
\begin{euleroutput}
  m
  n
  f
  y
  g
  vg
\end{euleroutput}
\begin{eulercomment}
Tabel berisi entri dari file dengan token yang diterjemahkan ke angka.

String khusus NA=“.” ditafsirkan sebagai “Tidak Tersedia”, dan
mendapatkan NAN (bukan angka) dalam tabel. Terjemahan ini dapat diubah
dengan parameter NA, dan NAval.
\end{eulercomment}
\begin{eulerprompt}
>MT[1]
\end{eulerprompt}
\begin{euleroutput}
  [1,  1,  30,  2,  NAN,  1.8,  2]
\end{euleroutput}
\begin{eulercomment}
Berikut ini adalah isi tabel dengan angka yang tidak diterjemahkan.
\end{eulercomment}
\begin{eulerprompt}
>writetable(MT,wc=5)
\end{eulerprompt}
\begin{euleroutput}
      1    1   30    2    .  1.8    2
      2    3   23    4    5  1.8    2
      3    3   26    4    5  1.8    4
      4    1   33    2    .  2.8    2
      5    1   37    2    .  1.8    2
      6    1   28    4    5  2.8    4
      7    3   31    4    6  2.8    2
      8    1   23    2    .  0.8    2
      9    3   24    4    6  1.8    4
     10    1   26    2    .  1.8    2
     11    3   23    4    6  1.8    4
     12    1   32    4    5  1.8    2
     13    1   29    4    6  1.8    4
     14    3   25    4    5  1.8    4
     15    3   31    4    5  0.8    2
     16    1   26    4    5  2.8    2
     17    1   37    2    .  3.8    2
     18    1   38    4    5    .    2
     19    3   29    2    .  3.8    2
     20    3   28    4    6  1.8    2
     21    3   28    4    1  2.8    4
     22    3   28    4    6  1.8    4
     23    3   38    4    5  2.8    2
     24    3   27    4    1  1.8    4
     25    1   27    2    .  2.8    4
\end{euleroutput}
\begin{eulercomment}
Untuk kenyamanan, Anda dapat menaruh output dari readtable() ke dalam
sebuah daftar.
\end{eulercomment}
\begin{eulerprompt}
>Table=\{\{readtable("table.dat",ctok=ctok)\}\};
\end{eulerprompt}
\begin{eulercomment}
Dengan menggunakan kolom token yang sama dan token yang dibaca dari
file, kita dapat mencetak tabel. Kita dapat menentukan ctok, tok, dll.
atau menggunakan daftar Tabel.
\end{eulercomment}
\begin{eulerprompt}
>writetable(Table,ctok=ctok,wc=5);
\end{eulerprompt}
\begin{euleroutput}
   Person  Sex  Age Titanic Evaluation  Tip Problem
        1    m   30       n          .  1.8       n
        2    f   23       y          g  1.8       n
        3    f   26       y          g  1.8       y
        4    m   33       n          .  2.8       n
        5    m   37       n          .  1.8       n
        6    m   28       y          g  2.8       y
        7    f   31       y         vg  2.8       n
        8    m   23       n          .  0.8       n
        9    f   24       y         vg  1.8       y
       10    m   26       n          .  1.8       n
       11    f   23       y         vg  1.8       y
       12    m   32       y          g  1.8       n
       13    m   29       y         vg  1.8       y
       14    f   25       y          g  1.8       y
       15    f   31       y          g  0.8       n
       16    m   26       y          g  2.8       n
       17    m   37       n          .  3.8       n
       18    m   38       y          g    .       n
       19    f   29       n          .  3.8       n
       20    f   28       y         vg  1.8       n
       21    f   28       y          m  2.8       y
       22    f   28       y         vg  1.8       y
       23    f   38       y          g  2.8       n
       24    f   27       y          m  1.8       y
       25    m   27       n          .  2.8       y
\end{euleroutput}
\begin{eulercomment}
Fungsi tablecol() mengembalikan nilai kolom dari tabel, melewatkan
setiap baris dengan nilai NAN (“.” dalam file), dan indeks kolom, yang
berisi nilai-nilai ini.
\end{eulercomment}
\begin{eulerprompt}
>\{c,i\}=tablecol(MT,[5,6]);
\end{eulerprompt}
\begin{eulercomment}
Kita dapat menggunakan ini untuk mengekstrak kolom dari tabel untuk
tabel baru.
\end{eulercomment}
\begin{eulerprompt}
>j=[1,5,6]; writetable(MT[i,j],labc=hd[j],ctok=[2],tok=tok)
\end{eulerprompt}
\begin{euleroutput}
      Person Evaluation       Tip
           2          g       1.8
           3          g       1.8
           6          g       2.8
           7         vg       2.8
           9         vg       1.8
          11         vg       1.8
          12          g       1.8
          13         vg       1.8
          14          g       1.8
          15          g       0.8
          16          g       2.8
          20         vg       1.8
          21          m       2.8
          22         vg       1.8
          23          g       2.8
          24          m       1.8
\end{euleroutput}
\begin{eulercomment}
Tentu saja, kita perlu mengekstrak tabel itu sendiri dari daftar Tabel
dalam kasus ini.
\end{eulercomment}
\begin{eulerprompt}
>MT=Table[1];
\end{eulerprompt}
\begin{eulercomment}
Tentu saja, kita juga dapat menggunakannya untuk menentukan nilai
rata-rata kolom atau nilai statistik lainnya.
\end{eulercomment}
\begin{eulerprompt}
>mean(tablecol(MT,6))
\end{eulerprompt}
\begin{euleroutput}
  2.175
\end{euleroutput}
\begin{eulercomment}
Fungsi getstatistics() mengembalikan elemen-elemen dalam sebuah
vektor, dan jumlahnya. Kita menerapkannya pada nilai “m” dan “f” pada
kolom kedua tabel kita.
\end{eulercomment}
\begin{eulerprompt}
>\{xu,count\}=getstatistics(tablecol(MT,2)); xu, count,
\end{eulerprompt}
\begin{euleroutput}
  [1,  3]
  [12,  13]
\end{euleroutput}
\begin{eulercomment}
Kita bisa mencetak hasilnya dalam tabel baru.
\end{eulercomment}
\begin{eulerprompt}
>writetable(count',labr=tok[xu])
\end{eulerprompt}
\begin{euleroutput}
           m        12
           f        13
\end{euleroutput}
\begin{eulercomment}
Fungsi selecttable() mengembalikan sebuah tabel baru dengan nilai
dalam satu kolom yang dipilih dari vektor indeks. Pertama, kita
mencari indeks dari dua nilai kita dalam tabel token.
\end{eulercomment}
\begin{eulerprompt}
>v:=indexof(tok,["g","vg"])
\end{eulerprompt}
\begin{euleroutput}
  [5,  6]
\end{euleroutput}
\begin{eulercomment}
Sekarang kita dapat memilih baris-baris dari tabel, yang memiliki
salah satu nilai dalam v di baris ke-5.
\end{eulercomment}
\begin{eulerprompt}
>MT1:=MT[selectrows(MT,5,v)]; i:=sortedrows(MT1,5);
\end{eulerprompt}
\begin{eulercomment}
Sekarang kita dapat mencetak tabel, dengan nilai yang diekstrak dan
diurutkan di kolom ke-5.
\end{eulercomment}
\begin{eulerprompt}
>writetable(MT1[i],labc=hd,ctok=ctok,tok=tok,wc=7);
\end{eulerprompt}
\begin{euleroutput}
   Person    Sex    Age Titanic Evaluation    Tip Problem
        2      f     23       y          g    1.8       n
        3      f     26       y          g    1.8       y
        6      m     28       y          g    2.8       y
       18      m     38       y          g      .       n
       16      m     26       y          g    2.8       n
       15      f     31       y          g    0.8       n
       12      m     32       y          g    1.8       n
       23      f     38       y          g    2.8       n
       14      f     25       y          g    1.8       y
        9      f     24       y         vg    1.8       y
        7      f     31       y         vg    2.8       n
       20      f     28       y         vg    1.8       n
       22      f     28       y         vg    1.8       y
       13      m     29       y         vg    1.8       y
       11      f     23       y         vg    1.8       y
\end{euleroutput}
\begin{eulercomment}
Untuk statistik berikutnya, kita ingin menghubungkan dua kolom tabel.
Jadi kita mengekstrak kolom 2 dan 4 dan mengurutkan tabel.
\end{eulercomment}
\begin{eulerprompt}
>i=sortedrows(MT,[2,4]);  ...
>  writetable(tablecol(MT[i],[2,4])',ctok=[1,2],tok=tok)
\end{eulerprompt}
\begin{euleroutput}
           m         n
           m         n
           m         n
           m         n
           m         n
           m         n
           m         n
           m         y
           m         y
           m         y
           m         y
           m         y
           f         n
           f         y
           f         y
           f         y
           f         y
           f         y
           f         y
           f         y
           f         y
           f         y
           f         y
           f         y
           f         y
\end{euleroutput}
\begin{eulercomment}
Dengan getstatistics(), kita juga dapat menghubungkan hitungan dalam
dua kolom tabel satu sama lain.
\end{eulercomment}
\begin{eulerprompt}
>MT24=tablecol(MT,[2,4]); ...
>\{xu1,xu2,count\}=getstatistics(MT24[1],MT24[2]); ...
>writetable(count,labr=tok[xu1],labc=tok[xu2])
\end{eulerprompt}
\begin{euleroutput}
                     n         y
           m         7         5
           f         1        12
\end{euleroutput}
\begin{eulercomment}
Tabel dapat ditulis ke sebuah file.
\end{eulercomment}
\begin{eulerprompt}
>filename="test.dat"; ...
>writetable(count,labr=tok[xu1],labc=tok[xu2],file=filename);
\end{eulerprompt}
\begin{eulercomment}
Kemudian kita dapat membaca tabel dari file tersebut.
\end{eulercomment}
\begin{eulerprompt}
>\{MT2,hd,tok2,hdr\}=readtable(filename,>clabs,>rlabs); ...
>writetable(MT2,labr=hdr,labc=hd)
\end{eulerprompt}
\begin{euleroutput}
                     n         y
           m         7         5
           f         1        12
\end{euleroutput}
\begin{eulercomment}
Dan hapus file tersebut.
\end{eulercomment}
\begin{eulerprompt}
>fileremove(filename);
\end{eulerprompt}
\eulerheading{Distribusi}
\begin{eulercomment}
Dengan plot2d, ada metode yang sangat mudah untuk memplot distribusi
data eksperimen.
\end{eulercomment}
\begin{eulerprompt}
>p=normal(1,1000); //1000 random normal-distributed sample p
>plot2d(p,distribution=20,style="\(\backslash\)/"); // plot the random sample p
>plot2d("qnormal(x,0,1)",add=1): // add the standard normal distribution plot
\end{eulerprompt}
\begin{eulercomment}
Perhatikan perbedaan antara plot batang (sampel) dan kurva normal
(distribusi sesungguhnya). Masukkan kembali ketiga perintah tersebut
untuk melihat hasil pengambilan sampel yang lain.
\end{eulercomment}
\begin{eulercomment}
Berikut ini adalah perbandingan 10 simulasi dari 1000 nilai
terdistribusi normal dengan menggunakan apa yang disebut plot kotak.
Plot ini menunjukkan median, kuartil 25\% dan 75\%, nilai minimal dan
maksimal, serta pencilan.
\end{eulercomment}
\begin{eulerprompt}
>p=normal(10,1000); boxplot(p):
\end{eulerprompt}
\begin{eulercomment}
Untuk menghasilkan bilangan bulat acak, Euler memiliki intrandom. Mari
kita simulasikan pelemparan dadu dan memplot distribusinya.

Kita menggunakan fungsi getmultiplicities(v,x), yang menghitung
seberapa sering elemen-elemen dari v muncul di dalam x. Kemudian kita
memplot hasilnya menggunakan columnsplot().
\end{eulercomment}
\begin{eulerprompt}
>k=intrandom(1,6000,6);  ...
>columnsplot(getmultiplicities(1:6,k));  ...
>ygrid(1000,color=red):
\end{eulerprompt}
\begin{eulercomment}
Meskipun intrandom(n,m,k) menghasilkan bilangan bulat yang
terdistribusi secara seragam dari 1 sampai k, adalah mungkin untuk
menggunakan distribusi bilangan bulat yang lain dengan randpint().

Pada contoh berikut, probabilitas untuk 1,2,3 adalah 0.4, 0.1, 0.5
secara berurutan.
\end{eulercomment}
\begin{eulerprompt}
>randpint(1,1000,[0.4,0.1,0.5]); getmultiplicities(1:3,%)
\end{eulerprompt}
\begin{euleroutput}
  [378,  102,  520]
\end{euleroutput}
\begin{eulercomment}
Euler dapat menghasilkan nilai acak dari lebih banyak distribusi.
Lihatlah ke dalam referensi.

Misalnya, kita mencoba distribusi eksponensial. Sebuah variabel acak
kontinu X dikatakan memiliki distribusi eksponensial, jika PDF-nya
diberikan oleh

\end{eulercomment}
\begin{eulerformula}
\[
f_X(x)=\lambda e^{-\lambda x},\quad x>0,\quad \lambda>0,
\]
\end{eulerformula}
\begin{eulercomment}
with parameter\\
\end{eulercomment}
\begin{eulerformula}
\[
\lambda=\frac{1}{\mu},\quad \mu \text{ is the mean, and denoted by } X \sim \text{Exponential}(\lambda).
\]
\end{eulerformula}
\begin{eulerprompt}
>plot2d(randexponential(1,1000,2),>distribution):
\end{eulerprompt}
\begin{eulercomment}
Untuk banyak distribusi, Euler dapat menghitung fungsi distribusi dan
kebalikannya.
\end{eulercomment}
\begin{eulerprompt}
>plot2d("normaldis",-4,4): 
\end{eulerprompt}
\begin{eulercomment}
Berikut ini adalah salah satu cara untuk memplot kuantil.
\end{eulercomment}
\begin{eulerprompt}
>plot2d("qnormal(x,1,1.5)",-4,6);  ...
>plot2d("qnormal(x,1,1.5)",a=2,b=5,>add,>filled):
\end{eulerprompt}
\begin{eulerformula}
\[
\text{normaldis(x,m,d)}=\int_{-\infty}^x \frac{1}{d\sqrt{2\pi}}e^{-\frac{1}{2}(\frac{t-m}{d})^2}\ dt.
\]
\end{eulerformula}
\begin{eulercomment}
Probabilitas untuk berada di area hijau adalah sebagai berikut.
\end{eulercomment}
\begin{eulerprompt}
>normaldis(5,1,1.5)-normaldis(2,1,1.5)
\end{eulerprompt}
\begin{euleroutput}
  0.248662156979
\end{euleroutput}
\begin{eulercomment}
Hal ini dapat dihitung secara numerik dengan integral berikut ini.\\
\end{eulercomment}
\begin{eulerformula}
\[
\int_2^5 \frac{1}{1.5\sqrt{2\pi}}e^{-\frac{1}{2}(\frac{x-1}{1.5})^2}\ dx.
\]
\end{eulerformula}
\begin{eulerprompt}
>gauss("qnormal(x,1,1.5)",2,5)
\end{eulerprompt}
\begin{euleroutput}
  0.248662156979
\end{euleroutput}
\begin{eulercomment}
Mari kita bandingkan distribusi binomial dengan distribusi normal
dengan rata-rata dan deviasi yang sama. Fungsi invbindis()
menyelesaikan interpolasi linier antara nilai bilangan bulat.
\end{eulercomment}
\begin{eulerprompt}
>invbindis(0.95,1000,0.5), invnormaldis(0.95,500,0.5*sqrt(1000))
\end{eulerprompt}
\begin{euleroutput}
  525.516721219
  526.007419394
\end{euleroutput}
\begin{eulercomment}
Fungsi qdis() adalah densitas dari distribusi chi-square. Seperti
biasa, Euler memetakan vektor ke fungsi ini. Dengan demikian kita
mendapatkan plot semua distribusi chi-kuadrat dengan derajat 5 hingga
30 dengan mudah dengan cara berikut.
\end{eulercomment}
\begin{eulerprompt}
>plot2d("qchidis(x,(5:5:50)')",0,50):
\end{eulerprompt}
\begin{eulercomment}
Euler memiliki fungsi-fungsi yang akurat untuk mengevaluasi
distribusi-distribusi. Mari kita periksa chidis() dengan sebuah
integral.

Penamaannya diusahakan untuk konsisten. Sebagai contoh,

- distribusi chi-kuadrat adalah chidis(),\\
- fungsi kebalikannya adalah invchidis(),\\
- densitasnya adalah qchidis().

Pelengkap dari distribusi (ekor atas) adalah chicdis().
\end{eulercomment}
\begin{eulerprompt}
>chidis(1.5,2), integrate("qchidis(x,2)",0,1.5)
\end{eulerprompt}
\begin{euleroutput}
  0.527633447259
  0.527633447259
\end{euleroutput}
\eulerheading{Distribusi Diskrit}
\begin{eulercomment}
Untuk menentukan distribusi diskrit Anda sendiri, Anda dapat
menggunakan metode berikut.

Pertama, kita tetapkan fungsi distribusinya.
\end{eulercomment}
\begin{eulerprompt}
>wd = 0|((1:6)+[-0.01,0.01,0,0,0,0])/6
\end{eulerprompt}
\begin{euleroutput}
  [0,  0.165,  0.335,  0.5,  0.666667,  0.833333,  1]
\end{euleroutput}
\begin{eulercomment}
Artinya, dengan probabilitas wd[i+1]-wd[i] kita menghasilkan nilai
acak i.

Ini hampir merupakan distribusi yang seragam. Mari kita definisikan
sebuah generator bilangan acak untuk ini. Fungsi find(v,x) menemukan
nilai x dalam vektor v. Fungsi ini juga dapat digunakan untuk vektor
x.
\end{eulercomment}
\begin{eulerprompt}
>function wrongdice (n,m) := find(wd,random(n,m))
\end{eulerprompt}
\begin{eulercomment}
Kesalahan ini sangat halus sehingga kita hanya bisa melihatnya setelah
melakukan iterasi yang sangat banyak.
\end{eulercomment}
\begin{eulerprompt}
>columnsplot(getmultiplicities(1:6,wrongdice(1,1000000))):
\end{eulerprompt}
\begin{eulercomment}
Berikut ini adalah fungsi sederhana untuk memeriksa distribusi seragam
dari nilai 1... K dalam v. Kami menerima hasilnya, jika untuk semua
frekuensi

\end{eulercomment}
\begin{eulerformula}
\[
\left|f_i-\frac{1}{K}\right| < \frac{\delta}{\sqrt{n}}.
\]
\end{eulerformula}
\begin{eulerprompt}
>function checkrandom (v, delta=1) ...
\end{eulerprompt}
\begin{eulerudf}
    K=max(v); n=cols(v);
    fr=getfrequencies(v,1:K);
    return max(fr/n-1/K)<delta/sqrt(n);
    endfunction
\end{eulerudf}
\begin{eulercomment}
Memang fungsi ini menolak distribusi seragam.
\end{eulercomment}
\begin{eulerprompt}
>checkrandom(wrongdice(1,1000000))
\end{eulerprompt}
\begin{euleroutput}
  0
\end{euleroutput}
\begin{eulercomment}
Dan ini menerima generator acak bawaan.
\end{eulercomment}
\begin{eulerprompt}
>checkrandom(intrandom(1,1000000,6))
\end{eulerprompt}
\begin{euleroutput}
  1
\end{euleroutput}
\begin{eulercomment}
Kita dapat menghitung distribusi binomial. Pertama, ada binomialsum(),
yang mengembalikan probabilitas i atau kurang dari n percobaan.
\end{eulercomment}
\begin{eulerprompt}
>bindis(410,1000,0.4)
\end{eulerprompt}
\begin{euleroutput}
  0.751401349654
\end{euleroutput}
\begin{eulercomment}
Fungsi Beta invers digunakan untuk menghitung interval kepercayaan
Clopper-Pearson untuk parameter p. Tingkat defaultnya adalah alpha.

Arti dari interval ini adalah jika p berada di luar interval, hasil
yang diamati sebesar 410 dalam 1000 jarang terjadi.
\end{eulercomment}
\begin{eulerprompt}
>clopperpearson(410,1000)
\end{eulerprompt}
\begin{euleroutput}
  [0.37932,  0.441212]
\end{euleroutput}
\begin{eulercomment}
Perintah berikut ini adalah cara langsung untuk mendapatkan hasil di
atas. Tetapi untuk n yang besar, penjumlahan langsung tidak akurat dan
lambat.
\end{eulercomment}
\begin{eulerprompt}
>p=0.4; i=0:410; n=1000; sum(bin(n,i)*p^i*(1-p)^(n-i))
\end{eulerprompt}
\begin{euleroutput}
  0.751401349655
\end{euleroutput}
\begin{eulercomment}
Omong-omong, invbinsum() menghitung kebalikan dari binomialsum().
\end{eulercomment}
\begin{eulerprompt}
>invbindis(0.75,1000,0.4)
\end{eulerprompt}
\begin{euleroutput}
  409.932733047
\end{euleroutput}
\begin{eulercomment}
Dalam Bridge, kita mengasumsikan 5 kartu yang terbuka (dari 52 kartu)
di dua tangan (26 kartu). Mari kita hitung probabilitas distribusi
yang lebih buruk dari 3:2 (misalnya 0:5, 1:4, 4:1, atau 5:0).
\end{eulercomment}
\begin{eulerprompt}
>2*hypergeomsum(1,5,13,26)
\end{eulerprompt}
\begin{euleroutput}
  0.321739130435
\end{euleroutput}
\begin{eulercomment}
Ada juga simulasi distribusi multinomial.
\end{eulercomment}
\begin{eulerprompt}
>randmultinomial(10,1000,[0.4,0.1,0.5])
\end{eulerprompt}
\begin{euleroutput}
            381           100           519 
            376            91           533 
            417            80           503 
            440            94           466 
            406           112           482 
            408            94           498 
            395           107           498 
            399            96           505 
            428            87           485 
            400            99           501 
\end{euleroutput}
\eulerheading{Memplot Data}
\begin{eulercomment}
Untuk memplot data, kami mencoba hasil pemilihan umum Jerman sejak
tahun 1990, yang diukur dalam kursi.
\end{eulercomment}
\begin{eulerprompt}
>BW := [ ...
>1990,662,319,239,79,8,17; ...
>1994,672,294,252,47,49,30; ...
>1998,669,245,298,43,47,36; ...
>2002,603,248,251,47,55,2; ...
>2005,614,226,222,61,51,54; ...
>2009,622,239,146,93,68,76; ...
>2013,631,311,193,0,63,64];
\end{eulerprompt}
\begin{eulercomment}
Untuk pesta, kami menggunakan serangkaian nama.
\end{eulercomment}
\begin{eulerprompt}
>P:=["CDU/CSU","SPD","FDP","Gr","Li"];
\end{eulerprompt}
\begin{eulercomment}
Mari kita cetak persentasenya dengan baik.

Pertama kita ekstrak kolom-kolom yang diperlukan. Kolom 3 sampai 7
adalah kursi masing-masing partai, dan kolom 2 adalah jumlah total
kursi. kolom adalah tahun pemilihan.
\end{eulercomment}
\begin{eulerprompt}
>BT:=BW[,3:7]; BT:=BT/sum(BT); YT:=BW[,1]';
\end{eulerprompt}
\begin{eulercomment}
Kemudian kita mencetak statistik dalam bentuk tabel. Kita menggunakan
nama sebagai judul kolom, dan tahun sebagai judul baris. Lebar default
untuk kolom adalah wc = 10, tetapi kami lebih suka output yang lebih
padat. Kolom-kolom akan diperluas untuk label-label kolom, jika perlu.
\end{eulercomment}
\begin{eulerprompt}
>writetable(BT*100,wc=6,dc=0,>fixed,labc=P,labr=YT)
\end{eulerprompt}
\begin{euleroutput}
         CDU/CSU   SPD   FDP    Gr    Li
    1990      48    36    12     1     3
    1994      44    38     7     7     4
    1998      37    45     6     7     5
    2002      41    42     8     9     0
    2005      37    36    10     8     9
    2009      38    23    15    11    12
    2013      49    31     0    10    10
\end{euleroutput}
\begin{eulercomment}
Perkalian matriks berikut ini mengekstrak jumlah persentase dua partai
besar yang menunjukkan bahwa partai-partai kecil telah memperoleh
suara di parlemen hingga tahun 2009.
\end{eulercomment}
\begin{eulerprompt}
>BT1:=(BT.[1;1;0;0;0])'*100
\end{eulerprompt}
\begin{euleroutput}
  [84.29,  81.25,  81.1659,  82.7529,  72.9642,  61.8971,  79.8732]
\end{euleroutput}
\begin{eulercomment}
Ada juga plot statistik sederhana. Kita menggunakannya untuk
menampilkan garis dan titik secara bersamaan. Alternatif lainnya
adalah memanggil plot2d dua kali dengan \textgreater{}add.
\end{eulercomment}
\begin{eulerprompt}
>statplot(YT,BT1,"b"):
\end{eulerprompt}
\begin{eulercomment}
Tentukan beberapa warna untuk masing-masing pihak.
\end{eulercomment}
\begin{eulerprompt}
>CP:=[rgb(0.5,0.5,0.5),red,yellow,green,rgb(0.8,0,0)];
\end{eulerprompt}
\begin{eulercomment}
Sekarang kita dapat memplot hasil pemilu 2009 dan perubahannya ke
dalam satu plot menggunakan figure. Kita dapat menambahkan vektor
kolom pada setiap plot.
\end{eulercomment}
\begin{eulerprompt}
>figure(2,1);  ...
>figure(1); columnsplot(BW[6,3:7],P,color=CP); ...
>figure(2); columnsplot(BW[6,3:7]-BW[5,3:7],P,color=CP);  ...
>figure(0):
\end{eulerprompt}
\begin{eulercomment}
Plot data menggabungkan baris data statistik dalam satu plot.
\end{eulercomment}
\begin{eulerprompt}
>J:=BW[,1]'; DP:=BW[,3:7]'; ...
>dataplot(YT,BT',color=CP);  ...
>labelbox(P,colors=CP,styles="[]",>points,w=0.2,x=0.3,y=0.4):
\end{eulerprompt}
\begin{eulercomment}
Plot kolom 3D menunjukkan deretan data statistik dalam bentuk kolom.
Kami menyediakan label untuk baris dan kolom. angle adalah sudut
pandang.
\end{eulercomment}
\begin{eulerprompt}
>columnsplot3d(BT,scols=P,srows=YT, ...
>  angle=30°,ccols=CP):
\end{eulerprompt}
\begin{eulercomment}
Representasi lainnya adalah plot mosaik. Perhatikan bahwa kolom-kolom
pada plot mewakili kolom-kolom pada matriks di sini. Karena panjangnya
label CDU/CSU, kita mengambil jendela yang lebih kecil dari biasanya.
\end{eulercomment}
\begin{eulerprompt}
>shrinkwindow(>smaller);  ...
>mosaicplot(BT',srows=YT,scols=P,color=CP,style="#"); ...
>shrinkwindow():
\end{eulerprompt}
\begin{eulercomment}
Kita juga bisa membuat diagram lingkaran. Karena warna hitam dan
kuning membentuk sebuah koalisi, kita menyusun ulang elemen-elemennya.
\end{eulercomment}
\begin{eulerprompt}
>i=[1,3,5,4,2]; piechart(BW[6,3:7][i],color=CP[i],lab=P[i]):
\end{eulerprompt}
\begin{eulercomment}
Berikut ini jenis plot yang lain.
\end{eulercomment}
\begin{eulerprompt}
>starplot(normal(1,10)+4,lab=1:10,>rays):
\end{eulerprompt}
\begin{eulercomment}
Beberapa plot di plot2d bagus untuk statika. Berikut ini adalah plot
impuls dari data acak, yang terdistribusi secara seragam dalam [0,1].
\end{eulercomment}
\begin{eulerprompt}
>plot2d(makeimpulse(1:10,random(1,10)),>bar):
\end{eulerprompt}
\begin{eulercomment}
Tetapi untuk data yang terdistribusi secara eksponensial, kita mungkin
memerlukan plot logaritmik.
\end{eulercomment}
\begin{eulerprompt}
>logimpulseplot(1:10,-log(random(1,10))*10):
\end{eulerprompt}
\begin{eulercomment}
Fungsi columnsplot() lebih mudah digunakan, karena hanya membutuhkan
sebuah vektor nilai. Selain itu, fungsi ini dapat mengatur labelnya
menjadi apa pun yang kita inginkan, kita telah mendemonstrasikan hal
ini dalam tutorial ini.

Berikut ini adalah aplikasi lain, di mana kita menghitung karakter
dalam sebuah kalimat dan memplot statistik.
\end{eulercomment}
\begin{eulerprompt}
>v=strtochar("the quick brown fox jumps over the lazy dog"); ...
>w=ascii("a"):ascii("z"); x=getmultiplicities(w,v); ...
>cw=[]; for k=w; cw=cw|char(k); end; ...
>columnsplot(x,lab=cw,width=0.05):
\end{eulerprompt}
\begin{eulercomment}
Anda juga dapat menetapkan sumbu secara manual.
\end{eulercomment}
\begin{eulerprompt}
>n=10; p=0.4; i=0:n; x=bin(n,i)*p^i*(1-p)^(n-i); ...
>columnsplot(x,lab=i,width=0.05,<frame,<grid); ...
>yaxis(0,0:0.1:1,style="->",>left); xaxis(0,style="."); ...
>label("p",0,0.25), label("i",11,0); ...
>textbox(["Binomial distribution","with p=0.4"]):
\end{eulerprompt}
\begin{eulercomment}
Berikut ini adalah cara untuk memplot frekuensi angka dalam vektor.

Kami membuat vektor angka acak bilangan bulat 1 hingga 6.
\end{eulercomment}
\begin{eulerprompt}
>v:=intrandom(1,10,10)
\end{eulerprompt}
\begin{euleroutput}
  [8,  5,  8,  8,  6,  8,  8,  3,  5,  5]
\end{euleroutput}
\begin{eulercomment}
Kemudian ekstrak nomor unik dalam v.
\end{eulercomment}
\begin{eulerprompt}
>vu:=unique(v)
\end{eulerprompt}
\begin{euleroutput}
  [3,  5,  6,  8]
\end{euleroutput}
\begin{eulercomment}
Dan memplot frekuensi dalam plot kolom.
\end{eulercomment}
\begin{eulerprompt}
>columnsplot(getmultiplicities(vu,v),lab=vu,style="/"):
\end{eulerprompt}
\begin{eulercomment}
Kami ingin mendemonstrasikan fungsi untuk distribusi nilai empiris.
\end{eulercomment}
\begin{eulerprompt}
>x=normal(1,20);
\end{eulerprompt}
\begin{eulercomment}
Fungsi empdist(x,vs) membutuhkan larik nilai yang telah diurutkan.
Jadi kita harus mengurutkan x sebelum dapat menggunakannya.
\end{eulercomment}
\begin{eulerprompt}
>xs=sort(x);
\end{eulerprompt}
\begin{eulercomment}
Kemudian kita memplot distribusi empiris dan beberapa batang kepadatan
ke dalam satu plot. Alih-alih plot batang untuk distribusi, kali ini
kami menggunakan plot gigi gergaji.
\end{eulercomment}
\begin{eulerprompt}
>figure(2,1); ...
>figure(1); plot2d("empdist",-4,4;xs); ...
>figure(2); plot2d(histo(x,v=-4:0.2:4,<bar));  ...
>figure(0):
\end{eulerprompt}
\begin{eulercomment}
Plot sebaran mudah dilakukan di Euler dengan plot titik biasa. Grafik
berikut ini menunjukkan bahwa X dan X+Y berkorelasi positif secara
jelas.
\end{eulercomment}
\begin{eulerprompt}
>x=normal(1,100); plot2d(x,x+rotright(x),>points,style=".."):
\end{eulerprompt}
\begin{eulercomment}
Sering kali, kita ingin membandingkan dua sampel dari distribusi yang
berbeda. Hal ini dapat dilakukan dengan plot kuantil-kuantil.

Untuk pengujian, kami mencoba distribusi student-t dan distribusi
eksponensial.
\end{eulercomment}
\begin{eulerprompt}
>x=randt(1,1000,5); y=randnormal(1,1000,mean(x),dev(x)); ...
>plot2d("x",r=6,style="--",yl="normal",xl="student-t",>vertical); ...
>plot2d(sort(x),sort(y),>points,color=red,style="x",>add):
\end{eulerprompt}
\begin{eulercomment}
Plot ini dengan jelas menunjukkan bahwa nilai yang terdistribusi
normal cenderung lebih kecil pada ujung yang ekstrim.

Jika kita memiliki dua distribusi dengan ukuran yang berbeda, kita
dapat memperluas distribusi yang lebih kecil atau memperkecil
distribusi yang lebih besar. Fungsi berikut ini bagus untuk keduanya.
Fungsi ini mengambil nilai median dengan persentase antara 0 dan 1.
\end{eulercomment}
\begin{eulerprompt}
>function medianexpand (x,n) := median(x,p=linspace(0,1,n-1));
\end{eulerprompt}
\begin{eulercomment}
Mari kita bandingkan dua distribusi yang sama.
\end{eulercomment}
\begin{eulerprompt}
>x=random(1000); y=random(400); ...
>plot2d("x",0,1,style="--"); ...
>plot2d(sort(medianexpand(x,400)),sort(y),>points,color=red,style="x",>add):
\end{eulerprompt}
\eulerheading{Regresi dan Korelasi}
\begin{eulercomment}
Regresi linier dapat dilakukan dengan fungsi polyfit() atau berbagai
fungsi kecocokan.

Sebagai permulaan, kita mencari garis regresi untuk data univariat
dengan polyfit(x,y,1).
\end{eulercomment}
\begin{eulerprompt}
>x=1:10; y=[2,3,1,5,6,3,7,8,9,8]; writetable(x'|y',labc=["x","y"])
\end{eulerprompt}
\begin{euleroutput}
           x         y
           1         2
           2         3
           3         1
           4         5
           5         6
           6         3
           7         7
           8         8
           9         9
          10         8
\end{euleroutput}
\begin{eulercomment}
Kami ingin membandingkan kecocokan tanpa bobot dan dengan bobot.
Pertama, koefisien dari kecocokan linier.
\end{eulercomment}
\begin{eulerprompt}
>p=polyfit(x,y,1)
\end{eulerprompt}
\begin{euleroutput}
  [0.733333,  0.812121]
\end{euleroutput}
\begin{eulercomment}
Sekarang, koefisien dengan bobot yang menekankan nilai terakhir.
\end{eulercomment}
\begin{eulerprompt}
>w &= "exp(-(x-10)^2/10)"; pw=polyfit(x,y,1,w=w(x))
\end{eulerprompt}
\begin{euleroutput}
  [4.71566,  0.38319]
\end{euleroutput}
\begin{eulercomment}
Kami menempatkan semuanya ke dalam satu plot untuk titik-titik dan
garis regresi, dan untuk bobot yang digunakan.
\end{eulercomment}
\begin{eulerprompt}
>figure(2,1);  ...
>figure(1); statplot(x,y,"b",xl="Regression"); ...
>  plot2d("evalpoly(x,p)",>add,color=blue,style="--"); ...
>  plot2d("evalpoly(x,pw)",5,10,>add,color=red,style="--"); ...
>figure(2); plot2d(w,1,10,>filled,style="/",fillcolor=red,xl=w); ...
>figure(0):
\end{eulerprompt}
\begin{eulercomment}
Untuk contoh lain, kita membaca survei tentang siswa, usia mereka,
usia orang tua mereka, dan jumlah saudara kandung dari sebuah file.

Tabel ini berisi “m” dan “f” pada kolom kedua. Kita menggunakan
variabel tok2 untuk mengatur terjemahan yang tepat dan bukannya
membiarkan readtable() mengumpulkan terjemahan.
\end{eulercomment}
\begin{eulerprompt}
>\{MS,hd\}:=readtable("table1.dat",tok2:=["m","f"]);  ...
>writetable(MS,labc=hd,tok2:=["m","f"]);
\end{eulerprompt}
\begin{euleroutput}
      Person       Sex       Age    Mother    Father  Siblings
           1         m        29        58        61         1
           2         f        26        53        54         2
           3         m        24        49        55         1
           4         f        25        56        63         3
           5         f        25        49        53         0
           6         f        23        55        55         2
           7         m        23        48        54         2
           8         m        27        56        58         1
           9         m        25        57        59         1
          10         m        24        50        54         1
          11         f        26        61        65         1
          12         m        24        50        52         1
          13         m        29        54        56         1
          14         m        28        48        51         2
          15         f        23        52        52         1
          16         m        24        45        57         1
          17         f        24        59        63         0
          18         f        23        52        55         1
          19         m        24        54        61         2
          20         f        23        54        55         1
\end{euleroutput}
\begin{eulercomment}
Bagaimana usia saling bergantung satu sama lain? Kesan pertama datang
dari scatterplot berpasangan.
\end{eulercomment}
\begin{eulerprompt}
>scatterplots(tablecol(MS,3:5),hd[3:5]):
\end{eulerprompt}
\begin{eulercomment}
Jelas bahwa usia ayah dan ibu saling bergantung satu sama lain. Mari
kita tentukan dan plot garis regresinya.
\end{eulercomment}
\begin{eulerprompt}
>cs:=MS[,4:5]'; ps:=polyfit(cs[1],cs[2],1)
\end{eulerprompt}
\begin{euleroutput}
  [17.3789,  0.740964]
\end{euleroutput}
\begin{eulercomment}
Ini jelas merupakan model yang salah. Garis regresinya adalah s = 17 +
0,74t, di mana t adalah usia ibu dan s adalah usia ayah. Perbedaan
usia mungkin sedikit bergantung pada usia, tetapi tidak terlalu
banyak.

Sebaliknya, kita menduga fungsi seperti s = a + t. Kemudian a adalah
rata-rata dari s-t. Ini adalah perbedaan usia rata-rata antara ayah
dan ibu.
\end{eulercomment}
\begin{eulerprompt}
>da:=mean(cs[2]-cs[1])
\end{eulerprompt}
\begin{euleroutput}
  3.65
\end{euleroutput}
\begin{eulercomment}
Mari kita plotkan ini ke dalam satu scatter plot.
\end{eulercomment}
\begin{eulerprompt}
>plot2d(cs[1],cs[2],>points);  ...
>plot2d("evalpoly(x,ps)",color=red,style=".",>add);  ...
>plot2d("x+da",color=blue,>add):
\end{eulerprompt}
\begin{eulercomment}
Berikut ini adalah plot kotak dari kedua usia tersebut. Ini hanya
menunjukkan, bahwa usia keduanya berbeda.
\end{eulercomment}
\begin{eulerprompt}
>boxplot(cs,["mothers","fathers"]):
\end{eulerprompt}
\begin{eulercomment}
Sangat menarik bahwa perbedaan dalam median tidak sebesar perbedaan
dalam mean.
\end{eulercomment}
\begin{eulerprompt}
>median(cs[2])-median(cs[1])
\end{eulerprompt}
\begin{euleroutput}
  1.5
\end{euleroutput}
\begin{eulercomment}
Koefisien korelasi menunjukkan korelasi positif.
\end{eulercomment}
\begin{eulerprompt}
>correl(cs[1],cs[2])
\end{eulerprompt}
\begin{euleroutput}
  0.7588307236
\end{euleroutput}
\begin{eulercomment}
Korelasi peringkat adalah ukuran untuk urutan yang sama dalam kedua
vektor. Korelasi ini juga cukup positif.
\end{eulercomment}
\begin{eulerprompt}
>rankcorrel(cs[1],cs[2])
\end{eulerprompt}
\begin{euleroutput}
  0.758925292358
\end{euleroutput}
\eulerheading{Membuat Fungsi baru}
\begin{eulercomment}
Tentu saja, bahasa EMT dapat digunakan untuk memprogram fungsi baru.
Misalnya, kita mendefinisikan fungsi kemiringan.

\end{eulercomment}
\begin{eulerformula}
\[
\text{sk}(x) = \dfrac{\sqrt{n} \sum_i (x_i-m)^3}{\left(\sum_i (x_i-m)^2\right)^{3/2}}
\]
\end{eulerformula}
\begin{eulercomment}
di mana m adalah rata-rata dari x.
\end{eulercomment}
\begin{eulerprompt}
>function skew (x:vector) ...
\end{eulerprompt}
\begin{eulerudf}
  m=mean(x);
  return sqrt(cols(x))*sum((x-m)^3)/(sum((x-m)^2))^(3/2);
  endfunction
\end{eulerudf}
\begin{eulercomment}
Seperti yang Anda lihat, kita dapat dengan mudah menggunakan bahasa
matriks untuk mendapatkan implementasi yang sangat singkat dan
efisien. Mari kita coba fungsi ini.
\end{eulercomment}
\begin{eulerprompt}
>data=normal(20); skew(normal(10))
\end{eulerprompt}
\begin{euleroutput}
  -0.198710316203
\end{euleroutput}
\begin{eulercomment}
Berikut ini adalah fungsi lain, yang disebut koefisien kemencengan
Pearson.
\end{eulercomment}
\begin{eulerprompt}
>function skew1 (x) := 3*(mean(x)-median(x))/dev(x)
>skew1(data)
\end{eulerprompt}
\begin{euleroutput}
  -0.0801873249135
\end{euleroutput}
\eulerheading{Simulasi Monte Carlo}
\begin{eulercomment}
Euler dapat digunakan untuk mensimulasikan kejadian acak. Kita telah
melihat contoh sederhana di atas. Berikut ini adalah contoh lainnya,
yang mensimulasikan 1000 kali pelemparan 3 dadu, dan menanyakan
distribusi dari jumlah tersebut.
\end{eulercomment}
\begin{eulerprompt}
>ds:=sum(intrandom(1000,3,6))';  fs=getmultiplicities(3:18,ds)
\end{eulerprompt}
\begin{euleroutput}
  [5,  17,  35,  44,  75,  97,  114,  116,  143,  116,  104,  53,  40,
  22,  13,  6]
\end{euleroutput}
\begin{eulercomment}
Kita bisa merencanakan ini sekarang.
\end{eulercomment}
\begin{eulerprompt}
>columnsplot(fs,lab=3:18):
\end{eulerprompt}
\begin{eulercomment}
Untuk menentukan distribusi yang diharapkan tidaklah mudah. Kami
menggunakan rekursi tingkat lanjut untuk hal ini.

Fungsi berikut ini menghitung jumlah cara angka k dapat
direpresentasikan sebagai jumlah n angka dalam rentang 1 hingga m.
Fungsi ini bekerja secara rekursif dengan cara yang jelas.
\end{eulercomment}
\begin{eulerprompt}
>function map countways (k; n, m) ...
\end{eulerprompt}
\begin{eulerudf}
    if n==1 then return k>=1 && k<=m
    else
      sum=0; 
      loop 1 to m; sum=sum+countways(k-#,n-1,m); end;
      return sum;
    end;
  endfunction
\end{eulerudf}
\begin{eulercomment}
Berikut ini adalah hasil dari tiga lemparan dadu.
\end{eulercomment}
\begin{eulerprompt}
>countways(5:25,5,5)
\end{eulerprompt}
\begin{euleroutput}
  [1,  5,  15,  35,  70,  121,  185,  255,  320,  365,  381,  365,  320,
  255,  185,  121,  70,  35,  15,  5,  1]
\end{euleroutput}
\begin{eulerprompt}
>cw=countways(3:18,3,6)
\end{eulerprompt}
\begin{euleroutput}
  [1,  3,  6,  10,  15,  21,  25,  27,  27,  25,  21,  15,  10,  6,  3,
  1]
\end{euleroutput}
\begin{eulercomment}
Kami menambahkan nilai yang diharapkan ke plot.
\end{eulercomment}
\begin{eulerprompt}
>plot2d(cw/6^3*1000,>add); plot2d(cw/6^3*1000,>points,>add):
\end{eulerprompt}
\begin{eulercomment}
Untuk simulasi lainnya, deviasi nilai rata-rata dari n variabel acak
berdistribusi normal 0-1 adalah 1/sqrt(n).
\end{eulercomment}
\begin{eulerprompt}
>longformat; 1/sqrt(10)
\end{eulerprompt}
\begin{euleroutput}
  0.316227766017
\end{euleroutput}
\begin{eulercomment}
Mari kita periksa hal ini dengan sebuah simulasi. Kami menghasilkan
10.000 kali 10 vektor acak.
\end{eulercomment}
\begin{eulerprompt}
>M=normal(10000,10); dev(mean(M)')
\end{eulerprompt}
\begin{euleroutput}
  0.319493614817
\end{euleroutput}
\begin{eulerprompt}
>plot2d(mean(M)',>distribution):
\end{eulerprompt}
\begin{eulercomment}
Median dari 10 bilangan acak berdistribusi normal 0-1 memiliki deviasi
yang lebih besar.
\end{eulercomment}
\begin{eulerprompt}
>dev(median(M)')
\end{eulerprompt}
\begin{euleroutput}
  0.374460271535
\end{euleroutput}
\begin{eulercomment}
Karena kita dapat dengan mudah menghasilkan jalan acak, kita dapat
mensimulasikan proses Wiener. Kami mengambil 1000 langkah dari 1000
proses. Kami kemudian memplot deviasi standar dan rata-rata dari
langkah ke-n dari proses-proses ini bersama dengan nilai yang
diharapkan dalam warna merah.
\end{eulercomment}
\begin{eulerprompt}
>n=1000; m=1000; M=cumsum(normal(n,m)/sqrt(m)); ...
>t=(1:n)/n; figure(2,1); ...
>figure(1); plot2d(t,mean(M')'); plot2d(t,0,color=red,>add); ...
>figure(2); plot2d(t,dev(M')'); plot2d(t,sqrt(t),color=red,>add); ...
>figure(0):
\end{eulerprompt}
\eulerheading{Tes}
\begin{eulercomment}
Tes adalah alat yang penting dalam statistik. Dalam Euler, banyak tes
yang diterapkan. Semua tes ini mengembalikan kesalahan yang kita
terima jika kita menolak hipotesis nol.

Sebagai contoh, kita menguji lemparan dadu untuk distribusi yang
seragam. Pada 600 lemparan, kita mendapatkan nilai berikut, yang kita
masukkan ke dalam uji chi-kuadrat.
\end{eulercomment}
\begin{eulerprompt}
>chitest([90,103,114,101,103,89],dup(100,6)')
\end{eulerprompt}
\begin{euleroutput}
  0.498830517952
\end{euleroutput}
\begin{eulercomment}
Uji chi-square juga memiliki mode, yang menggunakan simulasi Monte
Carlo untuk menguji statistik. Hasilnya seharusnya hampir sama.
Parameter \textgreater{}p menginterpretasikan vektor y sebagai vektor probabilitas.
\end{eulercomment}
\begin{eulerprompt}
>chitest([90,103,114,101,103,89],dup(1/6,6)',>p,>montecarlo)
\end{eulerprompt}
\begin{euleroutput}
  0.526
\end{euleroutput}
\begin{eulercomment}
Kesalahan ini terlalu besar. Jadi kita tidak bisa menolak distribusi
seragam. Ini tidak membuktikan bahwa dadu kita adil. Tetapi kita tidak
dapat menolak hipotesis kita.

Selanjutnya kita buat 1000 lemparan dadu dengan menggunakan generator
bilangan acak, dan lakukan pengujian yang sama.
\end{eulercomment}
\begin{eulerprompt}
>n=1000; t=random([1,n*6]); chitest(count(t*6,6),dup(n,6)')
\end{eulerprompt}
\begin{euleroutput}
  0.528028118442
\end{euleroutput}
\begin{eulercomment}
Mari kita uji nilai rata-rata 100 dengan uji-t.
\end{eulercomment}
\begin{eulerprompt}
>s=200+normal([1,100])*10; ...
>ttest(mean(s),dev(s),100,200)
\end{eulerprompt}
\begin{euleroutput}
  0.0218365848476
\end{euleroutput}
\begin{eulercomment}
The function ttest() needs the mean value, the deviation, the number
of data, and the mean value to test for.

Now let us check two measurements for the same mean. We reject the
hypothesis that they have the same mean, if the result is \textless{}0.05.
\end{eulercomment}
\begin{eulerprompt}
>tcomparedata(normal(1,10),normal(1,10))
\end{eulerprompt}
\begin{euleroutput}
  0.38722000942
\end{euleroutput}
\begin{eulercomment}
Jika kita menambahkan bias pada satu distribusi, kita akan mendapatkan
lebih banyak penolakan. Ulangi simulasi ini beberapa kali untuk
melihat efeknya.
\end{eulercomment}
\begin{eulerprompt}
>tcomparedata(normal(1,10),normal(1,10)+2)
\end{eulerprompt}
\begin{euleroutput}
  5.60009101758e-07
\end{euleroutput}
\begin{eulercomment}
Pada contoh berikut, kita membuat 20 lemparan dadu secara acak
sebanyak 100 kali dan menghitung jumlah dadu yang muncul. Rata-rata
harus ada 20/6 = 3,3 mata dadu.
\end{eulercomment}
\begin{eulerprompt}
>R=random(100,20); R=sum(R*6<=1)'; mean(R)
\end{eulerprompt}
\begin{euleroutput}
  3.28
\end{euleroutput}
\begin{eulercomment}
Sekarang kita bandingkan jumlah satu dengan distribusi binomial.
Pertama, kita memplot distribusi angka satu.
\end{eulercomment}
\begin{eulerprompt}
>plot2d(R,distribution=max(R)+1,even=1,style="\(\backslash\)/"):
>t=count(R,21);
\end{eulerprompt}
\begin{eulercomment}
Kemudian kami menghitung nilai yang diharapkan.
\end{eulercomment}
\begin{eulerprompt}
>n=0:20; b=bin(20,n)*(1/6)^n*(5/6)^(20-n)*100;
\end{eulerprompt}
\begin{eulercomment}
Kami harus mengumpulkan beberapa angka untuk mendapatkan kategori yang
cukup besar.
\end{eulercomment}
\begin{eulerprompt}
>t1=sum(t[1:2])|t[3:7]|sum(t[8:21]); ...
>b1=sum(b[1:2])|b[3:7]|sum(b[8:21]);
\end{eulerprompt}
\begin{eulercomment}
Uji chi-square menolak hipotesis bahwa distribusi kita adalah
distribusi binomial, jika hasilnya \textless{}0,05.
\end{eulercomment}
\begin{eulerprompt}
>chitest(t1,b1)
\end{eulerprompt}
\begin{euleroutput}
  0.53921579764
\end{euleroutput}
\begin{eulercomment}
Contoh berikut ini berisi hasil dari dua kelompok orang (laki-laki dan
perempuan, katakanlah) yang memberikan suara untuk satu dari enam
partai.
\end{eulercomment}
\begin{eulerprompt}
>A=[23,37,43,52,64,74;27,39,41,49,63,76];  ...
>  writetable(A,wc=6,labr=["m","f"],labc=1:6)
\end{eulerprompt}
\begin{euleroutput}
             1     2     3     4     5     6
       m    23    37    43    52    64    74
       f    27    39    41    49    63    76
\end{euleroutput}
\begin{eulercomment}
Kami ingin menguji independensi suara dari jenis kelamin. Uji tabel
chi\textasciicircum{}2 melakukan hal ini. Hasilnya terlalu besar untuk menolak
independensi. Jadi kita tidak dapat mengatakan, jika pemungutan suara
tergantung pada jenis kelamin dari data ini.
\end{eulercomment}
\begin{eulerprompt}
>tabletest(A)
\end{eulerprompt}
\begin{euleroutput}
  0.990701632326
\end{euleroutput}
\begin{eulercomment}
Berikut ini adalah tabel yang diharapkan, jika kita mengasumsikan
frekuensi pemungutan suara yang diamati.
\end{eulercomment}
\begin{eulerprompt}
>writetable(expectedtable(A),wc=6,dc=1,labr=["m","f"],labc=1:6)
\end{eulerprompt}
\begin{euleroutput}
             1     2     3     4     5     6
       m  24.9  37.9  41.9  50.3  63.3  74.7
       f  25.1  38.1  42.1  50.7  63.7  75.3
\end{euleroutput}
\begin{eulercomment}
Kita dapat menghitung koefisien kontingensi terkoreksi. Karena
koefisien ini sangat dekat dengan 0, kami menyimpulkan bahwa
pemungutan suara tidak bergantung pada jenis kelamin.
\end{eulercomment}
\begin{eulerprompt}
>contingency(A)
\end{eulerprompt}
\begin{euleroutput}
  0.0427225484717
\end{euleroutput}
\begin{eulercomment}
\begin{eulercomment}
\eulerheading{Beberapa Tes Lainnya}
\begin{eulercomment}
Selanjutnya kita menggunakan analisis varians (uji F) untuk menguji
tiga sampel data yang terdistribusi secara normal dengan nilai
rata-rata yang sama. Metode ini disebut ANOVA (analisis varians).
Dalam Euler, fungsi varanalysis() digunakan.
\end{eulercomment}
\begin{eulerprompt}
>x1=[109,111,98,119,91,118,109,99,115,109,94]; mean(x1),
\end{eulerprompt}
\begin{euleroutput}
  106.545454545
\end{euleroutput}
\begin{eulerprompt}
>x2=[120,124,115,139,114,110,113,120,117]; mean(x2),
\end{eulerprompt}
\begin{euleroutput}
  119.111111111
\end{euleroutput}
\begin{eulerprompt}
>x3=[120,112,115,110,105,134,105,130,121,111]; mean(x3)
\end{eulerprompt}
\begin{euleroutput}
  116.3
\end{euleroutput}
\begin{eulerprompt}
>varanalysis(x1,x2,x3)
\end{eulerprompt}
\begin{euleroutput}
  0.0138048221371
\end{euleroutput}
\begin{eulercomment}
Ini berarti, kami menolak hipotesis nilai rata-rata yang sama. Kami
melakukan ini dengan probabilitas kesalahan sebesar 1,3\%.

Ada juga uji median, yang menolak sampel data dengan distribusi
rata-rata yang berbeda dengan menguji median dari sampel gabungan.
\end{eulercomment}
\begin{eulerprompt}
>a=[56,66,68,49,61,53,45,58,54];
>b=[72,81,51,73,69,78,59,67,65,71,68,71];
>mediantest(a,b)
\end{eulerprompt}
\begin{euleroutput}
  0.0241724220052
\end{euleroutput}
\begin{eulercomment}
Uji lain tentang kesetaraan adalah uji peringkat. Uji ini jauh lebih
tajam daripada uji median.
\end{eulercomment}
\begin{eulerprompt}
>ranktest(a,b)
\end{eulerprompt}
\begin{euleroutput}
  0.00199969612469
\end{euleroutput}
\begin{eulercomment}
Dalam contoh berikut ini, kedua distribusi memiliki rata-rata yang
sama.
\end{eulercomment}
\begin{eulerprompt}
>ranktest(random(1,100),random(1,50)*3-1)
\end{eulerprompt}
\begin{euleroutput}
  0.129608141484
\end{euleroutput}
\begin{eulercomment}
Sekarang mari kita coba mensimulasikan dua perawatan a dan b yang
diterapkan pada orang yang berbeda.
\end{eulercomment}
\begin{eulerprompt}
>a=[8.0,7.4,5.9,9.4,8.6,8.2,7.6,8.1,6.2,8.9];
>b=[6.8,7.1,6.8,8.3,7.9,7.2,7.4,6.8,6.8,8.1];
\end{eulerprompt}
\begin{eulercomment}
Uji signum memutuskan, apakah a lebih baik daripada b.
\end{eulercomment}
\begin{eulerprompt}
>signtest(a,b)
\end{eulerprompt}
\begin{euleroutput}
  0.0546875
\end{euleroutput}
\begin{eulercomment}
Ini adalah kesalahan yang terlalu besar. Kita tidak dapat menolak
bahwa a sama baiknya dengan b.

Uji Wilcoxon lebih tajam daripada uji ini, tetapi bergantung pada
nilai kuantitatif dari perbedaan.
\end{eulercomment}
\begin{eulerprompt}
>wilcoxon(a,b)
\end{eulerprompt}
\begin{euleroutput}
  0.0296680599405
\end{euleroutput}
\begin{eulercomment}
Mari kita coba dua pengujian lagi dengan menggunakan rangkaian yang
dihasilkan.
\end{eulercomment}
\begin{eulerprompt}
>wilcoxon(normal(1,20),normal(1,20)-1)
\end{eulerprompt}
\begin{euleroutput}
  0.0068706451766
\end{euleroutput}
\begin{eulerprompt}
>wilcoxon(normal(1,20),normal(1,20))
\end{eulerprompt}
\begin{euleroutput}
  0.275145971064
\end{euleroutput}
\eulerheading{Bilangan Acak}
\begin{eulercomment}
Berikut ini adalah tes untuk generator bilangan acak. Euler
menggunakan generator yang sangat bagus, jadi kita tidak perlu
mengharapkan adanya masalah.

Pertama, kita akan membangkitkan sepuluh juta bilangan acak dalam
[0,1].
\end{eulercomment}
\begin{eulerprompt}
>n:=10000000; r:=random(1,n);
\end{eulerprompt}
\begin{eulercomment}
Selanjutnya, kami menghitung jarak antara dua angka yang kurang dari
0,05.
\end{eulercomment}
\begin{eulerprompt}
>a:=0.05; d:=differences(nonzeros(r<a));
\end{eulerprompt}
\begin{eulercomment}
Terakhir, kami memplot berapa kali, setiap jarak yang terjadi, dan
membandingkannya dengan nilai yang diharapkan.
\end{eulercomment}
\begin{eulerprompt}
>m=getmultiplicities(1:100,d); plot2d(m); ...
>  plot2d("n*(1-a)^(x-1)*a^2",color=red,>add):
\end{eulerprompt}
\begin{eulercomment}
Menghapus data.
\end{eulercomment}
\begin{eulerprompt}
>remvalue n;
\end{eulerprompt}
\begin{eulercomment}
\begin{eulercomment}
\eulerheading{Pengantar untuk Pengguna Proyek R}
\begin{eulercomment}
Jelas, EMT tidak bersaing dengan R sebagai sebuah paket statistik.
Namun, ada banyak prosedur dan fungsi statistik yang tersedia di EMT
juga. Jadi EMT dapat memenuhi kebutuhan dasar. Bagaimanapun, EMT hadir
dengan paket numerik dan sistem aljabar komputer.

Buku ini diperuntukkan bagi Anda yang sudah terbiasa dengan R, tetapi
perlu mengetahui perbedaan sintaks EMT dan R. Kami mencoba memberikan
gambaran umum mengenai hal-hal yang jelas dan kurang jelas yang perlu
Anda ketahui.

Selain itu, kami juga membahas cara-cara untuk bertukar data di antara
kedua sistem tersebut.
\end{eulercomment}
\begin{eulercomment}
Perhatikan bahwa ini adalah pekerjaan yang sedang berlangsung.
\end{eulercomment}
\eulerheading{Sintaks Dasar}
\begin{eulercomment}
Hal pertama yang Anda pelajari dalam R adalah membuat sebuah vektor.
Dalam EMT, perbedaan utamanya adalah bahwa operator : dapat mengambil
ukuran langkah. Selain itu, operator ini memiliki daya ikat yang
rendah.
\end{eulercomment}
\begin{eulerprompt}
>n=10; 0:n/20:n-1
\end{eulerprompt}
\begin{euleroutput}
  [0,  0.5,  1,  1.5,  2,  2.5,  3,  3.5,  4,  4.5,  5,  5.5,  6,  6.5,
  7,  7.5,  8,  8.5,  9]
\end{euleroutput}
\begin{eulercomment}
Fungsi c() tidak ada. Anda dapat menggunakan vektor untuk
menggabungkan beberapa hal.

Contoh berikut ini, seperti banyak contoh lainnya, berasal dari
“Interoduksi ke R” yang disertakan dengan proyek R. Jika Anda membaca
PDF ini, Anda akan menemukan bahwa saya mengikuti alurnya dalam
tutorial ini.
\end{eulercomment}
\begin{eulerprompt}
>x=[10.4, 5.6, 3.1, 6.4, 21.7]; [x,0,x]
\end{eulerprompt}
\begin{euleroutput}
  [10.4,  5.6,  3.1,  6.4,  21.7,  0,  10.4,  5.6,  3.1,  6.4,  21.7]
\end{euleroutput}
\begin{eulercomment}
Operator titik dua dengan ukuran langkah EMT digantikan oleh fungsi
seq() dalam R. Kita dapat menulis fungsi ini dalam EMT.
\end{eulercomment}
\begin{eulerprompt}
>function seq(a,b,c) := a:b:c; ...
>seq(0,-0.1,-1)
\end{eulerprompt}
\begin{euleroutput}
  [0,  -0.1,  -0.2,  -0.3,  -0.4,  -0.5,  -0.6,  -0.7,  -0.8,  -0.9,  -1]
\end{euleroutput}
\begin{eulercomment}
Fungsi rep() dari R tidak ada dalam EMT. Untuk input vektor, dapat
dituliskan sebagai berikut.
\end{eulercomment}
\begin{eulerprompt}
>function rep(x:vector,n:index) := flatten(dup(x,n)); ...
>rep(x,2)
\end{eulerprompt}
\begin{euleroutput}
  [10.4,  5.6,  3.1,  6.4,  21.7,  10.4,  5.6,  3.1,  6.4,  21.7]
\end{euleroutput}
\begin{eulercomment}
Perhatikan bahwa “=” atau “:=” digunakan untuk penugasan. Operator
“-\textgreater{}” digunakan untuk unit dalam EMT.
\end{eulercomment}
\begin{eulerprompt}
>125km -> " miles"
\end{eulerprompt}
\begin{euleroutput}
  77.6713990297 miles
\end{euleroutput}
\begin{eulercomment}
Operator “\textless{}-” untuk penugasan menyesatkan, dan bukan ide yang baik
untuk R. Berikut ini akan membandingkan a dan -4 dalam EMT.
\end{eulercomment}
\begin{eulerprompt}
>a=2; a<-4
\end{eulerprompt}
\begin{euleroutput}
  0
\end{euleroutput}
\begin{eulercomment}
Dalam R, “a\textless{}-4\textless{}3” bisa digunakan, tetapi “a\textless{}-4\textless{}-3” tidak. Saya juga
mengalami ambiguitas yang sama di EMT, tetapi saya mencoba untuk
menghilangkannya.

EMT dan R memiliki vektor dengan tipe boolean. Tetapi dalam EMT, angka
0 dan 1 digunakan untuk merepresentasikan salah dan benar. Dalam R,
nilai benar dan salah tetap dapat digunakan dalam aritmatika biasa
seperti dalam EMT.
\end{eulercomment}
\begin{eulerprompt}
>x<5, %*x
\end{eulerprompt}
\begin{euleroutput}
  [0,  0,  1,  0,  0]
  [0,  0,  3.1,  0,  0]
\end{euleroutput}
\begin{eulercomment}
EMT melempar kesalahan atau menghasilkan NAN tergantung pada flag
“kesalahan”.
\end{eulercomment}
\begin{eulerprompt}
>errors off; 0/0, isNAN(sqrt(-1)), errors on;
\end{eulerprompt}
\begin{euleroutput}
  NAN
  1
\end{euleroutput}
\begin{eulercomment}
String sama saja dalam R dan EMT. Keduanya berada di lokal saat ini,
bukan di Unicode.

Dalam R ada paket-paket untuk Unicode. Dalam EMT, sebuah string dapat
berupa string Unicode. Sebuah string Unicode dapat diterjemahkan ke
pengkodean lokal dan sebaliknya. Selain itu, u“...” dapat berisi
entitas HTML.
\end{eulercomment}
\begin{eulerprompt}
>u"&#169; Ren&eacut; Grothmann"
\end{eulerprompt}
\begin{euleroutput}
  © René Grothmann
\end{euleroutput}
\begin{eulercomment}
Berikut ini mungkin atau mungkin tidak ditampilkan dengan benar pada
sistem Anda sebagai A dengan titik dan tanda hubung di atasnya. Hal
ini tergantung pada jenis huruf yang Anda gunakan.
\end{eulercomment}
\begin{eulerprompt}
>chartoutf([480])
\end{eulerprompt}
\begin{euleroutput}
  Ǡ
\end{euleroutput}
\begin{eulercomment}
Penggabungan string dilakukan dengan “+” atau “\textbar{}”. Ini dapat
menyertakan angka, yang akan dicetak dalam format saat ini.
\end{eulercomment}
\begin{eulerprompt}
>"pi = "+pi
\end{eulerprompt}
\begin{euleroutput}
  pi = 3.14159265359
\end{euleroutput}
\eulerheading{Pengindeksan}
\begin{eulercomment}
Sebagian besar waktu, ini akan bekerja seperti pada R.

Tetapi EMT akan menginterpretasikan indeks negatif dari bagian
belakang vektor, sementara R menginterpretasikan x[n] sebagai x tanpa
elemen ke-n.
\end{eulercomment}
\begin{eulerprompt}
>x, x[1:3], x[-2]
\end{eulerprompt}
\begin{euleroutput}
  [10.4,  5.6,  3.1,  6.4,  21.7]
  [10.4,  5.6,  3.1]
  6.4
\end{euleroutput}
\begin{eulercomment}
Perilaku R dapat dicapai dalam EMT dengan drop().
\end{eulercomment}
\begin{eulerprompt}
>drop(x,2)
\end{eulerprompt}
\begin{euleroutput}
  [10.4,  3.1,  6.4,  21.7]
\end{euleroutput}
\begin{eulercomment}
Vektor logika tidak diperlakukan secara berbeda dengan indeks di EMT,
berbeda dengan R. Anda harus mengekstrak elemen-elemen yang bukan nol
terlebih dahulu di EMT.
\end{eulercomment}
\begin{eulerprompt}
>x, x>5, x[nonzeros(x>5)]
\end{eulerprompt}
\begin{euleroutput}
  [10.4,  5.6,  3.1,  6.4,  21.7]
  [1,  1,  0,  1,  1]
  [10.4,  5.6,  6.4,  21.7]
\end{euleroutput}
\begin{eulercomment}
Sama seperti di R, vektor indeks dapat berisi pengulangan.
\end{eulercomment}
\begin{eulerprompt}
>x[[1,2,2,1]]
\end{eulerprompt}
\begin{euleroutput}
  [10.4,  5.6,  5.6,  10.4]
\end{euleroutput}
\begin{eulercomment}
Namun pemberian nama untuk indeks tidak dimungkinkan dalam EMT. Untuk
paket statistik, hal ini mungkin sering diperlukan untuk memudahkan
akses ke elemen-elemen vektor.

Untuk meniru perilaku ini, kita dapat mendefinisikan sebuah fungsi
sebagai berikut.
\end{eulercomment}
\begin{eulerprompt}
>function sel (v,i,s) := v[indexof(s,i)]; ...
>s=["first","second","third","fourth"]; sel(x,["first","third"],s)
\end{eulerprompt}
\begin{euleroutput}
  
  Trying to overwrite protected function sel!
  Error in:
  function sel (v,i,s) := v[indexof(s,i)]; ... ...
               ^
  
  Trying to overwrite protected function sel!
  Error in:
  function sel (v,i,s) := v[indexof(s,i)]; ... ...
               ^
  
  Trying to overwrite protected function sel!
  Error in:
  function sel (v,i,s) := v[indexof(s,i)]; ... ...
               ^
  
  Trying to overwrite protected function sel!
  Error in:
  function sel (v,i,s) := v[indexof(s,i)]; ... ...
               ^
  
  Trying to overwrite protected function sel!
  Error in:
  function sel (v,i,s) := v[indexof(s,i)]; ... ...
               ^
  
  Trying to overwrite protected function sel!
  Error in:
  function sel (v,i,s) := v[indexof(s,i)]; ... ...
               ^
  [10.4,  3.1]
\end{euleroutput}
\eulerheading{Tipe Data}
\begin{eulercomment}
EMT memiliki lebih banyak tipe data yang tetap dibandingkan R. Jelas,
dalam R terdapat vektor yang berkembang. Anda bisa mengatur sebuah
vektor numerik kosong v dan memberikan sebuah nilai pada elemen v[17].
Hal ini tidak mungkin dilakukan dalam EMT.

Hal berikut ini sedikit tidak efisien.
\end{eulercomment}
\begin{eulerprompt}
>v=[]; for i=1 to 10000; v=v|i; end;
\end{eulerprompt}
\begin{eulercomment}
EMT sekarang akan membuat vektor dengan v dan i yang ditambahkan pada
tumpukan dan menyalin vektor tersebut kembali ke variabel global v.

Semakin efisien mendefinisikan vektor terlebih dahulu.
\end{eulercomment}
\begin{eulerprompt}
>v=zeros(10000); for i=1 to 10000; v[i]=i; end;
\end{eulerprompt}
\begin{eulercomment}
Untuk mengubah jenis tanggal di EMT, Anda dapat menggunakan fungsi
seperti complex().
\end{eulercomment}
\begin{eulerprompt}
>complex(1:4)
\end{eulerprompt}
\begin{euleroutput}
  [ 1+0i ,  2+0i ,  3+0i ,  4+0i  ]
\end{euleroutput}
\begin{eulercomment}
Konversi ke string hanya dapat dilakukan untuk tipe data dasar. Format
saat ini digunakan untuk penggabungan string sederhana. Tetapi ada
fungsi-fungsi seperti print() atau frac().

Untuk vektor, Anda dapat dengan mudah menulis fungsi Anda sendiri.
\end{eulercomment}
\begin{eulerprompt}
>function tostr (v) ...
\end{eulerprompt}
\begin{eulerudf}
  s="[";
  loop 1 to length(v);
     s=s+print(v[#],2,0);
     if #<length(v) then s=s+","; endif;
  end;
  return s+"]";
  endfunction
\end{eulerudf}
\begin{eulerprompt}
>tostr(linspace(0,1,10))
\end{eulerprompt}
\begin{euleroutput}
  [0.00,0.10,0.20,0.30,0.40,0.50,0.60,0.70,0.80,0.90,1.00]
\end{euleroutput}
\begin{eulercomment}
Untuk komunikasi dengan Maxima, ada sebuah fungsi convertmxm(), yang
juga dapat digunakan untuk memformat vektor untuk output.
\end{eulercomment}
\begin{eulerprompt}
>convertmxm(1:10)
\end{eulerprompt}
\begin{euleroutput}
  [1,2,3,4,5,6,7,8,9,10]
\end{euleroutput}
\begin{eulercomment}
Untuk Latex, perintah tex dapat digunakan untuk mendapatkan perintah
Latex.
\end{eulercomment}
\begin{eulerprompt}
>tex(&[1,2,3])
\end{eulerprompt}
\begin{euleroutput}
  \(\backslash\)left[ 1 , 2 , 3 \(\backslash\)right] 
\end{euleroutput}
\eulerheading{Faktor dan Tabel}
\begin{eulercomment}
Pada pengantar R terdapat sebuah contoh dengan apa yang disebut
faktor.

Berikut ini adalah daftar wilayah dari 30 negara bagian.
\end{eulercomment}
\begin{eulerprompt}
>austates = ["tas", "sa", "qld", "nsw", "nsw", "nt", "wa", "wa", ...
>"qld", "vic", "nsw", "vic", "qld", "qld", "sa", "tas", ...
>"sa", "nt", "wa", "vic", "qld", "nsw", "nsw", "wa", ...
>"sa", "act", "nsw", "vic", "vic", "act"];
\end{eulerprompt}
\begin{eulercomment}
Asumsikan, kita memiliki pendapatan yang sesuai di setiap negara
bagian.
\end{eulercomment}
\begin{eulerprompt}
>incomes = [60, 49, 40, 61, 64, 60, 59, 54, 62, 69, 70, 42, 56, ...
>61, 61, 61, 58, 51, 48, 65, 49, 49, 41, 48, 52, 46, ...
>59, 46, 58, 43];
\end{eulerprompt}
\begin{eulercomment}
Sekarang, kita ingin menghitung rata-rata pendapatan di wilayah
tersebut. Sebagai sebuah program statistik, R memiliki fungsi factor()
dan tappy() untuk hal ini.

EMT dapat melakukan hal ini dengan mencari indeks dari wilayah-wilayah
di dalam daftar unik dari wilayah-wilayah tersebut.
\end{eulercomment}
\begin{eulerprompt}
>auterr=sort(unique(austates)); f=indexofsorted(auterr,austates)
\end{eulerprompt}
\begin{euleroutput}
  [6,  5,  4,  2,  2,  3,  8,  8,  4,  7,  2,  7,  4,  4,  5,  6,  5,  3,
  8,  7,  4,  2,  2,  8,  5,  1,  2,  7,  7,  1]
\end{euleroutput}
\begin{eulercomment}
Pada titik ini, kita dapat menulis fungsi perulangan kita sendiri
untuk melakukan berbagai hal untuk satu faktor saja.

Atau kita dapat meniru fungsi tapply() dengan cara berikut.
\end{eulercomment}
\begin{eulerprompt}
>function map tappl (i; f$:call, cat, x) ...
\end{eulerprompt}
\begin{eulerudf}
  u=sort(unique(cat));
  f=indexof(u,cat);
  return f$(x[nonzeros(f==indexof(u,i))]);
  endfunction
\end{eulerudf}
\begin{eulercomment}
Ini sedikit tidak efisien, karena menghitung wilayah unik untuk setiap
i, tetapi berfungsi.
\end{eulercomment}
\begin{eulerprompt}
>tappl(auterr,"mean",austates,incomes)
\end{eulerprompt}
\begin{euleroutput}
  [44.5,  57.3333333333,  55.5,  53.6,  55,  60.5,  56,  52.25]
\end{euleroutput}
\begin{eulercomment}
Perhatikan bahwa ini bekerja untuk setiap vektor wilayah.
\end{eulercomment}
\begin{eulerprompt}
>tappl(["act","nsw"],"mean",austates,incomes)
\end{eulerprompt}
\begin{euleroutput}
  [44.5,  57.3333333333]
\end{euleroutput}
\begin{eulercomment}
Sekarang, paket statistik EMT mendefinisikan tabel seperti halnya di
R. Fungsi readtable() dan writetable() dapat digunakan untuk input dan
output.

Jadi kita dapat mencetak rata-rata pendapatan negara di wilayah dengan
cara yang ramah.
\end{eulercomment}
\begin{eulerprompt}
>writetable(tappl(auterr,"mean",austates,incomes),labc=auterr,wc=7)
\end{eulerprompt}
\begin{euleroutput}
      act    nsw     nt    qld     sa    tas    vic     wa
     44.5  57.33   55.5   53.6     55   60.5     56  52.25
\end{euleroutput}
\begin{eulercomment}
Kita juga dapat mencoba meniru perilaku R sepenuhnya.

Faktor-faktor tersebut harus disimpan dengan jelas dalam sebuah
koleksi dengan jenis dan kategorinya (negara bagian dan wilayah dalam
contoh kita). Untuk EMT, kita menambahkan indeks yang telah dihitung
sebelumnya.
\end{eulercomment}
\begin{eulerprompt}
>function makef (t) ...
\end{eulerprompt}
\begin{eulerudf}
  ## Factor data
  ## Returns a collection with data t, unique data, indices.
  ## See: tapply
  u=sort(unique(t));
  return \{\{t,u,indexofsorted(u,t)\}\};
  endfunction
\end{eulerudf}
\begin{eulerprompt}
>statef=makef(austates);
\end{eulerprompt}
\begin{eulercomment}
Sekarang elemen ketiga dari koleksi ini akan berisi indeks.
\end{eulercomment}
\begin{eulerprompt}
>statef[3]
\end{eulerprompt}
\begin{euleroutput}
  [6,  5,  4,  2,  2,  3,  8,  8,  4,  7,  2,  7,  4,  4,  5,  6,  5,  3,
  8,  7,  4,  2,  2,  8,  5,  1,  2,  7,  7,  1]
\end{euleroutput}
\begin{eulercomment}
Sekarang kita dapat meniru tapply() dengan cara berikut. Ini akan
mengembalikan sebuah tabel sebagai kumpulan data tabel dan judul
kolom.
\end{eulercomment}
\begin{eulerprompt}
>function tapply (t:vector,tf,f$:call) ...
\end{eulerprompt}
\begin{eulerudf}
  ## Makes a table of data and factors
  ## tf : output of makef()
  ## See: makef
  uf=tf[2]; f=tf[3]; x=zeros(length(uf));
  for i=1 to length(uf);
     ind=nonzeros(f==i);
     if length(ind)==0 then x[i]=NAN;
     else x[i]=f$(t[ind]);
     endif;
  end;
  return \{\{x,uf\}\};
  endfunction
\end{eulerudf}
\begin{eulercomment}
Kami tidak menambahkan banyak pemeriksaan tipe di sini. Satu-satunya
tindakan pencegahan adalah kategori (faktor) yang tidak memiliki data.
Tetapi kita harus memeriksa panjang t yang benar dan kebenaran koleksi
tf.

Tabel ini bisa dicetak sebagai sebuah tabel dengan writetable().
\end{eulercomment}
\begin{eulerprompt}
>writetable(tapply(incomes,statef,"mean"),wc=7)
\end{eulerprompt}
\begin{euleroutput}
      act    nsw     nt    qld     sa    tas    vic     wa
     44.5  57.33   55.5   53.6     55   60.5     56  52.25
\end{euleroutput}
\eulerheading{Larik}
\begin{eulercomment}
EMT hanya memiliki dua dimensi untuk array. Tipe datanya disebut
matriks. Akan lebih mudah untuk menulis fungsi untuk dimensi yang
lebih tinggi atau pustaka C untuk ini.

R memiliki lebih dari dua dimensi. Dalam R, larik adalah sebuah vektor
dengan sebuah bidang dimensi.

Dalam EMT, sebuah vektor adalah sebuah matriks dengan satu baris. Ini
bisa dibuat menjadi sebuah matriks dengan redim().
\end{eulercomment}
\begin{eulerprompt}
>shortformat; X=redim(1:20,4,5)
\end{eulerprompt}
\begin{euleroutput}
          1         2         3         4         5 
          6         7         8         9        10 
         11        12        13        14        15 
         16        17        18        19        20 
\end{euleroutput}
\begin{eulercomment}
Ekstraksi baris dan kolom, atau sub-matriks, sama seperti di R.
\end{eulercomment}
\begin{eulerprompt}
>X[,2:3]
\end{eulerprompt}
\begin{euleroutput}
          2         3 
          7         8 
         12        13 
         17        18 
\end{euleroutput}
\begin{eulercomment}
Namun, dalam R dimungkinkan untuk mengatur daftar indeks tertentu dari
vektor ke suatu nilai. Hal yang sama juga dapat dilakukan dalam EMT
hanya dengan sebuah perulangan.
\end{eulercomment}
\begin{eulerprompt}
>function setmatrixvalue (M, i, j, v) ...
\end{eulerprompt}
\begin{eulerudf}
  loop 1 to max(length(i),length(j),length(v))
     M[i\{#\},j\{#\}] = v\{#\};
  end;
  endfunction
\end{eulerudf}
\begin{eulercomment}
Kami mendemonstrasikan hal ini untuk menunjukkan bahwa matriks
diteruskan dengan referensi dalam EMT. Jika Anda tidak ingin mengubah
matriks asli M, Anda perlu menyalinnya dalam fungsi.
\end{eulercomment}
\begin{eulerprompt}
>setmatrixvalue(X,1:3,3:-1:1,0); X,
\end{eulerprompt}
\begin{euleroutput}
          1         2         0         4         5 
          6         0         8         9        10 
          0        12        13        14        15 
         16        17        18        19        20 
\end{euleroutput}
\begin{eulercomment}
Hasil kali luar dalam EMT hanya dapat dilakukan di antara vektor. Hal
ini otomatis karena bahasa matriks. Satu vektor harus berupa vektor
kolom dan vektor baris.
\end{eulercomment}
\begin{eulerprompt}
>(1:5)*(1:5)'
\end{eulerprompt}
\begin{euleroutput}
          1         2         3         4         5 
          2         4         6         8        10 
          3         6         9        12        15 
          4         8        12        16        20 
          5        10        15        20        25 
\end{euleroutput}
\begin{eulercomment}
Dalam pengantar PDF untuk R ada sebuah contoh, yang menghitung
distribusi ab-cd untuk a, b, c, d yang dipilih dari 0 sampai n secara
acak. Solusinya dalam R adalah membentuk sebuah matriks 4 dimensi dan
menjalankan table() di atasnya.

Tentu saja, ini bisa dicapai dengan sebuah perulangan. Tetapi
perulangan tidak efektif dalam EMT atau R. Dalam EMT, kita bisa
menulis perulangan dalam C dan itu adalah solusi tercepat.

Tetapi kita ingin meniru perilaku R. Untuk ini, kita perlu meratakan
perkalian ab dan membuat sebuah matriks ab-cd.
\end{eulercomment}
\begin{eulerprompt}
>a=0:6; b=a'; p=flatten(a*b); q=flatten(p-p'); ...
>u=sort(unique(q)); f=getmultiplicities(u,q); ...
>statplot(u,f,"h"):
\end{eulerprompt}
\begin{eulercomment}
Selain kelipatan yang tepat, EMT dapat menghitung frekuensi dalam
vektor.
\end{eulercomment}
\begin{eulerprompt}
>getfrequencies(q,-50:10:50)
\end{eulerprompt}
\begin{euleroutput}
  [0,  23,  132,  316,  602,  801,  333,  141,  53,  0]
\end{euleroutput}
\begin{eulercomment}
Cara yang paling mudah untuk memplot ini sebagai distribusi adalah
sebagai berikut.
\end{eulercomment}
\begin{eulerprompt}
>plot2d(q,distribution=11):
\end{eulerprompt}
\begin{eulercomment}
Tetapi juga memungkinkan untuk menghitung jumlah dalam interval yang
dipilih sebelumnya. Tentu saja, berikut ini menggunakan
getfrequencies() secara internal.

Karena fungsi histo() mengembalikan frekuensi, kita perlu
menskalakannya sehingga integral di bawah grafik batang adalah 1.
\end{eulercomment}
\begin{eulerprompt}
>\{x,y\}=histo(q,v=-55:10:55); y=y/sum(y)/differences(x); ...
>plot2d(x,y,>bar,style="/"):
\end{eulerprompt}
\eulerheading{Daftar}
\begin{eulercomment}
EMT memiliki dua jenis daftar. Pertama adalah daftar global yang dapat
diubah, dan yang kedua adalah jenis daftar yang tidak dapat diubah.
Kita tidak peduli dengan daftar global di sini.

Tipe daftar yang tidak dapat diubah disebut koleksi dalam EMT. Ia
berperilaku seperti struktur dalam C, tetapi elemen-elemennya hanya
diberi nomor dan tidak diberi nama.
\end{eulercomment}
\begin{eulerprompt}
>L=\{\{"Fred","Flintstone",40,[1990,1992]\}\}
\end{eulerprompt}
\begin{euleroutput}
  Fred
  Flintstone
  40
  [1990,  1992]
\end{euleroutput}
\begin{eulercomment}
Saat ini elemen-elemen tersebut tidak memiliki nama, meskipun nama
dapat ditetapkan untuk tujuan khusus. Elemen-elemen tersebut diakses
dengan angka.
\end{eulercomment}
\begin{eulerprompt}
>(L[4])[2]
\end{eulerprompt}
\begin{euleroutput}
  1992
\end{euleroutput}
\begin{eulercomment}
\begin{eulercomment}
\eulerheading{Input dan Output File (Membaca dan Menulis Data)}
\begin{eulercomment}
Anda mungkin sering ingin mengimpor matriks data dari sumber lain ke
EMT. Tutorial ini akan menjelaskan kepada Anda tentang berbagai cara
untuk melakukan hal tersebut. Fungsi yang sederhana adalah
writematrix() dan readmatrix().

Mari kita tunjukkan bagaimana cara membaca dan menulis sebuah vektor
real ke sebuah file.
\end{eulercomment}
\begin{eulerprompt}
>a=random(1,100); mean(a), dev(a),
\end{eulerprompt}
\begin{euleroutput}
  0.49815
  0.28037
\end{euleroutput}
\begin{eulercomment}
Untuk menulis data ke sebuah berkas, kita menggunakan fungsi
writematrix().

Karena pengenalan ini kemungkinan besar berada di sebuah direktori, di
mana pengguna tidak memiliki akses tulis, kita menulis data ke
direktori home pengguna. Untuk notebook sendiri, hal ini tidak
diperlukan, karena file data akan ditulis ke dalam direktori yang
sama.
\end{eulercomment}
\begin{eulerprompt}
>filename="test.dat";
\end{eulerprompt}
\begin{eulercomment}
Sekarang kita tuliskan vektor kolom a' ke dalam file. Hal ini akan
menghasilkan satu angka pada setiap baris file.
\end{eulercomment}
\begin{eulerprompt}
>writematrix(a',filename);
\end{eulerprompt}
\begin{eulercomment}
Untuk membaca data, kita menggunakan readmatrix().
\end{eulercomment}
\begin{eulerprompt}
>a=readmatrix(filename)';
\end{eulerprompt}
\begin{eulercomment}
Dan hapus file tersebut.
\end{eulercomment}
\begin{eulerprompt}
>fileremove(filename);
>mean(a), dev(a),
\end{eulerprompt}
\begin{euleroutput}
  0.49815
  0.28037
\end{euleroutput}
\begin{eulercomment}
Fungsi writematrix() atau writetable() dapat dikonfigurasi untuk
bahasa lain.

Sebagai contoh, jika Anda memiliki sistem bahasa Indonesia (titik
desimal dengan koma), Excel Anda membutuhkan nilai dengan koma desimal
yang dipisahkan oleh titik koma dalam file csv (defaultnya adalah
nilai yang dipisahkan dengan koma). File “test.csv” berikut ini akan
muncul di folder cuurent Anda.
\end{eulercomment}
\begin{eulerprompt}
>filename="test.csv"; ...
>writematrix(random(5,3),file=filename,separator=",");
\end{eulerprompt}
\begin{eulercomment}
Anda sekarang dapat membuka file ini dengan Excel Indonesia secara
langsung.
\end{eulercomment}
\begin{eulerprompt}
>fileremove(filename);
\end{eulerprompt}
\begin{eulercomment}
Terkadang kita memiliki string dengan token seperti berikut ini.
\end{eulercomment}
\begin{eulerprompt}
>s1:="f m m f m m m f f f m m f";  ...
>s2:="f f f m m f f";
\end{eulerprompt}
\begin{eulercomment}
Untuk menandai ini, kita mendefinisikan vektor token.
\end{eulercomment}
\begin{eulerprompt}
>tok:=["f","m"]
\end{eulerprompt}
\begin{euleroutput}
  f
  m
\end{euleroutput}
\begin{eulercomment}
Kemudian kita dapat menghitung berapa kali setiap token muncul dalam
string, dan memasukkan hasilnya ke dalam tabel.
\end{eulercomment}
\begin{eulerprompt}
>M:=getmultiplicities(tok,strtokens(s1))_ ...
>  getmultiplicities(tok,strtokens(s2));
\end{eulerprompt}
\begin{eulercomment}
Tulis tabel dengan tajuk token.
\end{eulercomment}
\begin{eulerprompt}
>writetable(M,labc=tok,labr=1:2,wc=8)
\end{eulerprompt}
\begin{euleroutput}
                 f       m
         1       6       7
         2       5       2
\end{euleroutput}
\begin{eulercomment}
Untuk statika, EMT dapat membaca dan menulis tabel.
\end{eulercomment}
\begin{eulerprompt}
>file="test.dat"; open(file,"w"); ...
>writeln("A,B,C"); writematrix(random(3,3)); ...
>close();
\end{eulerprompt}
\begin{eulercomment}
File terlihat seperti ini.
\end{eulercomment}
\begin{eulerprompt}
>printfile(file)
\end{eulerprompt}
\begin{euleroutput}
  A,B,C
  0.7003664386138074,0.1875530821001213,0.3262339279660414
  0.5926249243193858,0.1522927283984059,0.368140583062521
  0.8065535209872989,0.7265910840408142,0.7332619844597152
  
\end{euleroutput}
\begin{eulercomment}
Fungsi readtable() dalam bentuknya yang paling sederhana dapat membaca
ini dan mengembalikan sebuah koleksi nilai dan baris judul.
\end{eulercomment}
\begin{eulerprompt}
>L=readtable(file,>list);
\end{eulerprompt}
\begin{eulercomment}
Koleksi ini dapat dicetak dengan writetable() ke buku catatan, atau ke
sebuah file.
\end{eulercomment}
\begin{eulerprompt}
>writetable(L,wc=10,dc=5)
\end{eulerprompt}
\begin{euleroutput}
           A         B         C
     0.70037   0.18755   0.32623
     0.59262   0.15229   0.36814
     0.80655   0.72659   0.73326
\end{euleroutput}
\begin{eulercomment}
Matriks nilai adalah elemen pertama dari L. Perhatikan bahwa mean()
dalam EMT menghitung nilai rata-rata dari baris-baris matriks.
\end{eulercomment}
\begin{eulerprompt}
>mean(L[1])
\end{eulerprompt}
\begin{euleroutput}
    0.40472 
    0.37102 
    0.75547 
\end{euleroutput}
\eulerheading{File CSV}
\begin{eulercomment}
Pertama, mari kita tulis sebuah matriks ke dalam sebuah file. Untuk
keluarannya, kami membuat file di direktori kerja saat ini.
\end{eulercomment}
\begin{eulerprompt}
>file="test.csv";  ...
>M=random(3,3); writematrix(M,file);
\end{eulerprompt}
\begin{eulercomment}
Berikut ini adalah isi file ini.
\end{eulercomment}
\begin{eulerprompt}
>printfile(file)
\end{eulerprompt}
\begin{euleroutput}
  0.8221197733097619,0.821531098722547,0.7771240608094004
  0.8482947121863489,0.3237767724883862,0.6501422353377985
  0.1482301827518109,0.3297459716109594,0.6261901074210923
  
\end{euleroutput}
\begin{eulercomment}
CVS ini dapat dibuka di sistem bahasa Inggris ke Excel dengan klik dua
kali. Jika Anda mendapatkan file seperti itu pada sistem Jerman, Anda
perlu mengimpor data ke Excel dengan memperhatikan titik desimal.

Namun, titik desimal juga merupakan format default untuk EMT. Anda
dapat membaca sebuah matriks dari sebuah file dengan readmatrix().
\end{eulercomment}
\begin{eulerprompt}
>readmatrix(file)
\end{eulerprompt}
\begin{euleroutput}
    0.82212   0.82153   0.77712 
    0.84829   0.32378   0.65014 
    0.14823   0.32975   0.62619 
\end{euleroutput}
\begin{eulercomment}
Dimungkinkan untuk menulis beberapa matriks ke dalam satu file.
Perintah open() dapat membuka file untuk menulis dengan parameter “w”.
Standarnya adalah “r” untuk membaca.
\end{eulercomment}
\begin{eulerprompt}
>open(file,"w"); writematrix(M); writematrix(M'); close();
\end{eulerprompt}
\begin{eulercomment}
Matriks-matriks tersebut dipisahkan oleh sebuah baris kosong. Untuk
membaca matriks, buka file dan panggil readmatrix() beberapa kali.
\end{eulercomment}
\begin{eulerprompt}
>open(file); A=readmatrix(); B=readmatrix(); A==B, close();
\end{eulerprompt}
\begin{euleroutput}
          1         0         0 
          0         1         0 
          0         0         1 
\end{euleroutput}
\begin{eulercomment}
Di Excel atau spreadsheet serupa, Anda dapat mengekspor matriks
sebagai CSV (nilai yang dipisahkan dengan koma). Pada Excel 2007,
gunakan “save as” dan “format lain”, lalu pilih “CSV”. Pastikan, tabel
saat ini hanya berisi data yang ingin Anda ekspor.

Berikut ini adalah contohnya.
\end{eulercomment}
\begin{eulerprompt}
>printfile("excel-data.csv")
\end{eulerprompt}
\begin{euleroutput}
  0;1000;1000
  1;1051,271096;1072,508181
  2;1105,170918;1150,273799
  3;1161,834243;1233,67806
  4;1221,402758;1323,129812
  5;1284,025417;1419,067549
  6;1349,858808;1521,961556
  7;1419,067549;1632,31622
  8;1491,824698;1750,6725
  9;1568,312185;1877,610579
  10;1648,721271;2013,752707
\end{euleroutput}
\begin{eulercomment}
Seperti yang Anda lihat, sistem Jerman saya menggunakan titik koma
sebagai pemisah dan koma desimal. Anda dapat mengubahnya di pengaturan
sistem atau di Excel, tetapi tidak perlu untuk membaca matriks ke
dalam EMT.

Cara termudah untuk membaca ini ke dalam Euler adalah readmatrix().
Semua koma digantikan oleh titik dengan parameter \textgreater{}comma. Untuk CSV
bahasa Inggris, hilangkan saja parameter ini.
\end{eulercomment}
\begin{eulerprompt}
>M=readmatrix("excel-data.csv",>comma)
\end{eulerprompt}
\begin{euleroutput}
          0      1000      1000 
          1    1051.3    1072.5 
          2    1105.2    1150.3 
          3    1161.8    1233.7 
          4    1221.4    1323.1 
          5      1284    1419.1 
          6    1349.9      1522 
          7    1419.1    1632.3 
          8    1491.8    1750.7 
          9    1568.3    1877.6 
         10    1648.7    2013.8 
\end{euleroutput}
\begin{eulercomment}
Mari kita rencanakan ini.
\end{eulercomment}
\begin{eulerprompt}
>plot2d(M'[1],M'[2:3],>points,color=[red,green]'):
\end{eulerprompt}
\begin{eulercomment}
Ada beberapa cara yang lebih mendasar untuk membaca data dari file.
Anda dapat membuka file dan membaca angka baris demi baris. Fungsi
getvectorline() akan membaca angka dari sebuah baris data. Secara
default, fungsi ini mengharapkan sebuah titik desimal. Tetapi fungsi
ini juga dapat menggunakan koma desimal, jika Anda memanggil
setdecimaldot(“,”) sebelum menggunakan fungsi ini.

Fungsi berikut ini adalah contohnya. Fungsi ini akan berhenti pada
akhir file atau baris kosong.
\end{eulercomment}
\begin{eulerprompt}
>function myload (file) ...
\end{eulerprompt}
\begin{eulerudf}
  open(file);
  M=[];
  repeat
     until eof();
     v=getvectorline(3);
     if length(v)>0 then M=M_v; else break; endif;
  end;
  return M;
  close(file);
  endfunction
\end{eulerudf}
\begin{eulerprompt}
>myload(file)
\end{eulerprompt}
\begin{euleroutput}
    0.82212   0.82153   0.77712 
    0.84829   0.32378   0.65014 
    0.14823   0.32975   0.62619 
\end{euleroutput}
\begin{eulercomment}
Anda juga dapat membaca semua angka dalam file tersebut dengan
getvector().
\end{eulercomment}
\begin{eulerprompt}
>open(file); v=getvector(10000); close(); redim(v[1:9],3,3)
\end{eulerprompt}
\begin{euleroutput}
    0.82212   0.82153   0.77712 
    0.84829   0.32378   0.65014 
    0.14823   0.32975   0.62619 
\end{euleroutput}
\begin{eulercomment}
Dengan demikian, sangat mudah untuk menyimpan vektor nilai, satu nilai
di setiap baris dan membaca kembali vektor ini.
\end{eulercomment}
\begin{eulerprompt}
>v=random(1000); mean(v)
\end{eulerprompt}
\begin{euleroutput}
  0.50303
\end{euleroutput}
\begin{eulerprompt}
>writematrix(v',file); mean(readmatrix(file)')
\end{eulerprompt}
\begin{euleroutput}
  0.50303
\end{euleroutput}
\eulerheading{Menggunakan Tabel}
\begin{eulercomment}
Tabel dapat digunakan untuk membaca atau menulis data numerik. Sebagai
contoh, kita menulis tabel dengan judul baris dan kolom ke file.
\end{eulercomment}
\begin{eulerprompt}
>file="test.tab"; M=random(3,3);  ...
>open(file,"w");  ...
>writetable(M,separator=",",labc=["one","two","three"]);  ...
>close(); ...
>printfile(file)
\end{eulerprompt}
\begin{euleroutput}
  one,two,three
        0.09,      0.39,      0.86
        0.39,      0.86,      0.71
         0.2,      0.02,      0.83
\end{euleroutput}
\begin{eulercomment}
File ini dapat diimpor ke Excel.

Untuk membaca file di EMT, kita menggunakan readtable().
\end{eulercomment}
\begin{eulerprompt}
>\{M,headings\}=readtable(file,>clabs); ...
>writetable(M,labc=headings)
\end{eulerprompt}
\begin{euleroutput}
         one       two     three
        0.09      0.39      0.86
        0.39      0.86      0.71
         0.2      0.02      0.83
\end{euleroutput}
\eulerheading{Menganalisis Garis}
\begin{eulercomment}
Anda bahkan dapat mengevaluasi setiap baris dengan tangan. Misalkan,
kita memiliki baris dengan format berikut.
\end{eulercomment}
\begin{eulerprompt}
>line="2020-11-03,Tue,1'114.05"
\end{eulerprompt}
\begin{euleroutput}
  2020-11-03,Tue,1'114.05
\end{euleroutput}
\begin{eulercomment}
First we can tokenize the line.
\end{eulercomment}
\begin{eulerprompt}
>vt=strtokens(line)
\end{eulerprompt}
\begin{euleroutput}
  2020-11-03
  Tue
  1'114.05
\end{euleroutput}
\begin{eulercomment}
Kemudian, kita dapat mengevaluasi setiap elemen garis dengan
menggunakan evaluasi yang sesuai.
\end{eulercomment}
\begin{eulerprompt}
>day(vt[1]),  ...
>indexof(["mon","tue","wed","thu","fri","sat","sun"],tolower(vt[2])),  ...
>strrepl(vt[3],"'","")()
\end{eulerprompt}
\begin{euleroutput}
  7.3816e+05
  2
  1114
\end{euleroutput}
\begin{eulercomment}
Dengan menggunakan ekspresi reguler, Anda dapat mengekstrak hampir
semua informasi dari sebuah baris data.

Anggaplah kita memiliki baris dokumen HTML berikut ini.
\end{eulercomment}
\begin{eulerprompt}
>line="<tr><td>1145.45</td><td>5.6</td><td>-4.5</td><tr>"
\end{eulerprompt}
\begin{euleroutput}
  <tr><td>1145.45</td><td>5.6</td><td>-4.5</td><tr>
\end{euleroutput}
\begin{eulercomment}
Untuk mengekstrak ini, kita menggunakan ekspresi reguler, yang mencari

\end{eulercomment}
\begin{eulerttcomment}
 - tanda kurung tutup >,
 - setiap string yang tidak mengandung tanda kurung dengan
\end{eulerttcomment}
\begin{eulercomment}
sub-pencocokan “(...)”,\\
\end{eulercomment}
\begin{eulerttcomment}
 - kurung pembuka dan kurung penutup menggunakan solusi terpendek,
 - sekali lagi, semua string yang tidak mengandung tanda kurung,
 - dan sebuah kurung pembuka <.
\end{eulerttcomment}
\begin{eulercomment}

Ekspresi reguler agak sulit untuk dipelajari tetapi sangat kuat.
\end{eulercomment}
\begin{eulerprompt}
>\{pos,s,vt\}=strxfind(line,">([^<>]+)<.+?>([^<>]+)<");
\end{eulerprompt}
\begin{eulercomment}
Hasilnya adalah posisi kecocokan, string yang cocok, dan vektor string
untuk sub-cocokan.
\end{eulercomment}
\begin{eulerprompt}
>for k=1:length(vt); vt[k](), end;
\end{eulerprompt}
\begin{euleroutput}
  1145.5
  5.6
\end{euleroutput}
\begin{eulercomment}
Berikut ini adalah fungsi yang membaca semua item numerik antara \textless{}td\textgreater{}
dan \textless{}/td\textgreater{}.
\end{eulercomment}
\begin{eulerprompt}
>function readtd (line) ...
\end{eulerprompt}
\begin{eulerudf}
  v=[]; cp=0;
  repeat
     \{pos,s,vt\}=strxfind(line,"<td.*?>(.+?)</td>",cp);
     until pos==0;
     if length(vt)>0 then v=v|vt[1]; endif;
     cp=pos+strlen(s);
  end;
  return v;
  endfunction
\end{eulerudf}
\begin{eulerprompt}
>readtd(line+"<td>non-numerical</td>")
\end{eulerprompt}
\begin{euleroutput}
  1145.45
  5.6
  -4.5
  non-numerical
\end{euleroutput}
\eulerheading{Membaca dari Web}
\begin{eulercomment}
Situs web atau file dengan URL dapat dibuka di EMT dan dapat dibaca
baris demi baris.

Dalam contoh, kita membaca versi saat ini dari situs EMT. Kami
menggunakan ekspresi reguler untuk memindai “Versi ...” dalam judul.
\end{eulercomment}
\begin{eulerprompt}
>function readversion () ...
\end{eulerprompt}
\begin{eulerudf}
  urlopen("http://www.euler-math-toolbox.de/Programs/Changes.html");
  repeat
    until urleof();
    s=urlgetline();
    k=strfind(s,"Version ",1);
    if k>0 then substring(s,k,strfind(s,"<",k)-1), break; endif;
  end;
  urlclose();
  endfunction
\end{eulerudf}
\begin{eulerprompt}
>readversion
\end{eulerprompt}
\begin{euleroutput}
  Version 2024-01-12
\end{euleroutput}
\eulerheading{Masukan dan Keluaran Variabel}
\begin{eulercomment}
Anda dapat menulis variabel dalam bentuk definisi Euler ke file atau
ke baris perintah.
\end{eulercomment}
\begin{eulerprompt}
>writevar(pi,"mypi");
\end{eulerprompt}
\begin{euleroutput}
  mypi = 3.141592653589793;
\end{euleroutput}
\begin{eulercomment}
Untuk pengujian, kami membuat file Euler di direktori kerja EMT.
\end{eulercomment}
\begin{eulerprompt}
>file="test.e"; ...
>writevar(random(2,2),"M",file); ...
>printfile(file,3)
\end{eulerprompt}
\begin{euleroutput}
  M = [ ..
  0.5991820585590205, 0.7960280262224293;
  0.5167243983231363, 0.2996684599070898];
\end{euleroutput}
\begin{eulercomment}
Sekarang kita dapat memuat file tersebut. Ini akan mendefinisikan
matriks M.
\end{eulercomment}
\begin{eulerprompt}
>load(file); show M,
\end{eulerprompt}
\begin{euleroutput}
  M = 
    0.59918   0.79603 
    0.51672   0.29967 
\end{euleroutput}
\begin{eulercomment}
Sebagai catatan, jika writevar() digunakan pada sebuah variabel, maka
ia akan mencetak definisi variabel dengan nama variabel tersebut.
\end{eulercomment}
\begin{eulerprompt}
>writevar(M); writevar(inch$)
\end{eulerprompt}
\begin{euleroutput}
  M = [ ..
  0.5991820585590205, 0.7960280262224293;
  0.5167243983231363, 0.2996684599070898];
  inch$ = 0.0254;
\end{euleroutput}
\begin{eulercomment}
Kita juga dapat membuka file baru atau menambahkan ke file yang sudah
ada. Dalam contoh ini, kami menambahkan ke file yang telah dibuat
sebelumnya..
\end{eulercomment}
\begin{eulerprompt}
>open(file,"a"); ...
>writevar(random(2,2),"M1"); ...
>writevar(random(3,1),"M2"); ...
>close();
>load(file); show M1; show M2;
\end{eulerprompt}
\begin{euleroutput}
  M1 = 
    0.30287   0.15372 
     0.7504   0.75401 
  M2 = 
    0.27213 
   0.053211 
    0.70249 
\end{euleroutput}
\begin{eulercomment}
Untuk menghapus file, gunakan fileremove().
\end{eulercomment}
\begin{eulerprompt}
>fileremove(file);
\end{eulerprompt}
\begin{eulercomment}
Sebuah vektor baris dalam sebuah file tidak membutuhkan koma, jika
setiap angka berada dalam baris baru. Mari kita buat file seperti itu,
dengan menulis setiap baris satu per satu dengan writeln().
\end{eulercomment}
\begin{eulerprompt}
>open(file,"w"); writeln("M = ["); ...
>for i=1 to 5; writeln(""+random()); end; ...
>writeln("];"); close(); ...
>printfile(file)
\end{eulerprompt}
\begin{euleroutput}
  M = [
  0.344851384551
  0.0807510017715
  0.876519562911
  0.754157709472
  0.688392638934
  ];
\end{euleroutput}
\begin{eulerprompt}
>load(file); M
\end{eulerprompt}
\begin{euleroutput}
  [0.34485,  0.080751,  0.87652,  0.75416,  0.68839]
\end{euleroutput}
\eulersubheading{Latihan Soal}
\begin{eulercomment}
1. Analisis data statistika deskriptif
\end{eulercomment}
\begin{eulerprompt}
>mean([50,70,76,75,77,68,45,30,15,71,69,84,15,88,90])
\end{eulerprompt}
\begin{euleroutput}
  61.533
\end{euleroutput}
\begin{eulerprompt}
>data=[485, 360, 490, 900, 721, 840];
>urutan=sort(data)
\end{eulerprompt}
\begin{euleroutput}
  [360,  485,  490,  721,  840,  900]
\end{euleroutput}
\begin{eulerprompt}
>median([urutan])
\end{eulerprompt}
\begin{euleroutput}
  605.5
\end{euleroutput}
\begin{eulerprompt}
>data=[5, 9, 6, 10, 7, 8, 9,4,6,7];
>urut=sort(data)
\end{eulerprompt}
\begin{euleroutput}
  [4,  5,  6,  6,  7,  7,  8,  9,  9,  10]
\end{euleroutput}
\begin{eulerprompt}
>xbar=mean(urut)
\end{eulerprompt}
\begin{euleroutput}
  7.1
\end{euleroutput}
\begin{eulerprompt}
>dev= urut-xbar 
\end{eulerprompt}
\begin{euleroutput}
  [-3.1,  -2.1,  -1.1,  -1.1,  -0.1,  -0.1,  0.9,  1.9,  1.9,  2.9]
\end{euleroutput}
\begin{eulerprompt}
>varians=mean(dev^2)
\end{eulerprompt}
\begin{euleroutput}
  3.29
\end{euleroutput}
\begin{eulerprompt}
>data=[75,66,90,98,80,70,75,55,50,90];
>urut=sort(data)
\end{eulerprompt}
\begin{euleroutput}
  [50,  55,  66,  70,  75,  75,  80,  90,  90,  98]
\end{euleroutput}
\begin{eulerprompt}
>x=mean(urut)
\end{eulerprompt}
\begin{euleroutput}
  74.9
\end{euleroutput}
\begin{eulerprompt}
>dev=urut-x
\end{eulerprompt}
\begin{euleroutput}
  [-24.9,  -19.9,  -8.9,  -4.9,  0.1,  0.1,  5.1,  15.1,  15.1,  23.1]
\end{euleroutput}
\begin{eulerprompt}
>varians=mean(dev^2)
\end{eulerprompt}
\begin{euleroutput}
  213.49
\end{euleroutput}
\begin{eulerprompt}
>simpanganbaku= sqrt(varians)
\end{eulerprompt}
\begin{euleroutput}
  14.611
\end{euleroutput}
\begin{eulerprompt}
>x=[30,60,78,90,96,74,55,79,90]; max(x)- min(x)
\end{eulerprompt}
\begin{euleroutput}
  66
\end{euleroutput}
\begin{eulerprompt}
>data=[7,3,8,5,9,4,8,3,10,2,7,6,8,7,2,8,9,7,9,15,8,9];
>urut=sort(data)
\end{eulerprompt}
\begin{euleroutput}
  [2,  2,  3,  3,  4,  5,  6,  7,  7,  7,  7,  8,  8,  8,  8,  8,  9,  9,
  9,  9,  10,  15]
\end{euleroutput}
\begin{eulerprompt}
>quartiles(urut)
\end{eulerprompt}
\begin{euleroutput}
  [2,  5,  7.5,  9,  15]
\end{euleroutput}
\begin{eulercomment}
2. Uji statistik\\
contoh 1:\\
diberikan lemparan dadu acak sebanyak 1800 kali, tentukan nilai
chi-kuadratnya apabila nilai harapan dadunya memiliki distribusi
seragam
\end{eulercomment}
\begin{eulerprompt}
>throws=1800; sides=6;
>randomDice = intrandom(throws, sides);
>freqDice = multofsorted(sort(randomDice), 1:sides);
>expectDice = dup(throws/sides, sides)';
>chitest(freqDice, expectDice)
\end{eulerprompt}
\begin{euleroutput}
  0.84915
\end{euleroutput}
\begin{eulercomment}
contoh 2:\\
Diberikan hasil dari lemparan koin sebagai berikut, dengan G adalah
gambar dan A adalah angka

G, A, A, G, A, A, A, G, A, A

Apakah koin tersebut adil?
\end{eulercomment}
\begin{eulerprompt}
>coinRes = "G, A, A, G, A, A, A, G, A, A";
>function chisquarecustom(x, y)...
\end{eulerprompt}
\begin{eulerudf}
  sumSquare = 0;
  for i=1 to length(x) step i;
     sumSquareCurr = (x[i] - y[i])^2 / y[i];
     sumSquare = sumSquare + sumSquareCurr;
  end;
  return sumSquare;
  endfunction
\end{eulerudf}
\begin{eulerprompt}
>function replaceTokenToArr(strToken, separateToken, arrToken, arrSubsToken)...
\end{eulerprompt}
\begin{eulerudf}
  tokens = strtokens(strToken, separateToken);
  transformedArr = [];
  for i = 1 to length(tokens) step i
     token = tokens[i];
     indexToken = indexof(arrToken, token);
     subs = arrSubsToken[indexToken];
     transformedArr = [transformedArr, subs];
  end;
  return transformedArr;
  endfunction
\end{eulerudf}
\begin{eulerprompt}
>coinDis = replaceTokenToArr(coinRes, ", ", ["G", "A"], [1, 2])
\end{eulerprompt}
\begin{euleroutput}
  [1,  2,  2,  1,  2,  2,  2,  1,  2,  2]
\end{euleroutput}
\begin{eulerprompt}
>function getFrequencies(arr)...
\end{eulerprompt}
\begin{eulerudf}
  uniqueArr = unique(arr);
  sortArr = sort(arr);
  firstIndexes = [];
  freqs = [];
  for i = 1 to length(uniqueArr) step i
     currUnique = uniqueArr[i];
     currFirstIndex = indexof(sortArr, currUnique) - 1;
     firstIndexes = [firstIndexes,  currFirstIndex];
  end;
  firstIndexes = [firstIndexes, length(arr)];
  for i = 1 to (length(firstIndexes) - 1) step i
     currIndex = firstIndexes[i];
     nextIndex = firstIndexes[i + 1];
     currFreq = nextIndex - currIndex;
     freqs = [freqs, currFreq];
  end;
  return freqs;
  endfunction
\end{eulerudf}
\begin{eulerprompt}
>coinFreq = getFrequencies(coinDis)
\end{eulerprompt}
\begin{euleroutput}
  [3,  7]
\end{euleroutput}
\begin{eulerprompt}
>coinExpected = dup(sum(coinFreq)/length(coinFreq), length(coinFreq))'
\end{eulerprompt}
\begin{euleroutput}
  [5,  5]
\end{euleroutput}
\begin{eulerprompt}
>chitest(coinFreq, coinExpected)
\end{eulerprompt}
\begin{euleroutput}
  0.2059
\end{euleroutput}
\begin{eulercomment}
Soal

Penelitian dilakukan untuk mengetahui partisipasi warga dalam suatu
kepala desa yang dilihat dari jenis kelamin. Hasil penelitian tersebut
diperoleh data sebagai berikut

Jenis Kelamin \textbar{} Ikut \textbar{} Tidak\\
Pria          \textbar{} 15   \textbar{} 5\\
Wanita        \textbar{} 11   \textbar{} 6

Apakah ada pengaruh jenis kelamin terhadap keikutseraan dalam
pemilihan kepala desa?
\end{eulercomment}
\begin{eulerprompt}
>absenceGenders = [15,5;11,6]
\end{eulerprompt}
\begin{euleroutput}
         15         5 
         11         6 
\end{euleroutput}
\begin{eulercomment}
Buat tabel agar mudah melihatnya
\end{eulercomment}
\begin{eulerprompt}
>writetable(absenceGenders, wc = 15, labr=["Pria", "Wanita"], labc = ["Ikut", "Tidak"])
\end{eulerprompt}
\begin{euleroutput}
                            Ikut          Tidak
             Pria             15              5
           Wanita             11              6
\end{euleroutput}
\begin{eulercomment}
Memeriksa tabel untuk uji chi-kuadratnya
\end{eulercomment}
\begin{eulerprompt}
>tabletest(absenceGenders)
\end{eulerprompt}
\begin{euleroutput}
  0.49478
\end{euleroutput}
\begin{eulercomment}
Karena nilai terlalu besar, maka belum bisa menyimpulkan apakah
terdapat hubungan antara keikutsertaan dengan gender

Lalu, dibuat tabel nilai harapan untuk melihat nilai harapan pada
tabel tersebut
\end{eulercomment}
\begin{eulerprompt}
>writetable(expectedtable(absenceGenders), dc=1, wc=15, labr=["Pria", "Wanita"], labc=["Ikut", "Tidak"])
\end{eulerprompt}
\begin{euleroutput}
                            Ikut          Tidak
             Pria           14.1            5.9
           Wanita           11.9            5.1
\end{euleroutput}
\begin{eulercomment}
Terakhir, adalah menghitung nilai koefisien kontingensi
\end{eulercomment}
\begin{eulerprompt}
>contingency(absenceGenders)
\end{eulerprompt}
\begin{euleroutput}
  0.15774
\end{euleroutput}
\begin{eulercomment}
Contoh

diberikan nilai dua kelas berbeda, yaitu\\
Kelas A: 20, 30, 40, 50, 25, 35, 45, 55, 65, 45\\
Kelas B: 45, 74, 88, 91, 56, 68, 91, 31, 65, 68

Apakah nilai tengah antar keduanya sama?

Penyelesaian

Rumusan Hipotesis\\
Parameter populasi yang akan diuji adalah nilai antar kelas

\end{eulercomment}
\begin{eulerformula}
\[
\begin{array}{cc} \text{Hipotesis nol} & H_0: \mu_a = \mu_b \\ \text{Hipotesis alternatif} & H_1: \mu_a \ne \mu_b \\ \end{array}
\]
\end{eulerformula}
\begin{eulerprompt}
>function replaceTokenToArr2(str, replacement)...
\end{eulerprompt}
\begin{eulerudf}
  tokens = strtokens(str, replacement);
  transformedArr = [];
  for i = 1 to length(tokens) step i
     token = evaluate(tokens[i]);
     transformedArr = [transformedArr, token];
  end;
  return transformedArr;
  endfunction
\end{eulerudf}
\begin{eulerprompt}
>classAstr = "20, 30, 40, 50, 25, 35, 45, 55, 65, 45";
>classBstr = "45, 74, 88, 91, 56, 68, 91, 31, 65, 68";
>classA = replaceTokenToArr2(classAstr, ", ")
\end{eulerprompt}
\begin{euleroutput}
  [20,  30,  40,  50,  25,  35,  45,  55,  65,  45]
\end{euleroutput}
\begin{eulerprompt}
>classB = replaceTokenToArr2(classBstr, ", ")
\end{eulerprompt}
\begin{euleroutput}
  [45,  74,  88,  91,  56,  68,  91,  31,  65,  68]
\end{euleroutput}
\begin{eulerprompt}
>mediantest(classA, classB)
\end{eulerprompt}
\begin{euleroutput}
  0.011507
\end{euleroutput}
\begin{eulercomment}
Kesimpulan dari return fungsi:\\
Karena nilai return fungsi \textgreater{} 0.05 maka hipotesis nol diterima, artinya
bahwa nilai tengah dari kedua kelas sama.
\end{eulercomment}
\begin{eulerprompt}
>ranktest(classA, classB)
\end{eulerprompt}
\begin{euleroutput}
  0.0022932
\end{euleroutput}
\begin{eulercomment}
Untuk melakukan pengujian lebih baik, dapat menggunakan fungsi
ranktest

dimana hasil dari atas menunjukkan bahwa nilai tengah kedua kelas sama
Karena nilai koefisien kontingensi mendekati 0, maka dapat disimpulkan
bahwa hubungan antara keikutsertaan dengan jenis kelamin tidak ada
(sangat lemah). Karena nilai error dari chi-kuadratnya kecil, maka
kita dapat menolak hipotesis bahwa koin tersebut adil (setimbang).

3. Menggambar Grafik Statistika


Contoh Soal\\
Diketahui data ukuran sepatu siswa di suatu sekolahan
\end{eulercomment}
\begin{eulerprompt}
>A=[40,42,45,45,41,38,39,40,42,44,35,36,38,39,40,41,39,38,37,36,40,43,42,41,39,37,38,39,44,39,41,43]
\end{eulerprompt}
\begin{euleroutput}
  [40,  42,  45,  45,  41,  38,  39,  40,  42,  44,  35,  36,  38,  39,
  40,  41,  39,  38,  37,  36,  40,  43,  42,  41,  39,  37,  38,  39,
  44,  39,  41,  43]
\end{euleroutput}
\begin{eulerprompt}
>boxplot(A):
\end{eulerprompt}
\begin{eulercomment}
Penjelasan diagram kotak tersebut adalah. Dapat diketahui nilai
minimum sebesar 35, Q1=38, Median (Q2) = 40, Q3=42. Dan nilai
maksimumnya adalah 45. Sebaran data tersebut juga diketahui simetris,
karena si kuartil 1 dan 2 jaraknya sama dengan kuartil 2 dan 3.

Contoh lain perbandingan 15 simulasi 500 nilai terdistribusi normal
menggunakan box plot dan ditemukan pencilan sbb
\end{eulercomment}
\begin{eulerprompt}
>p=normal(15,500); boxplot(p):
\end{eulerprompt}
\begin{eulercomment}
Contoh soal : \\
Kita akan membuat diagram batang secara random 
\end{eulercomment}
\begin{eulerprompt}
>columnsplot(cumsum(random(6)),style="/",color=red):
>columnsplot(cumsum(random(15)),style="-",color=black):
>columnsplot(cumsum(random(3)),style="|",color=orange):
\end{eulerprompt}
\begin{eulercomment}
Kita akan membuat diagram batang dari penjualan warung watashi 
\end{eulercomment}
\begin{eulerprompt}
>day=["Senin","Selasa","Rabu","Kamis","Jumat","Sabtu","Minggu"];
>values=[30,40,20,10,50,60,70];
>columnsplot(values,lab=day,color=yellow);
>title("Data Penjualan Perhari di Toko Watashi")
>insimg
>CP:=[rgb(0.5,0.5,0.5),red,yellow,green,rgb(0.9,0,0)]
\end{eulerprompt}
\begin{euleroutput}
  [5.8753e+07,  2,  15,  3,  6.5405e+07]
\end{euleroutput}
\begin{eulerprompt}
>i=[1,2,3,4,5]; piechart(values[i],color=CP[i],lab=day[i]):
>starplot(normal(1,15)+16,lab=1:15,>rays):
>starplot(values,lab=day,>rays):
>plot2d(makeimpulse(1:20,random(1,20)),>bar):
>logimpulseplot(1:20,-log(random(1,20))*10):
>aspect(1); plot2d(random(100),>histogram):
>r=150:5:185; v=[22,71,136,150,139,71,32];
>plot2d(r,v,a=150,b=185,c=0,d=150,bar=1,style="/"):
>plot2d("qnormal(x,0,1)",-5,5); ...
>plot2d("qnormal(x,0,1)",a=1,b=4,>add,>filled):
\end{eulerprompt}
\begin{eulercomment}
Contoh kurva fungsi distribusi kumulatif kontinu terdiri atas tiga
bagian yaitu \\
1. Bernilai 0 untuk x dibawah minimal dari daerah rentang\\
2. Merupakan fungsi monoton naik pada daerah rentang \\
3. Mempunyai nilai konstan 1 diatas batas maksimum daerah rentangnya 

\end{eulercomment}
\begin{eulerprompt}
>x=normal(1,6);
\end{eulerprompt}
\begin{eulercomment}
Baris tersebut akan menghasilkan suatu nilai acak dari distribusi
normal dengan mean 1 dan standar deviasi 6, dan nilai tsb disimpan
dalam var x.
\end{eulercomment}
\begin{eulerprompt}
>xs=sort(x);
>plot2d("empdist",-3,5;xs):
\end{eulerprompt}
\begin{eulercomment}
Grafik fungsi distribusi kumulatif peubah acak diskrit merupakan
fungsi tangga dengan terendah 0 dan tertinggi 1. 
\end{eulercomment}
\begin{euleroutput}
  
\end{euleroutput}
\eulerheading{Rujukan Lengkap Fungsi plot2d()}
\begin{eulercomment}
\end{eulercomment}
\begin{eulerttcomment}
  function plot2d (xv, yv, btest, a, b, c, d, xmin, xmax, r, n,  ..
  logplot, grid, frame, framecolor, square, color, thickness, style, ..
  auto, add, user, delta, points, addpoints, pointstyle, bar, histogram,  ..
  distribution, even, steps, own, adaptive, hue, level, contour,  ..
  nc, filled, fillcolor, outline, title, xl, yl, maps, contourcolor, ..
  contourwidth, ticks, margin, clipping, cx, cy, insimg, spectral,  ..
  cgrid, vertical, smaller, dl, niveau, levels)
\end{eulerttcomment}
\begin{eulercomment}
Multipurpose plot function for plots in the plane (2D plots). This function can do
plots of functions of one variables, data plots, curves in the plane, bar plots, grids
of complex numbers, and implicit plots of functions of two variables.

Parameters
\\
x,y       : equations, functions or data vectors\\
a,b,c,d   : Plot area (default a=-2,b=2)\\
r         : if r is set, then a=cx-r, b=cx+r, c=cy-r, d=cy+r\\
\end{eulercomment}
\begin{eulerttcomment}
            r can be a vector [rx,ry] or a vector [rx1,rx2,ry1,ry2].
\end{eulerttcomment}
\begin{eulercomment}
xmin,xmax : range of the parameter for curves\\
auto      : Determine y-range automatically (default)\\
square    : if true, try to keep square x-y-ranges\\
n         : number of intervals (default is adaptive)\\
grid      : 0 = no grid and labels,\\
\end{eulercomment}
\begin{eulerttcomment}
            1 = axis only,
            2 = normal grid (see below for the number of grid lines)
            3 = inside axis
            4 = no grid
            5 = full grid including margin
            6 = ticks at the frame
            7 = axis only
            8 = axis only, sub-ticks
\end{eulerttcomment}
\begin{eulercomment}
frame     : 0 = no frame\\
framecolor: color of the frame and the grid\\
margin    : number between 0 and 0.4 for the margin around the plot\\
color     : Color of curves. If this is a vector of colors,\\
\end{eulercomment}
\begin{eulerttcomment}
            it will be used for each row of a matrix of plots. In the case of
            point plots, it should be a column vector. If a row vector or a
            full matrix of colors is used for point plots, it will be used for
            each data point.
\end{eulerttcomment}
\begin{eulercomment}
thickness : line thickness for curves\\
\end{eulercomment}
\begin{eulerttcomment}
            This value can be smaller than 1 for very thin lines.
\end{eulerttcomment}
\begin{eulercomment}
style     : Plot style for lines, markers, and fills.\\
\end{eulercomment}
\begin{eulerttcomment}
            For points use
            "[]", "<>", ".", "..", "...",
            "*", "+", "|", "-", "o"
            "[]#", "<>#", "o#" (filled shapes)
            "[]w", "<>w", "ow" (non-transparent)
            For lines use
            "-", "--", "-.", ".", ".-.", "-.-", "->"
            For filled polygons or bar plots use
            "#", "#O", "O", "/", "\(\backslash\)", "\(\backslash\)/",
            "+", "|", "-", "t"
\end{eulerttcomment}
\begin{eulercomment}
points    : plot single points instead of line segments\\
addpoints : if true, plots line segments and points\\
add       : add the plot to the existing plot\\
user      : enable user interaction for functions\\
delta     : step size for user interaction\\
bar       : bar plot (x are the interval bounds, y the interval values)\\
histogram : plots the frequencies of x in n subintervals\\
distribution=n : plots the distribution of x with n subintervals\\
even      : use inter values for automatic histograms.\\
steps     : plots the function as a step function (steps=1,2)\\
adaptive  : use adaptive plots (n is the minimal number of steps)\\
level     : plot level lines of an implicit function of two variables\\
outline   : draws boundary of level ranges.
\\
If the level value is a 2xn matrix, ranges of levels will be drawn\\
in the color using the given fill style. If outline is true, it\\
will be drawn in the contour color. Using this feature, regions of\\
f(x,y) between limits can be marked.
\\
hue       : add hue color to the level plot to indicate the function\\
\end{eulercomment}
\begin{eulerttcomment}
            value
\end{eulerttcomment}
\begin{eulercomment}
contour   : Use level plot with automatic levels\\
nc        : number of automatic level lines\\
title     : plot title (default "")\\
xl, yl    : labels for the x- and y-axis\\
smaller   : if \textgreater{}0, there will be more space to the left for labels.\\
vertical  :\\
\end{eulercomment}
\begin{eulerttcomment}
  Turns vertical labels on or off. This changes the global variable
  verticallabels locally for one plot. The value 1 sets only vertical
  text, the value 2 uses vertical numerical labels on the y axis.
\end{eulerttcomment}
\begin{eulercomment}
filled    : fill the plot of a curve\\
fillcolor : fill color for bar and filled curves\\
outline   : boundary for filled polygons\\
logplot   : set logarithmic plots\\
\end{eulercomment}
\begin{eulerttcomment}
            1 = logplot in y,
            2 = logplot in xy,
            3 = logplot in x
\end{eulerttcomment}
\begin{eulercomment}
own       :\\
\end{eulercomment}
\begin{eulerttcomment}
  A string, which points to an own plot routine. With >user, you get
  the same user interaction as in plot2d. The range will be set
  before each call to your function.
\end{eulerttcomment}
\begin{eulercomment}
maps      : map expressions (0 is faster), functions are always mapped.\\
contourcolor : color of contour lines\\
contourwidth : width of contour lines\\
clipping  : toggles the clipping (default is true)\\
title     :\\
\end{eulercomment}
\begin{eulerttcomment}
  This can be used to describe the plot. The title will appear above
  the plot. Moreover, a label for the x and y axis can be added with
  xl="string" or yl="string". Other labels can be added with the
  functions label() or labelbox(). The title can be a unicode
  string or an image of a Latex formula.
\end{eulerttcomment}
\begin{eulercomment}
cgrid     :\\
\end{eulercomment}
\begin{eulerttcomment}
  Determines the number of grid lines for plots of complex grids.
  Should be a divisor of the the matrix size minus 1 (number of
  subintervals). cgrid can be a vector [cx,cy].
\end{eulerttcomment}
\begin{eulercomment}

Overview

The function can plot

- expressions, call collections or functions of one variable,\\
- parametric curves,\\
- x data against y data,\\
- implicit functions,\\
- bar plots,\\
- complex grids,\\
- polygons.

If a function or expression for xv is given, plot2d() will compute\\
values in the given range using the function or expression. The\\
expression must be an expression in the variable x. The range must\\
be defined in the parameters a and b unless the default range\\
[-2,2] should be used. The y-range will be computed automatically,\\
unless c and d are specified, or a radius r, which yields the range\\
[-r,r] for x and y. For plots of functions, plot2d will use an\\
adaptive evaluation of the function by default. To speed up the\\
plot for complicated functions, switch this off with \textless{}adaptive, and\\
optionally decrease the number of intervals n. Moreover, plot2d()\\
will by default use mapping. I.e., it will compute the plot element\\
for element. If your expression or your functions can handle a\\
vector x, you can switch that off with \textless{}maps for faster evaluation.

Note that adaptive plots are always computed element for element. \\
If functions or expressions for both xv and for yv are specified,\\
plot2d() will compute a curve with the xv values as x-coordinates\\
and the yv values as y-coordinates. In this case, a range should be\\
defined for the parameter using xmin, xmax. Expressions contained
\end{eulercomment}
\begin{eulerudf}
  
\end{eulerudf}
\end{eulernotebook}
\end{document}
